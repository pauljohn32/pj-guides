%%%%%%%%%%%%%%%%%%%%%%%%%%%%%%%%%%%%%%%%%%%%%%%%%%%%%%%%%%%%%%%%%
% Filename: template.Rnw
%   Author: Paul Johnson
%
%%%%%%%%%%%%%%%%%%%%%%%%%%%%%%%%%%%%%%%%%%%%%%%%%%%%%%%%%%%%%%%%%
%%%%%%%%%%%%%%%%%%%%%%%%%%%%%%%%%%%%%%%%%%%%%%%%%%%%%%%%%%%%%%%%%
\documentclass[10pt,english]{beamer}
% use 'handout' to produce handouts
%\documentclass[handout]{beamer}

\usepackage{lmodern}
\renewcommand{\sfdefault}{lmss}
\renewcommand{\ttdefault}{lmtt}
\usepackage[T1]{fontenc}
\usepackage[latin9]{inputenc}
\usepackage{listings}
\setcounter{secnumdepth}{3}
\setcounter{tocdepth}{3}
\usepackage{url}
\usepackage{graphicx}

\makeatletter
%%%%%%%%%%%%%%%%%%%%%%%%%%%%%% Textclass specific LaTeX commands.
\usepackage{Sweavel}
 \newenvironment{topcolumns}{\begin{columns}[t]}{\end{columns}}

%%%%%%%%%%%%%%%%%%%%%%%%%%%%%% User specified LaTeX commands.
\usepackage{dcolumn}
\usepackage{booktabs}


\usepackage{wasysym}
\usepackage{pgfpages}
\newcommand{\vn}[1]{\mbox{{\it #1}}}\newcommand{\vb}{\vspace{\baselineskip}}\newcommand{\vh}{\vspace{.5\baselineskip}}\newcommand{\vf}{\vspace{\fill}}\newcommand{\splus}{\textsf{S-PLUS}}\newcommand{\R}{\textsf{R}}


\usepackage{graphicx}
\usepackage{listings}
\lstset{tabsize=2, breaklines=true,style=Rstyle}
%\usetheme{Warsaw}
% or ...

%\setbeamercovered{transparent}
% or whatever (possibly just delete it)

\mode<presentation>
{
  \usetheme{KU}
  \usecolortheme{dolphin} %dark blues
}

%=============================================================================
%



\title[CLT I] % (short title, use only with long paper titles)
{Central Limit Theorem}

\subtitle{The Deepest Thought Ever Thunk}

\author[Johnson] {Paul E. Johnson\inst{1,2}}

\institute[University of Kansas]{\inst{1} Department of Political Science \\
  University of Kansas \and \inst{2} Center for Research Methods and Data Analysis \\ University of Kansas} % (optional, but mostly needed)

\date[2020] % (optional, should be abbreviation of conference name)
{\today}

\subject{CLT}

%=============================================================================
%=============================================================================
\begin{document}

% In document Latex options:
\fvset{listparameters={\setlength{\topsep}{0em}}}
\def\Sweavesize{\normalsize}
\def\Rcolor{\color{black}}
\def\Rbackground{\color[gray]{0.95}}


% In document Latex options:
\fvset{listparameters={\setlength{\topsep}{0em}}}

\def\Sweavesize{\normalsize}
\def\Rcolor{\color{black}}
\def\Rbackground{\color[gray]{0.95}}

\begin{Schunk}
\begin{Sinput}
 ## Other settings I like
 options(device = pdf)
 options(useFancyQuotes = FALSE) 
 op <- par() 
 pjmar <- c(5.1, 5.1, 1.5, 2.1) 
 options(SweaveHooks=list(fig=function() par(mar=pjmar, ps=12)))
 pdf.options(onefile=F,family="Times",pointsize=12)
\end{Sinput}
\end{Schunk}





\begin{frame}
  \titlepage
\end{frame}



\begin{frame}
\frametitle{Outline}

\tableofcontents{}

\end{frame}









%____________________________________
%\begin{frame}
%  \frametitle{You Want to Write R Data To Excel?}
%  \begin{itemize}
%    \item That seems like a dubious mission
%    \item Google reports several proposed packages for Writing to Excel
%  \end{itemize}
%\end{frame}
%
%____________________________________

\section{The Difference Between A Sample and The Truth}

\begin{frame}
  \frametitle{We think of one survey, one estimate}

  \begin{itemize}
  \item Nature has a ``data generating
    mechanism:''
  \item Nature's probability density function is never fully
    revealed to us, we only get samples.
  \item Samples flucturate: no two
    samples are the same.
  \item From that one sample, we try to
    want to answer a LOT of questions.
    \begin{itemize}
    \item we calculate an \textbf{estimate}: a single
      number that represents something.
    \item We estimate a distribution's Expected Value, Variance,
      or other parameters
    \item Develop a model of the PDF of the estimator.  Almost NEVER
      are we interested in estimating Natures PDF that generates the
      data. Almost Always, we want to know the PDF of the estimator.
    \end{itemize}
  \end{itemize}
\end{frame}


\begin{frame}
  \frametitle{Normal Distribution PDF depends on $\mu$ and $\sigma^2$}
  \begin{columns}%{}

    \column{8cm}

    \includegraphics[width=8cm]{importfigs/Normal-2009}

    \column{4cm}

    \begin{itemize}
    \item Single Peaked
    \item Symmetric
    \item $E[x]=\mu$
    \item $Var[x]=\sigma^{2}$
    \item $SD[x]=\sigma$
    \end{itemize}
\end{columns}

\end{frame}


\begin{frame}
  \frametitle{$\mu$ and $\sigma^2$ are Parameters}
  \begin{itemize}
    \item Every distribution can have its ``own letters'' for parameters
    \item For generality, refer to them as $\theta$
    \item I say: The estimates from sample data have hats:
      \begin{displaymath}
        \hat{\mu}\ \ \ \ \ \ \widehat{\sigma^2}
      \end{displaymath}
    \item Some people prefer notation like:  $\bar{x}$ and $s^2$.
  \end{itemize}
\end{frame}


\begin{frame}[containsverbatim]
  \frametitle{The Theoretical PDF Is This:}

\includegraphics{figs/t-simple20}
\end{frame}



\begin{frame}[containsverbatim]
  \frametitle{Draw one Normal Sample from $N(5.353, 4.55^{2})$ }

\includegraphics{figs/t-simple10a}
\end{frame}



\begin{frame}[containsverbatim]
  \frametitle{But the Observed Sample (Kernel) Density Differs}

\includegraphics{figs/t-simple30}
\end{frame}

\begin{frame}[containsverbatim]
  \frametitle{Histogram with PDF and KDE Superimposed}

\includegraphics{figs/t-simple40}
\end{frame}


\begin{frame}[containsverbatim]
  \frametitle{Draw another}

\includegraphics{figs/t-simple10b}
\end{frame}



\begin{frame}[containsverbatim]
  \frametitle{Draw another}

\includegraphics{figs/t-simple10c}
\end{frame}


\section{Sampling Distribution}


\begin{frame}
  \frametitle{Important Term: Sampling Distribution}
  \begin{itemize}
  \item Definition: Sampling Distribution is the PDF of an estimator. like $\bar{x}$
  \item The ``true'' sampling distribution is a theoretical ``thing''
    (process?). Like the data generating process, it is never observed.
  \item Math can give us some ``exact'' characterizations about
    sampling distributions
  \item We can simulate repeated-sampling to visualize sampling distributions.
  \end{itemize}
\end{frame}


\begin{frame}
  \frametitle{Visualize a Sampling Distribution Via Simulation}
  \begin{itemize}
    \item Suppose you could repeatedly draw samples
    \item Calculate an estimate from each sample
      \begin{itemize}
        \item Perhaps $\hat{\mu} \equiv \bar{x}$ (the mean) for each sample
        \item More generally, any $\hat{\theta}$
   \item Create a histogram of those observed estimates
   \item We want to know
     \begin{itemize}
      \item Are the estimates close to the ``true'' value?
      \item Are the estimates symmetrically distributed?
      \item Are there any abstract patters worth finding in these
        distributions of estimates?
      \end{itemize}
    \end{itemize}
  \end{itemize}
\end{frame}

\begin{frame}
  \frametitle{General Claims about the Sampling Distribution of $\bar{x}$}
  This is true for the Normal distribution, AND ALL OTHER
  DISTRIBUTIONS we will work with!
  \begin{itemize}
  \item If the expected value is of $x$ is $\mu$, the expected value of the mean
    of a sample is also $\mu$.
    \begin{displaymath}
      If \ E[x] = \mu, then E[\bar{x}]=\mu
    \end{displaymath}

  \item If the variance of $x$ is $\sigma^2$, the Variance of the
    sampling distribution of the mean is $\frac{1}{N}Var[x]$
    \begin{displaymath}
      If Var[x] = \sigma^{2}, then Var[\bar{x}]=\frac{Var[x]}{N}
    \end{displaymath}
  \item Which implies $SD[\bar{x}]=\frac{SD[x]}{\sqrt(N)}$
  \end{itemize}
\end{frame}



\begin{frame}
  \frametitle{In Other Words...}

  The distribution of $\bar{x}$
  \begin{itemize}
  \item Is Centered on the same spot as $x_{i}$
  \item But $\bar{x}$ is clustered much  more ``tightly'
    than the distribution of $x_{i}$ itself.
  \end{itemize}

  That's impossibly easy to see

  \begin{itemize}
  \item Algebraically.
  \item By simulation.
  \end{itemize}

  I've moved the algebraic proof to the end of these notes, but
  have just one comment about it on the next 2 slides.
\end{frame}


\begin{frame}
  \frametitle{Spotlight on one Tricky Bit: One Observation $x_i$ Has Expected Value and
    Variance!}
  \begin{itemize}
  \item Think of a ``variable'' as one single observation from a distribution
    \begin{equation}
      x_i
    \end{equation}
  \item In past, we discussed $x={x_1,x_2,\ldots,x_N}$ as a collection
    of observations. Easy to think of the ``mean'' or ``variance'' of
    sample and expected value $E[x]$ and variance $Var[x]$
  \item We said $x$ is normally distributed, thinking of $x$ as a
    variety of outcomes. In your mind, a ``histogram'' with PDF curve.
    \item We can't calculate a sample mean from a single observation,
      but that one observation still has an ``expected value'' because
      its drawn from a data generating process.
  \end{itemize}
\end{frame}


\begin{frame}
  \frametitle{Spotlight on one Tricky Bit: One Observation $x_i$ Has Expected Value and
    Variance!}
  \begin{itemize}
  \item Now think of $x_1$, $x_2$ and so forth as separate variates
    from the same distribution.
  \item Appeal to Intuition. Each individual draw has the same
    expected value. So $E[x] = E[x_1] = E[x_2]= \ldots E[x_N]$
  \item Similarly, each draw has same variance.
  \item It should be obvious how we derive the claim that the expected
    value of an average is the expected value.
    \begin{displaymath}
      E[\bar{x}] = E[\frac{x_1 + x_2 + \ldots + x_N}{N}]
    \end{displaymath}

  \end{itemize}
\end{frame}


\begin{frame}
  \frametitle{And the Variance of the Estimated Mean is Manageable as well}

  \begin{itemize}
    \item Again, we are supposing we know $Var(x) = \sigma_{x}^2$.
    \item If we calculate the average of a sample,
      \begin{displaymath}
        \bar{x} = \frac{x_1 + x_2 + \ldots + x_N}{N}
      \end{displaymath}
    \item Apply the variance operator to both sides
      \begin{displaymath}
        Var(\bar{x}) = Var(\frac{x_1 + x_2 + \ldots + x_N}{N})
      \end{displaymath}
      its pretty easy to get the result we want.

   \end{itemize}
 \end{frame}


 \begin{frame}[allowframebreaks]
   \frametitle{The Variance of a Sampling Distribution is REALLY Important}
   \begin{itemize}

   \item We don't care about means, in particular. We care about
     all kinds of parameter estimates, $\hat{\theta}$

   \item We want to have precise estimates

   \item The route to a small variance of the estimate is especially
     clear in the case of the estimated mean:
     \begin{displaymath}
       Var(\bar{x}) = \frac{1}{N}Var(x).
     \end{displaymath}
     But not all estimators have such a clear, simple
     formula that makes it easy to see how to reduce the estimator's variance

     % \begin{itemize}

     % \item  If we had the PDF of the estimate $\hat{\theta}$, we could
     %   calculate the variance.

     % \item If $\hat{\theta}$ were a discrete variable, this would be
     %   easy to write down. By analogy to sample variance, the
     %   variance of $\hat{\theta}$ would be the
     %   average of squared errors (aka ``mean square error'')

     %   \begin{displaymath}
     %     Var(\hat{\theta}) = \sum Prob.(\hat{\theta_j})(\hat{\theta}_{j}-E[\hat{\theta}])^{2}\label{eq:-7}
     %   \end{displaymath}

     %   where $Prob()$ is the probability of observing a particular estimate.

     % \item Estimator distributions are not usually discrete, but the main idea is the same:
     %   estimators have uncertainty, we want to know how much.
     % \end{itemize}

   \item For just a few kinds of parameter estimates, we can actually
     know  $Var(\hat{\theta})$

   \item Most often, we have to estimate $Var(\hat{\theta})$. I'm
     entertained by the two hat notation:
     \begin{displaymath}
      \widehat{Var(\hat{\theta})}
    \end{displaymath}
    an estimate of our uncertainty about of an estimate.

  \item In Regression analysis, we won't write $\widehat{Var(\hat{\theta})}$
    very often. We instead talk about its square root, which by
    custom is called
    \begin{displaymath}
      s.e.(\hat{theta}): the standard\ error\ of\ \hat{\theta}
    \end{displaymath}
    i.e., standard error is an estimate of the standard deviation of a
    sampling distribution
  \end{itemize}
\end{frame}


\begin{frame}
  \frametitle{The Distribution of the Mean is ``Spike-ish''}

Please observe the illustration of the effect of sample size on the
variance of $\bar{x}$.

\end{frame}


\input{figs/t-p10}

\begin{frame}

\frametitle{Distribution of $x \sim Normal(0, 3^{2})$}

\includegraphics{figs/t-normal1}
\end{frame}



\begin{frame}[containsverbatim]
 \frametitle{Distribution of Mean, Sample=100 ($Normal(0,3^{2}/100)$)}
\includegraphics{figs/t-normal2}
\end{frame}



\begin{frame}
 \frametitle{Distribution of Mean, Sample=2000 ($Normal(0,3^{2}/2000)$)}
\includegraphics{figs/t-normal3}
\end{frame}


\section{Asymptotic Properties}

\begin{frame}
  \frametitle{Terms}
  \begin{itemize}
  \item Asymptotic: related to very large (tending to infinite)
    sample sizes
  \item Consistency: an estimator (formula's result) 'tends to' the
    correct value as sample size tends to infinity
  \end{itemize}
\end{frame}

\begin{frame}
 \frametitle{Law of Large Numbers}

As the Sample Size Increases, $\bar{x}$ tends to the Expected Value
(The True Mean)

This is the {}``law of large numbers''.

\end{frame}

\section{The Central Limit Theorem}

\begin{frame}
  \frametitle{The Basic Idea of the CLT}
  \begin{itemize}
    \item For ANY DISTRIBUTION (not just the normal) of $x$, the
      distribution of $\bar{x}$ approaches a normal distribution
      as the size of the sample upon which $\bar{x}$ is calculated
      tends to infinity.
    \item This one is difficult to prove algebraically, but it is
      quite easy to demonstrate with simulation
  \end{itemize}
\end{frame}


\begin{frame}
  \frametitle{The CLT with 0,1 data}

  Draw 50 observations where the probability of success on each one is 0.30.

\includegraphics[width=8cm]{figs/t-bin10}


\end{frame}




\begin{frame}
  \frametitle{Draw Another Sample}

\includegraphics[width=8cm]{figs/t-bin10b}


\end{frame}



\begin{frame}
  \frametitle{Draw Another Sample}

\includegraphics[width=8cm]{figs/t-bin10c}


\end{frame}



\begin{frame}
  \frametitle{Draw Another Sample}

\includegraphics[width=8cm]{figs/t-bin10d}


\end{frame}


\begin{frame}
  \frametitle{What about those means?}

  \begin{itemize}
    \item Do that over and over again.
    \item what do you guess the distribution of the means would look like?
    \item I'll make a guess.  It will be tightly clustered around
      ``0.30'' and it will be normally distributed
  \end{itemize}

\end{frame}




\begin{frame}
  \frametitle{Lots of means from 0,1 data}
100 samples, each including 50 random draws
\includegraphics[width=8cm]{figs/t-bin20a}

\end{frame}





\begin{frame}
  \frametitle{Lots of means from 0,1 data}
1000 samples, each including 2000 random draws
\includegraphics[width=8cm]{figs/t-bin20b}

\end{frame}



\begin{frame}
 \frametitle{Take the Poisson Distribution for another example}

\includegraphics{figs/t-p30}
\end{frame}






\begin{frame}
 \frametitle{Poisson(3), SampleSize=10 }
 \includegraphics{figs/t-pois10}
\end{frame}


\begin{frame}
 \frametitle{Poisson(3), SampleSize=100 }
 \includegraphics{figs/t-pois20}
\end{frame}


\begin{frame}
 \frametitle{Poisson(3), SampleSize=2000 }

 \includegraphics{figs/t-pois30}
\end{frame}


\begin{frame}
 \frametitle{Poisson(3), SampleSize=10000 }
 \includegraphics{figs/t-pois40}
\end{frame}


\begin{frame}
 \frametitle{Means of 1000 Poisson Samples, Sample Size 10\label{fig:PoissonMeans}.}

 \includegraphics{figs/t-pois81}

\end{frame}

\begin{frame}
  \frametitle{Means from 1000 Poissons, Sample Size=100}
 \includegraphics{figs/t-pois82}

\end{frame}

\begin{frame}
  \frametitle{Means from 1000 Poisson samples, Sample Size=2000}

 \includegraphics{figs/t-pois83}

\end{frame}

\begin{frame}
  \frametitle{Means from 1000 Poisson samples, Sample Size=10000}

 \includegraphics{figs/t-pois84}
\end{frame}



\begin{frame}
  \frametitle{Same thing, bigger picture (N=10000)}

 \includegraphics{figs/t-pois85}
\end{frame}


\begin{frame}
 \frametitle{Consider the Uniform Distribution}

\includegraphics{figs/t-U10}
\end{frame}


\begin{frame}
  \frametitle{Means from 1000 Uniform samples, Sample Size=30}

 \includegraphics{figs/t-u21}
\end{frame}


\begin{frame}
  \frametitle{Means from 1000 Uniform samples, Sample Size=500}

 \includegraphics{figs/t-u22}
\end{frame}


\begin{frame}
  \frametitle{Means from 1000 Uniform samples, Sample Size=2000}

 \includegraphics{figs/t-u23}
\end{frame}



\begin{frame}
 \frametitle{OK, Challenge Me With Your Beta(0.9,0.9)}

\includegraphics{figs/t-b10}
\end{frame}

\begin{frame}
  \frametitle{Means from 1000 Beta Samples, Sample Size=2000}

\includegraphics{figs/t-b23}
\end{frame}



\begin{frame}
 \frametitle{My Mantra}

From whatever distribution you pick, the Central Limit Theorem (CLT)
says the {}``Sampling Distribution of the Mean is Normal''.

\end{frame}

\section{Next Step}
\begin{frame}
  \frametitle{The CLT generalizes to any sum of random variables}
  \begin{itemize}
  \item We are not interested primarily in the distribution of the
    mean
  \item But we have many other estimators that are weighted sums
    of random variables.
  \item Example: the estimated slope of a regression line
  \end{itemize}
\end{frame}

\begin{frame}
  \frametitle{What is the Benefit of the CLT?}
  \begin{itemize}
  \item Consider an estimator $\hat{\theta}$ that follows the CLT.
  \item Suppose the $E[\hat{\theta}]$ = 0.
    \begin{itemize}
      \item Why? We usually think of $\hat{\theta}$ as fluctuations around an
      estimator around the true value. $\hat{\theta}$ is "unbiased".
    \end{itemize}
  \item IF (IF IF) we knew the standard deviation of the sampling
    distribution, then this would be a ``standardized Normal variable''
    \begin{equation}
       \frac{\hat{\theta}}{true\, std.dev.(\hat{\theta})}
    \end{equation}
  \item I mean, it would be $N(0,1)$, a VERY manageable quantity.
  \end{itemize}
\end{frame}



\begin{frame}
  \frametitle{T Distribution Fix: If we don't know the true standard deviation of $\hat{\theta}$ }
  \begin{itemize}
  \item Divide this (which is $N(0,1)$)
    \begin{equation}
      \frac{\hat{\theta}}{true\,std.dev(\hat{\theta})}
    \end{equation}
  \item By this

\begin{equation}
\frac{\widehat{std.dev(\hat{\theta})}}{true\, std.dev.(\hat{\theta})}
\end{equation}
  \item Gosset proved that the ratio follows a distribution that we
    now call T.  T depends on the number of cases used to calculate
    the estimate, a number we call degrees of freedom.
  \end{itemize}
\end{frame}


\begin{frame}
  \frametitle{Division cancels the unknown true $std.dev.(\hat{\theta})$ }

  \begin{equation}
\frac{\frac{\hat{\theta}}{true\, std.dev(\hat{\theta})}}{\frac{\widehat{std.dev(\hat{\theta})}}{true\, std.dev.(\hat{\theta})}}
\end{equation}
\begin{itemize}
  \item After division, it is
    \begin{equation}
      \frac{\hat{\theta}}{\widehat{std.dev(\hat{\theta}})}
    \end{equation}
  \item We give a special name to the estimated standard deviation of
    the sampling distribution. It is called \emph{standard error.}
  \end{itemize}
\end{frame}

\begin{frame}
  \frametitle{$\hat{\theta}/std.err.(\hat{\theta})$ is Everpresent in Stats}
  \begin{itemize}
  \item the T distribution is almost like the Normal(0,1).
  \item If the degrees of freedom is large (more than 1000), T and
    N(0,1) are virtually identical.
  \item Thus, the range $\hat{\theta} \pm
    1.96*std.err.(\hat{theta})$ contains about 95\% of the distribution
  \item By implication, outcomes outside that 95\% region are deemed
    ``unusual'' (2.5\% of cases at either tail of distribution)
  \item If degrees of freedom is smaller, we just replace 1.96 with a
    slightly larger magic number (see ``T table'').
  \end{itemize}
\end{frame}



\begin{frame}
  \frametitle{For Other Estimators, Much Detailed Research is Required }
  \begin{itemize}
  \item We (applied social scientists) usually don't have training
    or interest in developing new math for sampling distributions.
  \item We do have some simulation tools to approximate unknown
    sampling distributions
  \item Simulation based ideas
    \begin{itemize}
    \item Bootstrap: draw many samples from the data sample,
      re-calculate the estimate for each. The resulting
      distribution may approximate the sampling distribution.
    \item Markov Chain Monte Carlo (MCMC): a computer simulation
      model developed during WWII as a way to explore complex
      probability models. Described in my review essay.
    \end{itemize}
  \end{itemize}
\end{frame}


\section{Proof}

\subsection{Expected Value}
\begin{frame}
\frametitle{The Algebraic Argument for $E[\bar{x}]=E[x]$}

The average (estimate of mean) of a sample ${x_1, x_2, x_3, \ldots,x_N}$ is:

\begin{equation}
\bar{x}=\frac{1}{N} \sum_{i}^{N} x_{i}\label{eq:-2}
\end{equation}

If we have data on the frequency of each possible score $x_{j}$, calculate proportions

\begin{equation}
  Prop.(x_{j})=\frac{Frequency(x=x_{j})}{N}
 \end{equation}

\begin{equation}
Mean(x_{i})=\bar{x} = \sum_{j=1}^{m} Prop(x_{j})x_{j}\label{eq:-3}\end{equation}

where $Prop(x_{j})$ is the proportion of observations that
have value $x_{j}$. (sums across possible values of $x_j$, rather than
summing across all individuals observed).

\end{frame}


\begin{frame}
  \frametitle{The Expected Value of x, $E[x]$}

  \begin{itemize}
  \item EV sometimes thought of as the {}``population mean'' or ``true mean''
  \item  Recall, population=the random process that generates $x_{i}$.
  \item Consider a discrete distribution $f$. Note
    $\bar{x}$ and $E[x]$
    \begin{itemize}
    \item  $f$ is a {}``probability mass function''
      \begin{equation}
        Expected\, Value[x]=E[x]=\sum f(x_{j})x_{j}\label{eq:-4}\end{equation}
    \item Same as sample mean formula, except replace the {}``observed
      proportion'' ($Prop(x_{j})$) with the {}``theoretical probability'' $f(x_{j})$.
    \end{itemize}
  \item Similar for a continuous distribution with pdf $f(x)$
    \begin{equation}
      E[x] = \int_{-\infty}^{+\infty}f(x)\, x\, dx.
    \end{equation}

  \end{itemize}
\end{frame}

\begin{frame}
  \frametitle{Proof of claim that Expected Value of $\bar{x}$ equals
    Expected Value of $x$}
  Calculate the expected value of $\bar{x}$
  \begin{eqnarray*}
    \bar{x} = & \frac{x_1 + x_2 + x_3 + \ldots + x_N}{N}\\
    E[\bar{x}] = & E\left[\frac{x_1+x_2 +x_3+\ldots +x_N}{N}\right]\\
    & = \frac{1}{N} \left\{E[x_1]+E[x_2]+E[x_3]+\ldots + E[x_N] \right\}\\
    & = \frac{1}{N} \left\{N \cdot E[x] \right\} \\
    & = E[x]
  \end{eqnarray*}

   Conclusion: The expected value of the mean is the same as the
   expected value of one draw from a given distribution.

   Implication: $\bar{x}$ is an \textbf{unbiased estimator} of $E[x]$
\end{frame}

\subsection{Variance}

\begin{frame}
  \frametitle{Recall the Variance of A Sum}

The variance of a sum of two variables $x1$ and $x2$ can be found:

\begin{equation}
Var[x1+x2] = Var[x1]+ Var[x2] + 2Cov[x1,x2]\label{eq:VarianceX1X2}\end{equation}

And

\begin{equation}
Var[ax1+bx2]=a^{2} Var[x1]+b^{2} Var[x2] + 2abCov[x1,x2]\label{eq:VarianceOfSum}\end{equation}

Here $a$ and $b$ are constants.

We want a simple result, so we often assume the $Cov[x1,x2]=0$
on the grounds that the observations are {}``statistically independent.''

\end{frame}

\begin{frame}
  \frametitle{Calculate the Variance of the Mean}

What is the variance of the mean itself?

\begin{equation}
Var[\bar{x}]=Var[\frac{1}{N}x_{1}+\frac{1}{N}x_{2}+\ldots+\frac{1}{N}x_{N}]\label{eq:VarOfMean1}\end{equation}


Invoking the ``statistical independence'' principle to eliminate
the Covariance terms, we apply the ``Variance of a sum'' rule

\begin{eqnarray}
Var(\frac{1}{N}x_{1}+\frac{1}{N}x_{2}+\ldots+\frac{1}{N}x_{N})= \\
  \frac{1}{N^{2}}Var(x_{1})+\frac{1}{N^{2}}Var(x_{2})+\ldots+\frac{1}{N^{2}}Var(x_{N})\label{eq:VarOfMean2}\end{eqnarray}

\end{frame}

\begin{frame}
  \frametitle{}

If all the observations were drawn from the same random process--the
same population--then they all have the same variance, which is just
$Var(x_{i})$. So the previous instantly reduces to this:

\begin{equation}
Var(\bar{x})=\frac{1}{N^{2}}\frac{NVar(x_{i})}{1}\label{eq:-8}\end{equation}


\begin{equation}
=\frac{1}{N}Var(x_{i})\label{eq:varN2}
\end{equation}

\end{frame}

%
% ==========================================================
\end{document}
