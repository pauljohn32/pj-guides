\batchmode
\makeatletter
\def\input@path{{/home/pauljohn/SVN/SVN-guides/stat/Regression/ElementaryOLS//}}
\makeatother
\documentclass[10pt,english]{beamer}
\usepackage{lmodern}
\renewcommand{\sfdefault}{lmss}
\renewcommand{\ttdefault}{lmtt}
\usepackage[T1]{fontenc}
\usepackage[latin9]{inputenc}
\setcounter{secnumdepth}{3}
\setcounter{tocdepth}{3}
\usepackage{graphicx}

\makeatletter

%%%%%%%%%%%%%%%%%%%%%%%%%%%%%% LyX specific LaTeX commands.
%% Because html converters don't know tabularnewline
\providecommand{\tabularnewline}{\\}

%%%%%%%%%%%%%%%%%%%%%%%%%%%%%% Textclass specific LaTeX commands.
\usepackage{Sweavel}
 \def\lyxframeend{} % In case there is a superfluous frame end
 \long\def\lyxframe#1{\@lyxframe#1\@lyxframestop}%
 \def\@lyxframe{\@ifnextchar<{\@@lyxframe}{\@@lyxframe<*>}}%
 \def\@@lyxframe<#1>{\@ifnextchar[{\@@@lyxframe<#1>}{\@@@lyxframe<#1>[]}}
 \def\@@@lyxframe<#1>[{\@ifnextchar<{\@@@@@lyxframe<#1>[}{\@@@@lyxframe<#1>[<*>][}}
 \def\@@@@@lyxframe<#1>[#2]{\@ifnextchar[{\@@@@lyxframe<#1>[#2]}{\@@@@lyxframe<#1>[#2][]}}
 \long\def\@@@@lyxframe<#1>[#2][#3]#4\@lyxframestop#5\lyxframeend{%
   \frame<#1>[#2][#3]{\frametitle{#4}#5}}
 \newenvironment{topcolumns}{\begin{columns}[t]}{\end{columns}}

%%%%%%%%%%%%%%%%%%%%%%%%%%%%%% User specified LaTeX commands.
\usepackage{dcolumn}
\usepackage{booktabs}

% use 'handout' to produce handouts
%\documentclass[handout]{beamer}
\usepackage{wasysym}
\usepackage{pgfpages}
\newcommand{\vn}[1]{\mbox{{\it #1}}}\newcommand{\vb}{\vspace{\baselineskip}}\newcommand{\vh}{\vspace{.5\baselineskip}}\newcommand{\vf}{\vspace{\fill}}\newcommand{\splus}{\textsf{S-PLUS}}\newcommand{\R}{\textsf{R}}


\usepackage{graphicx}
\usepackage{listings}
\lstset{tabsize=2, breaklines=true,style=Rstyle}

\fvset{listparameters={\setlength{\topsep}{0em}}}
\def\Sweavesize{\normalsize} 
\def\Rcolor{\color{black}} 
\def\Rbackground{\color[gray]{0.95}}



\mode<presentation>

\usetheme{Antibes}

% In document Latex options:

\newcommand\makebeamertitle{\frame{\maketitle}}%

\expandafter\def\expandafter\insertshorttitle\expandafter{%
 \insertshorttitle\hfill\insertframenumber\,/\,\inserttotalframenumber}

\setbeamertemplate{frametitle continuation}[from second]
\renewcommand\insertcontinuationtext{...}

%\usepackage{handoutWithNotes}
%\pgfpagesuselayout{3 on 1 with notes}[letterpaper, border shrink=5mm]

\makeatother

\usepackage{babel}
\begin{document}

% In document Latex options:
\fvset{listparameters={\setlength{\topsep}{0em}}}

\def\Sweavesize{\scriptsize} 
\def\Rcolor{\color{black}} 
\def\Rbackground{\color[gray]{0.90}}

\begin{Schunk}
\begin{Sinput}
 ## Other settings I like
 options(device = pdf)
 options(useFancyQuotes = FALSE) 
 op <- par() 
 pjmar <- c(5.1, 5.1, 1.5, 2.1) 
 options(SweaveHooks=list(fig=function() par(mar=pjmar, ps=12)))
 pdf.options(onefile=F,family="Times",pointsize=12)
\end{Sinput}
\end{Schunk}



\title[Regression 1]{Elementary Regression 1}


\author{Paul E. Johnson\inst{1} \and \inst{2}}


\institute[K.U.]{\inst{1}Department of Political Science\and \inst{2}Psychology
}


\date[2020]{2020}

\makebeamertitle

\lyxframeend{}



\AtBeginSection[]{

  \frame<beamer>{ 

    \frametitle{Outline}   

    \tableofcontents[currentsection] 

  }

}

\begin{frame}

\frametitle{Elementary OLS}

\tableofcontents{}

\end{frame}


\lyxframeend{}\section{Introduction: Key Terms}


\lyxframeend{}\lyxframe{Data Set: Columns of Same Length}

\begin{center}
\begin{tabular}{|c|c|c|c|c|}
\hline 
row number &
respondent id &
$income$ &
$educ$  &
gender\tabularnewline
\hline 
\hline 
1 &
243223 &
4352.5 &
6 &
M\tabularnewline
\hline 
2 &
151512 &
6525.1 &
21 &
F\tabularnewline
\hline 
3 &
515131 &
4345.5 &
13 &
M\tabularnewline
\hline 
4 &
166122 &
3421.4 &
12 &
F\tabularnewline
\hline 
$\vdots$ &
 &
 &
 &
\tabularnewline
\hline 
\end{tabular}
\par\end{center}
\begin{itemize}
\item Variables are ``columns'' in a data frame
\item Rows are called ``observations'' or ``cases'' or ``respondents''
or ``subjects''
\item Talk about row ``i'' if you mean to say something that applies for
each row
\end{itemize}

\lyxframeend{}\lyxframe{Design Matrix}
\begin{itemize}
\item Regression is, inherently, a procedure for estimating effects of numeric
predictors
\item The data frame (in R, the ``model frame'') has to be converted from
data as we see it into a thing that has only numeric columns.
\item Categorical predictors are converted into ``indicator'' variables
(dummy variables, usually coded \{0,1\} or \{-1,1\}
\end{itemize}
\begin{center}
\begin{tabular}{|c|c|c|c|c|}
\hline 
row number &
respondent id &
$income$ &
$educ$  &
gender\tabularnewline
\hline 
\hline 
1 &
243223 &
4352.5 &
6 &
1\tabularnewline
\hline 
2 &
151512 &
6525.1 &
21 &
0\tabularnewline
\hline 
\end{tabular}
\par\end{center}


\lyxframeend{}\lyxframe{Scatterplot: One Input, One Output}

\begin{center}
\par\end{center}

\includegraphics[width=6cm]{plots/t-carinced10}

.

.

The ``Prestige'' dataset in the R package ``car'' by John Fox


\lyxframeend{}\lyxframe{Dependent, Independent}
\begin{itemize}
\item DV: Dependent Variable: The thing we are predicting

\begin{itemize}
\item The ``output'' variable, generally we call it $y_{i}$

\begin{itemize}
\item In this case ``$income_{i}$''.
\end{itemize}
\item Synonyms: ``endogenous variable'' ``outcome variable''
\end{itemize}
\item IV: Independent Variable

\begin{itemize}
\item The ``input'' variable, generally call it $x_{i}$, 

\begin{itemize}
\item In this case ``$education_{i}"$.
\end{itemize}
\item Synonyms: ``exogenous variable'' ``predictor'' ``covariate''
\end{itemize}
\item Regression allows several input variables, but for now we consider
only one.
\end{itemize}

\lyxframeend{}\lyxframe{Line of Best Fit }

\begin{center}
\par\end{center}
\begin{topcolumns}%{}


\column{6cm}


\includegraphics[width=6cm]{plots/t-carinced20}


\column{6cm}
\begin{itemize}
\item This is the Straight Line that ``best fits'' the data
\item Best fit = minimizes a criterion, here the ``sum of squared errors''
\item ``Predicted value'' synonym for ``fitted value'' or ``conditional
expected value''

\begin{itemize}
\item For any value of $education$, we predict an outcome on the line
\end{itemize}
\item Later, we will use diagnostics to test suitability of this model
\end{itemize}
\end{topcolumns}%{}

\lyxframeend{}

\begin{frame}
\frametitle{Typical Computer Printout Summarizing a Fitted Regression}

\begin{center}
\par\end{center}

\def\Sweavesize{\scriptsize}
\begin{Schunk}
\begin{Soutput}
Call:
lm(formula = income ~ education, data = Prestige)

Residuals:
    Min      1Q  Median      3Q     Max 
-5493.2 -2433.8   -41.9  1491.5 17713.1 

Coefficients:
            Estimate Std. Error t value Pr(>|t|)    
(Intercept)  -2853.6     1407.0  -2.028   0.0452 *  
education      898.8      127.0   7.075 2.08e-10 ***
---
Signif. codes:  0 '***' 0.001 '**' 0.01 '*' 0.05 '.' 0.1 ' ' 1

Residual standard error: 3483 on 100 degrees of freedom
Multiple R-squared:  0.3336,	Adjusted R-squared:  0.3269 
F-statistic: 50.06 on 1 and 100 DF,  p-value: 2.079e-10
\end{Soutput}
\end{Schunk}


\end{frame}


\lyxframeend{}\lyxframe{Make a Professionally Acceptable Regression Table}

\begin{tabular}{@{}l*{3}{l}@{}}
\hline
  &\multicolumn{2}{l}{M1  }\tabularnewline
 &\multicolumn{1}{c}{Estimate}&\multicolumn{1}{c}{(S.E.)}\tabularnewline
 \hline
 \hline
  (Intercept) & -2853.586* & (1407.039) \tabularnewline
  education & 898.813*** & ( 127.035) \tabularnewline
 \hline
 N&\multicolumn{1}{l}{102}  & \tabularnewline
 RMSE&3483.378\tabularnewline
 $R^2$&0.334\tabularnewline
 \hline
\hline
 
 \multicolumn{3}{l}{  ${*  p}\le 0.05$${*\!\!*  p}\le 0.01$${*\!\!*\!\!*  p}\le 0.001$}\tabularnewline
 \end{tabular}
\begin{itemize}
\item When we are finished, you will understand all of these details.
\end{itemize}

\lyxframeend{}

\begin{frame}[containsverbatim]
\frametitle{In R, after "lm", run follow-up functions}
\begin{topcolumns}%{}




\column{6cm}


There are many (at least 30) ``methods'' that can be used to explore
that fitted model.




\def\Sweavesize{\scriptsize}
\begin{Schunk}
\begin{Sinput}
 incedmod1 <- lm(income~education, data=Prestige)
 summary(incedmod1)
 anova(incedmod1, test="F")
 vcov(incedmod1)
 confint(incedmod1)
 plot(incedmod1)
 termplot(incedmod1, se=T, partial=T)
\end{Sinput}
\end{Schunk}



\column{6cm}
\begin{itemize}
\item lm: creates the regression model ``incedmod1''
\item summary: main regression table
\item anova: asks for sum of squares information
\item vcov: asks for the variance/covariance matrix of $\hat{\beta}$'s
\item confint: confidence intervals for intercept and slope
\item plot: creates diagnostic displays
\item termplot: plots the predictive line
\item many methods in the ``car'' package
\item rockchalk plotting and diagnostic routines
\end{itemize}
\end{topcolumns}%{}
\end{frame}


\lyxframeend{}\section{People Always Ask Me$\ldots$}


\lyxframeend{}\lyxframe{1. Can I Run Regression on This?}


\includegraphics[width=10cm]{plots/t-q10a}


\lyxframeend{}\lyxframe{1. As we say in Francais, Oui!}


\includegraphics[width=10cm]{plots/t-q10b}

2 numeric variables, passes the ``inter-occular trauma test''


\lyxframeend{}\lyxframe{2. Can I Run Regression on This?}


\includegraphics[width=10cm]{plots/t-q20a}


\lyxframeend{}\lyxframe{2. Sure, Why Not?}


\includegraphics[width=10cm]{plots/t-q20b}

The ``straight line'' prediction is not wrong. But not precise,
either.


\lyxframeend{}\lyxframe{3. Can I Run Regression on This?}


\includegraphics[width=10cm]{plots/t-q30a}


\lyxframeend{}\lyxframe{3. En Espanol, Si!}


\includegraphics[width=10cm]{plots/t-q30b}


\lyxframeend{}\lyxframe{4. Can I Run Regression on This?}


\includegraphics[width=10cm]{plots/t-q40a}


\lyxframeend{}\lyxframe{4. OK, I Don't Mind a Bit}


\includegraphics[width=10cm]{plots/t-q40b}

I don't know of any reason why you expect the predictor to be ``evenly
distributed'' or ``normal'' or whatnot


\lyxframeend{}\lyxframe{5. Can I Run Regression on This?}


\includegraphics[width=10cm]{plots/t-q50a}


\lyxframeend{}\lyxframe{5. No. Are You Joking?}


\includegraphics[width=10cm]{plots/t-q50b}

Straight line does not suit this data


\lyxframeend{}\lyxframe{What's the point so far?}
\begin{itemize}
\item We don't assume much about the predictor
\item We do assume a LOT about the outcome variable

\begin{itemize}
\item it is supposed to be scattered ``equally likely'' above and below
the line
\end{itemize}
\end{itemize}

\lyxframeend{}\lyxframe{6. Can I Run Regression on This?}


\includegraphics[width=10cm]{plots/t-q60a}


\lyxframeend{}\lyxframe{6. Maybe, But You'd Really Have to Stretch}


\includegraphics[width=10cm]{plots/t-q60b}

Its tough for me to see a ``regression line'' in there, but some
people do.


\lyxframeend{}\lyxframe{7. Can I Run Regression on This?}


\includegraphics[width=10cm]{plots/t-q70a}


\lyxframeend{}\lyxframe{7. Probably, if you recode the predictor as \{0,1\}}


\includegraphics[width=10cm]{plots/t-q70b}

The appropriate graph has ``steps'', rather than a line. Predictions
for discrete points.


\lyxframeend{}\lyxframe{8. Can I Run Regression on This?}


\includegraphics[width=10cm]{plots/t-q80a}


\lyxframeend{}\lyxframe{8. As Yoda says, ``Mistaken, This Appears''}


\includegraphics[width=10cm]{plots/t-q80b}


\lyxframeend{}\section{The Underlying Theory}


\lyxframeend{}\lyxframe{Assumption 1: Linear Relationship}
\begin{itemize}
\item For each ``case'' $i$, the following is true: 
\begin{equation}
y_{i}=\beta_{0}+\beta_{1}x_{i}+e_{i}\label{eq:linearModel}
\end{equation}

\item The \alert{parameters} are $\beta_{0}$, $\beta_{1}$, and $\sigma_{e}$

\begin{itemize}
\item $\beta_{0}$ is the ``constant'' or ``y intercept''. 
\item $\beta_{1}$ is the slope of the line.
\item $\sigma_{e}$ is the standard deviation of a ``random effect,''
$e_{i}$, that is uniquely drawn for each observation.
\end{itemize}
\item The subscript $i$ means $x_{i}$ and $y_{i}$ are individual specific.
Note no $i$ on $\beta$'s or $\sigma_{e}$
\item In the past, my notes used the letter $b$ for coefficients, not $\beta$,
mostly because $b$ was easier to type in MS Word. Now I use \LaTeX{},
I don't have that problem anymore. But I have not updated all of my
notes about everything.
\end{itemize}

\lyxframeend{}\lyxframe{Random and Deterministic Parts}
\begin{topcolumns}%{}


\column{4cm}
\begin{itemize}
\item The deterministic part is the ``true line'' $\beta_{0}+\beta_{1}x_{i}$
\item The stochastic (random part) ``throws'' observed scores up and down
\end{itemize}

\column{6cm}


\begin{center}
\includegraphics[bb=14bp 94bp 655bp 500bp,clip,width=7cm]{0_home_pauljohn_SVN_SVN-guides_stat_Regression_ElementaryOLS_importfigs_bivar1.pdf}
\par\end{center}

\end{topcolumns}%{}

\lyxframeend{}

\begin{frame}[containsverbatim]
\frametitle{Separate Deterministic and Stochastic Parts}
\begin{topcolumns}%{}


\column{6cm}
\begin{itemize}
\item Suppose $\beta_{0}=3$ and $\beta_{1}=1.3.$ 
\item The ``true relationship'':
\[
\mbox{\ensuremath{y_{i}}}=3+1.3\cdot x_{i}+e_{i}
\]

\item The deterministic part:
\[
3+1.3\cdot x_{i}
\]

\item The stochastic part is $e_{i}$.
\end{itemize}

\column{6cm}



\includegraphics[width=6cm]{plots/t-line20}
\end{topcolumns}%{}
\end{frame}

\begin{frame}[containsverbatim]
\frametitle{Refresher: Linear Equation}
\begin{topcolumns}%{}


\column{5cm}
\begin{itemize}
\item \textrm{$3+1.3\cdot x_{i}$}
\item The slope: 1.3 is the ``rise over run''

\begin{itemize}
\item For each 1 unit increase in $x_{i}$, the outcome increases by $1.3$.
\end{itemize}
\item The intercept: 3

\begin{itemize}
\item When $x_{i}=0,$ the outcome will be $3$.
\end{itemize}
\end{itemize}

\column{7cm}




\includegraphics[width=6cm]{plots/t-line90}

\end{topcolumns}%{}
\end{frame}


\lyxframeend{}\lyxframe{The Fitted Line in the Income Equation}
\begin{columns}%{}


\column{6cm}
\begin{itemize}
\item Note the difference between the theory and the estimate
\item Theory: $income_{i}=\beta_{0}+\beta_{1}education_{i}+e_{i}$
\item Estimated line: 
\end{itemize}

\begin{equation}
\widehat{income}_{i}=-2853.585+898.813\cdot education_{i}
\end{equation}

\begin{itemize}
\item There is no ``error term'' in the equation for the predicted line.
That's because we assume $E[e_{i}]=0$. 
\end{itemize}

\column{6cm}


\includegraphics[width=6cm]{plots/t-carinced20}

\end{columns}%{}

\lyxframeend{}\lyxframe{The Fitted Line in the Income Equation}
\begin{columns}%{}


\column{6cm}
\begin{itemize}
\item A 1 unit increase in $education_{i}$ ``is associated with''( causes?)
a 898.8 increase in $income_{i}$
\item The subscript $i$ is important. It helps us remember the assumption
that the same relationship applies for all cases, $i\in\{1,\ldots,N\}$
\item The regression model also summarizes the ``scatter'' above and below,
which is our next topic.
\end{itemize}

\column{6cm}


\includegraphics[width=6cm]{plots/t-carinced20}

\end{columns}%{}

\lyxframeend{}\lyxframe{Assumption 2: A ``Well Behaved Error Term''}
\begin{itemize}
\item We don't have to say $e_{i}$ is $Normal(0,\sigma_{e}^{2})$. But
we could. Some people do.
\item Well behaved means ``symmetric'' and ``homogeneous'', which is
not as strong as assuming Normal

\begin{itemize}
\item Assumption 2A: $e_{i}$ is ``on average'' 0: $E[e_{i}]=0$
\item Assumption 2B: all observations are drawn from the same distribution
with a constant variance, $\sigma_{e}^{2}$ (a.k.a ``homoskedasticity'')
\[
Var[e_{i}]=E[e_{i}^{2}]=\sigma_{e}^{2}
\]

\end{itemize}
\item Violations of these assumptions lead to re-specification and advanced
model-fitting techniques (nonlinear models, weighted least squares,
random effects models)
\end{itemize}

\lyxframeend{}\lyxframe{Assumption 2A: $E[e_{i}]=0$}
\begin{itemize}
\item The error term has an average value of $0$:
\end{itemize}
\begin{equation}
E(e_{i})=0
\end{equation}

\begin{itemize}
\item Thus $E[y_{i}|x_{i}]=E[\beta_{0}+\beta_{1}x_{i}+e_{i}]=\beta_{0}+\beta_{1}x_{i}+0$
\item You can guess where this leads, right? 

\begin{itemize}
\item If we had reasonable estimates $\hat{\beta}_{0}$ and $\hat{\beta}_{1}$,
the predicted value $\hat{y}_{i}=\hat{\beta}_{0}+\hat{\beta}_{1}x_{i}$
is a reasonable estimate of the expected value, given $x_{i}$.
\item In other words, it is not ridiculous to use predicted (or fitted)
value $\hat{y}_{i}=\hat{\beta}_{0}+\hat{\beta}_{1}x_{i}$ as an estimated
value for $y_{i}$
\end{itemize}
\end{itemize}

\lyxframeend{}\lyxframe{Assumption 2B: Homoskedasticity}
\begin{itemize}
\item The error term's variance is constant, i.e, the same for all cases
$i$
\end{itemize}
\begin{equation}
Variance[e_{i}]=\sigma_{e}^{2}\label{eq:Varesig2}
\end{equation}

\begin{itemize}
\item I.e., $\sigma_{e}^{2}$ \textbf{is the same for all cases}. It is
not subscripted by $i$.

\begin{itemize}
\item Every case's ``random effect'' comes from a distribution with the
same amount of uncertainty in it.
\end{itemize}
\item This assumption is vital in our understanding of uncertainty, or variance,
in the estimates.
\end{itemize}

\lyxframeend{}\lyxframe{Sidenote. Explain $E[e_{i}^{2}]=\sigma_{e}^{2}$}
\begin{itemize}
\item The Variance of the error term equals the expected value of $e_{i}^{2}$.
\item Many stats book will define ``homogeneous variance'' as: 
\[
E[e_{i}^{2}]=\sigma_{e}^{2}
\]



rather than the more obvious
\begin{equation}
Var[e_{i}]=\sigma_{e}^{2}
\end{equation}


\item While disconcerting, we can show these are the SAME definitions. Start
with the definition of variance
\end{itemize}
\[
V(e_{i})=\sigma_{e}^{2}=E[(e_{i}-E[e_{i}])^{2}]
\]

\begin{itemize}
\item Recall $E(e_{i})=0$, so
\end{itemize}
\[
V[e_{i}]=E[(e_{i}-0)^{2}]=E[e_{i}^{2}]
\]



\lyxframeend{}\lyxframe{In Maximum Likelihood Analysis, A Stronger Assumption Would be Required}
\begin{itemize}
\item In ML (including the generalized linear model), we would assume a
specific distribution for $e_{i}$, which amounts to saying that we
can state the distribution of $y_{i}$ given $x_{i}$ and the $\beta$'s. 
\item We would usually say $y_{i}$, depends on ``linear predictor'' ($\beta_{0}+\beta_{1}x_{i})$.
\item For example, given $x_{i}$, $y_{i}$ is Normal, i.e., drawn from
$N(\beta_{0}+\beta_{1}x_{i},\,\,\sigma_{e}^{2})$
\item Until the end of this class, we don't need to make that assumption,
but you can if you like it! 
\item When you get to GLM, you can assume that $y_{i}$ is Poisson, Gamma,
or whatever you like.
\end{itemize}

\lyxframeend{}\lyxframe{Roadmap of Ahead}
\begin{enumerate}
\item calculate estimates of $\beta_{0}$ and $\beta_{1}$ (which we will
call $\hat{\beta}_{0}$ and $\hat{\beta}_{1}$) 
\item evaluate our uncertainty about the $\hat{\beta}$'s by calculating
standard errors of the $\hat{\beta}$.
\item estimate the variance of $e_{i}$, $\widehat{\sigma_{e}^{2}}$
\item conduct some ``diagnostics'' to find out if we might fit a better
model.
\end{enumerate}

\lyxframeend{}\section{Estimate $\beta$'s}


\lyxframeend{}\lyxframe{Treat $\hat{\beta}_{0}$ and $\hat{\beta}_{1}$ as unknowns.}
\begin{itemize}
\item This week, we only use a ``straight line'' predicted value formula. 
\end{itemize}
\begin{equation}
\hat{y}_{i}=\hat{\beta}_{0}+\hat{\beta}_{1}\cdot x_{i}
\end{equation}

\begin{itemize}
\item The observed variables $x_{i}$ and $y_{i}$ are now treated as ``known
values'', 
\item The parameter estimates $\hat{\beta}_{0}$ and $\hat{\beta}_{1}$
become variables that we adjust to find the best prediction.
\end{itemize}

\lyxframeend{}\lyxframe{OLS: The Sum of Squares as a Criterion}
\begin{itemize}
\item Predicted: $\hat{y}_{i}=\hat{\beta}_{0}+\hat{\beta}_{1}x_{i}$ 
\item Residual: Difference between observed $y_{i}$ and predicted $\hat{y}_{i}$.
\item $S(\hat{\beta}_{0},\hat{\beta}_{1}):$Sum of Squared Residuals depends
on $\hat{\beta}_{0}$, $\hat{\beta}_{1}$
\end{itemize}
\begin{eqnarray}
S(\hat{\beta}_{0},\hat{\beta}_{1}) & = & \sum_{i=1}^{N}(y_{i}-\hat{y}_{i})^{2}\\
 & = & \,\sum_{i=1}^{N}(y_{i}-(\hat{\beta}_{0}+\hat{\beta}_{1}x_{i}))^{2}\nonumber \\
 & = & \,\sum_{i=1}^{N}(y_{i}-\hat{\beta}_{0}-\hat{\beta}_{1}x_{i})^{2}\nonumber 
\end{eqnarray}

\begin{itemize}
\item OLS Criterion: minimize the sum of squared residuals by adjusting
$\hat{\beta}_{0}$ and $\hat{\beta}_{1}$
\item Notation alert: Often also called ``sum of squared errors'', but
better to be clear: we never know ``true errors'', we only know
``residuals''. So I'm trying to remember to call it sum of squared
residuals.
\end{itemize}

\lyxframeend{}\lyxframe{Estimation process is outlined in the Appendix}
\begin{columns}%{}


\column{5cm}
\begin{itemize}
\item The sum of squared residuals is an objective function that we minimize
by adjusting $\hat{\beta}_{0}$ and $\hat{\beta}_{1}$
\item Because the sum of squares is a ``U'' shaped function, we can visualize
the solution.
\end{itemize}

\column{7cm}


\includegraphics[width=7cm]{1_home_pauljohn_SVN_SVN-guides_stat_Regression_ElementaryOLS_importfigs_OLSdemo-3d.pdf}

\end{columns}%{}

\lyxframeend{}\lyxframe{[allowframebreaks]The Solutions are the ``OLS Estimators''}
\begin{itemize}
\item We'd ordinarily use matrix algebra to solve this problem, but I don't
want to go into matrices at this point.
\item Thus I write out the solution in ``scalar'' format, using ordinary
summations and such.
\end{itemize}
\begin{equation}
\hat{\beta}_{1}^{OLS}=\frac{\sum_{i=1}^{N}(x_{i}-\bar{x})(y_{i}-\bar{y})}{\sum_{i=1}^{N}(x_{i}-\bar{x})^{2}}\label{bhat-1}
\end{equation}


$\bar{x}$ and $\bar{y}$ are sample means. 
\begin{itemize}
\item Note

\begin{itemize}
\item numerator terms: product of $x$ deviations and $y$ deviations about
their means
\item denominator terms: $x$ deviations squared.
\end{itemize}
\item If you have ``mean centered data'', this simplifies to
\begin{equation}
\hat{\beta}_{1}^{OLS}=\frac{\sum x_{i}y_{i}}{\sum x_{i}^{2}}
\end{equation}

\item And the intercept estimate: $\hat{\beta}_{0}^{OLS}=\bar{y}-\hat{\beta}_{1}^{OLS}\bar{x}$
\item If you were paying attention when we studied Variance and Covariance,
you notice the formula for $\hat{\beta}$ is $Cov(x,y)/Var(x)$. Interesting
co-incidence, there.
\end{itemize}

\lyxframeend{}\lyxframe{Gauss Markov Theorem: OLS is B.L.U.E.}
\begin{itemize}
\item $\hat{\beta}^{OLS}$ is an \alert{Unbiased} estimator, it is ``on
average'' correct: $E[\hat{\beta}^{OLS}]=\beta$
\item $\hat{\beta}^{OLS}$ is \alert{Consistent}, as $N\rightarrow\infty$,
$\hat{\beta}^{OLS}.\rightarrow\beta$. (the probability that the gap
$|\hat{\beta}^{OLS}-\beta|$ is bigger than any small number shrinks
toward $0$ as $N\rightarrow\infty$).
\item $\hat{\beta}^{OLS}$ is \alert{Efficient}: No linear unbiased estimating
formulae has lower variance than $\hat{\beta}^{OLS}$.
\end{itemize}

\lyxframeend{}\section{$\widehat{\sigma_{e}^{2}}$: Mean Square Error}


\lyxframeend{}\lyxframe{Define residual, as opposed to ``error''}
\begin{itemize}
\item $e_{i}$ is an ``error term'', it is unmeasured, unknown. 

\begin{itemize}
\item Its ``true mean'' (expected value) is assumed to be 0
\item Its ``true variance'' is $\sigma_{e}^{2}$, also unknown.
\end{itemize}
\item $\hat{e}_{i}$ is the ``residual'', $y_{i}-\hat{y}_{i}$. It serves
as an estimate of the error term.
\end{itemize}

\lyxframeend{}\lyxframe{MSE=Mean Square Error }
\begin{itemize}
\item Predict $\hat{y}_{i}$ from the best fitting model
\item The commonly-called MSE (Mean Squared Error) is the mean of squared
\textbf{residuals}.
\begin{equation}
MSE=\frac{\sum(y_{i}-\widehat{y_{i}})^{2}}{N-2}=\frac{\sum\widehat{e_{i}}^{2}}{N-2}\label{MSE}
\end{equation}

\item MSE = unbiased estimator of $\sigma_{e}^{2}$ (because of $N-2$ in
denominator). Unbiased means
\begin{equation}
E[MSE]=\sigma_{e}^{2}
\end{equation}

\item Other notation for MSE: $\widehat{\sigma_{e}^{2}}$,$\widehat{Var[e_{i}]}$,
$s_{e}^{2}$
\end{itemize}

\lyxframeend{}\lyxframe{RMSE=Root Mean Squared Error}
\begin{itemize}
\item RMSE (root MSE) is the SAS name for the square root of the MSE.
\item $\hat{\sigma}_{e}$ : The square root of MSE serves as an estimate
of the standard deviation of the error term.
\item Other names for root MSE:

\begin{itemize}
\item standard error of the estimate (in SPSS)
\item Residual standard error (in R)
\item $std.err.(e)$. 
\end{itemize}
\end{itemize}

\lyxframeend{}\section{Correlation and $R^{2}$}


\lyxframeend{}\subsection{The $R^{2}$}


\lyxframeend{}\lyxframe{$R^{2}$. The Coefficient of Determination}
\begin{itemize}
\item $R^{2}$ is the ``coefficient of determination'' 
\item $R^{2}$ has a minimum of 0 and a maximum of 1.
\item $R^{2}$ mostly about ``how big'' the error variance is compared
to the variance of $x$ and $y$.
\end{itemize}

\lyxframeend{}\lyxframe{The ``Proportion of Variance Explained''}
\begin{itemize}
\item Some people write that the $R^{2}$ represents the proportion of variance
in y explained by x. Where do they get that?
\item The Total Sum of Squares: $TSS=\sum(y_{i}-\bar{y})^{2}$
\item The Error Sum of Squares: $ESS=\sum(y_{i}-\hat{y}_{i})^{2}$
\item Regression Sum of Squares

\begin{itemize}
\item $RSS=TSS-ESS$
\item $RSS=\sum(\hat{y}_{i}-\bar{y}_{i})^{2}$
\end{itemize}
\end{itemize}

\lyxframeend{}\lyxframe{``Proportion of Variance''(cont)}
\begin{itemize}
\item Notice
\end{itemize}
\[
TSS=RSS+ESS
\]

\begin{itemize}
\item Divide all terms by $TSS$ and we see that the two ``proportions''
of variance add up to one
\end{itemize}
\[
1=\frac{RSS}{TSS}+\frac{ESS}{TSS}
\]

\begin{itemize}
\item That's
\begin{equation}
1=part\, accounted\, for\, by\, regression+part\, accounted\, for\, by\, error
\end{equation}

\end{itemize}

\lyxframeend{}\lyxframe{``Proportion of Variance''(cont)}
\begin{itemize}
\item Let the ``coefficient of determination'' be
\[
R^{2}=\frac{RSS}{TSS}
\]

\item which is the same as
\item 
\[
1-\frac{ESS}{TSS}
\]
 
\item Put that in words: $R^{2}$ is the proportion of variance left over
after we take out the part contributed by random error term. 
\item Calculate the 'anova' table for a regression model, you'll see for
yourself.
\end{itemize}

\lyxframeend{}\lyxframe{How Important is $R^{2}$}
\begin{itemize}
\item Experienced statisticians may have rules of thumb about $R^{2}$.
For example, $R^{2}$ should be bigger than 0.2 before a model is
worth reporting. 
\item For various reasons (next slides), I think that's silly.
\item Sometimes practitioners think a low $R^{2}$ is a general warning
sign that ``something is wrong.'' 
\item That's also mistaken: it might be there's not powerful predictive
relationship to be found. We shouldn't torture the data.
\item $R^{2}$ is partly dependent on the error term's variance, and we
will see later that big variance -> wide confidence intervals. I often
do wish error variance were smaller. 
\end{itemize}

\lyxframeend{}\lyxframe{Don't Over-Emphasize $R^{2}$ }
\begin{itemize}
\item A slope is a slope is a slope, no matter how big the error variance
might be. The same $b$'s underlie both, but $R^{2}=0.70$ on left
and $0.15$ on right:
\end{itemize}


\includegraphics[width=5cm]{plots/t-RSQ10.pdf}\includegraphics[width=5cm]{plots/t-RSQ20.pdf}


\lyxframeend{}\lyxframe{$R^{2}$continued}
\begin{itemize}
\item The $R^{2}$ reflects 3 factors that melt together

\begin{itemize}
\item The range of $x$
\item The size of the slope coefficient
\item The standard deviation of the error term.
\end{itemize}
\item Any of those 3 culprits can make the $R^{2}$ shrink.
\item Does not necessarily imply that some better regression model exists--it
may just be that the process under study has inherent uncertainty. 
\item Careful: Wrong to compare $R^{2}$ across models with different data.
(Both $Var[x_{i}]$ and $Var[e_{i}]$ can change across data sets.)
\end{itemize}

\lyxframeend{}


\lyxframeend{}\subsection{Correlations}

\begin{frame}[containsverbatim]
\frametitle{A Scatterplot: How Strongly Are These Variables Related?}


\includegraphics[width=10cm]{plots/t-cor10}

\end{frame}

\begin{frame}[containsverbatim]
\frametitle{Covariance: Consider the Quadrants}

\begin{topcolumns}%{}


\column{6cm}


\includegraphics[width=6cm]{plots/t-cor30}


\column{6cm}
\begin{itemize}
\item Covariance: For each point, calculate $(x_{i}-\bar{x})(y_{i}-\bar{y})$
\item Covariance: add those up, divide by N.
\item blue points have positive products
\item red points have negative products
\end{itemize}
\end{topcolumns}%{}
\end{frame}

\begin{frame}[containsverbatim]
\frametitle{How strong is this relationship?}


\includegraphics[width=7cm]{plots/t-cor61}

\end{frame}

\begin{frame}[containsverbatim]
\frametitle{Is this relationship stronger?}


\includegraphics[width=7cm]{plots/t-cor70}

\end{frame}

\begin{frame}[containsverbatim]
\frametitle{Correlation=scaled covariance}
\begin{itemize}
\item Question: How do you know if $\widehat{Cov[x,y]}$ is ``big'' or
``medium'' or ``small''
\item Karl Pearson's Answer: form a correlation coefficient by scaling the
covariance 
\begin{equation}
r=\frac{\widehat{Cov[x,y]}}{\widehat{Std.Dev.[x]}\cdot\widehat{Std.Dev.[y]}}
\end{equation}

\item $r\in[-1,1]$ . That's all I know for sure about Pearson's r.
\end{itemize}
\end{frame}


\lyxframeend{}\subsection{Understand $r$ from a Regression Point of View}

\begin{frame}
\frametitle{If there is One Input}
\begin{itemize}
\item The Pearson's $r$ squared equals the $R^{2}$ in a one-predictor
regression.
\item Since we already argued that $R^{2}$ has a ``proportion of variance
accounted for'' interpretation, that means Pearson's $r$ squared
has same meaning.
\item The $r_{yx}$ (and $R^{2})$ balance Covariance against the variance
of x and y.
\end{itemize}
\end{frame}

\begin{frame}[containsverbatim]
\frametitle{Simulate Data For Regression}

\begin{topcolumns}%{}


\column{6cm}

This has no ``random error term'' ($e_{i}=0$)
\begin{itemize}
\item $\beta_{0}=3$
\item $\beta_{1}=0.25$
\item $x_{i}\sim N(50,100)$, $i=\{1,2,\ldots100\}$
\item $y_{i}=\beta_{0}+\beta_{1}x_{i}$
\end{itemize}

\column{6cm}


\includegraphics[width=6cm]{plots/t-reg10}

There's no ``error term''
\end{topcolumns}%{}
\end{frame}

\begin{frame}[containsverbatim]
\frametitle{Add Some Error to $y_i$ to adjust $\sigma_e$ (and hence $R^2$)}


\begin{topcolumns}%{}




\column{6cm}
\begin{itemize}
\item Same $\beta_{0}$=3, $\beta_{1}=0.25$, $x_{i}$
\item $y_{i}=\beta_{0}+\beta_{1}x_{i}+e_{i}$
\item $e_{i}\sim N(0,5^{2})$
\end{itemize}

\def\Sweavesize{\scriptsize}
\begin{tabular}{@{}l*{3}{l}@{}}
\hline
  &\multicolumn{2}{l}{M1  }\tabularnewline
 &\multicolumn{1}{c}{Estimate}&\multicolumn{1}{c}{(S.E.)}\tabularnewline
 \hline
 \hline
  (Intercept) & 1.743 & (2.524) \tabularnewline
  x & 0.269*** & (0.047) \tabularnewline
 \hline
 N&\multicolumn{1}{l}{100}  & \tabularnewline
 RMSE&5.375\tabularnewline
 $R^2$&0.248\tabularnewline
 \hline
\hline
 
 \multicolumn{3}{l}{  ${*  p}\le 0.05$${*\!\!*  p}\le 0.01$${*\!\!*\!\!*  p}\le 0.001$}\tabularnewline
 \end{tabular}


\column{6cm}


\includegraphics[width=6cm]{plots/t-reg20}

Std. Deviation of error term is 5
\end{topcolumns}%{}
\end{frame}

\begin{frame}[containsverbatim]
\frametitle{Tune Up Std.Dev.(e) -> Shrink the Correlation}


\begin{topcolumns}%{}


\column{6cm}
\begin{itemize}
\item Same $\beta_{0}=3$, $\beta_{1}=0.25$, $x_{i}$
\item $y_{i}=\beta_{0}+\beta_{1}x_{i}+e_{i}$
\item $e_{i}\sim N(0,10^{2})$
\end{itemize}

\def\Sweavesize{\scriptsize}
\begin{tabular}{@{}l*{3}{l}@{}}
\hline
  &\multicolumn{2}{l}{M1  }\tabularnewline
 &\multicolumn{1}{c}{Estimate}&\multicolumn{1}{c}{(S.E.)}\tabularnewline
 \hline
 \hline
  (Intercept) & 0.487 & (5.047) \tabularnewline
  x & 0.289** & (0.095) \tabularnewline
 \hline
 N&\multicolumn{1}{l}{100}  & \tabularnewline
 RMSE&10.749\tabularnewline
 $R^2$&0.087\tabularnewline
 \hline
\hline
 
 \multicolumn{3}{l}{  ${*  p}\le 0.05$${*\!\!*  p}\le 0.01$${*\!\!*\!\!*  p}\le 0.001$}\tabularnewline
 \end{tabular}


\column{6cm}


\includegraphics[width=6cm]{plots/t-reg30}

Std. Deviation of error term is 10
\end{topcolumns}%{}
\end{frame}

\begin{frame}[containsverbatim]
\frametitle{A Massive Std.Dev.(e) Makes $R^2$ Even Smaller}


\begin{topcolumns}%{}


\column{6cm}
\begin{itemize}
\item Same $\beta_{0}$, $\beta_{1}$, $x$
\item $y_{i}=\beta_{0}+\beta_{1}x_{i}+e_{i}$
\item $e_{i}\sim N(0,50^{2})$
\end{itemize}

\def\Sweavesize{\scriptsize}
\input{plots/t-reg45mod}


\column{6cm}


\includegraphics[width=6cm]{plots/t-reg45}

Std. Deviation of error term is 50
\end{topcolumns}%{}
\end{frame}

\begin{frame}[containsverbatim]
\frametitle{Leave Std.Dev.(e) Large, but Raise $b_1$}


\begin{topcolumns}%{}


\column{6cm}
\begin{itemize}
\item Same $\beta_{0}$, $x$, and $e_{i}\sim N(0,50^{2})$
\item Make $\beta_{1}$ bigger
\end{itemize}

\def\Sweavesize{\scriptsize}
\begin{tabular}{@{}l*{3}{l}@{}}
\hline
  &\multicolumn{2}{l}{M1  }\tabularnewline
 &\multicolumn{1}{c}{Estimate}&\multicolumn{1}{c}{(S.E.)}\tabularnewline
 \hline
 \hline
  (Intercept) & -9.567 & (25.237) \tabularnewline
  x & 2.194*** & ( 0.474) \tabularnewline
 \hline
 N&\multicolumn{1}{l}{100}  & \tabularnewline
 RMSE&53.745\tabularnewline
 $R^2$&0.179\tabularnewline
 \hline
\hline
 
 \multicolumn{3}{l}{  ${*  p}\le 0.05$${*\!\!*  p}\le 0.01$${*\!\!*\!\!*  p}\le 0.001$}\tabularnewline
 \end{tabular}


\column{6cm}


\includegraphics[width=6cm]{plots/t-reg50}

Std. Deviation of error term is 50, $\beta_{1}=2$
\end{topcolumns}%{}
\end{frame}

\begin{frame}[containsverbatim]
\frametitle{Make $b_1$ Even Larger}


\begin{topcolumns}%{}


\column{6cm}
\begin{itemize}
\item Same $\beta_{0}$, $\beta_{1}$, $x$
\item $y_{i}=\beta_{0}+\beta_{1}x_{i}+e_{i}$
\item $e_{i}\sim N(0,50^{2})$
\end{itemize}

\def\Sweavesize{\scriptsize}
\input{plots/t-reg60mod}


\column{6cm}


\includegraphics[width=6cm]{plots/t-reg60}

Std. Deviation of error term is 50 and $\beta_{1}=10$
\end{topcolumns}%{}
\end{frame}

\begin{frame}[containsverbatim]
\frametitle{What are you Supposed to Conclude?}
\begin{itemize}
\item The slope and the error variance are ``balancing'' each other.
\item If the error variance is large, we need a steep slope to compensate
and keep $R^{2}$ in the same vicinity.
\item We can also fiddle with $R^{2}$ by adjusting the range of $x$ (shown
next).
\end{itemize}
\end{frame}

\begin{frame}[containsverbatim]
\frametitle{A Restricted $x$ Range Makes r Smaller}


\begin{topcolumns}%{}


\column{6cm}
\begin{itemize}
\item Chopped off the top half of the $x_{i}$ observations
\item Wow. The effect of $x$ on $y$ is the same, $\beta_{1}=10$ 
\item Smaller $Var(x)$$\rightarrow$ Smaller $R^{2}$ (``design'' implication)
\end{itemize}

\def\Sweavesize{\scriptsize}
\begin{tabular}{@{}l*{3}{l}@{}}
\hline
  &\multicolumn{2}{l}{M1  }\tabularnewline
 &\multicolumn{1}{c}{Estimate}&\multicolumn{1}{c}{(S.E.)}\tabularnewline
 \hline
 \hline
  (Intercept) & 91.217 & (48.138) \tabularnewline
  x2 & 7.709*** & ( 1.080) \tabularnewline
 \hline
 N&\multicolumn{1}{l}{56}  & \tabularnewline
 RMSE&48.419\tabularnewline
 $R^2$&0.485\tabularnewline
 \hline
\hline
 
 \multicolumn{3}{l}{  ${*  p}\le 0.05$${*\!\!*  p}\le 0.01$${*\!\!*\!\!*  p}\le 0.001$}\tabularnewline
 \end{tabular}


\column{6cm}


\includegraphics[width=6cm]{plots/t-reg70}

Std. Deviation of error term is 50 and $\beta_{1}=10$
\end{topcolumns}%{}
\end{frame}

\begin{frame}[containsverbatim]
\frametitle{Section Summary}
\begin{itemize}
\item Correlation depends on several components, $Var(x_{i})$, $b_{1}$,
and $Var(e_{i}$).
\item The ``correlation coefficient'' is not a ``parameter.'' It is
a description or a 'weighted summary' of the effect of parameters
on the data. 
\item Goldberger (1991, p.177) puts it the following way: \textquotedblleft{}Nothing
in the CR (Classical Regression) model requires that $R^{2}$ be high.
Hence, a high $R^{2}$ is not evidence in favor of the model, and
a low $R^{2}$ is not evidence against it.\textquotedblright{} 
\item Nevertheless, $R^{2}$ can be a persuasive tool because many people
think a model is ``wrong'' if the $R^{2}$ does not meet some subjective
standard.
\end{itemize}
\end{frame}


\lyxframeend{}\section{Show My Work: Derivation of $\hat{\beta}_{0}$ and $\hat{\beta}_{1}$}


\lyxframeend{}\lyxframe{SMW: Use Calculus to Minimize $S(\hat{\beta}_{0},\hat{\beta}_{1})$}
\begin{topcolumns}%{}


\column{5cm}
\begin{itemize}
\item Must find the minimum $S$, which is shaped like a bowl
\item Find combination of $(\hat{\beta}_{0}$ $\hat{\beta}_{1})$ where
the function is ``flat'', at bottom of bowl
\item First Order Conditions: 
\begin{equation}
\frac{\partial S(\hat{\beta}_{0},\hat{\beta}_{1})}{\partial\hat{\beta}_{0}}=0
\end{equation}
 and 
\begin{equation}
\frac{\partial S(\hat{\beta}_{0},\hat{\beta}_{1})}{\partial\hat{\beta}_{1}}=0
\end{equation}

\end{itemize}

\column{6cm}


Sketch something here:

\end{topcolumns}%{}

\lyxframeend{}\lyxframe{SMW: First Order Condition for $\hat{\beta}_{0}$:}

\[
\frac{\partial S(\hat{\beta}_{0},\hat{\beta}_{1})}{\partial\hat{\beta}_{0}}=-2\sum(y_{i}-\hat{\beta}_{0}-\hat{\beta}_{1}\cdot x_{i})=0
\]


\[
=\sum y_{i}-\sum\hat{\beta}_{0}-\sum\hat{\beta}_{1}\cdot x_{i}=0
\]
\begin{equation}
=\sum y_{i}-N\cdot\hat{\beta}_{0}-\hat{\beta}_{1}\cdot\sum x_{i}=0\label{N1}
\end{equation}



\lyxframeend{}\lyxframe{SMW: First Order Condition for $\hat{\beta}_{1}$:}

\[
\frac{\partial S(\hat{\beta}_{0},\hat{\beta}_{1})}{\partial\hat{\beta}_{1}}=-2\sum(y_{i}-\hat{\beta}_{0}-\hat{\beta}_{1}\cdot x_{i})x_{i}=0
\]


\begin{equation}
=\sum y_{i}-\sum\hat{\beta}_{0}\cdot x_{i}-\sum\hat{\beta}_{1}\cdot x_{i}^{2}=0\label{N2}
\end{equation}



\lyxframeend{}\lyxframe{SMW: Normal Equations.}

Equations \ref{N1} and \ref{N2} can be re-arranged as the so-called
``normal equations''.

\[
\sum y_{i}=N\hat{\beta}_{0}+\left(\sum x_{i}\right)\hat{\beta}_{1}
\]


and

\[
\sum x_{i}y_{i}=\left(\sum x_{i}\right)\hat{\beta}_{0}+\left(\sum x_{i}^{2}\right)\hat{\beta}_{1}
\]



\lyxframeend{}\lyxframe{SMW: Note that is a LINEAR Matrix Equation}

\begin{equation}
\begin{array}{cccc}
\begin{array}{c}
\left[\begin{array}{c}
\sum y_{i}\\
\sum x_{i}y_{i}
\end{array}\right]\end{array} & = & \left[\begin{array}{cc}
N & \sum x_{i}\\
\sum x_{i} & \sum x_{i}^{2}
\end{array}\right] & \left[\begin{array}{c}
\hat{\beta}_{0}\\
\hat{\beta}_{1}
\end{array}\right]\end{array}\label{matrix}
\end{equation}


Refer to the coefficient estimates as $\hat{\beta}$ :

$\hat{\beta}=\left[\begin{array}{c}
\hat{\beta}_{0}\\
\hat{\beta}_{1}
\end{array}\right]$, 


\lyxframeend{}\lyxframe{SMW: The Solution}
\begin{itemize}
\item The ``sum of squares minimizing'' estimate vector is
\begin{equation}
\hat{\beta}^{OLS}=(X^{T}X)^{-1}X^{T}y
\end{equation}

\item Definition: $X$ is predictor ``design matrix'', $X=\left[\begin{array}{cc}
1 & x_{1}\\
1 & x_{2}\\
\vdots & \vdots\\
1 & x_{N-1}\\
1 & x_{N}
\end{array}\right]$
\item And $y=\left[\begin{array}{c}
y_{1}\\
y_{2}\\
\ldots\\
y_{N-1}\\
y_{N}
\end{array}\right]$
\end{itemize}

\lyxframeend{}
\end{document}
