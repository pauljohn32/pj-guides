%% LyX 2.0.7 created this file.  For more info, see http://www.lyx.org/.
%% Do not edit unless you really know what you are doing.
\documentclass[10pt,english]{beamer}
\usepackage{lmodern}
\renewcommand{\sfdefault}{lmss}
\renewcommand{\ttdefault}{lmtt}
\usepackage[T1]{fontenc}
\usepackage[latin9]{inputenc}
\setcounter{secnumdepth}{3}
\setcounter{tocdepth}{3}
\usepackage{wasysym}

\makeatletter
%%%%%%%%%%%%%%%%%%%%%%%%%%%%%% Textclass specific LaTeX commands.
\usepackage{Sweavel}
 \long\def\lyxframe#1{\@lyxframe#1\@lyxframestop}%
 \def\@lyxframe{\@ifnextchar<{\@@lyxframe}{\@@lyxframe<*>}}%
 \def\@@lyxframe<#1>{\@ifnextchar[{\@@@lyxframe<#1>}{\@@@lyxframe<#1>[]}}
 \def\@@@lyxframe<#1>[{\@ifnextchar<{\@@@@@lyxframe<#1>[}{\@@@@lyxframe<#1>[<*>][}}
 \def\@@@@@lyxframe<#1>[#2]{\@ifnextchar[{\@@@@lyxframe<#1>[#2]}{\@@@@lyxframe<#1>[#2][]}}
 \long\def\@@@@lyxframe<#1>[#2][#3]#4\@lyxframestop#5\lyxframeend{%
   \frame<#1>[#2][#3]{\frametitle{#4}#5}}
 \def\lyxframeend{} % In case there is a superfluous frame end
 \newenvironment{topcolumns}{\begin{columns}[t]}{\end{columns}}

%%%%%%%%%%%%%%%%%%%%%%%%%%%%%% User specified LaTeX commands.
\usepackage{dcolumn}
\usepackage{booktabs}

%\usepackage{Sweavel}


% use 'handout' to produce handouts
%\documentclass[handout]{beamer}
\usepackage{wasysym}
\usepackage{pgfpages}
\newcommand{\vn}[1]{\mbox{{\it #1}}}\newcommand{\vb}{\vspace{\baselineskip}}\newcommand{\vh}{\vspace{.5\baselineskip}}\newcommand{\vf}{\vspace{\fill}}\newcommand{\splus}{\textsf{S-PLUS}}\newcommand{\R}{\textsf{R}}


%\setbeamercovered{transparent}
% or whatever (possibly just delete it)

% In document Latex options:
\fvset{listparameters={\setlength{\topsep}{0em}}}
\def\Sweavesize{\normalsize} 
\def\Rcolor{\color{black}} 
\def\Rbackground{\color[gray]{0.95}}

\usepackage{graphicx}
\usepackage{listings}
\lstset{tabsize=2, breaklines=true,style=Rstyle}
\usetheme{Antibes}
% or ...

%\setbeamercovered{transparent}
% or whatever (possibly just delete it)

%\mode<presentation>
%{
 % \usetheme{KU}
 % \usecolortheme{dolphin} %dark blues
%}


%%not for article, but for presentation
\mode<presentation>{
\newcommand\makebeamertitle{\frame{\maketitle}}}


%%only for presentation
\mode<presentation>{
\setbeamertemplate{frametitle continuation}[from second]
\renewcommand\insertcontinuationtext{...}
}

%\usepackage{handoutWithNotes}
%\pgfpagesuselayout{3 on 1 with notes}[letterpaper, border shrink=5mm]

%%for presentations
\mode<presentation>{
\expandafter\def\expandafter\insertshorttitle\expandafter{%
 \insertshorttitle\hfill\insertframenumber\,/\,\inserttotalframenumber}}

\makeatother

\usepackage{babel}
\begin{document}

% In document Latex options:
\fvset{listparameters={\setlength{\topsep}{0em}}}

\def\Sweavesize{\scriptsize} 
\def\Rcolor{\color{black}} 
\def\Rbackground{\color[gray]{0.90}}

\begin{Schunk}
\begin{Sinput}
 ## Other settings I like
 options(device = pdf)
 options(useFancyQuotes = FALSE) 
 op <- par() 
 pjmar <- c(5.1, 5.1, 1.5, 2.1) 
 options(SweaveHooks=list(fig=function() par(mar=pjmar, ps=12)))
 pdf.options(onefile=F,family="Times",pointsize=12)
\end{Sinput}
\end{Schunk}



\title[Heteroskedasticity]{Heteroskedasticity in Regression }


\author{Paul E. Johnson\inst{1} \and \inst{2}}


\institute[K.U.]{\inst{1}Department of Political Science\and \inst{2}Center for
Research Methods and Data Analysis, University of Kansas}


\date[2020]{2020}

\makebeamertitle


\AtBeginSection[]{

  \frame<beamer>{ 

    \frametitle{Outline}   

    \tableofcontents[currentsection,currentsubsection] 

  }

}


\lyxframeend{}\section{Introduction}


\lyxframeend{}\lyxframe{Remember the Theory}
\begin{itemize}
\item Linear Model 
\[
y_{i}=\beta_{0}+\beta_{1}x1_{i}+e_{i}
\]

\item About the error term, we assumed, for all $i$, 

\begin{itemize}
\item $E(e_{i})=0\, for\, all\, i$ 
\item $Var(e_{i})=E[(e_{i}-E(e_{i}))^{2}]=\sigma^{2}$ (Homoskedasticity). 
\item There is no $i$ subscript on $\sigma^{2}$. It is the same for all
rows.
\end{itemize}
\item Heteroskedasticity (or heteroscedasticity): the assumption of homogeneous
variance is violated. 
\end{itemize}

\lyxframeend{}\lyxframe{Homoskedasticity means}

\[
Var(e)=\left[\begin{array}{ccccc}
\sigma_{e}^{2} & 0 & 0 & 0 & 0\\
0 & \sigma_{e}^{2} & 0 & 0 & 0\\
0 & 0 & \sigma_{e}^{2} & 0 & 0\\
0 & 0 & 0 & \ddots & 0\\
0 & 0 & 0 & 0 & \sigma_{e}^{2}
\end{array}\right]
\]



\lyxframeend{}\lyxframe{Heteroskedasticity depicted in one of these ways}

\[
Var(e)=\left[\begin{array}{ccccc}
\sigma_{e}^{2}w_{1} &  &  &  & 0\\
 & \sigma_{e}^{2}w_{2}\\
 &  & \sigma_{e}^{2}w_{3}\\
 &  &  & \vdots\\
0 &  &  &  & \sigma_{e}^{2}w_{N}
\end{array}\right]\, or\,\left[\begin{array}{ccccc}
\frac{\sigma_{e}^{2}}{w_{1}} &  &  &  & 0\\
 & \frac{\sigma_{e}^{2}}{w_{2}}\\
 &  & \frac{\sigma_{e}^{2}}{w_{3}}\\
 &  &  & \vdots\\
0 &  &  &  & \frac{\sigma_{e}^{2}}{w_{N}}
\end{array}\right]
\]


I get confused a lot when comparing textbooks because of this problem!


\lyxframeend{}\lyxframe{And we are usually really interested the inverse of that}

\[
W_{i}=\left[\begin{array}{ccccc}
\frac{1}{Var(e_{1})} &  &  &  & 0\\
 & \frac{1}{Var(e_{2})}\\
 &  & \frac{1}{Var(e_{3})}\\
 &  &  & \vdots\\
0 &  &  &  & \frac{1}{Var(e_{N})}
\end{array}\right]\,
\]


Sometimes they call the variance of the errors $\Sigma$ and so the
weights are $\Sigma^{-1}$.


\lyxframeend{}\lyxframe{Why Bother With this Now?}
\begin{enumerate}
\item It justifies introduction of WLS estimator (good reason). That's one
step from GLS, which is neat.
\item It justifies ``robust'' (or ``heteroskedasticity consistent'')
covariance estimators (great reason)
\item Lets me sketch some proofs and demonstrate reasoning about estimators
(fabulous reason \smiley{})
\end{enumerate}

\lyxframeend{}

\begin{frame}[containsverbatim]
\frametitle{Consequences of Heteroskedasticity 1: $\hat{\beta}^{OLS}$ still unbiased, consistent}
\begin{itemize}
\item OLS Estimates of $\beta_{0}$ and $\beta_{1}$ are still unbiased
and consistent.

\begin{itemize}
\item Unbiased: $E[\hat{\beta}^{OLS}]=\beta$
\item Consistent: As $N\rightarrow\infty$, $\hat{\beta}^{OLS}$ tends to
$\beta$ in probability limit.
\end{itemize}
\item If the predictive line was ``right'' before, It's still right now.
\item Proof on next slide: Key in method is apply E{[}{]} to both sides
of formula for $\hat{\beta}_{1}$
\end{itemize}
\end{frame}


\lyxframeend{}\lyxframe{[allowframebreaks]Proof: $\hat{\beta}^{OLS}$ Still Unbiased}
\begin{itemize}
\item Easy! Begin with ``mean centered'' data. The slope estimate from
OLS with one variable:
\[
\hat{\beta}_{1}=\frac{\sum x_{i}\cdot y_{i}}{\sum x_{i}^{2}}=\frac{\sum x_{i}(b\cdot x_{i}+e_{i})}{\sum x_{i}^{2}}=\frac{\beta_{1}\sum x_{i}^{2}+\sum x_{i}\cdot e_{i}}{\sum x_{i}^{2}}=\beta_{1}+\frac{\sum x_{i}\cdot e_{i}}{\sum x_{i}^{2}}
\]

\item Apply the Expected value operator to both sides:
\[
E[\hat{\beta}_{1}]=E(\beta_{1})+E(\frac{\sum x_{i}\cdot e_{i}}{\sum x_{i}^{2}})
\]
\[
E[\hat{\beta}_{1}]=\beta_{1}+E(\frac{\sum x_{i}\cdot e_{i}}{\sum x_{i}^{2}})=\beta_{1}+(\frac{\sum E[x_{i}\cdot e_{i}]}{\sum x_{i}^{2}})
\]

\item Regression assumes errors uncorrelated with $x_{i}$, $E[x_{i}e_{i}]=0$,
the work is done
\end{itemize}
\[
E(\hat{\beta}_{1})=\beta_{1}
\]



\lyxframeend{}\lyxframe{Consequence 2. OLS formula for $\widehat{Var(\hat{\beta})}$ is wrong}
\begin{enumerate}
\item The derivation of the ``true variance'' of the OLS estimator, $Var(\hat{\beta})$,
assumed $Var[e_{i}]$ was same for all $i$.
\item Thus our formula that estimated $Var(\hat{\beta})$, $\widehat{Var(\hat{\beta})}$
is wrong. And it's square root, the $std.err.(\hat{\beta})$ is wrong.
\item Thus t-tests from OLS, $\hat{\beta}/std.err.(\hat{\beta})$ are WRONG
(too big).
\item And your p-values are WRONG (too small).
\end{enumerate}

\lyxframeend{}\lyxframe{Proof: OLS $\widehat{Var(\hat{\beta})}$ Wrong }
\begin{itemize}
\item Variance of $e_{i}$: $Var(e_{i})$. 
\item The variance of the OLS slope estimator, $Var(\hat{b}_{1})$, in ``mean-centered
(or deviations) form'':
\end{itemize}
\begin{equation}
Var(\hat{\beta}_{1})=Var\left[\frac{\sum x_{i}\cdot e_{i}}{\sum x_{i}^{2}}\right]=\frac{Var[\sum x_{i}e_{i}]}{\left(\sum x_{i}^{2}\right)^{2}}=\frac{\sum Var(x_{i}e_{i})}{\left(\sum x_{i}^{2}\right)^{2}}=\frac{\sum x_{i}^{2}\cdot Var(e_{i})}{\left(\sum x_{i}^{2}\right)^{2}}\label{eq:Varbhat}
\end{equation}

\begin{itemize}
\item We assume all $Var(e_{i})$ are equal, and we put in the MSE as an
estimate of it.
\end{itemize}
\begin{equation}
\widehat{Var(\hat{\beta}_{1})}=\frac{MSE}{\sum x_{i}^{2}}
\end{equation}



\lyxframeend{}\lyxframe{Proof: OLS $Var(\hat{\beta})$Wrong (page 2)}
\begin{itemize}
\item Instead, suppose the ``true'' variance


\begin{equation}
Var(e_{i})=s^{2}+s_{i}^{2}
\end{equation}
 (a common minimum variance $s^{2}$ plus an additional individualized
variance $s_{i}^{2}$). 

\item Plug this into (\ref{eq:Varbhat}):
\begin{equation}
\frac{\sum x_{i}^{2}(s^{2}+s_{i}^{2})}{\left(\sum x_{i}^{2}\right)^{2}}=\frac{s^{2}}{\sum x_{i}^{2}}+\frac{\sum x_{i}\cdot s_{i}^{2}}{\left(\sum x_{i}^{2}\right)^{2}}
\end{equation}

\item The first term is \textquotedbl{}roughly\textquotedbl{} what OLS would
calculate for the variance of $\hat{\beta}_{1}$. 
\item The second term is the additional \textquotedbl{}true variance\textquotedbl{}
in $\hat{\beta}_{1}$ that the OLS formula $\widehat{V(\hat{\beta}_{1})}$
does not include.
\end{itemize}

\lyxframeend{}\lyxframe{Consequence 3: $\hat{\beta}^{OLS}$ is Inefficient}
\begin{enumerate}
\item $\hat{\beta}_{i}^{OLS}$is \textit{inefficient}: It has higher variance
than the ``weighted'' estimator. 
\item Note that to prove an estimator is ``inefficient'', it is necessary
to provide an alternative estimator that has lower variance.
\item WLS: Weighted Least Squares estimator, $\hat{\beta}_{1}^{WLS}$.
\item The Sum of Squares to be minimized now includes a weight for each
case
\begin{equation}
SS(\hat{\beta}_{0},\hat{\beta}_{1})=\sum_{i=1}^{N}W_{i}(y-\hat{y}_{i})^{2}
\end{equation}

\item The weights chosen to ``undo'' the heteroskedasticity. 
\begin{equation}
W_{i}^{2}=1/Var(e_{i})
\end{equation}

\end{enumerate}

\lyxframeend{}\lyxframe{[allowframebreaks]Some insight from matrix view}
\begin{itemize}
\item Think of the prediction errors (residuals) as a vector
\end{itemize}
\begin{equation}
y-\hat{y}=\left[\begin{array}{c}
y_{1}-\hat{y}_{1}\\
y_{2}-\hat{y}_{2}\\
y_{3}-\hat{y}_{3}\\
\vdots\vdots\\
y_{N}-\hat{y}_{N}
\end{array}\right]
\end{equation}

\begin{itemize}
\item Think of the Sum of squares in OLS as product of $[y-\hat{y}]$ transpose
and $[y-\hat{y}]$
\end{itemize}
\begin{equation}
\sum_{i=1}^{N}(y_{i}-\hat{y}_{i})^{2}=[y-\hat{y}]^{T}[y-\hat{y}]
\end{equation}


\begin{equation}
\left[\begin{array}{ccccc}
y_{1}-\hat{y}_{1} & y_{2}-\hat{y}_{2} &  & \cdots & y_{N}-\hat{y}_{N}\end{array}\right]\left[\begin{array}{c}
y_{1}-\hat{y}_{1}\\
y_{2}-\hat{y}_{2}\\
\vdots\vdots\\
y_{N}-\hat{y}_{N}
\end{array}\right]
\end{equation}

\begin{itemize}
\item Weighted least squares can be seen as the insertion of a square diagonal
matrix in the middle.
\end{itemize}
{\footnotesize{}
\begin{equation}
\left[\begin{array}{ccccc}
y_{1}-\hat{y}_{1} & y_{2}-\hat{y}_{2} &  & \cdots & y_{N}-\hat{y}_{N}\end{array}\right]\left[\begin{array}{cccc}
\frac{1}{Var(e_{1})} &  &  & 0\\
 & \frac{1}{Var(e_{2})}\\
 &  & \ddots\\
0 &  &  & \frac{1}{Var(e_{N})}
\end{array}\right]\left[\begin{array}{c}
y_{1}-\hat{y}_{1}\\
y_{2}-\hat{y}_{2}\\
y_{3}-\hat{y}_{3}\\
\vdots\vdots\\
y_{N}-\hat{y}_{N}
\end{array}\right]
\end{equation}
}{\footnotesize \par}
\begin{itemize}
\item WLS always proceeds with the assumption that the off-diagonals are
0, so none of the errors are correlated with each other. 
\item GLS: Generalized Least Squares introduces those off diagonal components.
GLS is used in time series analysis.
\end{itemize}

\lyxframeend{}\lyxframe{Covariance matrix of error terms}
\begin{itemize}
\item This thing is a weighting matrix
\end{itemize}
\textrm{
\[
\left[\begin{array}{cccc}
\frac{1}{Var(e_{1})} &  &  & 0\\
 & \frac{1}{Var(e_{2})}\\
 &  & \ddots\\
0 &  &  & \frac{1}{Var(e_{N})}
\end{array}\right]
\]
}

Is usually simplified in various ways. 
\begin{itemize}
\item Factor out a common parameter, so each individual's error variance
is proportional 
\end{itemize}
\[
\frac{1}{\sigma_{e}^{2}}\left[\begin{array}{cccc}
\frac{1}{w_{1}} &  &  & 0\\
 & \frac{1}{w_{2}}\\
 &  & \ddots\\
0 &  &  & \frac{1}{w_{N}}
\end{array}\right]
\]



\lyxframeend{}

\begin{frame}
\frametitle{Example. Suppose variance proportional to $x_{i}^2$}

\begin{topcolumns}%{}


\column{5cm}


The ``truth'' is 
\begin{equation}
y_{i}=3+0.25x_{i}+e_{i}
\end{equation}


Homoskedastic: 
\begin{equation}
Std.Dev.(e_{i})=\sigma_{e}=10
\end{equation}


Heteroskedastic: 

\begin{equation}
Std.Dev.(e_{i})=0.05*(x_{i}-min(x_{i}))*\sigma_{e}
\end{equation}


\column{7cm}



\includegraphics[width=6cm]{plots/t-sim10}

\end{topcolumns}%{}
\end{frame}



\begin{frame}[containsverbatim]
\frametitle{Draw 1000 Heteroskedastic Regressions}
\begin{topcolumns}%{}


\column{5cm}
\begin{itemize}
\item 1 example scatter
\item 1000 fitted lines in gray
\item True regression line in red
\end{itemize}

\column{7cm}


\includegraphics[width=6cm]{plots/t-sim21}

\end{topcolumns}%{}
\end{frame}



\begin{frame}[containsverbatim]
\frametitle{Compare Lines of 1000 Fits (Homo vs Heteroskedastic)}
\begin{topcolumns}%{}


\column{6cm}


\includegraphics[width=6cm]{plots/t-sim21}


\column{6cm}


\includegraphics[width=6cm]{plots/t-sim41}

\end{topcolumns}%{}
\end{frame}


\begin{frame}[containsverbatim]
\frametitle{Histograms of Slope Estimates (w/Kernel Density Lines) }

\includegraphics[width=10cm]{plots/t-sim50}

\end{frame}

\begin{frame}
\frametitle{So, the Big Message Is}
\begin{itemize}
\item Heteroskedasticity leaves slope estimates correct on average
\item Heteroskedasticity inflates the amount of uncertainty in the estimates.
\item Weird, confusing effect on t-ratios.
\end{itemize}

\end{frame}


\lyxframeend{}\section{Fix \#1: Robust Standard Errors}


\lyxframeend{}\lyxframe{Robust estimate of the variance of $\hat{b}$ }
\begin{itemize}
\item We should not use the OLS formula for $\widehat{Var(\hat{b})}$ anymore
(and hence the $std.err.(\hat{b}$) from OLS is no good either).
\item Robust ``heteroskedasticity consistent'' variance estimator.
\item Wish these were accurate for small samples, but they generally are
not
\item Best we can hope for is 

\begin{itemize}
\item consistent standard errors 
\item asymptotically valid t-tests
\end{itemize}
\item Note: This does not ``fix'' $\hat{b}^{OLS}$. It just gives us more
accurate t-ratios by correcting $std.err(\hat{b}).$
\end{itemize}

\lyxframeend{}\lyxframe{Robust Std.Err. in a Nutshell}
\begin{itemize}
\item Recall: the variance-covariance matrix of the errors assumed by OLS.
\begin{equation}
Var(e)=E(e\cdot e'|X)=\left[\begin{array}{ccccc}
\sigma_{e}^{2} & 0 & 0 & 0 & 0\\
0 & \sigma_{e}^{2} & 0 & 0 & 0\\
0 & 0 & \sigma_{e}^{2} & 0 & 0\\
\dots &  & \dots & \dots & 0\\
0 & 0 & 0 & 0 & \sigma_{e}^{2}
\end{array}\right]\label{eq:Vare_homo-1}
\end{equation}

\end{itemize}
\begin{equation}
=\sigma_{e}^{2}\left[\begin{array}{ccccc}
1 & 0 & 0 & 0 & 0\\
0 & 1 & 0 & 0 & 0\\
0 & 0 & 1 & 0 & 0\\
\dots &  & \dots & \dots & 0\\
0 & 0 & 0 & 0 & 1
\end{array}\right]=\sigma_{e}^{2}I\label{eq:Vare_homo-1-1}
\end{equation}



\lyxframeend{}\lyxframe{Heteroskedastic Covariance Matrix}
\begin{itemize}
\item If there's heteroskedasticity, we have to allow the possibility like
this:
\end{itemize}
\[
Var(e)=E[e\cdot e'|X]=\left[\begin{array}{ccccc}
\sigma_{1}^{2} & 0 & 0 & 0 & 0\\
0 & \sigma_{2}^{2} & 0 & 0 & 0\\
0 & 0 & \ddots & \cdots & 0\\
0 & 0 & 0 & \sigma_{N-1}^{2} & 0\\
0 & 0 & 0 & 0 & \sigma_{N}^{2}
\end{array}\right]
\]



\lyxframeend{}\lyxframe{Robust Std.Err. : Use Variance Estimates}
\begin{itemize}
\item Fill in estimates for the case-specific error variances
\end{itemize}
\[
\widehat{Var(e)}=\left[\begin{array}{ccccc}
\widehat{\sigma_{1}^{2}} & 0 & 0 & 0 & 0\\
0 & \widehat{\sigma_{2}^{2}} & 0 & 0 & 0\\
0 & 0 & \ddots & \cdots & 0\\
0 & 0 & 0 & \widehat{\sigma_{N-1}^{2}} & 0\\
0 & 0 & 0 & 0 & \widehat{\sigma_{N}^{2}}
\end{array}\right]
\]

\begin{itemize}
\item Embed those estimates into the larger formula that is used to calculate
the robust standard errors.
\item Famous paper


White, Halbert. (1980). A Heteroskedasticity-Consistent Covariance
Matrix Estimator and a Direct Test for Heteroskedasticity. \emph{Econometrica},
48(4), 817-838.


Robust estimator originally proposed by Huber (1967), but was forgotten

\end{itemize}

\lyxframeend{}\lyxframe{Calculating The Robust Estimator of $Var(\hat{b})$}
\begin{itemize}
\item The true variance of the OLS estimator is
\end{itemize}
\begin{equation}
Var(\hat{b}_{1})=\frac{\sum x_{i}^{2}Var(e_{i})}{(\sum x_{i}^{2})^{2}}
\end{equation}


\textrm{Assuming Homoskedasticity, estimate $\ensuremath{\sigma_{e}^{2}}$
with MSE}.
\begin{equation}
\widehat{Var(\hat{b}_{1})}=\frac{MSE}{\sum x_{i}^{2}}\,\mbox{and the square root of that is \ensuremath{std.err.(\hat{b}_{1})}}
\end{equation}

\begin{itemize}
\item The robust versions replace $Var(e_{i})$ with other estimates. White's
suggestion was 
\begin{equation}
Robust\,\widehat{Var(\hat{b}_{1})}=\frac{\sum x_{i}^{2}\cdot\hat{e}_{i}^{2}}{(\sum x_{i}^{2})^{2}}\label{eq:WhiteBiVar}
\end{equation}



$\hat{e_{i}}^{2}$ : the ``squared residual'', used in place of
the unknown error variance.

\end{itemize}

\lyxframeend{}


\lyxframeend{}\section{Weighted Least Squares}


\lyxframeend{}\lyxframe{WLS is Efficient}
\begin{itemize}
\item WLS has lower variance than OLS
\item However, WLS is more difficult to estimate. 
\item WLS requires the Weights for minimizing the sum of squares.
\end{itemize}
\[
minimize\, SS(\hat{b})=\sum_{i=1}^{N}W_{i}(y_{i}-\hat{y}_{i})^{2}
\]

\begin{itemize}
\item And the weights depend on the (unknown) standard deviation of the
error term 
\begin{equation}
W_{i}=\frac{1}{\sigma_{i}}
\end{equation}

\item We need a model for the variance of the error term, $\sigma_{i}^{2}$
\end{itemize}

\lyxframeend{}\lyxframe{Feasible Weighted Least Squares. }
\begin{itemize}
\item Analysis proceeds in 2 steps.

\begin{itemize}
\item Regression is estimated to gather information about the error variance.
\item That information is used to fill in the Weight matrix with WLS
\end{itemize}
\item May revise the weights, re-fit the WLS, repeatedly until convergence.
\end{itemize}

\lyxframeend{}\subsection{Combine Subsets of a Sample}


\lyxframeend{}\lyxframe{Data From Categorical Groups}
\begin{itemize}
\item Suppose you separately investigate data for men and women
\begin{equation}
men:\, y_{i}=\beta_{0}+\beta_{1}x_{i}+e_{i}
\end{equation}

\end{itemize}
\begin{equation}
women:\, y_{i}=c_{0}+c_{1}x_{i}+u_{i}
\end{equation}

\begin{itemize}
\item Then you wonder, ``can I combine the data for men and women to estimate
one model''
\begin{equation}
humans:\, y_{i}=\beta_{0}+\beta_{1}x_{i}+\beta_{2}sex_{i}+\beta_{3}sex_{i}x_{i}+e_{i}
\end{equation}

\item This ``manages'' the differences of intercept and slope for men
and women by adding coefficients $\beta_{2}$ and $\beta_{3}$.
\item But this ASSUMED that $Var(e_{i})=Var(u_{i})$.
\item We should have tested for homoskedasticity (the ability to pool the
2 samples).
\end{itemize}

\lyxframeend{}\subsection{Random coefficient model}


\lyxframeend{}\lyxframe{Methods Synonyms}

The basic idea is to say that the linear model has ``extra'' random
error terms. 
\begin{columns}%{}


\column{6cm}


Synonyms
\begin{itemize}
\item Random effects models
\item Mixed Models
\item Hierarchical Linear Models (HLM)
\item Multi-level Models (MLM)
\end{itemize}

\column{6cm}


This ``Laird and Ware'' notation has now become a standard. Let
the ``fixed'' coefficients be $\beta$'s, but suppose in addition
there are random coefficients $b\sim N(0,\sigma_{b}^{2})$. 
\begin{equation}
y=X\beta+Zb+e
\end{equation}

\begin{itemize}
\item I'll probably write something on the board.
\end{itemize}
\end{columns}%{}

\lyxframeend{}\lyxframe{Simple Random Coefficient Model}
\begin{itemize}
\item Start with the regression model that has a different slope for each
case:
\begin{equation}
y_{i}=\beta_{0}+\beta_{i}x_{i}+e_{i}
\end{equation}

\item Slope is a \textquotedbl{}random coefficient'' with 2 parts
\[
\beta_{i}=\beta_{1}+u_{i}
\]


\begin{itemize}
\item $\beta_{1}$ is the ``same'' for all cases
\item $u_{i}$ is noise in the slope that is individually assigned. It has
expected value 0 and a variance $\sigma_{u}^{2}$.
\end{itemize}
\end{itemize}

\lyxframeend{}\lyxframe{Simple Random Coefficient Model}
\begin{itemize}
\item The regression model becomes
\[
y_{i}=\beta_{0}+(\beta_{1}+u_{i})x_{i}+e_{i}
\]
\[
=\beta_{0}+\beta_{1}x_{i}+u_{i}x_{i}+e_{i}
\]

\item Note: My ``new'' error term is $u_{i}x_{i}+e_{i}$. NOT homoskedastic
\item What's the variance of that? Apply the usual rule: 
\[
Var[u_{i}x_{i}+e_{i}]=x_{i}^{2}Var(u_{i})+Var(e_{i})+2x_{i}Cov(u_{i},e_{i})
\]

\item Get rid of the last part by asserting that the 2 random effects are
uncorrelated, so we have
\end{itemize}
\[
=x_{i}^{2}\sigma_{u}^{2}+\sigma_{e}^{2}
\]



\lyxframeend{}\subsection{Aggregate Data}


\lyxframeend{}\lyxframe{With Aggregated Data, the Variance is Almost Never Homogeneous.}
\begin{itemize}
\item Each row in the data set represents a collection of observations (``group
averages'' like ``mean education'' or ``mean salary'')
\item The averaging process causes heteroskedasticity.
\item The mean $\bar{y}=\frac{\sum y_{i}}{N}$ and standard deviation $\sigma_{y}^{2}$
imply the variance of the mean is 
\[
Var(\bar{y})=\frac{Var(y_{i})}{N}=\frac{\sigma_{y}^{2}}{N}
\]

\item Regression Weights proportional to $\sqrt{N_{group}}$ should be used.
\end{itemize}

\lyxframeend{}\section{Testing for heteroskedasticity }


\lyxframeend{}\subsection{Categorical Heteroskedasticity}

\begin{frame}[containsverbatim,allowframebreaks]
\frametitle{Adapt tests from Analysis of Variance}
\begin{description}
\item [{Idea:}] Estimate the error variances for the subgroups, try to
find out if they are different. \end{description}
\begin{itemize}
\item Bartlett's test: Assuming normality of observations, derives a statistic
that is distributed as a $\chi^{2}$. 


\begin{Sinput}
library(lmtest)
plot(count ~ spray, data = InsectSprays)
bartlett.test(count ~ spray, data = InsectSprays)
\end{Sinput}

\end{itemize}
Bartlett, M. S. (1937). Properties of sufficiency and statistical
tests. \emph{Proceedings of the Royal Society of London Series A}
160, 268\textendash{}282. 
\begin{itemize}
\item Fligner-Killeen Test : Robust against non-normality (less likely to
confuse non-normality for heteroskedasticity)


\begin{Sinput}
library(lmtest)
fligner.test(count ~ spray, data = InsectSprays)
\end{Sinput}

\end{itemize}
William J. Conover, Mark E. Johnson and Myrle M. Johnson (1981). A
comparative study of tests for homogeneity of variances, with applications
to the outer continental shelf bidding data. \emph{Technometrics}
23, 351\textendash{}361. 
\begin{itemize}
\item Levene's test
\end{itemize}
\begin{Sinput}
library(car)
leveneTest (y~x*z, data=dat) ##x and z must be factors
\end{Sinput}

\end{frame}


\lyxframeend{}\subsection{Checking for Continuous Heteroskedasticity}

\begin{frame}[containsverbatim, allowframebreaks]
\frametitle{Goldfield Quandt test}

S.M. Goldfeld \& R.E. Quandt (1965), Some Tests for Homoskedasticity.
\emph{Journal of the American Statistical Association} 60, 539\textendash{}547 
\begin{itemize}
\item Consider a continuous numeric predictor. Exclude observations ``in
the middle'' and then compare observed variances of the left and
right. 
\item Draw a picture on Board here!
\item HOWTO: compare the Error Sum of Squares for 2 chunks of data. 
\[
F=\frac{ESS_{2}}{ESS_{1}}=\frac{\mbox{the "lower set" ESS}}{\mbox{the "upper set" ESS}}
\]
 and the degrees of freedom for both numerator and denominator are
$(N-d-4)/2$ , where $d$ is the number of excluded observations.
\item The more observations you exclude, the smaller will be your degrees
of freedom, meaning your F value must be larger.
\end{itemize}
\begin{Sinput}
library(lmtest)
gqtest(y ~ x, fraction=0.2, order.by=c(z))
gqtest(y ~ x, point=0.4, order.by=c(z))
\end{Sinput}

\end{frame}

\begin{frame}[containsverbatim, allowframebreaks]
\frametitle{Example of Goldfield-Quandt Test: Continuous X}
\begin{itemize}
\item Use heteroskedastic model from previous illustration.\end{itemize}
\begin{topcolumns}%{}


\column{5cm}




\def\Sweavesize{\scriptsize}
\begin{Schunk}
\begin{Sinput}
 mymod <- lm(y~x)
 gqtest(mymod, fraction=0.2, order.by= ~ x)
\end{Sinput}
\begin{Soutput}
	Goldfeld-Quandt test

data:  mymod
GQ = 4.497, df1 = 198, df2 = 198, p-value < 2.2e-16
alternative hypothesis: variance increases from segment 1 to 2
\end{Soutput}
\end{Schunk}


This excludes 20\% of the cases from the middle, and compares the
variances of the outer sections.

\column{7cm}




\includegraphics[width=7cm]{plots/t-gq11}

\end{topcolumns}%{}
\end{frame}


\lyxframeend{}\subsection{Toward a General Test for Heteroskedasticity}

\begin{frame}[containsverbatim, allowframebreaks]
\frametitle{Test for Predictable Squared Residuals}
\begin{itemize}
\item Versions of this test were proposed in Breusch \& Pagan (1979) and
White (1980). 
\item Basic Idea: If errors are homogeneous, the variance of the residuals
should not be predictable with the use of input variables. 
\item T.S. Breusch \& A.R. Pagan (1979), A Simple Test for Heteroscedasticity
and Random Coefficient Variation. \emph{Econometrica} 47, 1287\textendash{}1294 
\end{itemize}
\end{frame}

\begin{frame}[containsverbatim, allowframebreaks]
\frametitle{Breusch-Pagan test}
\begin{itemize}
\item Model the squared residuals with the other predictors ($Z1_{i},$
etc) like this:
\[
\frac{\widehat{e_{i}}^{2}}{\widehat{\sigma^{2}}}=\gamma_{o}+\gamma_{1}Z1_{1}+\gamma_{2}Z2_{i}
\]

\end{itemize}
Here, $\widehat{\sigma^{2}}=MSE$. 
\begin{itemize}
\item If the error is homoskedastic/Normal, the coefficients $\gamma_{0}$,
$\gamma_{1}$, and $\gamma_{2}$ will all equal zero. The input variables
$Z$ can be the same as the original regression, but may are include
squared values of those variables. 
\item BP contend that $\frac{1}{2}RSS$ (the regression sum of squares)
should be distributed as $\chi^{2}$ with degrees of freedom equal
to the number of Z variables.
\end{itemize}
\begin{Sinput}
library(lmtest)
mod <- lm (y ~ x1 + x2 +x3, data=dat)
bptest( mod , studentize=F) ##for the classic bp test
\end{Sinput}

\end{frame}

\begin{frame}[containsverbatim, allowframebreaks]
\frametitle{A Robust Version of the Test}
\begin{itemize}
\item The original form of the BP test assumed Normally distributed errors.
Non-normal, but homoskedastic, error, might cause the test to indicate
there is heteroskedasticity.
\item A ``studentized'' version of the test was proposed by Koenker (1981),
that's what lmtest's bptest uses by default.
\end{itemize}
\begin{Sinput}
library(lmtest)
mod <- lm (y ~ x1 + x2 +x3, data=dat)
bptest( mod ) ## Koenker's robust version
\end{Sinput}

\end{frame}

\begin{frame}[containsverbatim, allowframebreaks]
\frametitle{White's Version of the Test}
\begin{itemize}
\item White's general test for heteroskedasticity is another view of the
same exercise. Run the regression
\end{itemize}
\[
\widehat{e_{i}}^{2}=\gamma_{o}+\gamma_{1}Z1_{1}+\gamma_{2}Z2_{i}+\ldots
\]


The $Z$'s should include the predictors, their squares, and cross
products.
\begin{itemize}
\item Under the assumption of homoskedasticity, $N\cdot R^{2}$ is asymptotically
distributed as $\chi_{p}^{2}$, where N is the sample size, $R^{2}$
is the coefficient of determination from the fitted model, and $p$
is the number of $Z$ variables used in the regression.
\item Algebraically equivalent to robust version of bp test (Waldman, 1983). 
\end{itemize}
\end{frame}


\lyxframeend{}\section{Appendix: Robust Variance Estimator Derivation}

\begin{frame}[allowframebreaks]
\frametitle{Where Robust $Var(\hat{\beta})$ Comes From}
\begin{itemize}
\item The OLS estimator in matrix form
\begin{equation}
\hat{b}=(X'X)^{-1}X'Y\label{bhat}
\end{equation}

\item If $e_{i}$ is homoskedastic, the ``true variance'' of the estimates
of the $b$'s is~
\begin{equation}
Var(\hat{b})=\sigma^{2}\cdot(X'X)^{-1}\label{eq:trueVarb1}
\end{equation}
 Replace $\sigma^{2}$, with the Mean Squared Error (MSE). 
\begin{equation}
\widehat{Var(\hat{b})}=MSE\cdot(X'X)^{-1}\label{eq:varbhat_homo}
\end{equation}

\end{itemize}
\end{frame}

\begin{frame}[allowframebreaks]
\frametitle{Where OLS Exploits Homoskedastic Assumption}
\begin{itemize}
\item In the OLS derivation of (\ref{eq:varbhat_homo}), one arrives at
this intermediate step:
\end{itemize}
\begin{equation}
OLS:\, Var(\hat{b})=(X'X)^{-1}(X'Var(e)X)(X'X)^{-1}\label{eq:trueVarB2}
\end{equation}

\begin{itemize}
\item The OLS derivation exploits homoskedasticity, which appears as 
\begin{equation}
Var(e)=E(e\cdot e'|X)=\left[\begin{array}{ccccc}
\sigma^{2} & 0 & 0 & 0 & 0\\
0 & \sigma^{2} & 0 & 0 & 0\\
0 & 0 & \sigma^{2} & 0 & 0\\
\dots &  & \dots & \dots & 0\\
0 & 0 & 0 & 0 & \sigma^{2}
\end{array}\right]\label{eq:Vare_homo}
\end{equation}

\end{itemize}
\begin{equation}
=\sigma^{2}\left[\begin{array}{ccccc}
1 & 0 & 0 & 0 & 0\\
0 & 1 & 0 & 0 & 0\\
0 & 0 & 1 & 0 & 0\\
\dots &  & \dots & \dots & 0\\
0 & 0 & 0 & 0 & 1
\end{array}\right]=\sigma^{2}\cdot I
\end{equation}


\[
OLS\, Var(\hat{b})=(X'X)^{-1}(X'\cdot\sigma^{2}\cdot X)(X'X)^{-1}=\sigma^{2}(X'X)^{-1}
\]


\end{frame}

\begin{frame}
\frametitle{But Heteroskedasticity Implies}
\begin{itemize}
\item If there's heteroskedasticity, we have to allow the possibility like
this:
\end{itemize}
\[
Var(e)=E[e\cdot e'|X]=\left[\begin{array}{ccccc}
\sigma_{1}^{2} & 0 & 0 & 0 & 0\\
0 & \sigma_{2}^{2} & 0 & 0 & 0\\
0 & 0 & \ddots & \cdots & 0\\
0 & 0 & 0 & \sigma_{N-1}^{2} & 0\\
0 & 0 & 0 & 0 & \sigma_{N}^{2}
\end{array}\right]
\]

\begin{itemize}
\item Those ``true variances'' are unknown. How can we estimate $Var(\hat{b})$?
\end{itemize}
\end{frame}

\begin{frame}
\frametitle{White's Idea}
\begin{itemize}
\item The variance of $e_{1}$, for example, is never observed, but the
best estimate we have for it is the mean square for that one case:
\[
\widehat{e_{1}}^{2}=(y_{1}-X_{1}\hat{b})(y_{1}-X_{1}\hat{b})
\]

\item Hence, Replace $Var(e)$ with a matrix of estimates like this:
\[
\widehat{Var(e)}=\left[\begin{array}{ccccc}
\widehat{e_{1}}^{2}\\
 & \widehat{e_{2}}^{2}\\
\\
 &  &  & \widehat{e_{N-1}}^{2}\\
 &  &  &  & \widehat{e_{N}}^{2}
\end{array}\right]
\]

\end{itemize}
\end{frame}

\begin{frame}
\frametitle{Heteroskedasticity Consistent Covariance Matrix}
\begin{itemize}
\item The ``heteroskedasticity consistent covariance matrix of $\hat{b}$''
uses $\widehat{Var(e)}$ in the formula to calculate estimated variance.
\[
hccm\, Var(\hat{b})=(X'X)^{-1}(X'\widehat{Var(e)}X)(X'X)^{-1}
\]

\item White proved that the estimator is consistent, i.e, for large samples,
the value converges to the true $Var(\hat{b})$.
\item Sometimes called an ``information sandwich'' estimator. The matrix
$(X'X)^{-1}$ is the ``information matrix''. This equation gives
us a ``sandwich'' of $X'Var(e)X$ between two pieces of information
matrix.
\end{itemize}
\end{frame}

\include{Heteroskedasticity-WLS-lecture-problems}


\lyxframeend{}
\end{document}
