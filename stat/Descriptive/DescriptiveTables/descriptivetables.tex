%% LyX 2.2.3 created this file.  For more info, see http://www.lyx.org/.
%% Do not edit unless you really know what you are doing.
\documentclass[11pt,letterpaper,english]{extarticle}
\usepackage{lmodern}
\renewcommand{\sfdefault}{lmss}
\renewcommand{\ttdefault}{lmtt}
\usepackage[T1]{fontenc}
\usepackage[utf8]{inputenc}
\setlength{\parskip}{\medskipamount}
\setlength{\parindent}{0pt}
\usepackage{babel}
\usepackage{float}
\usepackage{graphicx}
\usepackage[authoryear]{natbib}
\usepackage[unicode=true,pdfusetitle,
 bookmarks=true,bookmarksnumbered=false,bookmarksopen=false,
 breaklinks=false,pdfborder={0 0 1},backref=false,colorlinks=false]
 {hyperref}
\usepackage{breakurl}

\makeatletter

%%%%%%%%%%%%%%%%%%%%%%%%%%%%%% LyX specific LaTeX commands.
\special{papersize=\the\paperwidth,\the\paperheight}


%%%%%%%%%%%%%%%%%%%%%%%%%%%%%% Textclass specific LaTeX commands.

\@ifundefined{date}{}{\date{}}
%%%%%%%%%%%%%%%%%%%%%%%%%%%%%% User specified LaTeX commands.
% Following comment is required. Do not delete it.
%\usepackage{Sweave}

\usepackage[includehead,includefoot]{geometry}
\geometry{
lmargin=1in,
rmargin=1in,
tmargin=0.75in,
bmargin=1.0in,
headheight=0pt,
headsep=0pt,
marginparwidth=0pt,
footskip=1.5\baselineskip,
}

\usepackage{booktabs}
\usepackage{dcolumn}


%% Paste guidePreambleHeader
%%BEGIN: This begins the custom header
\usepackage{array}
\usepackage{xparse}
\usepackage{babel}
\usepackage{datetime}
\usepackage{xcolor}
\usepackage{colortbl}
\usepackage{graphicx}
\usepackage{multicol}

\definecolor{darkblue}{HTML}{1e2277}
\definecolor{kublue81186}{RGB}{0,81,186}
\newdateformat{monddyyyy}{\shortmonthname[\THEMONTH].~\THEDAY,~\THEYEAR}


%% BEGIN custom header
\ExplSyntaxOn
\NewDocumentCommand \guidesetup { m }
 {
  \keys_set:nn { guide } { #1 }
 }

\int_new:N \g_guide_authors_int
\prop_new:N \g_guide_authors_prop
\tl_new:N \l__guide_authors_tl

\keys_define:nn { guide }
 {
  author .code:n =
   {
    \int_gincr:N \g_guide_authors_int
    \keys_set:nn { guide / author } { #1 }
   },
  author/firstname .code:n = { \guide_add_author:nn { firstname } { #1 } },
  author/lastname .code:n = { \guide_add_author:nn { lastname } { #1 } },
  author/affiliation .code:n = { \guide_add_author:nn { affiliation } { #1 } },
  author/email .code:n = { \guide_add_author:nn { email } { #1 } },
  leftlogo .tl_set:N = \l_guide_leftlogo_tl,
  rightlogo .tl_set:N = \l_guide_rightlogo_tl,
  number .tl_set:N = \l_guide_number_tl,
  title .tl_set:N = \l_guide_title_tl,
  subtitle .tl_set:N = \l_guide_subtitle_tl,
  keywords .tl_set:N = \l_guide_keywords_tl,
  url .tl_set:N = \l_guide_url_tl,
  title .initial:n = {TITLE~NEEDED},
}
\NewDocumentCommand \guidehdr { +O{} }
 {
  \group_begin:
  \keys_set:nn { guide } { #1 }
  \guide_hdr:
  \group_end:
 }
\cs_new_protected:Nn \guide_hdr:
 {
   \centering
   \noindent\fcolorbox{kublue81186}{white}{
  \begin{tabular}
   {
    @{}c
    >{\centering}m{ \dim_eval:n { \textwidth - 2.5in - 6\tabcolsep - 2\arrayrulewidth } }
    c@{}
   }
   &&\tabularnewline[-\dim_eval:n { -\normalbaselineskip + 3ex }]
  \begin{tabular}{@{}>{\centering}p{1.25in}@{}}
  \includegraphics[width=.75in]{\l_guide_leftlogo_tl}
  \end{tabular}
  &
  \centering
    {\textsc{\large{}\l_guide_title_tl}}
    \tl_if_empty:NF \l_guide_subtitle_tl
    {
      {\\\textsc{\large{}\l_guide_subtitle_tl}}
    }
  \\
  \textcolor{kublue81186}{\rule[0.5ex]{1\linewidth}{0.7pt}}\\
  \guide_print_authors:
  &
  \begin{tabular}{@{}>{\centering}p{1.25in}@{}}
  \includegraphics[width=1.25in]{\l_guide_rightlogo_tl}
  \end{tabular}
  \tabularnewline
  \tl_if_empty:NTF \l_guide_number_tl
   { \leavevmode\vphantom{G}}
   {\small\bfseries { Guide~No: ~ \tl_use:N \l_guide_number_tl }} & 
   \tl_if_empty:NF \l_guide_keywords_tl
   {\centering
    \textbf{Keywords}: ~ \l_guide_keywords_tl}
   & 
   \small\bfseries\monddyyyy\today \tabularnewline
   \tl_if_empty:NF \l_guide_url_tl
   {
   \multicolumn{3}{c}{
    See ~ \url{\l_guide_url_tl} ~ for ~ updates.}
     }
  \tabularnewline
  \end{tabular}
  }
}

\cs_new_protected:Nn \guide_add_author:nn
 {
  \prop_gput:Nxn \g_guide_authors_prop
   { #1 \int_to_arabic:n { \g_guide_authors_int } }
   { #2 }
 }
\cs_generate_variant:Nn \prop_gput:Nnn { Nx }
\cs_new_protected:Nn \guide_print_authors:
 {
  \tl_clear:N \l__guide_authors_tl
  \int_step_inline:nnnn { 1 } { 1 } { \g_guide_authors_int }
   {
    \tl_put_right:Nn \l__guide_authors_tl
     {
      \textbf
       {
        \prop_item:Nn \g_guide_authors_prop { firstname ##1 }~
        \prop_item:Nn \g_guide_authors_prop { lastname ##1 },~
       }
       \prop_item:Nn \g_guide_authors_prop { affiliation##1 },~
       \href{mailto:\prop_item:Nn \g_guide_authors_prop { email##1 } }
       {\url{\prop_item:Nn \g_guide_authors_prop { email##1 }}}
       \\
     }
   }
   \tl_use:N \l__guide_authors_tl
 }
\ExplSyntaxOff
%%END This ends the custom header


%%Now the hyperref colors

\hypersetup{
 unicode=true,pdfusetitle,bookmarks=true,
 bookmarksnumbered=false, bookmarksopen=false,
 breaklinks=true, pdfborder={0 0 0},
 pdfborderstyle={}, backref=false,
 colorlinks=true, linkcolor=darkblue,
 urlcolor=darkblue, citecolor=darkblue}

%% Was previously in "preambleFloatCenter.tex"
%%BEGIN: graphics and such inside floats are centered by default
\renewenvironment{figure}[1][]{%
 \ifthenelse{\equal{#1}{}}{%
   \@float{figure}
 }{%
   \@float{figure}[#1]%
 }%
 \centering
}{%
 \end@float
}
\renewenvironment{table}[1][]{%
 \ifthenelse{\equal{#1}{}}{%
   \@float{table}
 }{%
   \@float{table}[#1]%
 }%

 \centering
}{%
 \end@float
}
%%END: graphics and such inside floats are centered by default
%% END guidePreambleHeader

%%\input{theme/preambleFooter.tex}
%% BEGIN: Footer
\usepackage{fancyhdr}
\usepackage{lastpage}
\usepackage{xparse}

\ExplSyntaxOn
\NewDocumentCommand \footersetup { +m }
{
\keys_set:nn { footer } { #1 }
}
\keys_define:nn { footer }
{	
    leftside .code:n = {
\keys_set:nn { footer / leftside } { #1 }
},
rightside .code:n = {
\keys_set:nn { footer / rightside } { #1 }
},
    leftside / lone .tl_set:N = \l_footer_leftside_lone_tl,
    leftside / ltwo .tl_set:N = \l_footer_leftside_ltwo_tl,
    leftside / lthree .tl_set:N = \l_footer_leftside_lthree_tl,
    rightside / rone .tl_set:N = \l_footer_rightside_rone_tl,
    rightside / rtwo .tl_set:N = \l_footer_rightside_rtwo_tl,
    rightside / rthree .tl_set:N = \l_footer_rightside_rthree_tl,
    leftside .initial:n = {lone = , ltwo = ~, lthree = ~ },
    rightside .initial:n = {rone = ~, rtwo = ~,  rthree = ~},
    leftside .default:n = {lone = , ltwo = ~, lthree = ~ },
    rightside .default:n = {rone = ~, rtwo = ~,  rthree = ~},
}
\NewDocumentCommand \guideftr { +o }
{
  \IfValueT { #1 } 
  {
    \keys_set:nn { footer } { #1 }
  }
  \guide_ftr:
}
\cs_new_protected:Nn \guide_ftr:
{
  % Now fancy header/footer settings
  \lhead{}%
  \chead{}%
  \rhead{}%
  \renewcommand{\headrulewidth}{0pt}%
  \lfoot{}%
  \cfoot{{Page} ~ \thepage {} ~ { of } ~ {} \pageref{LastPage}}%
  \rfoot{}%

  \fancypagestyle{firststyle}%
  {%
    \lhead{}%
    \chead{}%
    \rhead{}%
    \lfoot{}%
    \cfoot{}%
    \rfoot{}%
    \renewcommand{\headrulewidth}{0pt}%
    \fancyfoot[L]{\l_footer_leftside_lone_tl {}\\%
      \l_footer_leftside_ltwo_tl {}\\%
      \l_footer_leftside_lthree_tl {}}%
    \fancyfoot[R]{\l_footer_rightside_rone_tl {}\\%
      \l_footer_rightside_rtwo_tl {}\\%
      \l_footer_rightside_rthree_tl {}}%
      \renewcommand{\footrulewidth}{.2pt}%
  }%
  \thispagestyle{firststyle}
}
\ExplSyntaxOff

\pagestyle{fancy}
%% End of fancy footer setup

\input{theme/addressFooter.tex}
%% end preambleFooter

%%\input{theme/guidePreambleSweavel.tex}
%%% From beamer slide:
\usepackage{Sweave}
%% 
%% This controls display of code chunks
\usepackage{ae,fancyvrb,relsize,listings}

\providecommand{\Sweavesize}{\normalsize}
\providecommand{\Rsize}{}
\renewcommand{\Rsize}{\normalsize}
\providecommand{\Routsize}{\scriptsize}

\providecommand{\Rcolor}{\color[rgb]{0.1, 0.1, 0.1}}
\providecommand{\Routcolor}{\color[rgb]{0.2, 0.2, 0.2}}
\providecommand{\Rcommentcolor}{\color[rgb]{0.101, 0.43, 0.432}}

\providecommand{\Rbackground}{\color[gray]{0.91}}
\providecommand{\Routbackground}{\color[gray]{0.935}}
% Can specify \color[gray]{1} for white background or just \color{white}

\lstdefinestyle{Rinput}{
  language=R,
  escapechar=`,
  fancyvrb=false,%
  tabsize=2,%
  breaklines=true,
  breakatwhitespace=true,%
  captionpos=b,%
  frame=single,%
  framerule=0.2pt,%
  framesep=1pt,%
  showstringspaces=false,%
  basicstyle=\Rsize\Rcolor\ttfamily,%
  columns=fixed%,
  \lst@ifdisplaystyle\scriptsize\fi,%,
  commentstyle=\Rcommentcolor\ttfamily,%
  identifierstyle=,%
  keywords=\bfseries,%
  keywordstyle=\color[rgb]{0, 0.5, 0.5},
  escapeinside={(*}{*)},
  literate={~}{{$\sim$}}1{==}{{=\,=}}2{--}{{-\,-}}2,
  alsoother={$},
  alsoletter={.<-},%
  otherkeywords={!,!=,~,$$,*,\&,\%/\%,\%*\%,\%\%,<-,<<-,/},%
  backgroundcolor=\Rbackground,%
  numbers=left,%
  %numberblanklines=false,%
  stepnumber=5,
  firstnumber=1,
  numberstyle={\tiny}
}%

% Other options of interest:
% frame=single,framerule=0.1pt,framesep=1pt,rulecolor=\color{blue},
% numbers=left,numberstyle=\tiny,stepnumber=1,numbersep=7pt,
% keywordstyle={\bf\Rcolor}

\lstdefinestyle{Routput}{fancyvrb=false,
  literate={~}{{$\sim$}}1{R^2}{{$R^{2}$}}2{^}{{$^{\scriptstyle\wedge}$}}1{R-squared}{{$R^{2}$}}2,%
  basicstyle=\Routcolor\Routsize\ttfamily,%
  backgroundcolor=\Routbackground,
  language=R,
  escapechar=`,
  fancyvrb=false,%
  tabsize=2,%
  breaklines=true,
  breakatwhitespace=true,%
  captionpos=b,%
  frame=single,%
  framerule=0.2pt,%
  framesep=1pt,%
  showstringspaces=false,%
  columns=fixed%,
  \lst@ifdisplaystyle\scriptsize\fi,%
  identifierstyle=,%
  keywords=\bfseries,%
  keywordstyle=\color[rgb]{0, 0.5, 0.5},
  escapeinside={(*}{*)},
  literate={~}{{$\sim$}}1 {==}{{=\,=}}2,
  alsoother={$},
  alsoletter={.<-},%
  otherkeywords={!,!=,~,$,*,\&,\%/\%,\%*\%,\%\%,<-,<<-,/},
  numbers=left,
  %numberblanklines=false,%
  stepnumber=5,
  firstnumber=1,
  numberstyle={\tiny}
}

\renewenvironment{Schunk}{}{}
\renewenvironment{Sinput}{}{}
\let\Sinput\relax
\let\Scode\relax
\let\Soutput\relax
\lstnewenvironment{Sinput}{\lstset{style=Rinput}}{}
\lstnewenvironment{Scode}{\lstset{style=Rinput}}{}
\lstnewenvironment{Soutput}{\lstset{style=Routput}}{}
%%end paste in from guidePreambleSweavel.tex


\lstset{tabsize=2, breaklines=true, style=Rinput, breakatwhitespace=true}

\fvset{listparameters={\setlength{\topsep}{0em}}}

\usepackage{xcolor}
\definecolor{light-gray}{gray}{0.90}
\usepackage{realboxes}
\providecommand*{\code}[1]{\texttt{#1}}
\renewcommand{\code}[1]{%
\Colorbox{light-gray}{#1}%
}%

\makeatother

\begin{document}
%% Fill in values of the arguments here, 
%% If blanks are needed, must insert value " ~ "
%% If comma needed inside value, wrap in {}.
%% Delete secondauthor and thirdauthor if not needed
\guidesetup{%
  author={
    lastname=Johnson, 
    firstname=Paul, 
    affiliation=CRMDA,
    email=pauljohn@ku.edu},
  url={https://pj.freefaculty.org/guides},
  keywords={R},
  title={Descriptive data tables},
  leftlogo={theme/logoleft.pdf},
  rightlogo={theme/logo-vert.pdf},
  number=0,
}
\guidehdr

% tmpout directory must exist first

% In document Latex options:
\fvset{listparameters={\setlength{\topsep}{0em}}}




%The listings class used here allows within-document style
%changes. R input boxes are governed by 
% "Rsize", "Rbackground" and "Rcolor", while R output boxes depend on
% "Routsize", "Routbackground", and "Routcolor". Colors
% can be specified in many ways, as shown here
%\def\Rsize{\huge\ttfamily}
%\def\Routsize{\huge}
%\def\Rbackground{\color[gray]{0.90}}
%\def\Routbackground{\color[gray]{0.40}}
%\def\Rcolor{\color[gray]{0.60}
%\def\Routcolor{\color[rgb]{0.9, 0.1, 0.1}]}
%\def\Rcommentcolor{\color{green}}
\begin{abstract}
\noindent This is an abstract. Please include a terse, yet descriptive
statement here of less than 200 words. It should avoid colloquialisms
and polysyllabic profundities.
\end{abstract}
\tableofcontents{}

\section{Reviewers want descriptive variable tables}

People who read our models usually want to know about the variables
that are included in the model. In the very simplest case, it might
require only the mean and standard deviation of each predictor. That
kind of table is very easy to produce, as we will see.

A more complicated scenario arises when there are different sorts
of variables to be summarized and the researcher wants to knit them
together.

\subsection{The Social Welfare Project}

We had a funded project that replicated a previous research project.

In Table \ref{tab:nam2008}, please see an example from an article
by Nam (2008) that we were required to replicate. From updated data,
we generated Table \ref{tab:ournamtable}. Our output table is reasonable.
I checked the R code that was used to generate it and it is very complicated;
it is not the sort of simple, elegant code I'd want to teach you to
use today. That example is complicated because it is flexible to deal
with different yearly data sets and there are several variable types.

\begin{table}[H]

\caption{Summary table on social welfare project\label{tab:nam2008}}

\includegraphics{importfigs/descriptive-Nam2008}

\end{table}

\begin{table}[H]
\caption{Updated social welfare data (project 663)\label{tab:ournamtable}}

% latex table generated in R 3.3.2 by xtable 1.8-2 package
% Tue Mar  7 09:48:17 2017
\begin{tabular}{lrrr}
  \toprule
  & Full Sample & Target Group & Comparison Group \\ 
  \midrule
On welfare in 2002 & 0.03 & 0.19 & 0.00 \\ 
  On welfare between 2002 and 2012 & 0.03 & 0.12 & 0.00 \\ 
  Age & 34.03 & 31.18 & 34.61 \\ 
  African American & 0.40 & 0.83 & 0.31 \\ 
  Head's education in 2003 &  &  &  \\ 
  --Less than high school & 0.14 & 0.27 & 0.11 \\ 
  --High school degree & 0.44 & 0.40 & 0.45 \\ 
  --Some College & 0.42 & 0.33 & 0.44 \\ 
  Household size in 2003 & 3.20 & 3.44 & 3.15 \\ 
  Change in household size (2003-2013) & -0.02 & -0.02 & -0.01 \\ 
  Number of children & 1.35 & 2.19 & 1.18 \\ 
  Average family income (2002-2012) &  &  &  \\ 
  --Mean & \$54,607.45 & \$22,265.05 & \$61,133.81 \\ 
  --Median & \$44,449.76 & \$18,266.60 & \$52,327.77 \\ 
  Change in family income (2002-2012) &  &  &  \\ 
  --Mean & \$3,819.98 & \$2,925.92 & \$4,000.85 \\ 
  --Median & \$3,160.13 & \$4,474.76 & \$4,232.61 \\ 
  Change in state unemployment rate (2003-2013) & 1.42 & 1.54 & 1.40 \\ 
  Per capita GSP in 2003 & \$43,951.73 & \$42,827.91 & \$44,179.07 \\ 
  Financial assets in 2003: &  &  &  \\ 
  --Mean & \$4,926.50 & \$879.27 & \$5,766.16 \\ 
  --Median & \$500.00 & \$0.00 & \$1,000.00 \\ 
  Change in financial assets (2003-2013) &  &  &  \\ 
  --Mean & \$1,630.13 & \$-400.15 & \$2,029.70 \\ 
  --Median & \$-26.08 & \$0.00 & \$-52.15 \\ 
  Possessed savings in 2003 & 0.66 & 0.34 & 0.72 \\ 
  Possessed savings in 2013 & 0.62 & 0.28 & 0.69 \\ 
  Owned a vehicle in 2003 & 0.87 & 0.65 & 0.91 \\ 
  Owned a vehicle in 2013 & 0.86 & 0.73 & 0.89 \\ 
  N & 1153 & 194 & 959 \\ 
   \bottomrule
\end{tabular}

\end{table}

As I look at these examples, I realize there are a couple of very
important details.

\begin{enumerate}
\item For which rows of data should the summary be calculated?
  
  As you may know, many stats programs use ``listwise deletion'' of cases
  with missing values. That means the rows of data that are used may 
  differ between models. 
  
  Probably, we want the summary to reflect the data that is actually
  used in a model. Possibly/Probably we also might want a summary of
  the whole data set.
  
\item Which summary indicators are required?
  
  The social welfare project is tricky because they wanted different
  summary items for different kinds of predictors
  
\item What are we supposed to do about categorical variables?
\end{enumerate}    

\section{The Workflow Problem}

Clearly, we want the software to write a table on disk in a format
that can be inserted into a document without trouble. 

There are two scenarios.

\begin{enumerate}
\item Three step process
  \begin{itemize} 
  \item Run an R file that writes a file on disk (write into
    output folder)
  \item Copy that file into the writeup folder, \emph{possibly
      with some hand editing}.
  \item Create the writeup document that imports that table.
  \end{itemize}
\item One step process: reproducible research document  
  \begin{itemize} 
    \item The code in the document produces tables and figures that
     \emph{do NOT need to hand editing}. They are ready as is!
  \item The documents in the stationery package demonstrate this idea.
  \item See stationery vignette ``Code Chunks'' because it shows how
    \emph{as is} material can be used in different kinds of document
  \end{itemize}
\end{enumerate}

\subsection{Formatting for tables}

The desired output format depends on the output document format.
If we are creating a PDF document, best to have:

LaTeX formatting

If we are creating a Web Page, then best to have:

HTML formatting

If we are completely lost, we may have to take the worst possible
avenue: 

raw ``CSV'' formatting

This last is the worst option because the output is not structured
and cannot be used in a document without a lot of hand editing.
  
This document is a LaTeX/NoWeb document that has the PDF back end, so
I'm saving table files as LaTeX tabular objects.

Authors have some control over the features and formatting that will
be inserted into the table. In my documents, I prefer to create the
floating table and figure objects and then insert the results in
them. Hence, I do not want my LaTeX table output file to include
caption or label. That is a matter of taste, sometimes I will write
the table (or figure) captions into the files.

The LaTeX table writing function I prefer is a traditional one, 
\code{xtable}. It can create either LaTeX or HTML output. It has worked well
for many years, I (honestly) don't understand why so many R package
writers want to re-invent this ability. 

If I'm writing a regression table, I generally use my function,
\code{outreg} in the \code{rockchalk} package. However, that works for
only the standard regression models provided with R. Some
user-contributed regression packages do not create the required
structures from which \code{outreg} will work. 

\section{Implementation: Numeric Predictors}

Lets test with R \citep{RCore}. There are many
data summary tools in R. I'm partial to the function ``summarize''
in the rockchalk package, but I expect you can find many packages
that get the same work done. 

In short, 3 steps are needed.
\begin{enumerate}
\item Tabulate data summaries. The will often be in R data.frames, 
  matrices, or similar. Sometimes there will be a list of summaries by
  subgroups.
\item Inspect the data in the R session, make sure it looks correct. 
\item Use one of the R functions that can write the file in the needed
  format.
\end{enumerate}


\subsection{First, check what the rockchalk::summarize function is doing}

Read ?summarize, it explains the output printed on screen is just
the pretty rendition, but there is actually a list provided as the
return. Within the list, the first 2 items are a numeric summary and
a summary of the categorical variables.

\subsection{Mean and Standard Deviations}

We hope guide authors will choose carefully thought out titles for
sections and that material will be grouped meaningfully into sections. 



\input{tmpout/t-hsbget}

The output from summarize in R is wide, does not fit into this
document unless I make the font small.
\input{tmpout/t-hsbsumry10}

The output object \code{sum.hsb} has 3 parts, the third one is that
thing you are looking at. It is the formatted, ``textified'' 
version.  The first 2 parts are the ``numeric'' and 
``categorical'' predictors, which are not rounded or summarized. 

Take a look. My numeric summary has the variables on the rows,
not on the columns as shown in the printed output within the R session:

\begin{Schunk}
\begin{Soutput}
          [,1]       [,2]       [,3]       [,4]
[1,] 21.982553  6.7582014 -3.6663810  7.1137074
[2,]  6.758201 41.7811616 -0.6594455 -2.7048963
[3,] -3.666381 -0.6594455 28.0277308  0.2415728
[4,]  7.113707 -2.7048963  0.2415728 42.3190939
\end{Soutput}
\end{Schunk}


The easy thing is to summarize the numeric variables. 
In the output folder, it will write ``hsbsumry20.tex''.

\input{tmpout/t-hsbsumry20}

%%\def\Routsize{\tiny}
\input{tmpout/t-hsbsumry21}

I'm going to create a ``floating'' table, Table \ref{tab:HSBsumry20}.

\begin{table}[H]

\caption{HSB summary information \label{tab:HSBsumry20}}

% latex table generated in R 3.5.1 by xtable 1.8-2 package
% Tue Sep 11 14:12:40 2018
\begin{tabular}{rrr}
  \hline
 & mean & sd \\ 
  \hline
ses & 0.00 & 0.78 \\ 
  mathach & 12.75 & 6.88 \\ 
  size & 1056.86 & 604.17 \\ 
  pracad & 0.53 & 0.25 \\ 
  disclim & -0.13 & 0.94 \\ 
  himinty & 0.28 & 0.45 \\ 
  schoolid & 5277.90 & 2499.58 \\ 
  mean & 12.75 & 3.01 \\ 
  sd & 6.20 & 0.86 \\ 
  sdalt & 6.26 & 0.00 \\ 
  junk & 47.32 & 48.90 \\ 
  sdalt2 & 48.39 & 0.00 \\ 
  num & 48.02 & 10.82 \\ 
  se & 0.92 & 0.20 \\ 
  sealt & 0.92 & 0.13 \\ 
  sealt2 & 1.03 & 0.14 \\ 
  t2 & 14.66 & 26.42 \\ 
  t2alt & 8.54 & 11.06 \\ 
  pickone & 0.02 & 0.15 \\ 
  mmses & 0.00 & 0.41 \\ 
  mnses & 0.00 & 0.41 \\ 
  xb & 12.69 & 2.42 \\ 
  resid & 0.06 & 6.46 \\ 
   \hline
\end{tabular}


\end{table}


\section{Categorical Variables: more difficult}

\subsection{Likert variables: I have that worked out!}

In the \code{kutils} package, I wrote a function called
likert that can line up responses to many Likert scale questions
and then assemble them into a table.  The output here is
actually showing 2 tables, because there are too many
columns to fit in 1 table. 

\begin{table}[H]

\caption{Likert variable summary from TFA project}

\includegraphics{importfigs/descriptive-likert}

\end{table}

\subsection{More generally, more difficult}

The output from rockchalk::summarize has a separate piece for
the categorical variables. It is a list, with one table for each
variable.

\begin{Schunk}
\begin{Sinput}
 curve(log(x), 0.1, 30, ylab = "", lty = 1, col = "black", ylim = c(-3, 7))
 curve(sqrt(x), 0.1, 30, lty = 2, col = "green", add = TRUE)
 legend("topleft", legend = c("log(x)", "sqrt(x)"), lty = c(1,2), col = c("black", "green"))
\end{Sinput}
\end{Schunk}


I notice that the output from this command is not handled well
by the LaTeX to PDF transition, so I won't run it.

\begin{Schunk}
\begin{Sinput}
 #  class(myObject)
\end{Sinput}
\end{Schunk}


If we want to summarize the different tables, we have to make
some document preparation decisions. 

\begin{itemize}
  \item Which summary information do we require?
  
  \item Are we trying to squash all of the tables into 
    one big summary table?
\end{itemize}

The way I created those summary tables causes complications
throughout. I need to think about how to fix.

\subsection{Pretty easy to make separate summaries}

Here, we will be like people who lived in caves and write summary 
code for each of the first 3 table objects. I am NOT writing those
summaries in separate files. I am using the LaTeX results ``as is'',
they are printing directly into the document.

\begin{Schunk}
\begin{Sinput}
 confint(mod1)
\end{Sinput}
\begin{Soutput}
                   2.5 %     97.5 %
(Intercept) 27.165443886 40.2839844
GRIP         0.007898305  0.1242813
\end{Soutput}
\end{Schunk}


\begin{Schunk}
\begin{Sinput}
 xvals <- seq(0,20, length.out=200)
 gam <- dgamma(xvals, shape=2, scale=1)
 igaus <- dinvgauss(xvals,mu=2,lambda=5)
 plot(xvals,gam,type="l",lty=1,main="")
 lines(xvals,igaus,lty=2)
 legend(6,.2,c("gamma(sh=2,sc=1)","inv gauss(mu=2,lambda=5)"),lty=1:2,)
 
\end{Sinput}
\end{Schunk}


\begin{Schunk}
\begin{Sinput}
 par(mfcol=c(3,1))
 for (i in 1:3){
      minx <- 20 + 50 * (i-1)
      xvals <- seq(minx,300, length.out=1000)
      gam <- dgamma(xvals, shape = 2, scale = 1)
      igaus <- dinvgauss(xvals, mu = 2, lambda = 5)
      plot(xvals,gam, type="l", lty = 1, main="")
      lines(xvals, igaus, lty = 2)
      legend(150, 0.7*max(gam), c("gamma(sh=2,sc=1)", "inv gauss(mu=2,lambda=5)"), lty = 1:2,)
  }
\end{Sinput}
\end{Schunk}


\section{Get out of Jail Free Card? Regression Model Matrix?}

In R, a regression model will, by default, have the abilty to give
back a model design matrix. The function for this is
\code{model.matrix}.

Suppose we fit a multiple regression, and then use model.matrix
to recover the data and the recoded variables that were actually used.

I'm not claiming this is a reasonable regression, but it will run and
demonstrate the purpose.

\begin{Schunk}
\begin{Sinput}
 xpoly <- poly(dat$x, 2)
 head(xpoly)
\end{Sinput}
\begin{Soutput}
               1         2
[1,] -0.09043700 0.1234184
[2,] -0.08802246 0.1136961
[3,] -0.08764475 0.1121995
[4,] -0.08763258 0.1121514
[5,] -0.08749189 0.1115957
[6,] -0.08724543 0.1106244
\end{Soutput}
\begin{Sinput}
 (pcorr <- cor(xpoly[, 1], xpoly[ ,2]))
\end{Sinput}
\begin{Soutput}
[1] -1.599071e-16
\end{Soutput}
\end{Schunk}


The raw output of the summary.lm method function is like this:
\begin{Schunk}
\begin{Sinput}
 head(model.matrix(mod1))
\end{Sinput}
\begin{Soutput}
  (Intercept) schtypeprivate
1           1              1
3           1              0
5           1              0
6           1              1
7           1              0
8           1              0
\end{Soutput}
\end{Schunk}


In Table \ref{tab:outreg10}, I have a nicer-looking summary.

\begin{table}[H]
 
\caption{Outreg from rockchalk package \label{tab:outreg10}}

\begin{Schunk}
\begin{Sinput}
 newdf <- data.frame(income = mean(Chile$income, na.rm=TRUE), age=median(Chile$age, na.rm = TRUE))
\end{Sinput}
\end{Schunk}


\end{table}
\begin{Schunk}
\begin{Sinput}
 par()
\end{Sinput}
\end{Schunk}


Note that all of the columns in the \code{model.matrix} output are
numeric.  Even the categorical predictors are reduced to 1's and 0's.

\begin{Schunk}
\begin{Sinput}
 plot( prestige ~ education, data=Prestige, col=Prestige$type)
 m7 <- lm(prestige ~ type / education, data=Prestige)
 nd2 <- expand.grid(education=range(Prestige$education), type = levels(Prestige$type))
 nd2$pred <- predict(m7, newdata=nd2)
 for(i in 1:3){
  with(nd2[nd2$type %in% levels(nd2$type)[i], ], 
     lines(education, pred, col=i, lty=i, lwd=2))
  }
 legend("bottomright", legend=levels(Prestige$type), lty=1:3, col=1:3, lwd=2)
 outreg(m7, tight=F)
\end{Sinput}
\end{Schunk}


Why not grab the numerical summaries of all of those predictors
as seen from the \code{model.matrix} point of view?

\begin{Schunk}
\begin{Sinput}
 attributes(s3d)
\end{Sinput}
\begin{Soutput}
$names
[1] "xyz.convert" "points3d"    "plane3d"     "box3d"      
\end{Soutput}
\end{Schunk}


Note there are no factors there. Lets see if output in Table \ref{tab:sumry2}
is close to good.

\begin{Schunk}
\begin{Sinput}
 plot(salary ~ age, data = mydf)
 plot(salary ~ age, data = mydf, cex = 2)
 plot(salary ~ age, data = mydf, col = "pink", cex = 2, pch = 16)
 plot(salary ~ age, data = mydf, col = as.numeric(sex), cex = salary, pch = 16)
 plot(salary ~ age, data = mydf, type = "l")
 plot(salary ~ age, data = mydf, type = "b")
 plot(salary ~ age, data = mydf, type = "h")
 plot(salary ~ age, data = mydf, type = "s")
 plot(salary ~ age, data = mydf, type = "c")
 text(salary ~ age, data = mydf, labels = 1:8)
\end{Sinput}
\end{Schunk}


If that's close to good, I suggest we write it in a file, and
then hand edit the variable names, then include it in a document:

\begin{Schunk}
\begin{Sinput}
 dat <- genCorrelatedData(1000, stde=5)
 dat$x3 <- rnorm(1000)
 dat$x4 <- rnorm(1000)
 dat$x5 <- factor(c( rep("A", 700), rep("B", 300)))
 m2 <- lm( y ~ x1 + x2 + x3 + x4 + x5, data=dat)
 m2.data <- model.data(m2)
 newdata1 <- newdata(m2, predVals = list(x1 = "quantile"))
 newdata1$fit <- predict(m2, newdata=newdata1)
 print(newdata1, digits = 3)
\end{Sinput}
\begin{Soutput}
    x1   x2     x3       x4 x5  fit
1 43.3 50.5 0.0162 -0.00544  A 18.6
2 50.2 50.5 0.0162 -0.00544  A 20.0
3 57.2 50.5 0.0162 -0.00544  A 21.4
\end{Soutput}
\end{Schunk}


Manually copy that into some other folder where we can cultivate
and perfect it.  I often use a folder named ``importfigs'' for that
kind of thing.

The output in Table \ref{tab:sumry3} is almost good enough. Each of the
``dummy variables'' in the regression has a mean equal the proportion
of 1's in that predictor column. If we had a multi-category predictor,
then we'd see more than one row for each variable.  Still, there's a
problem that it ``leaves out'' one category because it is ``in the
intercept.'' We can think on that.  

Actually, I thought about that, here's my fix in Table \ref{tab:sumry.noint}.
If that has all of the right numbers, then we should write it in a file,
fix the labels, and be done with it.

\begin{Schunk}
\begin{Sinput}
 fit <- list(call = call, call1 = object$call, call2 = object$call1, nmis = object$nmis, m = m, qhat = qhat, u = u, qbar = qbar, ubar = ubar, b = b, t = t, r = r, dfcom = dfcom, df = df,         fmi = fmi, lambda = lambda)
\end{Sinput}
\end{Schunk}


\begin{center}
 \begin{tabular}{*{2}{l}}
 \hline
               & Estimate  \\
              & (S.E.)  \\
\hline 
 \hline
  (Intercept)   &    40.591* \\
    &   ( 1.24) \\
  instcst   &    0.002* \\
    &   ( 0) \\
 \hline 
N   &    1155 \\
 $RMSE$       & 16.41  \\
 $R^2$       & 0.203  \\
 adj $R^2$       & 0.202  \\
 \hline\hline
* $p \le 0.05$\end{tabular}
\end{center}


\subsection{Create 1 function to do approximately the correct thing}

\begin{Schunk}
\begin{Sinput}
 plot(y ~ x, data = dat)
\end{Sinput}
\end{Schunk}


\input{tmpout/t-makedesc50}

\input{tmpout/t-makedesc60}

\input{tmpout/t-makedesc100}


\input{tmpout/t-makedesc160}

Check contents of m4.desc2

\input{tmpout/t-makedesc180}

\input{tmpout/t-makedesc200}


Many problems still remain and while fixing them I have to revise the
rockchalk::summarize function itself. Thus, when that is finished,
I'll have to come back here and make descriptiveTable compatible
again. 

So check back soon, as I will fix this up to be Great Again, just like America.



\bibliographystyle{apalike2}
\phantomsection\addcontentsline{toc}{section}{\refname}\bibliography{theme/R}

\section*{Replication Information\label{sec:Session-Info}}

Please leave this next code chunk if you are producing a guide document. 

\begin{Schunk}
\begin{Soutput}
R version 3.5.1 (2018-07-02)
Platform: x86_64-pc-linux-gnu (64-bit)
Running under: Ubuntu 18.04.1 LTS

Matrix products: default
BLAS: /usr/lib/x86_64-linux-gnu/blas/libblas.so.3.7.1
LAPACK: /usr/lib/x86_64-linux-gnu/lapack/liblapack.so.3.7.1

locale:
 [1] LC_CTYPE=en_US.UTF-8       LC_NUMERIC=C               LC_TIME=en_US.UTF-8       
 [4] LC_COLLATE=en_US.UTF-8     LC_MONETARY=en_US.UTF-8    LC_MESSAGES=en_US.UTF-8   
 [7] LC_PAPER=en_US.UTF-8       LC_NAME=C                  LC_ADDRESS=C              
[10] LC_TELEPHONE=C             LC_MEASUREMENT=en_US.UTF-8 LC_IDENTIFICATION=C       

attached base packages:
[1] stats     graphics  grDevices utils     datasets  methods   base     

other attached packages:
[1] rockchalk_1.8.118 memisc_0.99.14.11 MASS_7.3-50       lattice_0.20-35   foreign_0.8-70   

loaded via a namespace (and not attached):
 [1] Rcpp_0.12.17      magrittr_1.5      splines_3.5.1     rlang_0.2.1       minqa_1.2.4      
 [6] carData_3.0-1     car_3.0-0         tools_3.5.1       grid_3.5.1        nlme_3.1-137     
[11] data.table_1.11.4 rio_0.5.10        htmltools_0.3.6   lme4_1.1-17       abind_1.4-5      
[16] readxl_1.1.0      digest_0.6.15     tibble_1.4.2      Matrix_1.2-14     zip_1.0.0        
[21] nloptr_1.0.4      repr_0.15.0       base64enc_0.1-3   curl_3.2          haven_1.1.1      
[26] openxlsx_4.1.0    compiler_3.5.1    pillar_1.2.3      cellranger_1.1.0  forcats_0.3.0    
\end{Soutput}
\end{Schunk}


\end{document}
