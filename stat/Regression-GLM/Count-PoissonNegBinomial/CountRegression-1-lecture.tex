\batchmode
\makeatletter
\def\input@path{{/home/pauljohn/SVN/SVN-guides/stat/Regression-GLM/Count-PoissonNegBinomial//}}
\makeatother
\documentclass[10pt,english]{beamer}
\usepackage{lmodern}
\renewcommand{\sfdefault}{lmss}
\renewcommand{\ttdefault}{lmtt}
\usepackage[T1]{fontenc}
\usepackage[latin9]{inputenc}
\setcounter{secnumdepth}{3}
\setcounter{tocdepth}{3}
\synctex=-1
\usepackage{graphicx}
\usepackage{esint}

\makeatletter
%%%%%%%%%%%%%%%%%%%%%%%%%%%%%% Textclass specific LaTeX commands.
 % this default might be overridden by plain title style
 \newcommand\makebeamertitle{\frame{\maketitle}}%
 \AtBeginDocument{
   \let\origtableofcontents=\tableofcontents
   \def\tableofcontents{\@ifnextchar[{\origtableofcontents}{\gobbletableofcontents}}
   \def\gobbletableofcontents#1{\origtableofcontents}
 }
 \long\def\lyxframe#1{\@lyxframe#1\@lyxframestop}%
 \def\@lyxframe{\@ifnextchar<{\@@lyxframe}{\@@lyxframe<*>}}%
 \def\@@lyxframe<#1>{\@ifnextchar[{\@@@lyxframe<#1>}{\@@@lyxframe<#1>[]}}
 \def\@@@lyxframe<#1>[{\@ifnextchar<{\@@@@@lyxframe<#1>[}{\@@@@lyxframe<#1>[<*>][}}
 \def\@@@@@lyxframe<#1>[#2]{\@ifnextchar[{\@@@@lyxframe<#1>[#2]}{\@@@@lyxframe<#1>[#2][]}}
 \long\def\@@@@lyxframe<#1>[#2][#3]#4\@lyxframestop#5\lyxframeend{%
   \frame<#1>[#2][#3]{\frametitle{#4}#5}}
 \def\lyxframeend{} % In case there is a superfluous frame end
 \newenvironment{topcolumns}{\begin{columns}[t]}{\end{columns}}

%%%%%%%%%%%%%%%%%%%%%%%%%%%%%% User specified LaTeX commands.
\usepackage{dcolumn}
\usepackage{Sweavel}
\usepackage{booktabs}

% use 'handout' to produce handouts
%\documentclass[handout]{beamer}
\usepackage{wasysym}
\usepackage{pgfpages}
\newcommand{\vn}[1]{\mbox{{\it #1}}}\newcommand{\vb}{\vspace{\baselineskip}}\newcommand{\vh}{\vspace{.5\baselineskip}}\newcommand{\vf}{\vspace{\fill}}\newcommand{\splus}{\textsf{S-PLUS}}\newcommand{\R}{\textsf{R}}


\usepackage{graphicx}
\usepackage{listings}
\lstset{tabsize=2, breaklines=true,style=Rstyle}
\usetheme{Antibes}
% or ...

%\setbeamercovered{transparent}
% or whatever (possibly just delete it)

%\mode<presentation>
%{
 % \usetheme{KU}
 % \usecolortheme{dolphin} %dark blues
%}

% In document Latex options:
\fvset{listparameters={\setlength{\topsep}{0em}}}
\def\Sweavesize{\normalsize} 
\def\Rcolor{\color{black}} 
\def\Rbackground{\color[gray]{0.95}}



%%only for presentation
\mode<presentation>{
\setbeamertemplate{frametitle continuation}[from second]
\renewcommand\insertcontinuationtext{...}
}

%\usepackage{handoutWithNotes}
%\pgfpagesuselayout{3 on 1 with notes}[letterpaper, border shrink=5mm]

%%for presentations
\mode<presentation>{
\expandafter\def\expandafter\insertshorttitle\expandafter{%
 \insertshorttitle\hfill\insertframenumber\,/\,\inserttotalframenumber}}

\makeatother

\usepackage{babel}
\begin{document}

% In document Latex options:
\fvset{listparameters={\setlength{\topsep}{0em}}}

\def\Sweavesize{\scriptsize} 
\def\Rcolor{\color{black}} 
\def\Rbackground{\color[gray]{0.90}}

\begin{Schunk}
\begin{Sinput}
 ## Other settings I like
 options(device = pdf)
 options(useFancyQuotes = FALSE) 
 op <- par() 
 pjmar <- c(5.1, 5.1, 1.5, 2.1) 
 options(SweaveHooks=list(fig=function() par(mar=pjmar, ps=12)))
 pdf.options(onefile=F,family="Times",pointsize=12)
\end{Sinput}
\end{Schunk}



\title{Count Regression Introduction}


\author{Paul Johnson <pauljohn@ku.edu>}

\makebeamertitle


\AtBeginSection[]{

  \frame<beamer>{ 

    \frametitle{Outline}   

    \tableofcontents[currentsection,currentsubsection] 

  }

}


\lyxframeend{}\lyxframe{Welcome}
\begin{itemize}
\item This is not ready yet
\end{itemize}

\lyxframeend{}

\begin{frame}

\frametitle{Outline}

\tableofcontents{}

\end{frame}

 


\lyxframeend{}\section{Motivation}


\lyxframeend{}\lyxframe{Count means}
\begin{itemize}
\item integer valued, 0, 1, 2,...
\item must be positive (0 or greater).
\end{itemize}
Well, if the expected number of observed cases is large, and the distribution
of the count data is drawn from the Poisson distribution, then OLS
might be OK. The Poisson is not all that different from a Normal distribution.
So sometimes the Normal case can be thought of as an approximation.

But if the expected number of counts is small, the Poisson distribution
is not even a little bit like a normal distribution.


\lyxframeend{}\lyxframe{Alternatives might not be as good.}

There are alternatives
\begin{itemize}
\item tobit (OLS, but with a truncation at 0)
\item ordinal logit/probit
\end{itemize}

\lyxframeend{}\lyxframe{Generalized Linear Model Approach }
\begin{itemize}
\item Recall. We asserted the dependent variable $y_{i}$ as a sum of a
``predictable part'' and a ``random'' part: $y_{i}=b_{0}+b_{1}x_{i}+e_{i}$.
\item If $e_{i}$ is ``Normal with a mean of 0 and standard deviation of
$\sigma_{e}$'', 

\begin{itemize}
\item then $y_{i}$ is ``Normal with a mean of $b_{0}+b_{1}x_{i}$ and
a standard deviation of $\sigma_{e}$.'' 
\end{itemize}
\item So OLS with an assumed Normal error implies
\end{itemize}
\[
y_{i}\sim N(X_{i}b,\,\,\sigma_{e}^{2})
\]


The symbol ``$\sim$'' means ``is distributed as'' or ``is drawn
from''. $X_{i}b$ is shorthand matrix terminology for the ``linear
predictor'', $b_{o}+b_{1}x1_{i}+b_{2}x2_{i}$. 


\lyxframeend{}\lyxframe{What's the big point here?}
\begin{quote}%{}
We think of the predictor as determining a parameter in a distribution
from which observations are drawn. 
\end{quote}%{}

\lyxframeend{}\lyxframe{You can use any distribution you want}
\begin{itemize}
\item $y_{i}$ can be any distribution you want
\item Let the properties of that distribution depend on input variables
and parameters. 
\item For a ``count'' model, all you absolutely need is an integer-valued
distribution for which $y_{i}\geq0$. 2 possibilities:

\begin{itemize}
\item Poisson 
\item Negative Binomial
\end{itemize}
\end{itemize}

\lyxframeend{}\section{Poisson.}


\lyxframeend{}\section{Poisson: $\mbox{\ensuremath{\lambda}}$}
\begin{itemize}
\item The Poisson is a ``one parameter'' distribution. 
\item The parameter is usually called $\lambda$, and that parameter determines
the expected value and the variance.
\[
Pr(y_{i}|input_{i})=\frac{exp(-\lambda)\lambda^{y_{i}}}{y_{i}!}
\]

\end{itemize}

\lyxframeend{}\lyxframe{Relabel $\lambda$ as ``input'' For Interpretation}
\begin{itemize}
\item Instead of the greek letter $\lambda$, let's call it what we mean:
``input''.
\item 
\[
Pr(y_{i}|input_{i})=\frac{exp(-input_{i})input_{i}^{y_{i}}}{y_{i}!}
\]
For any $y_{i}$ you put in here, this tells you how likely you are
to count that many ``things'' if the input is ``input''. 
\item When I write ``input'', I mean the combined impact of parameters
and variables. 
\end{itemize}
Input is not necessarily simply $X_{i}b$. In fact, we usually have
to ``translate'' or ``curve'' the linear predictor so it fits
``within boundaries.'' 

So ``input'' is typically some function that depends on $X_{i}b$,
for generality, $g(X_{i}b)$.


\lyxframeend{}


\lyxframeend{}\subsection{Graph the Poisson Distribution to Get a ``Feeling''.}

\begin{frame}[containsverbatim]
\frametitle{Poisson Sample, small lambda}
\includegraphics[width=12cm]{plots/t-pois01}
\end{frame}

\begin{frame}[containsverbatim]
\frametitle{Poisson Sample, large lambda}
\includegraphics[width=12cm]{plots/t-pois01}
\end{frame}


\lyxframeend{}\lyxframe{Noteworthy }
\begin{enumerate}
\item The Expected Value of the Poisson is $\lambda="input"$.
\item The Variance of the Poisson is $\lambda="input"$.
\item The shape changes and gets ``more normal'' as ``input'' gets bigger.


Implication: If your count data has high values, then the OLS Normal
model may serve about as well as a Poisson model


However, there are 2 problems.
\begin{enumerate}
\item Nonlinearity.
\item Heteroskedasticity.
\end{enumerate}
\end{enumerate}

\lyxframeend{}\lyxframe{Nonlinear Transformation of $X_{i}b$ Required for Poisson}

\textbf{The $input_{i}$ must be positive!} We are considering a ``count
variable,'' something that is always POSITIVE. The expected value
of a Poisson variable has to be positive. Since the expected value
equals the value of ``$input_{i}$'', then $X_{i}b$ cannot serve
as $input_{i}$ because it may be negative. 

All kinds of transformations have been considered to make sure input
is positive. A common way is to say that the input should be exponentiated,
because exp(anything) is positive. 

\[
input_{i}=exp(X_{i}b)
\]


Now, that results in the stupid looking exp(exp) appearance of the
Poisson regression model:
\[
Pr(y|Xb)=\frac{exp(-exp(Xb))(exp(Xb))^{y}}{y!}
\]


or it looks slightly less ugly (not much) if we write:

\[
Pr(y|Xb)=\frac{exp(-e^{Xb})(e^{Xb})^{y}}{y!}
\]



\lyxframeend{}\lyxframe{King called it the ``Exponential Poisson'' model, others call it
the ``log link''.}

If
\[
input_{i}=exp(X_{i}b)
\]


then
\[
log(input_{i})=X_{i}b
\]


In the Generalized Linear Model literature, they think of the transformation
happing to the left hand side, they call it the link function. 

So the exponential on the right is the ``inverse link'' function.


\lyxframeend{}\lyxframe{Estimation: straight forward ML}

Adjust the b's to maximize the product of the probabilities of the
observations:
\[
L(b;y,X)=Pr(y_{1}|Xb)*Pr(y_{2}|Xb)*...*Pr(y_{N}|Xb)
\]


Usually, one would take logs, and maximize the log likelihood, which
would be a sum.


\lyxframeend{}\lyxframe{Interpretation}
\begin{itemize}
\item Recall the expected value of $y_{i}$ given input is just the input
itself. 
\[
E(y_{i}|X_{i})=exp(X_{i}\hat{b})
\]

\item So if the k'th variable changes, the impact is
\end{itemize}
\[
\frac{\partial E(y_{i}|X_{i})}{\partial x_{k}}=\hat{b}_{k}*E(y_{i}|X_{i})
\]


\[
=\hat{b}_{k}*exp(X_{i}\hat{b})
\]

\begin{itemize}
\item Long discusses the calculation of the percent change in expected y,
i.e.
\end{itemize}
\[
\frac{E(y_{i}|X_{i},x_{k}+\delta)}{E(y_{i}|X_{i},x_{k})}=exp(\hat{b}*\delta)
\]



\lyxframeend{}

\begin{frame}
\frametitle{See poisson-1.R for this example}
\includegraphics[width=10cm]{plots/t-pois20}
\end{frame}

\begin{frame}[containsverbatim]
\frametitle{Poisson GLM Fit}
\begin{Schunk}
\begin{Sinput}
  hist(0.5+rpois(2000, lambda=3), xlim = c(0.5,10.5), prob = TRUE, xlab = "x", main = "Poisson Sample N=2000")
   points(xseq+0.5, xdpois, lty=4)
\end{Sinput}
\end{Schunk}

\end{frame}

\begin{frame}[containsverbatim]
\frametitle{Just for Curiosity, fit OLS}
\begin{Schunk}
\begin{Sinput}
 lm1 <- lm(y ~ x1 + x2, data=dat)
 summary(lm1)
\end{Sinput}
\begin{Soutput}
Call:
lm(formula = y ~ x1 + x2, data = dat)

Residuals:
       Min         1Q     Median         3Q        Max 
-133325385  -81548444  -29144773   26306378 1114657874 

Coefficients:
              Estimate Std. Error t value Pr(>|t|)    
(Intercept) -366588914   85770267  -4.274 4.53e-05 ***
x1            19986500    3681228   5.429 4.25e-07 ***
x2              621443    4685999   0.133    0.895    
---
Signif. codes:  0 '***' 0.001 '**' 0.01 '*' 0.05 '.' 0.1 ' ' 1

Residual standard error: 160200000 on 96 degrees of freedom
  (1 observation deleted due to missingness)
Multiple R-squared:  0.2384,	Adjusted R-squared:  0.2225 
F-statistic: 15.03 on 2 and 96 DF,  p-value: 2.102e-06
\end{Soutput}
\end{Schunk}

\end{frame}

\begin{frame}[containsverbatim, allowframebreaks]
\frametitle{See poisson-1.R for this example}
\begin{Schunk}
\begin{Sinput}
 plot(x1, y, main="Ugly Poisson Data")
 library(rockchalk)
 newdat <- expand.grid(x1=plotSeq(dat$x1, length.out=50), x2=mean(dat$x2))
 newdat$p1 <- predict(glm1, newdata=newdat, type="response")
 lines(newdat$x1, newdat$p1, lwd=3, col="red")
 newdat$lmp1 <- predict(lm1, newdata=newdat)
 lines(newdat$x1, newdat$lmp1, lwd=3, col="green")
 legend("topleft", legend=c("Exp. Poisson", "OLS"), lwd=c(3,3), col=c("red","green"))
 
\end{Sinput}
\end{Schunk}

\includegraphics[width=10cm]{plots/t-pois35}
\end{frame}

\begin{frame}[containsverbatim, allowframebreaks]
\frametitle{From x2's point of view}
\input{plots/t-pois36}
\includegraphics[width=10cm]{plots/t-pois36}
\end{frame}


\lyxframeend{}\section{Negative Binomial}


\lyxframeend{}\lyxframe{Poisson Weaknesses}
\begin{enumerate}
\item Poisson is a ``one parameter'' model. The variance is not separately
under our control. Maybe we could find a two parameter distribution
with a more well-suited variance parameter.
\item Repeat the same point: The Poisson may not fit the data because the
variance predicted by the Poisson may be too small for the observed
data.
\end{enumerate}

\lyxframeend{}\lyxframe{Negative Binomial Derivation: Overdispersion}
\begin{itemize}
\item The Negative Binomial can be described in a number of ways. 
\item I think the ``extra randomness'' interpretation is the simplest. 
\item $input_{i}$ additional random error that causes ``heterogeneity''
(sometimes the term ``frailty'' is used) in the outputs for cases
that have the same observed values of $X_{i}$. 
\item Suppose the Poisson process has an expected value:
\end{itemize}
\[
new\, input_{i}=input_{i}*\delta_{i}
\]


Note that if

\[
\delta_{i}=1
\]
\\
then this thing just degenerates back to the original Poisson model. 


\lyxframeend{}\lyxframe{Log Link and Multiplicative Error}
\begin{itemize}
\item In the most common version of the Poisson model, we use the ``log
link'' 
\end{itemize}
\[
input_{i}=exp(X_{i}b)
\]

\begin{itemize}
\item Supplement with an additional error term $u_{i}$:
\item 
\[
new\, input_{i}=exp(X_{i}b+u_{i})
\]

\end{itemize}

\lyxframeend{}\lyxframe{Multiplicative <-> Additive}
\begin{itemize}
\item Easy:
\item 
\[
new\, input_{i}=exp(X_{i}b+u_{i})=exp(X_{i}b)\times exp(u_{i})
\]

\item So one can either think of the new error as an additive bit of noise
with the linear predictor ($+u_{i})$ or a multiplicative effect applied
to the transformed linear predictor ($\times\delta_{i}=exp(u_{i})$). 
\item Obviously, we can convert ``back and forth'' 
\[
u_{i}=log(\delta_{i})
\]

\end{itemize}

\lyxframeend{}\lyxframe{Vital to Pick $\delta_{i}$ Distribution Properly}
\begin{itemize}
\item It is necessary to assume that this new noise is ``neutral'', in
the sense that it causes more uncertainty, but it does not change
the average outcome. 
\item That is true if


\[
E[\delta_{i}]=1\Longrightarrow E(exp(u_{i}))=1
\]
\\
or, equivalently, 
\[
E[u_{i}]=0
\]


\item ``On average'' the extra error term has ``no effect''.
\end{itemize}

\lyxframeend{}\lyxframe{Output is Conditional Poisson Model}
\begin{itemize}
\item The maximum likelihood estimation has to be amended to incorporate
a new likelihood component for each case. 
\item Hence, our theory says that GIVEN $X_{i}$ and an additional perturbation
$u_{i}$, the probability model is a a Poisson process.
\end{itemize}
\[
P(y_{i}|X_{i},u_{i})=\frac{exp(-new\, input_{i})\times new\, input_{i}^{y_{i}}}{y!}.
\]


The input on the right side includes the additional frailty.


\lyxframeend{}\subsection{Gamma distributed heterogeneity}


\lyxframeend{}\lyxframe{Gamma is most Common Frailty Distribution}
\begin{itemize}
\item Gamma is common probability distribution for $\delta_{i}=exp(u_{i})$
\item The full Gamma distribution has two parameters, but we are going to
simplify them so we only need to worry about one, $v=shape$, which
determines the variance. This simplification of the gamma can be done
in several ways, which will be outlined later. 
\item The key think is this: If $\delta_{i}$ is drawn from ``a properly
selected'' gamma distribution, then $E(\delta_{i})=1$ and $Var[\delta_{i}]=1/some\, parameter\, we\, choose$.
\end{itemize}

\lyxframeend{}\subsection{Gamma distribution background}


\lyxframeend{}\lyxframe{Gamma Density Illustration}

The Gamma describes the probability of a continuous variable on $[0,\infty].$
It can look like a ``ski slope'' or it can look single-peaked.

\begin{figure}
\begin{centering}
\includegraphics[width=3in]{0_home_pauljohn_SVN_SVN-guides_stat_Regression-GLM_Count-PoissonNegBinomial_importfigs_GammaPDF.pdf}
\par\end{centering}

\caption{Gamma Distribution\label{fig:Gamma-Distribution}}
\end{figure}



\lyxframeend{}\lyxframe{Gamma PDF}
\begin{itemize}
\item 2 parameters, $shape$ and $scale$. In some books, the $scale$ parameter
is replaced by a parameter called $rate$, which is equal to $1/scale$.
\item If $\delta_{i}$ is Gamma distributed, the probability density function
is:
\end{itemize}
\[
f(\delta_{i})=\frac{1}{scale^{shape}\Gamma(shape)}\delta_{i}^{(shape-1)}e^{-(\frac{\delta_{i}}{scale})}
\]



\lyxframeend{}\lyxframe{What is that Gamma function?}
\begin{topcolumns}%{}

\end{topcolumns}%{}

\column{6cm}
\begin{itemize}
\item The function $\Gamma(shape)$ is the Gamma function (which is a complicated
math thing I've never looked into very much). It is $\Gamma(s)=\int_{0}^{\infty}t^{s-1}e^{-t}dt$
if $s>0$. If you pick s as an integer, $\Gamma(s)$ is very easy
to calculate:


\[
\Gamma(s)=(s-1)!\,\,\, s=1,2,...
\]

\begin{itemize}
\item So, the value of $\Gamma(1)=1.$ And $\Gamma(2)=1.$ And $\Gamma(20)$
is some impossibly huge number. 
\end{itemize}

\column{6cm}


\begin{center}
\includegraphics[width=2in]{1_home_pauljohn_SVN_SVN-guides_stat_Regression-___nt-PoissonNegBinomial_importfigs_GammaFunct.pdf}
\par\end{center}

\end{itemize}

\lyxframeend{}\lyxframe{Adjust the Gamma PDF to create the right kind of heterogeneity.}
\begin{itemize}
\item The two parameter Gamma probability distribution has these interesting
properties:
\end{itemize}
\[
E(\delta_{i})=shape*scale
\]


\[
Var(\delta_{i})=shape*scale^{2}
\]

\begin{itemize}
\item Simplify: $scale=1/shape$
\item The expected value and variance are 
\[
E[\delta_{i}]=shape/shape=1
\]

\end{itemize}
\[
Var[\delta_{i}]=shape/shape^{2}=1/shape
\]



\lyxframeend{}\lyxframe{Suppose shape is the Same for All Observations}
\begin{itemize}
\item The $shape$ parameter is assumed to exist, we need to estimate it. 
\item In this formulation, it is easy to see that if $shape$ is very large,
then the variance of $\delta_{i}$ is very small. The extra heterogeneity
has only a minor effect, and, in fact, as $shape$ tends to $\infty$,
the value of $\delta_{i}$ collapses around $1.0$.
\end{itemize}

\lyxframeend{}\lyxframe{Another Derivation that ends up at the same place}
\begin{itemize}
\item Fix the scale = 1. 
\item Draw a random variable $m_{i}$ with gamma(shape, 1). Then the probability
density formula with scale=1 simplifies to:
\end{itemize}
\[
f(m_{i})=\frac{1}{\Gamma(shape)}m_{i}^{(shape-1)}e^{-m{}_{i}}\,\,\,\, shape>0
\]

\begin{itemize}
\item If shape=1, then this is an exponential distribution (because $\Gamma(1)=1).$ 
\item The expected value and variance are:


\[
E[m_{i}]=shape
\]
and

\end{itemize}
\[
Var[m_{i}]=shape.
\]
\\



\lyxframeend{}\lyxframe{Not Better formulation, just Different}
\begin{itemize}
\item The advantage of this formulation is that we can easily see what we
need to do to convert $m_{i}$ into our final result:
\[
\delta_{i}=\frac{m_{i}}{shape}
\]

\item Notice that after diving each draw by $shape,$ we have a variable
with just the same properties as the other formulation.
\end{itemize}
\[
E(\delta_{i})=E(\frac{m_{i}}{shape})=\frac{1}{shape}E(m_{i})=\frac{shape}{shape}=1
\]


and also

\[
V(\delta_{i})=V(\frac{m_{i}}{shape})=\frac{1}{shape^{2}}V(m_{i})=\frac{shape}{shape^{2}}=\frac{1}{shape}
\]

\begin{itemize}
\item If you go back and forth between books, you get a headache because
no two books seem to write this down in exactly the same way. But
I'm pretty sure I've written it down correctly.
\end{itemize}

\lyxframeend{}\lyxframe{Illustrate $m_{i}/shape$}

.

\begin{center}
\includegraphics[width=8cm]{2_home_pauljohn_SVN_SVN-guides_stat_Regression-___oissonNegBinomial_importfigs_GammaDivShape1.pdf}
\par\end{center}


\lyxframeend{}\lyxframe{Illustrate $log(\frac{v_{i}}{shape})$}

\begin{center}
\includegraphics[width=8cm]{3_home_pauljohn_SVN_SVN-guides_stat_Regression-___sonNegBinomial_importfigs_logGammaDivShape1.pdf}
\par\end{center}

Please note that, as the math above indicates, the center of the distribution
of $log(\frac{v_{i}}{shape})$ is centered on 0, and that as the shape
increases, the variance of the distribution shrinks dramatically.


\lyxframeend{}\lyxframe{About that ``shape'' parameter}

\begin{center}
\includegraphics[width=8cm]{4_home_pauljohn_SVN_SVN-guides_stat_Regression-___-PoissonNegBinomial_importfigs_GammaDistrib.pdf}
\par\end{center}


\lyxframeend{}\lyxframe{Estimating }
\begin{itemize}
\item Fitting is an iterative, two-stage process. 

\begin{itemize}
\item The shape estimate is chosen
\item Then the slope parameters are estimated. 
\end{itemize}
\item Repeat until estimates converge to stable values.
\item The MASS package for R provides a procedure ``glm.nb'' which will
do maximum likelihood to estimate the b's and the shape parameter.
(In Venables \& Ripley, p. 207, the ``shape'' parameter is called
$\theta$. 
\end{itemize}

\lyxframeend{}\lyxframe{The Negative Binomial Distribution ``Pops Out''}
\begin{itemize}
\item If you start with a Poisson model, and then add random noise with
multiplicative Gamma error, 
\end{itemize}
\[
Y\,|\,\delta_{i}\sim Poisson(input_{i}*\delta_{i})
\]

\begin{itemize}
\item The result is known (in probability theory) to be a Negative Binomial
Distribution. 
\end{itemize}
\[
f_{y}(y|shape,input)=\frac{\Gamma(shape+y)}{\Gamma(shape)y!}\bullet\frac{input^{y}shape^{shape}}{(input+shape)^{shape+y}}
\]


(Venables and Ripley, 4th ed, p. 206)

\[
E(y_{i})=input
\]


\[
Var(y_{i})=input+input^{2}/shape
\]


Note that if $shape=\infty$, then the variance of $y_{i}$ is just
$input_{i}$, meaning the original Poisson model is back! But for
other values of the shape parameter, the variance of $y_{i}$ is greater
than in the Poisson model.


\lyxframeend{}\subsection{NB estimation}

The results indicate one surprise, that the expected value of y is
the same in the Poisson and the NB model. However, the variance is
different. In the NB model, the variance is
\[
Var(y_{i}|X)=exp(X_{i}b)\left(1+\frac{exp(X_{i}b)}{v_{i}}\right)
\]



\lyxframeend{}\subsection{Overdispersion}

Estimates from a Poisson model are inefficient and have bad standard
errors if the data is really produced by a heterogeneous process of
the NB sort.

Note the Poisson model is really ``nested'' inside the NB model.
If we do a significance test of $H_{o}:\alpha=0$ and cannot reject
it, then it means we ought to go back to the Poisson. Long p. 237
discusses other tests. See the R package pscl for a test that can
be used.


\lyxframeend{}\section{Zero inflated models}

The Poisson or NB models might not match the data because they don't
have enough observed 0's.

The ``fix'' is to think of the probability process as a two step
thing. First, the observed y is either 0 or a number $y_{i}.$ Whether
it is observed or not is modeled by any dichotomous regression model,
such as logit or probit. Second, if it is observed, the count is given
by one of the models above.

All kinds of details flow forth if you get into writing out one of
these models. Should the predictors in the dichotomous regression
exercise be the same ones that are used in the Poisson or NB regression?
Should we insist the predictive part of the probit model is proportional
to the count model?

Now, how can a probability process give back a 0?

Either through the failure of the probit stage or a predicted 0 from
the count stage, so in the Poisson 
\[
P(y_{i}=0|X_{i})=\psi_{i}+(1-\psi_{i})*exp(-exp(Xb))
\]


(Write out the poisson for y=0 to understand the last term).

And the probability of any other value is given by the regular poisson,
multiplied by $(1-\psi_{i}):$

\[
P(y_{i}|X_{i})=(1-\psi_{i})*Poisson(Xb)
\]



\lyxframeend{}\section{Additional Readings}

For more reading on Count models, consult the following, probably
in this order:

Scott Long, Regression Models for Categorical and Limited Dependent
Variables, Chapter 8, ``Count outcomes''

Gary King. 1988. Statistical Models for Political Science Event Counts:
Bias in Conventional Procedures and Evidence for the Exponential Poisson
Regression Model, \emph{American Journal of Political Science}, 32(3):
838-863.

Gary King. 1989. Variance Specification in Event Count Models: From
Restrictive Assumptions to a Generalized Estimator. \emph{American
Journal of Political Science} 33(3): 762-784

Cameron and Trivedi, 1998. \emph{Regression Analysis of Count Data},
Cambridge University Press.


\lyxframeend{}
\end{document}
