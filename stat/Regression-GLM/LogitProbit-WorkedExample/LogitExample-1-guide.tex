%% LyX 2.3.0 created this file.  For more info, see http://www.lyx.org/.
%% Do not edit unless you really know what you are doing.
\documentclass[11pt,letterpaper,english]{extarticle}
\usepackage{lmodern}
\renewcommand{\sfdefault}{lmss}
\renewcommand{\ttdefault}{lmtt}
\usepackage[T1]{fontenc}
\usepackage[utf8]{inputenc}
\setlength{\parskip}{\medskipamount}
\setlength{\parindent}{0pt}
\usepackage{babel}
\usepackage{url}
\usepackage[authoryear]{natbib}
\usepackage[unicode=true,pdfusetitle,
 bookmarks=true,bookmarksnumbered=false,bookmarksopen=false,
 breaklinks=false,pdfborder={0 0 1},backref=false,colorlinks=false]
 {hyperref}
\usepackage{breakurl}

\makeatletter

%%%%%%%%%%%%%%%%%%%%%%%%%%%%%% LyX specific LaTeX commands.
\special{papersize=\the\paperwidth,\the\paperheight}

%% Special footnote code from the package 'stblftnt.sty'
%% Author: Robin Fairbairns -- Last revised Dec 13 1996
\let\SF@@footnote\footnote
\def\footnote{\ifx\protect\@typeset@protect
    \expandafter\SF@@footnote
  \else
    \expandafter\SF@gobble@opt
  \fi
}
\expandafter\def\csname SF@gobble@opt \endcsname{\@ifnextchar[%]
  \SF@gobble@twobracket
  \@gobble
}
\edef\SF@gobble@opt{\noexpand\protect
  \expandafter\noexpand\csname SF@gobble@opt \endcsname}
\def\SF@gobble@twobracket[#1]#2{}
%% Because html converters don't know tabularnewline
\providecommand{\tabularnewline}{\\}

%%%%%%%%%%%%%%%%%%%%%%%%%%%%%% Textclass specific LaTeX commands.
\providecommand*{\code}[1]{\texttt{#1}}

\@ifundefined{date}{}{\date{}}
%%%%%%%%%%%%%%%%%%%%%%%%%%%%%% User specified LaTeX commands.
% Following comment is required. Do not delete it.
%\usepackage{Sweave}

\usepackage[includehead,includefoot]{geometry}
\geometry{
lmargin=1in,
rmargin=1in,
tmargin=0.75in,
bmargin=1.0in,
headheight=0pt,
headsep=0pt,
marginparwidth=0pt,
footskip=1.5\baselineskip,
}

\usepackage{booktabs}
\usepackage{dcolumn}


% $ This nullifies emacs subscripting in following%
%%BEGIN: This begins the custom header
\usepackage{array}
\usepackage{xparse}
\usepackage{babel}
\usepackage{datetime}
\usepackage{xcolor}
\usepackage{colortbl}
\usepackage{graphicx}
\usepackage{multicol}

\definecolor{darkblue}{HTML}{1e2277}
\definecolor{kublue81186}{RGB}{0,81,186}
\newdateformat{monddyyyy}{\shortmonthname[\THEMONTH].~\THEDAY,~\THEYEAR}


%% BEGIN custom header
\ExplSyntaxOn
\NewDocumentCommand \guidesetup { m }
 {
  \keys_set:nn { guide } { #1 }
 }

\int_new:N \g_guide_authors_int
\prop_new:N \g_guide_authors_prop
\tl_new:N \l__guide_authors_tl

\keys_define:nn { guide }
 {
  author .code:n =
   {
    \int_gincr:N \g_guide_authors_int
    \keys_set:nn { guide / author } { #1 }
   },
  author/firstname .code:n = { \guide_add_author:nn { firstname } { #1 } },
  author/lastname .code:n = { \guide_add_author:nn { lastname } { #1 } },
  author/affiliation .code:n = { \guide_add_author:nn { affiliation } { #1 } },
  author/email .code:n = { \guide_add_author:nn { email } { #1 } },
  leftlogo .tl_set:N = \l_guide_leftlogo_tl,
  rightlogo .tl_set:N = \l_guide_rightlogo_tl,
  number .tl_set:N = \l_guide_number_tl,
  title .tl_set:N = \l_guide_title_tl,
  subtitle .tl_set:N = \l_guide_subtitle_tl,
  keywords .tl_set:N = \l_guide_keywords_tl,
  url .tl_set:N = \l_guide_url_tl,
  title .initial:n = {TITLE~NEEDED},
}
\NewDocumentCommand \guidehdr { +O{} }
 {
  \group_begin:
  \keys_set:nn { guide } { #1 }
  \guide_hdr:
  \group_end:
 }
\cs_new_protected:Nn \guide_hdr:
 {
   \centering
   \noindent\fcolorbox{kublue81186}{white}{
  \begin{tabular}
   {
    @{}c
    >{\centering}m{ \dim_eval:n { \textwidth - 2.5in - 6\tabcolsep - 2\arrayrulewidth } }
    c@{}
   }
   &&\tabularnewline[-\dim_eval:n { -\normalbaselineskip + 3ex }]
  \begin{tabular}{@{}>{\centering}p{1.25in}@{}}
  \includegraphics[width=.75in]{\l_guide_leftlogo_tl}
  \end{tabular}
  &
  \centering
    {\textsc{\large{}\l_guide_title_tl}}
    \tl_if_empty:NF \l_guide_subtitle_tl
    {
      {\\\textsc{\large{}\l_guide_subtitle_tl}}
    }
  \\
  \textcolor{kublue81186}{\rule[0.5ex]{1\linewidth}{0.7pt}}\\
  \guide_print_authors:
  &
  \begin{tabular}{@{}>{\centering}p{1.25in}@{}}
  \includegraphics[width=1.25in]{\l_guide_rightlogo_tl}
  \end{tabular}
  \tabularnewline
  \tl_if_empty:NTF \l_guide_number_tl
   { \leavevmode\vphantom{G}}
   {\small\bfseries { Guide~No: ~ \tl_use:N \l_guide_number_tl }} & 
   \tl_if_empty:NF \l_guide_keywords_tl
   {\centering
    \textbf{Keywords}: ~ \l_guide_keywords_tl}
   & 
   \small\bfseries\monddyyyy\today \tabularnewline
   \tl_if_empty:NF \l_guide_url_tl
   {
   \multicolumn{3}{c}{
    See ~ \url{\l_guide_url_tl} ~ for ~ updates.}
     }
  \tabularnewline
  \end{tabular}
  }
}

\cs_new_protected:Nn \guide_add_author:nn
 {
  \prop_gput:Nxn \g_guide_authors_prop
   { #1 \int_to_arabic:n { \g_guide_authors_int } }
   { #2 }
 }
\cs_generate_variant:Nn \prop_gput:Nnn { Nx }
\cs_new_protected:Nn \guide_print_authors:
 {
  \tl_clear:N \l__guide_authors_tl
  \int_step_inline:nnnn { 1 } { 1 } { \g_guide_authors_int }
   {
    \tl_put_right:Nn \l__guide_authors_tl
     {
      \textbf
       {
        \prop_item:Nn \g_guide_authors_prop { firstname ##1 }~
        \prop_item:Nn \g_guide_authors_prop { lastname ##1 },~
       }
       \prop_item:Nn \g_guide_authors_prop { affiliation##1 },~
       \href{mailto:\prop_item:Nn \g_guide_authors_prop { email##1 } }
       {\url{\prop_item:Nn \g_guide_authors_prop { email##1 }}}
       \\
     }
   }
   \tl_use:N \l__guide_authors_tl
 }
\ExplSyntaxOff
%%END This ends the custom header


%%Now the hyperref colors

\hypersetup{
 unicode=true,pdfusetitle,bookmarks=true,
 bookmarksnumbered=false, bookmarksopen=false,
 breaklinks=true, pdfborder={0 0 0},
 pdfborderstyle={}, backref=false,
 colorlinks=true, linkcolor=darkblue,
 urlcolor=darkblue, citecolor=darkblue}

%% Was previously in "preambleFloatCenter.tex"
%%BEGIN: graphics and such inside floats are centered by default
\renewenvironment{figure}[1][]{%
 \ifthenelse{\equal{#1}{}}{%
   \@float{figure}
 }{%
   \@float{figure}[#1]%
 }%
 \centering
}{%
 \end@float
}
\renewenvironment{table}[1][]{%
 \ifthenelse{\equal{#1}{}}{%
   \@float{table}
 }{%
   \@float{table}[#1]%
 }%

 \centering
}{%
 \end@float
}
%%END: graphics and such inside floats are centered by default



%% $ nullify Emacs underscores
%% BEGIN: Footer
\usepackage{fancyhdr}
\usepackage{lastpage}
\usepackage{xparse}

\ExplSyntaxOn
\NewDocumentCommand \footersetup { +m }
{
\keys_set:nn { footer } { #1 }
}
\keys_define:nn { footer }
{	
    leftside .code:n = {
\keys_set:nn { footer / leftside } { #1 }
},
rightside .code:n = {
\keys_set:nn { footer / rightside } { #1 }
},
    leftside / lone .tl_set:N = \l_footer_leftside_lone_tl,
    leftside / ltwo .tl_set:N = \l_footer_leftside_ltwo_tl,
    leftside / lthree .tl_set:N = \l_footer_leftside_lthree_tl,
    rightside / rone .tl_set:N = \l_footer_rightside_rone_tl,
    rightside / rtwo .tl_set:N = \l_footer_rightside_rtwo_tl,
    rightside / rthree .tl_set:N = \l_footer_rightside_rthree_tl,
    leftside .initial:n = {lone = , ltwo = ~, lthree = ~ },
    rightside .initial:n = {rone = ~, rtwo = ~,  rthree = ~},
    leftside .default:n = {lone = , ltwo = ~, lthree = ~ },
    rightside .default:n = {rone = ~, rtwo = ~,  rthree = ~},
}
\NewDocumentCommand \guideftr { +o }
{
  \IfValueT { #1 } 
  {
    \keys_set:nn { footer } { #1 }
  }
  \guide_ftr:
}
\cs_new_protected:Nn \guide_ftr:
{
  % Now fancy header/footer settings
  \lhead{}%
  \chead{}%
  \rhead{}%
  \renewcommand{\headrulewidth}{0pt}%
  \lfoot{}%
  \cfoot{{Page} ~ \thepage {} ~ { of } ~ {} \pageref{LastPage}}%
  \rfoot{}%

  \fancypagestyle{firststyle}%
  {%
    \lhead{}%
    \chead{}%
    \rhead{}%
    \lfoot{}%
    \cfoot{}%
    \rfoot{}%
    \renewcommand{\headrulewidth}{0pt}%
    \fancyfoot[L]{\l_footer_leftside_lone_tl {}\\%
      \l_footer_leftside_ltwo_tl {}\\%
      \l_footer_leftside_lthree_tl {}}%
    \fancyfoot[R]{\l_footer_rightside_rone_tl {}\\%
      \l_footer_rightside_rtwo_tl {}\\%
      \l_footer_rightside_rthree_tl {}}%
      \renewcommand{\footrulewidth}{.2pt}%
  }%
  \thispagestyle{firststyle}
}
\ExplSyntaxOff

\pagestyle{fancy}
%% End of fancy footer setup

%% Specify fully the left and right sides of footer, 3 lines
%% If blanks are needed, must insert value " ~ "
%% If comma needed inside value, wrap in {}.
\urlstyle{tt}
\footersetup{
  leftside={
    lone={Address line 1}, 
    ltwo={Address line 2}, 
    lthree={City State Zipcode}},
  rightside={
    rone=Web: \url{https://crmda.ku.edu},
    rtwo=Email: \href{mailto:you@where.edu}{\url{you@where.edu}},
    rthree=Phone: 123-345-5678}
}
\guideftr



%%% From beamer slide:
\usepackage{Sweave}
%% 
%% This controls display of code chunks
\usepackage{ae,fancyvrb,relsize,listings}

\providecommand{\Sweavesize}{\normalsize}
\providecommand{\Rsize}{}
\renewcommand{\Rsize}{\normalsize}
\providecommand{\Routsize}{\scriptsize}

\providecommand{\Rcolor}{\color[rgb]{0.1, 0.1, 0.1}}
\providecommand{\Routcolor}{\color[rgb]{0.2, 0.2, 0.2}}
\providecommand{\Rcommentcolor}{\color[rgb]{0.101, 0.43, 0.432}}

\providecommand{\Rbackground}{\color[gray]{0.91}}
\providecommand{\Routbackground}{\color[gray]{0.935}}
% Can specify \color[gray]{1} for white background or just \color{white}

\lstdefinestyle{Rinput}{
  language=R,
  escapechar=`,
  fancyvrb=false,%
  tabsize=2,%
  breaklines=true,
  breakatwhitespace=true,%
  captionpos=b,%
  frame=single,%
  framerule=0.2pt,%
  framesep=1pt,%
  showstringspaces=false,%
  basicstyle=\Rsize\Rcolor\ttfamily,%
  columns=fixed%,
  \lst@ifdisplaystyle\scriptsize\fi,%,
  commentstyle=\Rcommentcolor\ttfamily,%
  identifierstyle=,%
  keywords=\bfseries,%
  keywordstyle=\color[rgb]{0, 0.5, 0.5},
  escapeinside={(*}{*)},
  literate={~}{{$\sim$}}1{==}{{=\,=}}2{--}{{-\,-}}2,
  alsoother={$},
  alsoletter={.<-},%
  otherkeywords={!,!=,~,$$,*,\&,\%/\%,\%*\%,\%\%,<-,<<-,/},%
  backgroundcolor=\Rbackground,%
  numbers=left,%
  %numberblanklines=false,%
  stepnumber=5,
  firstnumber=1,
  numberstyle={\tiny}
}%

% Other options of interest:
% frame=single,framerule=0.1pt,framesep=1pt,rulecolor=\color{blue},
% numbers=left,numberstyle=\tiny,stepnumber=1,numbersep=7pt,
% keywordstyle={\bf\Rcolor}

\lstdefinestyle{Routput}{fancyvrb=false,
  literate={~}{{$\sim$}}1{R^2}{{$R^{2}$}}2{^}{{$^{\scriptstyle\wedge}$}}1{R-squared}{{$R^{2}$}}2,%
  basicstyle=\Routcolor\Routsize\ttfamily,%
  backgroundcolor=\Routbackground,
  language=R,
  escapechar=`,
  fancyvrb=false,%
  tabsize=2,%
  breaklines=true,
  breakatwhitespace=true,%
  captionpos=b,%
  frame=single,%
  framerule=0.2pt,%
  framesep=1pt,%
  showstringspaces=false,%
  columns=fixed%,
  \lst@ifdisplaystyle\scriptsize\fi,%
  identifierstyle=,%
  keywords=\bfseries,%
  keywordstyle=\color[rgb]{0, 0.5, 0.5},
  escapeinside={(*}{*)},
  literate={~}{{$\sim$}}1 {==}{{=\,=}}2,
  alsoother={$},
  alsoletter={.<-},%
  otherkeywords={!,!=,~,$,*,\&,\%/\%,\%*\%,\%\%,<-,<<-,/},
  numbers=left,
  %numberblanklines=false,%
  stepnumber=5,
  firstnumber=1,
  numberstyle={\tiny}
}

\renewenvironment{Schunk}{}{}
\renewenvironment{Sinput}{}{}
\let\Sinput\relax
\let\Scode\relax
\let\Soutput\relax
\lstnewenvironment{Sinput}{\lstset{style=Rinput}}{}
\lstnewenvironment{Scode}{\lstset{style=Rinput}}{}
\lstnewenvironment{Soutput}{\lstset{style=Routput}}{}
%%end paste in from guidePreambleSweavel.tex


\lstset{tabsize=2, breaklines=true, style=Rinput, breakatwhitespace=true}

\fvset{listparameters={\setlength{\topsep}{0em}}}

\usepackage{xcolor}
\definecolor{light-gray}{gray}{0.90}
\usepackage{realboxes}
\providecommand*{\code}[1]{\texttt{#1}}
\renewcommand{\code}[1]{%
\Colorbox{light-gray}{#1}%
}%


\makeatother

\begin{document}
%% Fill in values of the arguments here, 
%% If blanks are needed, must insert value " ~ "
%% If comma needed inside value, wrap in {}.
%% Delete secondauthor and thirdauthor if not needed
\guidesetup{%
  author={
    lastname=Johnson, 
    firstname=Paul, 
    affiliation=CRMDA,
    email=pauljohn@ku.edu},
  url={https://crmda.ku.edu/guides},
  keywords={single-authoring, just one},
  title={A Title for Skeleton Template: rnw2pdf-guide-sweave},
  leftlogo={theme/logoleft.pdf},
  rightlogo={theme/logo-vert.pdf},
}
\guidehdr

% tmpout directory must exist first

% In document Latex options:
\fvset{listparameters={\setlength{\topsep}{0em}}}



%The listings class used here allows within-document style
%changes. R input boxes are governed by 
% "Rsize", "Rbackground" and "Rcolor", while R output boxes depend on
% "Routsize", "Routbackground", and "Routcolor". Colors
% can be specified in many ways, as shown here
%\def\Rsize{\huge\ttfamily}
%\def\Routsize{\huge}
%\def\Rbackground{\color[gray]{0.90}}
%\def\Routbackground{\color[gray]{0.40}}
%\def\Rcolor{\color[gray]{0.60}
%\def\Routcolor{\color[rgb]{0.9, 0.1, 0.1}]}
%\def\Rcommentcolor{\color{green}}
\begin{abstract}
\noindent Some variables from the American National Election Study
of 2002 are used to demonstrate logistic regression in R. The same
old SAS export file that we created 2003 still works, all these years
later. 
\end{abstract}
\tableofcontents{}

\section{Get that ANES 2002 dataset}

In 2003, some students and I used SAS to pull variables from the American
National Election Study. That effort was immortalized in a post on
the KU server, then called lark. When lark closed, I copied that material
onto my personal web server, \url{http://pj.freefaculty.org/DataSets/ANES/2002}.
The result of the importation was a SAS export file, which I called
``PJTEST.sasxport'' because I did not think it would work. But it
did. 

The data import process today might be handled differently, but this
one is still to be cherished because of its special place in time.
That was the semester when I quit teaching stats with SAS and asked
the students to use R (\citet{RCore}).

There is a SAS export dataset that is the primary source material
for this exercise. 

There is one very important thing to explain about this data. The
variable labels were lost, we have just the numeric values for them.
Hence, to create factors in R, we have to do a little work. 

Now, in 2018, I'd be telling everybody ``don't bother with that,
use the variable key in the kutils package.'' But I realize that's
an unfamiliar thing to most people, so I'll do the recoding the old
fashioned way here.

\subsection{Check the Codebook}

The full variable list of the original survey is oneline:

\url{http://pj.freefaculty.org/DataSets/ANES/2002/RELEASE/nes02var.txt}

Some of the highlights are as follows.

\textbf{V023131} Y3x. Summary: R Education 

What is the highest grade of school or year of college you completed? 

1. 8 grades or less and no diploma or equivalency 

2. 9-11 grades, no further schooling 

3. High school diploma or equivalency test 

4. More than 12 years of schooling, no higher degree 

5. Junior or community college level degrees (AA degrees) 

6. BA level degrees; 17+ years, no advanced degree 

7. Advanced degree, including LLB

9. Refused 

0. NA in Y3, Y3a or Y3b

V023131 Frequency ------------------- 

0 2 

1 36 

2 70 

3 399 

4 313 

5 155 

6 347 

7 178 

9 11

===============================

\textbf{V023111} Q1a. Who did R vote for in 2000 Pres Election 

Which one did you vote for? 

1. Al Gore 

3. George W. Bush 

5. Ralph Nader 

7. Other \{SPECIFY\}

8. Dont know 

9. Refused 0. NA

INAP. Fresh Cross-Section respondent; 5,8,9 in Q1

V023111 Frequency ------------------- . 

526 0 1 1 431 3 502 5 32 7 8 8 1 9 10

============================== 

\textbf{V023036} J1. R Consider Self Dem/Rep/Ind Numeric Missing eq
0, ge 8

J1.

Generally speaking, do you usually think of yourself as a REPUBLICAN,
a DEMOCRAT, an INDEPENDENT, or what? ---------------------------------------

1. Democrat 

2. Republican 

3. Independent 

4. Other Party \{VOL\} \{SPECIFY\} 

5. No Preference \{VOL\}

8. Don't know 

9. Refused 0. NA

V023036 Frequency ------------------- 

0 13 

1 502 

2 474 

3 429 

4 27 

5 59 

8 5 

9 2

============================== 

\textbf{V023022} R 7Pt Scale Lib-Con Self-Placement 

We hear a lot of talk these days about liberals and conservatives.
When it comes to politics, do you usually think of yourself as EXTREMELY
LIBERAL, LIBERAL, SLIGHTLY LIBERAL, MODERATE OR MIDDLE OF THE ROAD,
SLIGHTLY CONSERVATIVE, CONSERVATIVE, EXTREMELY CONSERVATIVE, or haven't
you thought much about this? --------------------------------------------------------------------------

01. Extremely Liberal 

02. Liberal 

03. Slightly Liberal 

04. Moderate; Middle of the Road 

05. Slightly Conservative 

06. Conservative 

07. Extremely Conservative

08. Don't know 

09. Refused 

90. Haven't thought much {[}Do Not Probe{]} 

00. NA

V023022 Frequency ------------------- 

0 11 

1 23 

2 181 

3 135 

4 340 

5 186 

6 315 

7 65 

8 8 

9 3 

90 244

============================== 

\textbf{V023027} H1. US Economy Better/Worse in Last Yr

Now thinking about the economy in the country as a whole, would you
say that over the past year the nation's economy has gotten BETTER,
STAYED ABOUT THE SAME, or gotten WORSE? ---------

1. Better 

3. Same 

5. Worse

8. Don't know 

9. Refused 

0. NA

V023027 Frequency ------------------- 

0 6 

1 69 

3 321 

5 1112 

8 3

===================================

\textbf{V023010}: Where on that thermometer would you rate George
W. Bush? (on a scale from 0 to 100) 

====================

\subsection{Use foreign::read.xport to import the data}

R's package ``foreign'' has a number of procedures to import data
from other packages. Happily, this data imports correctly.

First, I'll retrieve a copy of the file from the web server. I'll
rename it ``anes2002.sasxport''. To keep this simple, I'll just
drop it in the current working directory. (In real projects, I'd put
it in a separate folder.)

\input{tmpout/t-2a}

The SAS export file is easy to import

\begin{Schunk}
\begin{Sinput}
 library(rockchalk)
 summarize(dat)
\end{Sinput}
\begin{Soutput}
$numerics
     elevation latitude longitude   tann
0%        2.00   42.050  -124.600  3.200
25%      94.25   43.600  -123.200  8.700
50%     462.00   44.460  -121.700 10.400
75%     863.80   45.280  -119.600 11.200
100%   1974.00   46.150  -117.000 12.600
mean    557.00   44.320  -121.300  9.885
sd      498.30    1.116     2.278  1.874
var  248300.00    1.245     5.191  3.511
NA's      0.00    0.000     0.000  0.000
N        92.00   92.000    92.000 92.000

$factors
          station      
 ANT          : 1.000  
 ARL          : 1.000  
 ASH          : 1.000  
 AST          : 1.000  
 (All Others) :88.000  
 NA's         : 0.000  
 entropy      : 6.524  
 normedEntropy: 1.000  
 N            :92.000  
\end{Soutput}
\end{Schunk}


I've got 1511 rows and 732
columns. 

Unfortunately, all of the variable labels were lost in the process,
I have only the numeric scores. I found that out by using the str()
function:

\begin{Schunk}
\begin{Sinput}
 x7 <- seq(0, 100)
 xprob7 <- dnbinom(x7, size = 50, prob = 0.5)
 plot(xprob7, type="s")
\end{Sinput}
\end{Schunk}


\subsection{Recodes}

I've got a few challenges. For categorical variables, it is best to
use R factors, rather than integers to record the information. This
would have been easier if I had made a variable key, I'm sorry I did
not. But I'll make an informal variable table to help trace this through.

\begin{tabular}{|c|c|c|c|c|}
\hline 
name\_old & name\_new &  & class\_new & value\_new\tabularnewline
\hline 
\hline 
V023111 & bushvotef &  & factor & ``Gore'', ``Bush''\tabularnewline
\hline 
V023111 & bushvote &  & integer & 0=Gore,1=Bush\tabularnewline
\hline 
V023036 & partyid &  & factor & ``D'',''R'',''I''\tabularnewline
\hline 
V023036 & repub &  & integer & 1=''R'',0=''D'',''I'',or''NA''\tabularnewline
\hline 
V023036 & democ &  & integer & 1=''D'',0=''R'',''I'',or''NA''\tabularnewline
\hline 
V023022 & ideolf &  & factor & ``EL'', ``L'',...,''EC''\tabularnewline
\hline 
V023022 & ideoln &  & integer & 1-7\tabularnewline
\hline 
\end{tabular}
\begin{enumerate}
\item Create ``bushvote'', a binary variable for Gore vs Bush, and all
of the other candidates are treated as missing.

\input{tmpout/t-rec10}

\input{tmpout/t-rec15}
\item For political party, it looks like I featured the simple 3 category
variable, V023036, not the 7 point ``strength of identification''
scale that you see in some projects. First I'll assign labels for
the 3 most widely used party labels.

\input{tmpout/t-rec20}

In the analysis below, I have some illustrations using democrat and
republican dummy variables. 

\input{tmpout/t-rec30}

As I look at that now, I have some misgivings about it because of
the way missing values are handled. Can you see why I might worry? 

(Concern is that when we say democ = 0, we might mean everybody who
is not a democrat, EVEN including missing values on party. In this
current coding, we don't have that, we have missings on democ unless
party is D, R, or I.

The most I think about that, the more I think I ought to fix it. So
here's a fix:

\input{tmpout/t-rec31}

The important thing about this ``fix'' is that we mean to say that
democ is 1 if a person is a democrat, and they are 0 no matter what
otherwise. They are 0 even if they did not answer the party identification
question. This is aggressive coding, I think. It is the way we always
used to do it in SAS, however.
\item Fix Ideology. Some authors treat ideology as a numeric variable, as
if 1 through 7 are numerically meaningful scores. Some authors say
``that's a categorical variable, don't do that.'' We can try both
ways. Hence we have ideolf and ideoln

\input{tmpout/t-rec50}

\input{tmpout/t-rec51}
\end{enumerate}
In the discussion below, I compare an ideology variable coded 1-2-3-4-5-6-7
against one that is coded with labels.

\section{Ideology and the vote}

I decided to ``be stupid'' and simply treat Ideology as a 7 point
numerical variable (even though it is actually an ordered factor).

\subsection*{Plot The Vote's Dependence on Ideology}

Consider the very uninformative plot in Figure \ref{cap:scatter1}.
That's really quite pathetic. The visual impact is not quite so depressing
in Figure \ref{cap:jittered1}, which uses ``jittered'' values to
display overlapped points. That figure includes the predicted values
from the OLS regression (discussed next).

\begin{figure}[h]
\caption{Scatterplot showing impact of ideology on vote choice in 2000\label{cap:scatter1}}

\begin{Schunk}
\begin{Sinput}
 plot(nes2002$V023022,nes2002$bushvote,ylim=c(0,1),xlab="ideology",ylab="Vote (Bush=1, Gore=0)")
\end{Sinput}
\end{Schunk}

\includegraphics{tmpout/t-3b}
\end{figure}


\subsection*{Ordinary Least Squares}

The ordinary least squares result is obtained with the lm (short for
``linear model'') function in R. The fitted model is saved as an
object that we call ols1. The summary results are obtained by applying
the summary function to the result. 

\begin{Schunk}
\begin{Sinput}
 ols1 <- lm(bushvote~V023022,data=nes2002) 
 summary(ols1)
\end{Sinput}
\begin{Soutput}
Call:
lm(formula = bushvote ~ V023022, data = nes2002)

Residuals:
    Min      1Q  Median      3Q     Max 
-1.0167 -0.2943  0.1639  0.3445  1.0670 

Coefficients:
             Estimate Std. Error t value Pr(>|t|)    
(Intercept) -0.247610   0.045310  -5.465 6.18e-08 ***
V023022      0.180620   0.009624  18.768  < 2e-16 ***
---
Signif. codes:  0 '***' 0.001 '**' 0.01 '*' 0.05 '.' 0.1 ' ' 1

Residual standard error: 0.4149 on 807 degrees of freedom
  (702 observations deleted due to missingness)
Multiple R-squared:  0.3038,	Adjusted R-squared:  0.303 
F-statistic: 352.2 on 1 and 807 DF,  p-value: < 2.2e-16
\end{Soutput}
\end{Schunk}


\begin{figure}[h]
\caption{Jittered values reveal more information\label{cap:jittered1}}

\begin{Schunk}
\begin{Sinput}
 with(nes2002, plot(jitter(V023022),jitter(bushvote),ylim=c(0,1), 
  xlab="ideology",ylab="Vote (Bush=1,Gore=0)"))
 abline(ols1)
\end{Sinput}
\end{Schunk}

\includegraphics{tmpout/t-3d}
\end{figure}


\subsection*{Logistic estimation with glm}

The R team prefers that we think of the logit model as a Generalized
Linear Model (GLM). We are supposed to think of the outcomes as draws
from a Binomial distribution (coin flips) with probability equal to
the following formula:

\[
P(Y_{i}=1)=\frac{1}{1+e^{-(\beta_{0}+\beta_{1}X_{i})}}
\]

Imagine searching through the possible combinations of $\hat{\beta}_{0}$
and $\hat{\beta}_{1}$. For each combination, we can calculate $\hat{p}_{i}$,
an estimate of the probability that the $i$'th observation will be
1. 

\[
\hat{p}_{i}=\frac{1}{1+e^{-(\hat{\beta}_{0}+\hat{\beta}_{1}X_{i})}}
\]
The estimator will tune $\hat{\beta}_{0}$ and $\hat{\beta}_{1}$
up and down, trying to make the observed data as likely as possible.

The ``raw'' output from R is as follows. 

\begin{Schunk}
\begin{Sinput}
 #first, use the glm method from R's builtin functions
 bushideoglm1 <- glm(bushvote~V023022,family=binomial,data=nes2002) 
 summary(bushideoglm1)
\end{Sinput}
\begin{Soutput}
Call:
glm(formula = bushvote ~ V023022, family = binomial, data = nes2002)

Deviance Residuals: 
    Min       1Q   Median       3Q      Max  
-2.3568  -0.7596   0.5632   0.8583   2.4691  

Coefficients:
            Estimate Std. Error z value Pr(>|z|)    
(Intercept) -3.95168    0.31728  -12.46   <2e-16 ***
V023022      0.95209    0.07001   13.60   <2e-16 ***
---
Signif. codes:  0 '***' 0.001 '**' 0.01 '*' 0.05 '.' 0.1 ' ' 1

(Dispersion parameter for binomial family taken to be 1)

    Null deviance: 1110.80  on 808  degrees of freedom
Residual deviance:  833.91  on 807  degrees of freedom
  (702 observations deleted due to missingness)
AIC: 837.91

Number of Fisher Scoring iterations: 4
\end{Soutput}
\end{Schunk}


\subsection*{Presentable Tables}

\input{tmpout/t-3e2}

The first package I know of that incorporated a high-quality LaTeX
table was ``memisc'', authored by Martin Elff. The regression table
has some interesting features. The LaTeX rendering from the use of
the memisc package is displayed in Table \ref{tab:memisc1}:

\begin{table}

\caption{Logistic Regression Tables (memisc)\label{tab:memisc1}}

\begin{Schunk}
\begin{Sinput}
 toLatex(mtable(bushideoglm1))
\end{Sinput}
\begin{tabular}{lD{.}{.}{3}}
\toprule
(Intercept)          & -3.952^{***}\\
                     & (0.317)     \\
V023022              &  0.952^{***}\\
                     & (0.070)     \\
\midrule
Aldrich-Nelson R-sq. &    0.255\\
McFadden R-sq.       &    0.249\\
Cox-Snell R-sq.      &    0.290\\
Nagelkerke R-sq.     &    0.388\\
phi                  &    1.000\\
Likelihood-ratio     &  276.885\\
p                    &    0.000\\
Log-likelihood       & -416.956\\
Deviance             &  833.913\\
AIC                  &  837.913\\
BIC                  &  847.304\\
N                    &  809    \\
\bottomrule
\end{tabular}\end{Schunk}

\end{table}

The strength of this is the profuse listing of summary statistics. 

I don't want student papers to include a profusion of summary statistics
that they don't understand. That's why rockchalk::outreg's output,
which is displayed in the bottom part of Table . 
\begin{center}
\begin{table}
\caption{Logistic Regression Table (outreg)\label{tab:outreg1}}
\begin{Schunk}
\begin{Sinput}
 outreg(bushideoglm1, tight=F)
\end{Sinput}
\begin{tabular}{*{3}{l}}
\hline
     & \multicolumn{2}{l}{ M1}\tabularnewline
      & Estimate  & (S.E.) \tabularnewline
 \hline
 \hline
  (Intercept)    &-3.952*** &   (0.317) \tabularnewline
  V023022    &0.952*** &   (0.070) \tabularnewline
 \hline
 N & 809 &  \tabularnewline
   Deviance         &833.913         & \tabularnewline
 $-2LLR (Model \chi^2)$  &   276.885*** &   \tabularnewline
 \hline
\hline
 
 \multicolumn{3}{c}{${*  p}\le 0.05$${*\!\!*  p}\le 0.01$${*\!\!*\!\!*  p}\le 0.001$}\tabularnewline
 \end{tabular}\end{Schunk}

hello
\end{table}
\par\end{center}

\subsection{How to interpret these estimates.}

Most of our emphasis is on the estimates of the $\beta$'s. We wonder
``does the predictor significantly increase the chances that the
outcome will be 1?'' 

The probability is defined in this two step procedure.

1. Calculate the Linear Predictor

\begin{equation}
\hat{\eta}_{i}=\hat{\beta}_{0}+\hat{\beta}_{1}x_{i}+more\,\hat{\beta}\cdot x's
\end{equation}

2. ``Bend'' the Linear Predictor into a Probability curve
\begin{equation}
\hat{p}_{i}=Prob(Y_{i}=1|\hat{\eta}_{i})=\frac{1}{1+e^{-\hat{\eta}}}
\end{equation}

The intercept and the predictor variables all ``blend together''
in the linear predictor to control the chances that the outcome will
be 1. 

The best way to understand the $\beta$'s is to calculate $\hat{p}_{i}$
for ``interesting'' values of the input variables.

Please remember this very important thing:
\begin{quote}
To calculate a predicted value, one must include values for all input
variables and all parameters.
\end{quote}
That means the ``effect of x'' does really depend on the values
of all of the other input variables.

The usual approach is to set all of the variables at their ``means''
or ``modes'' and then adjust the values of a particular input to
findout how the predicted probability changes.

\subsection{Use the predict function}

In the past, I have calculated predicted values ``manually'' by
taking the estimated coefficients and using them in the logistic equation.
R offers a function ``predict'' to handle that for us. 

In this case, since ideology takes on values from 1 to 7, we need
only calculate 7 predictions. To obtain predicted values from the
glm output, one uses the following approach:

\begin{Schunk}
\begin{Sinput}
 pp3<-predict(bushideoglm1,newdata=data.frame(V023022=1:7),type="response",se.fit=T)
 pp3
\end{Sinput}
\begin{Soutput}
$fit
         1          2          3          4          5          6          7 
0.04744458 0.11430586 0.25060247 0.46423380 0.69185001 0.85331917 0.93778746 

$se.fit
         1          2          3          4          5          6          7 
0.01132256 0.01885170 0.02403794 0.02210266 0.02046537 0.01784950 0.01185117 

$residual.scale
[1] 1
\end{Soutput}
\end{Schunk}


The calculation of the confidence intervals for those predictions
is a bit more complicated. The predict method for glm has no argument
interval=''confidence'', as does the predict for linear models.
One must then construct a confidence interval in the un-transformed
``log odds'' scale and then map that onto the logistic scale. That
is done in Figure \ref{fig:LogitWConfInt}.

\begin{figure}
\caption{Fitted Values and Confidence Interval of Logistic Regression\label{fig:LogitWConfInt}}

\begin{Schunk}
\begin{Sinput}
 dat <- genCorrelatedData(1000, stde=5)
 dat$x3 <- rnorm(1000)
 dat$x4 <- rnorm(1000)
 dat$x5 <- factor(c( rep("A", 700), rep("B", 300)))
 m2 <- lm( y ~ x1 + x2 + x3 + x4 + x5, data=dat)
 m2.data <- model.data(m2)
 newdata1 <- newdata(m2, predVals = list(x1 = "quantile"))
 newdata1$fit <- predict(m2, newdata=newdata1)
 print(newdata1, digits = 3)
\end{Sinput}
\begin{Soutput}
    x1   x2     x3       x4 x5  fit
1 43.3 50.5 0.0162 -0.00544  A 18.6
2 50.2 50.5 0.0162 -0.00544  A 20.0
3 57.2 50.5 0.0162 -0.00544  A 21.4
\end{Soutput}
\end{Schunk}

\includegraphics{tmpout/t-022}
\end{figure}


\subsection*{rockchalk to make that automatic? Maybe}

Because some students have trouble putting together the sequence of
calculations, the rockchalk package has functions that are intended
to do this work. These are described in the vignette rockchalk, but
here's the high level overview. 

The predictOMatic function is supposed to look at the regression,
find the predictors, get ``interesting'' example values, and then
calculate the predictions (possibly with a confidence interval). It
turns out that deciding what is ``interesting'' is subjective, so
the predVals parameter has a \emph{whole lot} of options. At the current
time, the options are difficult to understand and it is necessary
to study the help page and the examples to truly grasp the power and
beauty of this fully operational death star. Here I take the easy
road by simply listing the values for the one predictor.

\begin{Schunk}
\begin{Sinput}
 plot(1:10, 1:10, type="n")
 text(4, 5, expression(paste( alpha  ," is alpha")))
 text(7, 3, expression(paste( beta * alpha, " is beta * alpha")))
 text(3, 6, expression(paste(frac(beta, alpha), " is frac(beta,alpha)")))
 text(2,9, expression(paste(y[i] == beta[0]+beta[1]*x[1]+hat(e))))
\end{Sinput}
\end{Schunk}


I think it is useful to run \code{predictOMatic}, especially if an
output table of predicted probabilities is needed. These can be used
to create a predicted value plot.

On the other hand, it is not necessary for the user to run \code{predictOMatic}
to get a nice looking plot. The \code{plotCurves} function will run
predictOMatic in the background and draw a plot. An example of plotCurves
is found in Figure \ref{fig:plotCurves-ideoln}. 

\begin{figure}[h]
\begin{Schunk}
\begin{Sinput}
 x <- 10
 plot( 1:10, 1:10, type="n")
 anexpression <-   bquote(x == .(x))
 text( 4, 5, anexpression)
 mnorm <- round(mean(rnorm(1000, m=10, sd=20)), 2 )
 text( 7,5, bquote(mu == .(mnorm)))
 text(4, 2, bquote(lambda == .(round(mean(rpois(100,lambda=7))))))
\end{Sinput}
\end{Schunk}

\includegraphics{tmpout/t-pm20}

\caption{plotCurves demonstration with ideology\label{fig:plotCurves-ideoln}}
\end{figure}

\code{predictOMatic} might fail, \code{plotCurves} might fail. These
fail when the regression model being estimated has some unusual characteristic,
something I did not plan for.

In which case I'd fall back to the two step procedure. 
\begin{enumerate}
\item make a ``newdata'' data frame that has one row for each hypothetical
case for which I want to make a prediction, and
\item run predict (as shown in Figure \ref{fig:LogitWConfInt}) and then
fiddle around with the fit and se.fit values to manufacture the confidence
intervals.
\end{enumerate}
Because making the newdata object can be frustrating, I created a
function called \code{newdata} in \code{rockchalk}.

\section{Put party into the logistic regression}

In 2003, I include this section only because I thought Figure \ref{cap:DandR}
was really neat. I believe this shows that the effect of ideology
is easier to spot if you contrast Democrats and Republicans. Now,
in 2018, I have a better way to estimate the model and make the plot. 

But I think it is somewhat unfair to students if I give you the awesome
result we have at the current time without admitting that I blundered
about while I was finding the better way. So I'll include the old
demonstration, then the new way. 

\subsection*{The 2003 way I included party id in the model}

The party variable was recoded into two ``dummy'' variables, ``repub''
id 1 for people who are in the Republican party, 0 for others, and
``democ'', represents membership in the Democratic party.

\begin{Schunk}
\begin{Sinput}
 ps3b <- plotSlopes(mod1, plotx = "x1", modx = "x3", modxVals = c("right"))
\end{Sinput}
\end{Schunk}


To estimate the impact of ideology for Democrats and Republicans,
I went through an inconvenient process of making up 3 separate new
data objects, and pulling predictions for each one. Today, I'd make
one newdata object here, but I have to live with the silliness of
the past here:

\begin{Schunk}
\begin{Sinput}
 lnth <- length(nes2002$V023022) 
 newdfD1 <-data.frame(V023022=seq(1,7), repub=0, democ=1)
 newdfD1
\end{Sinput}
\begin{Soutput}
  V023022 repub democ
1       1     0     1
2       2     0     1
3       3     0     1
4       4     0     1
5       5     0     1
6       6     0     1
7       7     0     1
\end{Soutput}
\begin{Sinput}
 predvalsD1 <- predict(bushideoglm2, newdata=newdfD1, type="response")
 newdfR1 <-data.frame(V023022=seq(1,7), repub=1, democ=0)
 newdfR1
\end{Sinput}
\begin{Soutput}
  V023022 repub democ
1       1     1     0
2       2     1     0
3       3     1     0
4       4     1     0
5       5     1     0
6       6     1     0
7       7     1     0
\end{Soutput}
\begin{Sinput}
 predvalsR1 <- predict(bushideoglm2, newdata=newdfR1, type="response")
 newdfI1 <-data.frame(V023022=seq(1,7), repub=0, democ=0)
 predvalsI1 <- predict(bushideoglm2, newdata=newdfI1, type="response")
\end{Sinput}
\end{Schunk}


\begin{figure}
\caption{Democrats and Republicans\label{cap:DandR}}

\begin{Schunk}
\begin{Sinput}
 plot(seq(1,7),predvalsD1,type="n",ylim=c(0,1),xlim=c(0,8),xlab="V023022: ideology", ylab="Vote (Bush=1, Gore=0)") 
 text(seq(1,7),predvalsR1,labels=rep("r",7))
 text(seq(1,7),predvalsD1,labels=rep("d",7))
 text(seq(1,7),predvalsI1,labels=rep("i",7))
 lines(seq(1,7),predvalsR1)
 lines(seq(1,7),predvalsD1)
 lines(seq(1,7),predvalsI1)
\end{Sinput}
\end{Schunk}

\includegraphics{tmpout/t-623}
\end{figure}


\subsection*{Making a better party \& ideology model}

There was no reason to use the dummies repub and democ. I have a factor
variable partyid and I might as well use it. The additive model is:

\input{tmpout/t-ideol200}

\input{tmpout/t-ideol210}

\begin{figure}
\caption{plotCurves for ideology with moderator party identification}

\input{tmpout/t-ideol220}
\includegraphics{tmpout/t-ideol220}
\end{figure}


\subsection*{Should it be treated as an interaction model?}

We assumed party id was an additive element, but perhaps it is interactive. 

There was no reason to use the dummies repub and democ. I have a factor
variable partyid and I might as well use it. The additive model is:

\input{tmpout/t-ideol240}

\begin{figure}[p]
\caption{plotCurves for ideology with interaction\label{fig:ideologyinteraction1}}

\input{tmpout/t-ideol250}
\includegraphics{tmpout/t-ideol250}
\end{figure}


\subsection*{Note the problem of working in the S-shaped predictions}

One problem I have not emphasized much in class is that all glm have
an inherent ``interaction'' effect, even if we don't estimate models
with interaction terms. In logistic regression, that's easy to see
because the S-shaped curve bends everything. Party ID is important,
so much so that it appears that ideology has not much effect on Republicans
or Democrats in Figure \ref{fig:ideologyinteraction1}. 

\input{tmpout/t-ideol260}

\begin{figure}[p]
\caption{plotCurves in the linear predictor space to better inspect interactions}

\input{tmpout/t-ideol300}
\includegraphics{tmpout/t-ideol300}
\end{figure}


\section{The Bush Thermometer variable is interesting. }

This is introduced because it is \emph{as close to a numeric variable
as we can get} in this data. I believe it is actually an ordinal variable
masquerading as a number, but we'll treat it as a number, just for
practice.

The variable in question is
\begin{itemize}
\item V023010: Where on that thermometer would you rate George W. Bush?
(on a scale from 0 to 100) 
\end{itemize}
That one does not need to be recoded. Its histogram is displayed in
Figure \ref{fig:HistBushTherm}.

\begin{figure}

\caption{Histogram of the Bush feeling thermometer\label{fig:HistBushTherm}}

\input{tmpout/t-bushtherm}
\includegraphics{tmpout/t-bushtherm}
\end{figure}

The OLS estimates for the model including only the Bush feeling thermometer
are as follows:

\begin{Schunk}
\begin{Sinput}
 olsthermo1 <- lm(bushvote~V023010,data=nes2002) 
 summary(olsthermo1)
\end{Sinput}
\begin{Soutput}
Call:
lm(formula = bushvote ~ V023010, data = nes2002)

Residuals:
    Min      1Q  Median      3Q     Max 
-0.9629 -0.3299  0.1004  0.2903  1.2779 

Coefficients:
              Estimate Std. Error t value Pr(>|t|)    
(Intercept) -0.3031905  0.0327025  -9.271   <2e-16 ***
V023010      0.0126614  0.0004567  27.726   <2e-16 ***
---
Signif. codes:  0 '***' 0.001 '**' 0.01 '*' 0.05 '.' 0.1 ' ' 1

Residual standard error: 0.3691 on 929 degrees of freedom
  (580 observations deleted due to missingness)
Multiple R-squared:  0.4528,	Adjusted R-squared:  0.4522 
F-statistic: 768.7 on 1 and 929 DF,  p-value: < 2.2e-16
\end{Soutput}
\end{Schunk}


The results from logistic estimates from glm are as follows (see Figure
\ref{cap:LogisticAndOLSglmthermo1}).

\begin{Schunk}
\begin{Sinput}
 numV023010 <- as.numeric(nes2002$V023010)
 glmthermo1 <- glm(bushvote~numV023010,data=nes2002,family=binomial) 
 summary(glmthermo1) 
\end{Sinput}
\begin{Soutput}
Call:
glm(formula = bushvote ~ numV023010, family = binomial, data = nes2002)

Deviance Residuals: 
    Min       1Q   Median       3Q      Max  
-2.4632  -0.6273   0.3141   0.5978   3.4230  

Coefficients:
             Estimate Std. Error z value Pr(>|z|)    
(Intercept) -6.035992   0.416477  -14.49   <2e-16 ***
numV023010   0.090204   0.005713   15.79   <2e-16 ***
---
Signif. codes:  0 '***' 0.001 '**' 0.01 '*' 0.05 '.' 0.1 ' ' 1

(Dispersion parameter for binomial family taken to be 1)

    Null deviance: 1284.91  on 930  degrees of freedom
Residual deviance:  745.54  on 929  degrees of freedom
  (580 observations deleted due to missingness)
AIC: 749.54

Number of Fisher Scoring iterations: 5
\end{Soutput}
\end{Schunk}


A presentable table for these models is provided in Table \ref{tab:ThermoOutreg},
which also includes the probit model for comparison.

\begin{Schunk}
\begin{Sinput}
 numV023010 <- as.numeric(nes2002$V023010)
 glmthermo1 <- glm(bushvote~numV023010,data=nes2002,family=binomial) 
 summary(glmthermo1) 
\end{Sinput}
\begin{Soutput}
Call:
glm(formula = bushvote ~ numV023010, family = binomial, data = nes2002)

Deviance Residuals: 
    Min       1Q   Median       3Q      Max  
-2.4632  -0.6273   0.3141   0.5978   3.4230  

Coefficients:
             Estimate Std. Error z value Pr(>|z|)    
(Intercept) -6.035992   0.416477  -14.49   <2e-16 ***
numV023010   0.090204   0.005713   15.79   <2e-16 ***
---
Signif. codes:  0 '***' 0.001 '**' 0.01 '*' 0.05 '.' 0.1 ' ' 1

(Dispersion parameter for binomial family taken to be 1)

    Null deviance: 1284.91  on 930  degrees of freedom
Residual deviance:  745.54  on 929  degrees of freedom
  (580 observations deleted due to missingness)
AIC: 749.54

Number of Fisher Scoring iterations: 5
\end{Soutput}
\end{Schunk}


\begin{table}

\caption{Bush Thermometer Models\label{tab:ThermoOutreg}}

\input{tmpout/t-6123}

\end{table}

Figure \ref{cap:LogisticAndOLSglmthermo1} illustrates how, in 2003,
I plotted the relationship between the respondent's ``thermometer
scale score'' for Pres Bush and the vote. The OLS and Logistic regression
lines are superimposed on the plot.

\begin{figure}
\caption{Thermometer model: Logistic and OLS Predicted Probabilities\label{cap:LogisticAndOLSglmthermo1}}

\begin{Schunk}
\begin{Sinput}
 glmpred1 <- predict(glmthermo1, newdata=data.frame(numV023010 =seq(0,100)), type="response")
 plot(nes2002$V023010, nes2002$bushvote, ylim=c(-0.2,1.2), xlim=c(0,100), xlab="Bush Thermometer",ylab="Predic6.1.2.5ted Probability of Bush Vote")
 lines(seq(0,100),glmpred1)
 abline(olsthermo1,lty=2)
\end{Sinput}
\end{Schunk}

\includegraphics{tmpout/t-6124}
\end{figure}


\section{The ``Full Model''??}

Consider a ``full'' (garbage can?) model. The full model includes
the Bush thermometer variable(V0234010), ideology (V023027), the 2
party variables,  education (V023131), and the state of the economy
(V023027).

\begin{Schunk}
\begin{Sinput}
 bushideolrm5 <- glm(bushvote ~ V023010+V023022+repub+democ+V023027+ V023131, data=nes2002, family=binomial(link=logit), model=T)
 summary(bushideolrm5)
\end{Sinput}
\begin{Soutput}
Call:
glm(formula = bushvote ~ V023010 + V023022 + repub + democ + 
    V023027 + V023131, family = binomial(link = logit), data = nes2002, 
    model = T)

Deviance Residuals: 
    Min       1Q   Median       3Q      Max  
-2.7853  -0.2821   0.1527   0.3575   3.3384  

Coefficients:
             Estimate Std. Error z value Pr(>|z|)    
(Intercept) -4.813647   0.996164  -4.832 1.35e-06 ***
V023010      0.067983   0.007713   8.814  < 2e-16 ***
V023022      0.354252   0.105927   3.344 0.000825 ***
repub        1.167235   0.328048   3.558 0.000374 ***
democ       -2.692004   0.317822  -8.470  < 2e-16 ***
V023027     -0.042292   0.121325  -0.349 0.727403    
V023131     -0.105637   0.083003  -1.273 0.203130    
---
Signif. codes:  0 '***' 0.001 '**' 0.01 '*' 0.05 '.' 0.1 ' ' 1

(Dispersion parameter for binomial family taken to be 1)

    Null deviance: 1098.89  on 799  degrees of freedom
Residual deviance:  422.07  on 793  degrees of freedom
  (711 observations deleted due to missingness)
AIC: 436.07

Number of Fisher Scoring iterations: 6
\end{Soutput}
\end{Schunk}


Outreg summarizes that as:

\begin{Schunk}
\begin{Sinput}
 predictOMatic(m1mc, predVals = c("educationc"))
\end{Sinput}
\begin{Soutput}
  educationc type      fit
1  -4.415102   bc 26.09849
2  -2.350102   bc 35.93543
3  -0.190102   bc 46.22492
4   1.959898   bc 56.46677
5   5.174898   bc 71.78190
\end{Soutput}
\end{Schunk}


\section{What about using factors as predictors?\protect\footnote{This is the Ian Ostrander memorial section}}


\subsection{Education}

I expected that education, V023131, would not have an effect, and
only after carelessly throwing it into the model did I remember that
it should be treated as a factor variable. 

If we treat it as an unordered factor variable, we can get estimates
of one parameter for each of the 6 different levels.

\begin{Schunk}
\begin{Sinput}
 nes2002$educFactor <- as.factor(nes2002$V023131)
 bushideolrm6 <- glm(bushvote ~ V023010+ V023022+ repub+democ+ V023027+ educFactor, data=nes2002, family=binomial(link=logit))
 summary(bushideolrm6)
\end{Sinput}
\begin{Soutput}
Call:
glm(formula = bushvote ~ V023010 + V023022 + repub + democ + 
    V023027 + educFactor, family = binomial(link = logit), data = nes2002)

Deviance Residuals: 
    Min       1Q   Median       3Q      Max  
-2.7569  -0.2847   0.1399   0.3475   3.3258  

Coefficients:
             Estimate Std. Error z value Pr(>|z|)    
(Intercept) -7.048866   1.394368  -5.055  4.3e-07 ***
V023010      0.068760   0.007817   8.796  < 2e-16 ***
V023022      0.368200   0.108234   3.402 0.000669 ***
repub        1.172697   0.333312   3.518 0.000434 ***
democ       -2.792379   0.330018  -8.461  < 2e-16 ***
V023027     -0.082868   0.125623  -0.660 0.509473    
educFactor2  1.393612   1.349233   1.033 0.301655    
educFactor3  2.176177   1.107965   1.964 0.049516 *  
educFactor4  2.459526   1.117383   2.201 0.027725 *  
educFactor5  1.516711   1.141784   1.328 0.184056    
educFactor6  1.308992   1.098129   1.192 0.233253    
educFactor7  1.872912   1.119012   1.674 0.094186 .  
---
Signif. codes:  0 '***' 0.001 '**' 0.01 '*' 0.05 '.' 0.1 ' ' 1

(Dispersion parameter for binomial family taken to be 1)

    Null deviance: 1098.89  on 799  degrees of freedom
Residual deviance:  409.45  on 788  degrees of freedom
  (711 observations deleted due to missingness)
AIC: 433.45

Number of Fisher Scoring iterations: 6
\end{Soutput}
\end{Schunk}


What does this show? Should we conclude that education plays a role,
or not? A couple of categories ``seem to matter'', but what is the
bottom line? 

If we state the null hypothesis that ``all education coefficients
are 0'' and let the alternative be ``at least one of the education
coefficients is not 0,'' then we should conduct a Likelihood Ratio
test comparing the two models. The built-in function drop1 will do
all of the work:

\begin{Schunk}
\begin{Sinput}
 drop1(bushideolrm6,test="Chisq")
\end{Sinput}
\begin{Soutput}
Single term deletions

Model:
bushvote ~ V023010 + V023022 + repub + democ + V023027 + educFactor
           Df Deviance    AIC     LRT  Pr(>Chi)    
<none>          409.45 433.45                      
V023010     1   523.06 545.06 113.613 < 2.2e-16 ***
V023022     1   421.16 443.16  11.712 0.0006210 ***
repub       1   422.54 444.54  13.091 0.0002967 ***
democ       1   502.44 524.44  92.986 < 2.2e-16 ***
V023027     1   409.89 431.89   0.438 0.5079183    
educFactor  6   423.71 435.71  14.256 0.0269040 *  
---
Signif. codes:  0 '***' 0.001 '**' 0.01 '*' 0.05 '.' 0.1 ' ' 1
\end{Soutput}
\end{Schunk}


Note that the 'drop1' groups together all of the levels of the factor
variable and it addresses the question ``does this variable make
a difference \emph{at all}.'' In other words, it is conducting the
Likelihood Ratio test that you really want if you are deciding whether
education plays a role in predicting voting. This test says that it
does.

The effect is not linear, as is clearly indicated by the predicted
values.

\begin{Schunk}
\begin{Sinput}
 termplot(bushideolrm6,terms="educFactor")
\end{Sinput}
\end{Schunk}

\includegraphics{tmpout/t-821}

This indicates that ``education matters,'' and so the first model
that was estimated, the one that assumed the effect of education was
a linear 1-2-3 step sequence, is probably wrong. Compare the ``residual
deviance'' values of the two models. What do you see? We can formally
test that conclusion, however. Conduct a likelihood ratio test, comparing
the ``full'' model that uses the factor coding of education with
the ``reduced'' model which assumes that the different levels have
the same effects. Note how ironic it is that the restricted model
in this comparison is the one I called the full model in a previous
section.

\begin{Schunk}
\begin{Sinput}
 wildGuess <- lm(xsq ~ x, data = dat) 
 wildGuessResid <- resid(wildGuess)
\end{Sinput}
\end{Schunk}


\subsection{Ideology}

I found myself wondering if the ideology variable should also be given
the ``factor'' consideration. The effect data is as follows.

\begin{Schunk}
\begin{Sinput}
 nes2002$ideolFactor <- as.factor(nes2002$V023022)
 bushideolrm7 <- glm(bushvote ~ V023010 + ideolFactor + repub + democ + V023027 + educFactor, data=nes2002,family=binomial(link=logit))
\end{Sinput}
\end{Schunk}


A formal test that compares the ``factor coding'' against the ``linear
coding'' is as follows:

\begin{Schunk}
\begin{Sinput}
 cor(wildGuessResid, xpoly[,2])
\end{Sinput}
\begin{Soutput}
[1] 1
\end{Soutput}
\end{Schunk}

significant, so we probably ought to go ahead and use the ``linear
coding''. 

Would you stop with that conclusion? You'd wonder why the lowest level
of ideology appears to have such a strange spike. Look back at the
descriptive table for the ideology variable. Notice that there are
12 respondents, only 12, who have the lowest score, which is 1. The
anova test indicates that the aberrant estimate for that group is
not ``statistically out of line,'' at least not far enough out of
line to make us think we need to worry about treating ideology as
a factor.

\section{Compare Logit and Probit links}

\subsection{Estimating a Probit model: glm with binomial(probit)}

With R's glm, the distinction between logit and probit is found in
the ``link'' function, the transformation that is applied to the
left hand side to go from ``probability'' into the scale of the
linear predictor. The Logistic model was estimated with the command

\begin{Sinput}
bushideoglm1 <- glm(bushvote~V023022,family=binomial,data=nes2002) 
\end{Sinput}

To fit a probit model, one in which the random error is normal, we
make only one small change in the command.

\begin{tabular}{*{2}{l}}
\hline
                  & M1   \tabularnewline
                   &Estimate \tabularnewline
                 &(S.E.) \tabularnewline
 \hline
 \hline
   regionC         &-0.030 \tabularnewline
                 &(0.040)  \tabularnewline
   regionM         &0.287** \tabularnewline
                 &(0.099)  \tabularnewline
   regionN         &0.136* \tabularnewline
                 &(0.055)  \tabularnewline
   regionS         &0.165*** \tabularnewline
                 &(0.037)  \tabularnewline
   regionSA        &-0.180*** \tabularnewline
                 &(0.032)  \tabularnewline
 \hline
 N                 &2683       \tabularnewline
 RMSE             &0.989   \tabularnewline
 $R^2$             &0.024   \tabularnewline
 adj $R^2$         &0.022   \tabularnewline
 \hline
\hline
 
 \multicolumn{2}{c}{${*  p}\le 0.05$${*\!\!*  p}\le 0.01$${*\!\!*\!\!*  p}\le 0.001$}\tabularnewline
 \end{tabular}

\subsection{OLS, Logit and Probit Side by Side}

\subsubsection*{Ideology as the only predictor.}

\begin{Schunk}
\begin{Sinput}
 newdf$gdplog <- log(newdf$gdp)
\end{Sinput}
\end{Schunk}


\subsubsection*{The Full Model fitted with all three methods}

Compare the OLS, Logit, and Probit fits to the larger model.

\begin{Schunk}
\begin{Sinput}
 bushideoProbit5 <- glm(bushvote ~ V023010 + V023022 + repub + democ + V023027 + V023131, data=nes2002, family=binomial(link=probit), model=T)
 summary(bushideoProbit5)
\end{Sinput}
\begin{Soutput}
Call:
glm(formula = bushvote ~ V023010 + V023022 + repub + democ + 
    V023027 + V023131, family = binomial(link = probit), data = nes2002, 
    model = T)

Deviance Residuals: 
    Min       1Q   Median       3Q      Max  
-2.8579  -0.3064   0.1186   0.3717   3.5602  

Coefficients:
             Estimate Std. Error z value Pr(>|z|)    
(Intercept) -2.459054   0.523643  -4.696 2.65e-06 ***
V023010      0.036153   0.003957   9.138  < 2e-16 ***
V023022      0.198724   0.056489   3.518 0.000435 ***
repub        0.626346   0.172427   3.633 0.000281 ***
democ       -1.503168   0.168354  -8.929  < 2e-16 ***
V023027     -0.016886   0.064345  -0.262 0.792995    
V023131     -0.081039   0.044730  -1.812 0.070023 .  
---
Signif. codes:  0 '***' 0.001 '**' 0.01 '*' 0.05 '.' 0.1 ' ' 1

(Dispersion parameter for binomial family taken to be 1)

    Null deviance: 1098.89  on 799  degrees of freedom
Residual deviance:  425.93  on 793  degrees of freedom
  (711 observations deleted due to missingness)
AIC: 439.93

Number of Fisher Scoring iterations: 6
\end{Soutput}
\end{Schunk}



Outreg summarizes that as:

\begin{Schunk}
\begin{Sinput}
 rockchalk::summarize(myx) ## or summary(myx)
 hist(myx)
 dev.new(height=3, width=8) ## fiddle values
 hist(myx)
 hist(myx, breaks = seq(min(myx), max(myx), length = 5))
 hist(myx, breaks = c(min(myx), 0, 40, 50, max(myx)))
 hist(myx, xlim = c(-200, 200))
 hist(myx, xlim = c(-200, 200), ylim = c(0, 350))
 text(-200, 300, pos = 4, labels = c("If you typed all this \nin I think you are the \nmost dedicated and well \nadjusted person who ever \ntook my course"))
\end{Sinput}
\end{Schunk}


The ratio of the logit to the probit estimates depends on the assumed
value of standard deviation of the error term (as that term arises
in the CDF interpretation of the logit and probit models). That standard
deviation cannot be estimated, it has to be set by the user to some
arbitrary value, and all of the coefficients adjust in response (as
in $\hat{\beta}=b/s.d.(e_{i})$). I extracted the estimated coefficients
from the 2 models and then calculated the ratio of the logit to the
probit estimates. I expected that the ratios would be almost exactly
the same, and that made me suspect I had done something incorrectly.
But I don't see the mistake.

\begin{Schunk}
\begin{Sinput}
 logitcoef <- coef(bushideolrm5)
 probitcoef <- coef(bushideoProbit5)
 cbind(logitcoef, probitcoef, ratio=logitcoef/probitcoef)
\end{Sinput}
\begin{Soutput}
              logitcoef  probitcoef    ratio
(Intercept) -4.81364708 -2.45905355 1.957520
V023010      0.06798295  0.03615298 1.880424
V023022      0.35425202  0.19872397 1.782634
repub        1.16723481  0.62634619 1.863562
democ       -2.69200372 -1.50316795 1.790887
V023027     -0.04229175 -0.01688572 2.504587
V023131     -0.10563718 -0.08103940 1.303529
\end{Soutput}
\end{Schunk}


Lets compare the ratio of b/se(b) and the p values to see what the
differences might be:

\begin{Schunk}
\begin{Sinput}
 options(digits=3)
 logitsummary <- summary(bushideolrm5)
 probitsummary <- summary(bushideoProbit5)
 zcompare <- cbind(logitsummary$coefficients[, c(3,4)], probitsummary$coefficients[, c(3,4)])
 zcompare <- as.data.frame(zcompare)
 colnames(zcompare) <- c("logit-z","logit-p", "probit-z", "probit-p")
 zcompare$ratioz <- zcompare[ ,1] / zcompare[ ,3]
 zcompare
\end{Sinput}
\begin{Soutput}
            logit-z  logit-p probit-z probit-p ratioz
(Intercept)  -4.832 1.35e-06   -4.696 2.65e-06  1.029
V023010       8.814 1.21e-18    9.138 6.39e-20  0.965
V023022       3.344 8.25e-04    3.518 4.35e-04  0.951
repub         3.558 3.74e-04    3.633 2.81e-04  0.980
democ        -8.470 2.45e-17   -8.929 4.31e-19  0.949
V023027      -0.349 7.27e-01   -0.262 7.93e-01  1.328
V023131      -1.273 2.03e-01   -1.812 7.00e-02  0.702
\end{Soutput}
\end{Schunk}


In practice, I have never fitted a model for dichotomous data for
which I felt that the choice of the logit or probit link made a substantial
difference. The differences here in the estimated ratios of $\hat{\beta}/se(\hat{\beta})$
are not trivial, but they also are not so great as to make us change
our conclusions. 

Finally, lets get the predicted values from the models and plot them
against each other.

\begin{Schunk}
\begin{Sinput}
 logitpred <- predict(bushideolrm5, type="response")
 probitpred <- predict(bushideoProbit5, type="response")
 plot(logitpred, probitpred, xlab="Predicted Probabilities from Logit Model", ylab="Predicted Probabilties from Probit Model", main="Scatterplot of Predictions from Logit and Probit")
 pmod1 <- lm(probitpred ~ logitpred)
 summary(pmod1)
\end{Sinput}
\begin{Soutput}
Call:
lm(formula = probitpred ~ logitpred)

Residuals:
     Min       1Q   Median       3Q      Max 
-0.03754 -0.01067  0.00039  0.00623  0.05724 

Coefficients:
            Estimate Std. Error t value Pr(>|t|)    
(Intercept) 0.009604   0.000842    11.4   <2e-16 ***
logitpred   0.988270   0.001219   810.9   <2e-16 ***
---
Signif. codes:  0 '***' 0.001 '**' 0.01 '*' 0.05 '.' 0.1 ' ' 1

Residual standard error: 0.0141 on 798 degrees of freedom
Multiple R-squared:  0.999,	Adjusted R-squared:  0.999 
F-statistic: 6.58e+05 on 1 and 798 DF,  p-value: <2e-16
\end{Soutput}
\begin{Sinput}
 abline(pmod1, col="red", lwd=2)
 legend("topleft", legend=c("OLS regression"), col="red", lwd=2)
\end{Sinput}
\end{Schunk}

\includegraphics{tmpout/t-130}

\bibliographystyle{apalike2}
\phantomsection\addcontentsline{toc}{section}{\refname}\bibliography{theme/R}


\section*{Replication Information\label{sec:Session-Info}}

Please leave this next code chunk if you are producing a guide document. 

\begin{Schunk}
\begin{Sinput}
 sessionInfo()
\end{Sinput}
\begin{Soutput}
R version 3.5.1 (2018-07-02)
Platform: x86_64-pc-linux-gnu (64-bit)
Running under: Ubuntu 18.10

Matrix products: default
BLAS: /usr/lib/x86_64-linux-gnu/blas/libblas.so.3.8.0
LAPACK: /usr/lib/x86_64-linux-gnu/lapack/liblapack.so.3.8.0

locale:
 [1] LC_CTYPE=en_US.UTF-8       LC_NUMERIC=C              
 [3] LC_TIME=en_US.UTF-8        LC_COLLATE=en_US.UTF-8    
 [5] LC_MONETARY=en_US.UTF-8    LC_MESSAGES=en_US.UTF-8   
 [7] LC_PAPER=en_US.UTF-8       LC_NAME=C                 
 [9] LC_ADDRESS=C               LC_TELEPHONE=C            
[11] LC_MEASUREMENT=en_US.UTF-8 LC_IDENTIFICATION=C       

attached base packages:
[1] stats     graphics  grDevices utils     datasets  methods   base     

other attached packages:
[1] foreign_0.8-71      pscl_1.5.2          lmtest_0.9-36      
[4] zoo_1.8-2           MASS_7.3-51         faraway_1.0.7      
[7] rockchalk_1.8.128   stationery_0.98.5.5

loaded via a namespace (and not attached):
 [1] zip_1.0.0       Rcpp_0.12.17    compiler_3.5.1  nloptr_1.0.4   
 [5] plyr_1.8.4      tools_3.5.1     digest_0.6.15   lme4_1.1-17    
 [9] evaluate_0.10.1 nlme_3.1-137    lattice_0.20-35 openxlsx_4.1.0 
[13] Matrix_1.2-14   pbivnorm_0.6.0  stringr_1.3.1   knitr_1.20     
[17] stats4_3.5.1    rprojroot_1.3-2 grid_3.5.1      rmarkdown_1.10 
[21] lavaan_0.6-1    carData_3.0-1   minqa_1.2.4     magrittr_1.5   
[25] backports_1.1.2 htmltools_0.3.6 kutils_1.51     splines_3.5.1  
[29] mnormt_1.5-5    xtable_1.8-2    stringi_1.2.3  
\end{Soutput}
\end{Schunk}


\end{document}
