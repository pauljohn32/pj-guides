%% LyX 2.3.1 created this file.  For more info, see http://www.lyx.org/.
%% Do not edit unless you really know what you are doing.
\documentclass[english,a0,portrait]{a0poster}
\usepackage{mathpazo}
\usepackage{berasans}
\usepackage{beramono}
\usepackage[T1]{fontenc}
\usepackage[latin1]{inputenc}
\usepackage{geometry}
\geometry{verbose,tmargin=1cm,bmargin=1cm,lmargin=1cm,rmargin=1cm}
\pagestyle{empty}
\setcounter{secnumdepth}{0}
\setcounter{tocdepth}{0}
\setlength{\parskip}{\bigskipamount}
\setlength{\parindent}{0pt}
\usepackage{color}
\definecolor{page_backgroundcolor}{rgb}{0.917969, 0.890625, 0.921875}
\pagecolor{page_backgroundcolor}
\usepackage{array}
\usepackage{multicol}
\usepackage{multirow}
\usepackage{tcolorbox}
\usepackage{setspace}
\onehalfspacing

\makeatletter

%%%%%%%%%%%%%%%%%%%%%%%%%%%%%% LyX specific LaTeX commands.
\providecommand{\LyX}{L\kern-.1667em\lower.25em\hbox{Y}\kern-.125emX\@}
\newcommand{\noun}[1]{\textsc{#1}}
\DeclareRobustCommand*{\lyxarrow}{%
\@ifstar
{\leavevmode\,$\triangleleft$\,\allowbreak}
{\leavevmode\,$\triangleright$\,\allowbreak}}
%% Because html converters don't know tabularnewline
\providecommand{\tabularnewline}{\\}

%%%%%%%%%%%%%%%%%%%%%%%%%%%%%% Textclass specific LaTeX commands.
\providecommand*{\code}[1]{\texttt{#1}}
\providecommand*{\strong}[1]{\textbf{#1}}

%%%%%%%%%%%%%%%%%%%%%%%%%%%%%% User specified LaTeX commands.
\usepackage{xcolor}

%KU color pallette
%primary
\definecolor{kublue}{RGB}{0,81,186}
\definecolor{kucrimson}{RGB}{232,0,13}
\definecolor{kujayhawkyellow}{RGB}{255,200,45}
\definecolor{kusignaturegrey}{RGB}{133,137,138}
%secondary
\definecolor{kunight}{RGB}{0,52,89}
\definecolor{kulake}{RGB}{39,103,255}
\definecolor{kusky}{RGB}{115,203,242}
\definecolor{kubrick}{RGB}{151,27,47}
\definecolor{kufire}{RGB}{255,48,66}
\definecolor{kuwheat}{RGB}{242,169,0}
\definecolor{kufog}{RGB}{142,159,188}
\definecolor{kusteam}{RGB}{221,229,237}
\definecolor{kuterracotta}{RGB}{192,110,78}
\definecolor{kulimestone}{RGB}{215,210,203}


\usepackage{tcolorbox}
% tcolorbox skins support
\tcbuselibrary{skins}

% define the box color
\colorlet{myboxcolor}{kublue} 

% For \arrayrulecolor
\usepackage{colortbl}

\usepackage{xcolor}
\definecolor{light-gray}{gray}{0.90}
\usepackage{realboxes}
\providecommand*{\code}[1]{\texttt{#1}}
\renewcommand{\code}[1]{%
%\Colorbox{light-gray}{#1}%
\Colorbox{kulimestone}{#1}%
}%

\makeatother

\usepackage{babel}
\usepackage{listings}
\renewcommand{\lstlistingname}{Listing}

\begin{document}
\tcbset{colframe=myboxcolor, fonttitle=\Large\bfseries, boxsep=5mm}

\tcbox[colback=myboxcolor, coltext=white]{

\begin{tabular}{>{\raggedright}m{0.15\linewidth}>{\centering}p{0.68\linewidth}>{\raggedright}m{0.15\linewidth}}
\multirow{2}{0.15\linewidth}{{[}Logo?{]}} & \textbf{{\VERYHuge{}\textbf{Poster Heading is 3 col tabular}}} & \multirow{2}{0.15\linewidth}{\centering{}~\hfill{}{[}Logo?{]}}\tabularnewline
 & {\veryHuge{}tcolorbox and colortbl \par} & \tabularnewline
\end{tabular}}
\begin{center}
\begin{tcolorbox}[title=A Page Wide Color Box]
In this document, we have logical markup. Here is \emph{emphasis},
\noun{noun}, \code{computer code}, and \strong{strong}. The computer
code background is just barely darker than the poster background.
Maybe we should think that over, maybe it is OK/subtle. Lets insert
an equation to make sure it works
\begin{equation}
y_{i}=\beta_{0}+\beta_{1}x1_{i}+\beta_{2}x2_{i}+\varepsilon_{i}\label{eq:one}
\end{equation}

As we see in equation (\ref{eq:one}), all is well.

\begin{lstlisting}[numbers=left,numberstyle={\tiny},basicstyle={\ttfamily},breaklines=true,tabsize=2]
A program listing using the LaTeX listing environment. Seems OK! It is set in typewriter font, standard for code listings. 
There are tiny line numbers on left. The preamble does not have any listings details,  so each individual listings object will have to add any desired decorations.
\end{lstlisting}
\end{tcolorbox}
\par\end{center}

\vspace*{-42pt}
\begin{minipage}[t]{0.49\columnwidth}%
\begin{tcolorbox}
Left text

with several paragraphs

with several paragraphs
\end{tcolorbox}
%
\end{minipage}\hfill{}%
\begin{minipage}[t]{0.49\columnwidth}%
\begin{tcolorbox}
Right text

with several paragraphs

with several paragraphs
\end{tcolorbox}
%
\end{minipage}

\begin{tcolorbox}[title=Multiple Columns with the ``Multiple Columns'' Module]

\setlength{\columnsep}{2.2cm}

\begin{multicols}{4}

Wide blocks on a poster usually have too long lines (more than 60
letters per line is difficult to read).

One way to solve this is using columns. To this end, you can use the
``Multiple Columns'' module, as done here. It provides the ``Columns''
inset this text is written in. The space betweThis is adapted from
the LyX template a0poster-simple. It is a plain poster based on LaTeX
boxes (minipages and parboxes). I've made the background the official
blue of KU, and within the boxes, the color is a light gray. It is
the simplest approach, the least likely to fail. The more elaborate
types of boxes are shown in the companion document, a0poster-colorbox.en
the columns can be adjusted via the \LaTeX{} code \texttt{\textbackslash setlength\{\textbackslash columnsep\}}
(see above in the \LyX{} work area). The ``Multiple Columns'' module
is documented in detail in \textsf{Help\lyxarrow Additional Features}.
Here's a manual column break: \columnbreak

Lorem ipsum dolor sit amet, consetetur sadipscing elitr, sed diam
nonumy eirmod tempor invidunt ut labore et dolore magna aliquyam erat,
sed diam voluptua. At vero eos et accusam et justo duo dolores et
ea rebum. Stet clita kasd gubergren, no sea takimata sanctus est Lorem
ipsum dolor sit amet. Lorem ipsum dolor sit amet, consetetur sadipscing
elitr, sed diam nonumy eirmod tempor invidunt ut labore et dolore
magna aliquyam erat, sed diam voluptua. At vero eos et accusam et
justo duo dolores et ea rebum. Stet clita kasd gubergren, no sea takimata
sanctus est Lorem ipsum dolor sit amet. Lorem ipsum dolor sit amet,
consetetur sadipscing elitr, sed diam nonumy eirmod tempor invidunt
ut labore et dolore magna aliquyam erat, sed diam voluptua. At vero
eos et accusam et justo duo dolores et ea rebum. Stet clita kasd gubergren,
no sea takimata sanctus est Lorem ipsum dolor sit amet.

Duis autem vel eum iriure dolor in hendrerit in vulputate velit esse
molestie consequat, vel illum dolore eu feugiat nulla facilisis at
vero eros et accumsan et iusto odio dignissim qui blandit praesent
luptatum zzril delenit augue duis dolore te feugait nulla facilisi.
Lorem ipsum dolor sit amet, consectetuer adipiscing elit, sed diam
nonummy nibh euismod tincidunt ut laoreet dolore magna aliquam erat
volutpat.

Ut wisi enim ad minim veniam, quis nostrud exerci tation ullamcorper
suscipit lobortis nisl ut aliquip ex ea commodo consequat. Duis autem
vel eum iriure dolor in hendrerit in vulputate velit esse molestie
consequat, vel illum dolore eu feugiat nulla facilisis at vero eros
et accumsan et iusto odio dignissim.
\end{multicols}
\end{tcolorbox}

\begin{tcolorbox}[title=Available Font Sizes (with corresponding \LaTeX\ command),
enhanced, watermark text=Font Sizes]

\begin{tabular}{ccc}
{\tiny{}Tiny (}\texttt{\tiny{}\textbackslash tiny}{\tiny{})} & {\scriptsize{}Smallest (}\texttt{\scriptsize{}\textbackslash scriptsize}{\scriptsize{})} & {\footnotesize{}Smaller (}\texttt{\footnotesize{}\textbackslash footnotesize}{\footnotesize{})}\tabularnewline
Normal (\texttt{\textbackslash normalsize}) & {\large{}Large (}\texttt{\large{}\textbackslash large}{\large{})} & {\Large{}Larger (}\texttt{\Large{}\textbackslash Large}{\Large{})}\tabularnewline
{\LARGE{}Largest (}\texttt{\LARGE{}\textbackslash LARGE}{\LARGE{})} & {\huge{}Huge (}\texttt{\huge{}\textbackslash huge}{\huge{})} & {\Huge{}Huger (}\texttt{\Huge{}\textbackslash Huge}{\Huge{})}\tabularnewline
{\veryHuge{}Giant (\texttt{\textbackslash veryHuge})\par} & \multicolumn{2}{c}{{\VERYHuge{}Gigantic (\textbackslash VERYHuge)\par}}\tabularnewline
 &  & \tabularnewline
\end{tabular}
\end{tcolorbox}

\begin{tcolorbox}[title=Boxes with Two Levels, collower=myboxcolor]

By means of the \textsf{Color Box Separator} paragraph style, a color
box can be split in an upper \ldots{}

\tcblower{}

\ldots{} and a lower part. This can only done once per box. The \textsf{Color
Box Line} paragraph style, which just draws the separation line without
actually separating the box, can be used repeatedly.
\end{tcolorbox}

\begin{center}
\tcbox[title=Look: tcbox enclosing a tabular with colored dividers!, left=0mm,
right=0mm, top=0mm, bottom=0mm, boxsep=0mm, toptitle=5mm, lefttitle=3mm,
bottomtitle=5mm]{\arrayrulecolor{myboxcolor}\renewcommand{\arraystretch}{1.2}%

\Large%
\begin{tabular}{c|c|c|c|c|c|c|c|c|c|c|c}
One  & Two  & Three  & Four & Five & Six & Seven & Eight & Nine & Ten & Eleven & Twelve\tabularnewline
\hline 
\hline 
A  & B  & C  & D & E & F & G & H & I & J & K & L\tabularnewline
\hline 
I  & II & III  & IV & V & VI & VII & VIII & IX & X & XI & XII\tabularnewline
\end{tabular}}
\par\end{center}

\begin{tcolorbox}[title=Themes, skin=beamer]

Color boxes can be themed (``skinned'').

\tcblower{}

This is the ``beamer'' skin.
\end{tcolorbox}

\begin{tcolorbox}[title=Tweaking the Boxes, leftrule=1mm, colback=black!50!kublue, colframe=green,
arc=7mm, boxsep=1cm, height=10cm, valign=center, enhanced, interior
style={left color=myboxcolor!04, right color=myboxcolor!14}, sharp
corners=downhill, frame style={right color=kublue, left color=black!90},
coltext=kujayhawkyellow, fontupper=\Large]
\textcolor{black}{Color box appearance can be customized in many
different ways.}\textsf{\textcolor{white}{\hfill{}}}Please refer
to the \textsf{tcolorbox} manual for details. Find out why text changes
color here (seriously). 
\end{tcolorbox}

\vfill{}

\begin{center}
\begin{tcolorbox}[colback=myboxcolor, coltext=white]

\begin{center}
Poster Footer Can Have Anything You Like Here
\par\end{center}
\end{tcolorbox}
\par\end{center}
\end{document}
