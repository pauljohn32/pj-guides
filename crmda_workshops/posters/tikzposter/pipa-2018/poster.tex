\documentclass[25pt,a0paper,portrait,margin=0pt]{tikzposter}

%\usepackage{latexsym}
%\usepackage{amsmath}
%\usepackage{amssymb}
\usepackage{tabularx}
\usepackage{multicol}
\usepackage{graphicx}
\usepackage{amssymb,amstext,amsmath}
%\usepackage{tikz}
\newcommand{\itGamma}{ {\varGamma} }                                    % italic \Gamma
%\newcommand{\itGamma}{ {\mathit{\Gamma}} }                             % staro italic \Gamma
\newcommand{\mitGamma}{ {\itGamma} }
\newcommand{\bra}[1]{{\langle{#1}\vert}}
\newcommand{\ket}[1]{{\vert{#1}\rangle}}
\newcommand{\bracket}[2]{\langle #1 \vert #2 \rangle}
\newcommand{\Tr}{\mathop{\rm Tr}\nolimits}
\newcommand{\realni}{\ensuremath{\mathbb{R}}}
\newcommand{\tabindent}{\hphantom{em}}
\newcommand{\ds}{\displaystyle}

\newcommand{\cC}{{\cal C}}
\newcommand{\cD}{{\cal D}}
\newcommand{\cH}{{\cal H}}
\newcommand{\cK}{{\cal K}}
\newcommand{\cL}{{\cal L}}
\newcommand{\cM}{{\cal M}}
\newcommand{\cN}{{\cal N}}
\newcommand{\cO}{{\cal O}}

\newcommand{\nablaEnt}{{\pmb{\nabla}}}
\newcommand{\GammaEnt}[2]{ \pmb{\mitGamma}^{#1}{}_{#2} }
\newcommand{\gEnt}{{\pmb{g}}}
\newcommand{\TEnt}{{\pmb{T}}}
\newcommand{\PsiEnt}{{\pmb{\Psi}}}
\newcommand{\barg}{h}
\newcommand{\orto}{\bot}


\newcommand{\barT}{t}

\usepackage{amssymb,amstext,amsmath,bm}
\usepackage{tikz-cd}
\usepackage{cite}


\renewcommand{\Re}{\mathop{\rm Re}\nolimits}
\newcommand{\prirodni}{\ensuremath{\mathbb{N}}}
\newcommand{\del}{\partial}
\newcommand{\nablalr}{\stackrel{\leftrightarrow}{\nabla}\!\!{}}
\newcommand{\nablar}{\stackrel{\rightarrow}{\nabla}\!\!{}}
\newcommand{\nablal}{\stackrel{\leftarrow}{\nabla}\!\!{}}
\newcommand{\diag}{\mathop{\rm diag}\nolimits}



\newcommand{\tst}{\textstyle}

\newcommand{\rmd}{{\rm d}}

                                    % italic \Gamma
%\newcommand{\itGamma}{ {\mathit{\Gamma}} }                             % staro italic \Gamma





%%%%%%%%%%%%%%%%%%%%%%%%%%%%%%%%%%%%%%%%%%%%%%%%%%%%%%%%%%%%%%%%%
%%%%%%%%%%%%%%%%%%%%%%%%%%%%%%%%%%%%%%%%%%%%%%%%%%%%%%%%%%%%%%%%%
%%%%%%%%%%%%%%%%%%%%%%%%%%%%%%%%%%%%%%%%%%%%%%%%%%%%%%%%%%%%%%%%%
%% Definitions for the commands for "entangled" quantities.
%%
%% Note that these are subject to change, should we decide to change the notation.
%%
%% Usage:
%% \GammaEnt has two arguments, the upper and lower index, for example:
%%
%%  \GammaEnt{\lambda}{\mu\nu}.
%%
%% The other commands do not have arguments, and indices and other stuff should be explicitly added as needed, for example:
%%
%%  \gEnt_{mu\nu}, \TEnt^{\mu\nu}, \ket{\PsiEnt}
%%

%\newcommand{\GammaEnt}[2]{ \mitGamma^{\ #1}_{\!\rm ent}{}_{#2} }    %% old definition, obsolete for now

%\newcommand{\GammaEnt}[2]{ \pmb{\mitGamma}^{#1}{}_{#2} }
%\newcommand{\gEnt}{\pmb{g}}
%\newcommand{\DeltagEnt}{{\pmb{\Delta g}}}
%\newcommand{\TEnt}{{\pmb{T}}}
%\newcommand{\DeltaTEnt}{{\pmb{\Delta T}}}
%\newcommand{\PsiEnt}{{\pmb{\Psi}}}
%\newcommand{\nablaEnt}{{\pmb{\pmb{\nabla}}}\!}
%\newcommand{\BEnt}{{\pmb{B}}}
%% And another kludge command that may change in the future:

\newcommand{\DeltagEnt}{{\bm{\Delta g}}}
\newcommand{\DeltaTEnt}{{\bm{\Delta T}}}
\newcommand{\mEnt}{{\bm{m}}}
%% And another kludge command that may change in the future:


\title{

\begin{tabular}{c}
The effects of entanglement between gravity and matter \\ on the motion of localized massive particles\
\end{tabular}
}

\author{

\begin{tabular}{lcr}

\raisebox{-2cm}{\parbox[b]{8.4cm}{\includegraphics[height=2.5cm]{IT-amblem.jpg}  $\vphantom{\int}$ \includegraphics[height=2.5cm]{IST-amblem.png}}}
 &

\hspace*{4cm}
Francisco Pipa$^{1,2}$

\hspace*{4cm}
Nikola Paunkovi\'c$^{3}$

\hspace*{4cm}
Marko Vojinovi\'c$^{4}$

\hspace*{4cm}
 &

\raisebox{-2cm}{\includegraphics[height=5.4cm]{GPF-amblem.png}} \\
\end{tabular}
}

\institute{

\ \vspace*{-0.5cm} \\

\begin{tabular}{rcl}
\normalsize $^1$ Department of Philosophy, University of Kansas & \hspace*{2cm} &
\normalsize  $^3$ Instituto de Telecomunica\~coes, Departamento de Matem\'atica and CeFEMA, IST, Universidade de Lisboa\\[-0.8cm]
\normalsize $^2$ Departamento de de F\'isica, IST, Universidade de Lisboa & &
\normalsize $^4$ Group for Gravitation, Particles and Fields (GPF), Institute of Physics, University of Belgrade \\
\normalsize FP acknowledges the financial support of the Department of Philosophy at the University of Kansas & \hspace*{2cm} &
\normalsize  NP acknowledges the financial support of the IT Research Unit, ref. UID/EEA/50008/2013\\[-0.8cm]
\normalsize  & &
\normalsize and the IT project QbigD funded by FCT PEst-OE/EEI/LA0008/2013
\end{tabular}
}

\usetheme{Autumn}

\definecolor{nekaboja}{RGB}{255, 250, 250}
\definecolor{drugaboja}{RGB}{125, 171, 114}
\definecolor{crvena}{RGB}{250, 95, 95}

\colorlet{notebgcolor}{nekaboja}

\begin{document}

\maketitle

\block{Abstract}{
\begin{center}
\parbox[t]{65cm}{We derive an effective equation of motion for a pointlike particle in the framework of quantum gravity from simple basic assumptions \cite{fra:1}. The geodesic motion of a classical particle can be deduced by coupling a classical field theory to general relativity. We use a similar method to obtain an effective equation of motion, starting from an abstract quantum gravity description. We find that entanglement between gravity and matter leads to modifications of the geodesic trajectory, mainly because of nonzero overlap terms between gravity-matter coherent states. Lastly, we discuss a possible violation of the weak equivalence principle due to the nongeodesic motion. We show that in the Newtonian limit, the acceleration of the particle depends on its mass and the inertial and gravitational masses are not equal.
\vspace*{1cm}
}
\end{center}
}

%\begin{columns}
%
%% Argument komande \column je ili duzina tipa 25cm, ili procenat sirine strane,
%% tipa 0.3 = 30% ukupne sirine. Pazi da ne predjes 100%. ;-)
%
%\column{0.5}
%
%\block{Weak Equivalence Principle (WEP)}{The local effects of particle motion in a gravitational field are indistinguishable from those of an accelerated observer in flat spacetime.}
%%\block{Naslov bloka 2}{Tekst bloka 2.}
%
%\column{0.5}
%
%\block{Consequence}{A particle in a gravitational field should follow the geodesic, since this is how the straight line in flat space looks like from the accelerated frame.}
%%\block{Naslov bloka 4}{Tekst bloka 4.}
%
%\end{columns}

\begin{columns}

% Argument komande \column je ili duzina tipa 25cm, ili procenat sirine strane,
% tipa 0.3 = 30% ukupne sirine. Pazi da ne predjes 100%. ;-)

\column{0.5}



\block{Derivation of the geodesic equation \\ from general relativity}{
Single pole approximation:
\begin{equation}
\huge
\label{eq:single-pole}
	T^{\mu\nu}(x) = \int_{\cC} d\tau \, B^{\mu\nu}(\tau) \frac{\delta^{(4)}(x-z(\tau))}{\sqrt{-g}}
\end{equation}

Assuming the local Poincar\'e invariance for {\em both}  $S_G[g]$ and $S_M[g,\phi]$:
{\huge
\begin{equation}
\label{eq:noether}
	\nabla_{\nu} T^{\mu\nu} = 0
\end{equation}
}
Substituting~\eqref{eq:single-pole} into~\eqref{eq:noether}, we obtain the geodesic equation, with $u^\lambda \nabla_\lambda \equiv \nabla$, $u^{\mu} \equiv \frac{dz^{\mu}(\tau)}{d\tau}$ and $u^{\mu}u_{\mu} \equiv -1$ (Mathisson and Papapetrou \cite{mat:37,pap:51}; see also~\cite{vas:voj:07})\medskip:
{\huge
\begin{equation*} \label{EquationForZ}
\nabla u^{\mu} = 0
\end{equation*}
}%\\
%or, using Cristoffel symbols	, $\frac{d^2 z^{\lambda}(\tau)}{d\tau^2} + \mitGamma^{\lambda}{}_{\mu\nu} \frac{d z^{\mu}(\tau)}{d\tau} \frac{d z^{\nu}(\tau)}{d\tau} = 0\,.$
}
\block{Quantising gravity}{Given the fundamental gravitational degrees of freedom $\hat{g}$ and $\hat{\pi}_g$: \huge 
\begin{equation*}
\Delta \hat{g} \Delta \hat{\pi}_g \geq \frac{\hbar}{2} \qquad
\Delta \hat{\phi} \Delta \hat{\pi}_{\phi} \geq \frac{\hbar}{2}
\end{equation*}

\begin{equation*} \label{SeparableState}
\ket{\Psi} = \ket{g} \otimes \ket{\phi}
\end{equation*}

\begin{equation*}
g_{\mu\nu} \equiv \bra{\Psi} \hat{g}_{\mu\nu} \ket{\Psi} \qquad T_{\mu\nu} \equiv \bra{\Psi} \hat{T}_{\mu\nu} \ket{\Psi}
\end{equation*}
}

\note[targetoffsetx=8cm,
      targetoffsety=-1cm,
      connection,
      angle=0,
      radius=5cm,
      width=10cm]
{$\ket g$ and $\ket\phi$ -- coherent states of gravity and matter.} 


\column{0.5}

\block{Deviation of the geodesic motion due to entanglement}{\huge
\begin{equation*}
\label{eq:ent_st_1}
\ket{\PsiEnt} = \kappa \ket{\Psi} + \eta \ket{\Psi^{\orto}}
\end{equation*}

\begin{equation*}
\gEnt_{\mu\nu} = g_{\mu\nu} + \eta \, \barg_{\mu\nu} + \cO(\eta^2)\,%,\qquad 
%\TEnt_{\mu\nu} = T_{\mu\nu} + \beta \, \bar{T}_{\mu\nu} + \cO(\beta^2)\,.
\end{equation*}
\begin{equation*}
\TEnt_{\mu\nu} = T_{\mu\nu} + \eta \, \barT_{\mu\nu} + \cO(\eta^2)\,    
\end{equation*}

\begin{equation*}
\barg_{\mu\nu}=2 \Re \left( \kappa \bra{\Psi^{\orto}} \hat{g}_{\mu\nu} \ket{\Psi} \vphantom{\hat{T}} \right)+\cO(\eta) 
\end{equation*}
\begin{equation*} \label{StressEnergyOverlap}
\barT_{\mu\nu}=2 \Re \left( \kappa \bra{\Psi^{\orto}} \hat{T}_{\mu\nu} \ket{\Psi} \right)+\cO(\eta)
\end{equation*}




%\begin{equation}
%\frac{d^2 z^{\mu}(\tau)}{d\tau^2} + \left[ \mitGamma^{\mu}{}_{\rho\nu} +  \beta(\bar{\mitGamma}^{\mu}{}_{\rho\nu} - \check{\mitGamma}^{\mu}{}_{\rho\nu})\right]\frac{d z^{\rho}(\tau)}{d\tau} \frac{d z^{\nu}(\tau)}{d\tau} + \cO(\beta^2) = 0\,.	
%\end{equation}
{\large From the quantum analogue of (2), $\nablaEnt_{\nu} \TEnt^{\mu\nu} = 0$, we get:}

{\large An effective mass parameter $m(\tau)$,}
\begin{equation*} \label{MassNonConservation}
(B+\eta\bar{B})  u^{\mu} u^{\nu} \equiv m(\tau) u^{\mu} u^{\nu}
\end{equation*}

\hfill \break
{\large The proper time evolution of the mass parameter,}
\begin{equation*} \label{MassNonConservation}
\nabla m = \eta m u^{\sigma} \left( u^{\nu} u_{\lambda} F^{\lambda}{}_{\nu\sigma} - F^{\nu}{}_{\nu\sigma}\right)\,,
\end{equation*}
{\large The effective equation of motion of the
particle (modified geodesic trajectory),}
\begin{equation*}
\nabla u^{\mu}+\eta u^{\nu}u^{\sigma}  F_{\orto}^{\mu}{}_{\nu\sigma}=0
\end{equation*}
}

\note[targetoffsetx=-12cm,
      targetoffsety=8.1cm,
      connection,
      angle=130,
      radius=5cm,
      width=9cm]
{``Entangled'' metric and stress-energy: 
\begin{equation*}
\gEnt_{\mu\nu} = \bra{\PsiEnt}\hat{g}_{\mu\nu}\ket{\PsiEnt}
\end{equation*}
\begin{equation*}
\TEnt_{\mu\nu} = \bra{\PsiEnt}\hat{T}_{\mu\nu}\ket{\PsiEnt}
\end{equation*}
 }

\note[targetoffsetx=8.5cm,
      targetoffsety=17.5cm,
      connection,
      angle=0,
      radius=5cm,
      width=8.8cm]
{Orthogonal state.
$|\kappa|^2 = 1-\eta^2$, with $\eta$ small
 }
 
 \note[targetoffsetx=8cm,
      targetoffsety=-8.8cm,
      connection,
      angle=0,
      radius=5cm,
      width=15cm]
{
  \begin{equation*} \label{DefinitionOfTheTbarScale}
\barT^{\mu\nu} = \int_{\cC} d\tau \, \bar{B}^{\mu\nu}(\tau) \frac{\delta^{(4)}(x-z(\tau))}{\sqrt{-g}} 
\end{equation*}
} 
 
 
\note[targetoffsetx=10cm,
      targetoffsety=-20.5cm,
      connection,
      angle=0,
      radius=5cm,
      width=12cm]
{
\begin{equation*}
F_{\orto}^{\mu}{}_{\nu\sigma} = P_{\orto}^{\mu}{}_{\lambda}F^{\lambda}{}_{\nu\sigma}.
\end{equation*}
\begin{equation*}
F^{\lambda}{}_{\nu\sigma} \equiv \nabla_{(\sigma} \barg^{\lambda}{}_{\nu)} -\frac{1}{2} \nabla^{\lambda} \barg_{\nu\sigma}
\end{equation*}
\begin{equation*}
A_{(\sigma\nu)} \equiv \frac{1}{2} \left( A_{\sigma\nu} + A_{\nu\sigma} \right)
\end{equation*}

 }
 
\note[targetoffsetx=-15cm,
      targetoffsety=-34.5cm,
      connection,
      angle=0,
      radius=0cm,
      width=18cm]
{
\begin{equation*}
\barg^i{}_i & = & \ds 2 \delta^{ij} \Re \left( \kappa \bracket{\phi}{\tilde{\phi}} \frac{\epsilon_G}{\epsilon}\bra{g^{\orto}} \hat{g}_{ij} \ket{g} \right) + \cO(\eta)
\end{equation*}

It depends on the overlap of the matter fields, $\bracket{\phi}{\tilde{\phi}}$.
 }
 
\end{columns}




\block{Equivalence principle and quantum theory      \hphantom{mmm} Newtonian limit of the effective equation}
{

\begin{multicols}{2}[\columnsep2em] 
\includegraphics[width=0.6\linewidth]{diagrama}
\columnbreak
\begin{equation*}
\mbox{The acceleration depends on mass and the inertial and gravitational masses are not equal:}
\end{equation*}
\begin{equation*}
m_I \frac{d^2 z^k}{d\tau^2} = - m_I \left( 1-\frac{1}{3} \eta \barg^i{}_i \right) \frac{G M}{r^3} z^k - \eta m_I \left[ \del_0 \barg_{0k} - \frac{1}{2} \del_k \barg_{00} - \frac{GM}{r^3} z^j \tilde{\barg}_{jk} \right] \\
\end{equation*}{}
\begin{equation*}
    \frac{m_G}{m_I} \equiv \left( 1-\frac{1}{3} \eta \barg^i{}_i \right)
\end{equation*}

\end{multicols}



}




\block{Conclusions and discussion}{\vspace{-1.2cm}
\parbox[t]{48cm}{
\begin{itemize}
\item We derived an effective equation of motion within an abstract framework of quantum gravity, in particular in the case where both matter and gravity are in a quantum superposition of macroscopic states. This equation contains a non-geodesic term, giving rise to an effective force acting on the particle, as a consequence  of overlap terms.
\item We found a violation of the weak equivalence principle relative to the dominant metric $g_{\mu \nu}$, in the context of field theory and deriving a point-particle mechanics as a consequence of this theory, working in a fully general-relativistic regime, and making use of proper relativistic observables (formally quantized).
\item It would be interesting to estimate the magnitude of the nongeodesic term to compare different quantum gravity models. We would need to specify the concrete quantum gravity models. Then, use the equation of motion or the equation that relates the gravitational and inertial mass.


 
\end{itemize}
}
\vspace*{1cm}
}

{\colorlet{notebgcolor}{white}
\note[targetoffsetx=3cm,
      targetoffsety=-5.6cm,
      connection,
      angle=0,
      radius=0cm,
      width=25cm
      ]
{
\begin{center}
\large \textbf{\textsl{RQI-N 2018, 24-27 September, Vienna}}
\end{center}
}
}
{\colorlet{notebgcolor}{white}
\note[targetoffsetx=25cm,
      targetoffsety=2.cm,
      angle=0,
      radius=0cm,
      width=30.3cm]
{
\begin{thebibliography}{99}
\bibitem{fra:1} F. Pipa, N. Paunkovi\'c and M. Vojinovi\'c, (2018), arXiv:1801.03207
\bibitem{mat:37}  M. Mathisson, {\em Acta Phys. Polon.} 6, 163 (1937)
\bibitem{pap:51} A. Papapetrou, {\em Proc. R. Soc.} A 209, 248 (1951)
%\bibitem{voj:08} M. Vojinovi\'c, PhD Thesis (2008), http://www.markovojinovic.com/professional/ \\ pdf/PhD-thesis.pdf %{\em Motion of extended objects in gravitational field background,}
\bibitem{vas:voj:07} M. Vasili\'c and M. Vojinovi\'c, JHEP , 0707, 028 (2007), arXiv:0707.3395
\end{thebibliography}
}
}

\end{document}

\endinput