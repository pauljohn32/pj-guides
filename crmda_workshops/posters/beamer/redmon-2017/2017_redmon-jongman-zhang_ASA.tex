\documentclass[serif, mathserif, final, xcolor=table]{beamer}
\usetheme{KU1}
\usepackage{amsmath,amsfonts,amssymb,pxfonts,bm,multirow,
            booktabs,eulervm,xspace,epstopdf,tipa, tabularx}
\usepackage{graphicx}
\usepackage[orientation=landscape, size=custom, 
            width=122, height=92, scale=1.22]{beamerposter}

\newcommand{\colsep}{\hspace{1.5em}}

\usepackage{hyperref}
\hypersetup{
    colorlinks=true,
    linkcolor=blue,
    filecolor=magenta,      
    urlcolor=blue,
}

\setbeamertemplate{itemize item}{\raisebox{0.35ex}{\footnotesize $\blacktriangleright$}}
\setbeamertemplate{itemize subitem}{\raisebox{0.35ex}{\tiny $\blacktriangleright$}}

\renewcommand{\arraystretch}{1.2}
\newcommand{\ssref}[1]{\textsuperscript{\color{blue}#1}}

%-- Header information 
\title{Distributional factors in Telugu sibilant production}
\author{Charles Redmon,~ Allard Jongman,~ and Jie Zhang}
\institute{Department of Linguistics,~ University of Kansas; Lawrence, KS, USA}


%-- Header information
\newcommand{\footleft}{http://redmonc.github.io/dravidian}
\newcommand{\footcenter}{174th Meeting of the Acoustical Society of America, New Orleans, December 2017}
\newcommand{\footright}{redmon@ku.edu}


%-- Main Document 
\begin{document}
\begin{frame}[t]{}
\vspace{-1cm}
  \begin{columns}[t]

    %-- Column 1 ---------------------------------------------------
    \begin{column}{0.24\linewidth}

      \begin{block}{background}
        \small
        \begin{itemize}
          \item Telugu is a Dravidian language spoken in South India
          \item Unlike many languages in the region which lost the three-way distinction between alveolar, palatal, and retroflex sibilants present in Sanskrit, Telugu purportedly preserves the contrast\ssref{1-4}
          \item Such dense systems are typologically rare and have been shown (e.g., in Polish and Mandarin) to be acoustically unstable\ssref{5-7}
        \end{itemize}

      \end{block}

      \begin{block}{goal of the study}
      \small
We seek to characterize the acoustics of the sibilant contrast system in Telugu, information which is largely absent from the literature.
      \end{block}

	\vspace{-1.25cm}

      \begin{block}{participants}
        \small
        \begin{itemize}
          \item 16 native speakers of Telugu (8 female, 8 male) recorded in Hyderabad at the English and Foreign Languages University
          \item 14/16 from Telangana (8 of whom were from Hyderabad)
        \end{itemize}
      \end{block}

      \begin{block}{materials}
        \small
        \begin{itemize}
          \item 240 stimuli (120 words $\times$ 2 reps)
            \begin{itemize}
              \item 3 sibilant fricatives (alveolar, retroflex, palatal)
              \item 60 word-initial (CV), 60 word-medial/final (VC)
              \item Critical vowel contexts: 12 /a/, 2 each of /i, e, o, u/
              \item Half of the /a/-context items have 2nd-order neighbors (near-minimal pairs) contrasting in sibilant place; half do not
            \end{itemize}
          \item \textbf{We focus in this presentation on studying the contrast in the /aCa/ context, because (1) it is the most common environment in which all three sibilants occur, and (2) word-initial retroflex sibilants are largely limited to English loanwords}
        \end{itemize}
      \end{block}

      \begin{block}{sibilant spectra}
        \small
        The following are sample spectra from Speaker F01, where the dotted 
palatal line illustrates the occasional alveolar-like realization observed 
in many speakers' data.

        \bigskip

        \begin{center}
          \includegraphics[width=\linewidth]{figures/spectra1.pdf}
        \end{center}
      \end{block}

      \begin{block}{measurements}
        
        \begin{itemize}
          \item Noise amplitude (RMS)
          \item Spectral peak frequency (PeakF)
          \item Spectral tilt below (LFT) and above PeakF (HFT)
          \item Spectral moments at consonant midpoint (M1--M4)
          \item F2 and F3 transitions (modeled with coefficients of quadratic polynomial fits to VC/CV transitions; for simplicity the table in the next panel shows F2/F3 at vowel midpoint and offset/onset)
        \end{itemize}
      \end{block}

    \end{column}


    %-- Column 2/3 (wide column spanning cols 2 and 3)--------------

    \begin{column}{0.72\linewidth}
      \vspace{-1.4cm}
      \begin{columns}[t]
        \begin{column}{0.69\linewidth}

          % -- Block 2-1
          \begin{block}{acoustic features}
            \begin{columns}
              \begin{column}{0.35\linewidth}
                \bigskip
                \begin{tabular}{c}
                  \quad Consonantal Parameters (aCa) \\
                  \includegraphics[width=\linewidth]{figures/paramBP1.pdf} \\
                  \includegraphics[width=\linewidth]{figures/paramBP2.pdf} \\
                \end{tabular}
              \end{column}
              \begin{column}{0.35\linewidth}

                \vspace{0.6cm}

                \begin{tabular}{c}
                  \quad\quad Coarticulatory Information \\
                  \includegraphics[width=0.95\linewidth]{figures/aTransitions1.pdf} \\[0.5cm]
                  \includegraphics[width=0.95\linewidth]{figures/locusEq1.pdf} \\
                \end{tabular}
              \end{column}
              \begin{column}{0.3\linewidth}

                Univariate tests of place effects

                \bigskip

                \small
                \begin{tabular}{lr|rrr}
                  \toprule
                        & \multicolumn{1}{c}{} & \multicolumn{3}{c}{\textbf{Sibilant Means}} \\
                  \cmidrule(lr){3-5}
                  \textbf{Param.} & \multicolumn{1}{r}{$\boldsymbol{\Delta LL_{Sib.}}$} & \textbf{alv.} & \textbf{ret.} & \textbf{pal.} \\
                  \midrule
                  M1 (Hz)    & 477*\textsuperscript{$\dagger$} & 7047  & 4450 & 4594 \\
                  M3         & 329*\textsuperscript{$\dagger$} & -1.11 & 0.81 & 0.68 \\
                  PeakF (Hz) & 266*\phantom{*} & 6969  & 3743 & 3868 \\
                  F2\textsubscript{CV} (Hz)  & 80*\phantom{*} & 1513 & 1702 & 1678 \\
                  F2\textsubscript{VC} (Hz) & 80*\phantom{*} & 1539 & 1757 & 1752 \\
                  HFT        & 27*\phantom{*}  & -4e-3 & -2e-3 & -2e-3 \\
                  F2\textsubscript{V1} (Hz) & 27*\phantom{*} & 1405 & 1332 & 1416 \\
                  RMS (dB)   & 26*\phantom{*}  & 54.5  & 56.4 & 56.2 \\
                  F3\textsubscript{V2} (Hz)  & 13*\phantom{*} & 2877 & 2759 & 2791 \\
                  F2\textsubscript{V2} (Hz) & 9*\phantom{*} & 1434 & 1452 & 1394 \\
                  F3\textsubscript{V1} (Hz) & 9*\phantom{*} & 2892 & 2788 & 2807 \\
                  M2 (Hz)    & 5*\phantom{*} & 1835 & 1732 & 1771 \\
                  M4         & 4*\phantom{*} & 2.96 & 1.69 & 1.79 \\
                  F3\textsubscript{VC} (Hz) & 3*\phantom{*} & 2981 & 2920 & 2926 \\
                  LFT        & 2\phantom{**} & 3e-3 & 4e-3 & 4e-3 \\
                  F3\textsubscript{CV} (Hz) & 2\phantom{**} & 2923 & 2888 & 2874 \\
                  \bottomrule
                  \end{tabular}

                  \bigskip

                  {\footnotesize
                   \textbf{Model:} Linear mixed effects regression with 
                   Speaker as random intercept \\
                    *significant omnibus effect of sibilant place \\
                   \textsuperscript{$\dagger$}all pairwise differences significant}
                  \bigskip
              \end{column}

            \end{columns}

          \end{block}
       
        \end{column}

        \begin{column}{0.001\linewidth}        
        \end{column}

        \begin{column}{0.29\linewidth}

          \begin{block}{contrast separation}
            \begin{tabular}{c}
              \hspace{1.4cm}\includegraphics[width=0.85\linewidth]{figures/densities1.pdf} \\[0.2cm]
              \hspace{-1.5cm}\includegraphics[width=0.97\linewidth]{figures/mds1.pdf} \\
            \end{tabular}
          \end{block}

        \end{column}

      \end{columns}
		\vspace{-1cm}
      \begin{block}{pattern of palatal sibilant misclassifications (\%) by speaker in the \MakeLowercase{a}C\MakeLowercase{a} context}
        \large
        \begin{tabular}{rrrrrrrrrrrrrrrrr}
                    & \colsep F01 & \colsep F02 & \colsep F03 & \colsep F04 & \colsep F05 & \colsep F06 & \colsep F07 & \colsep F08 & \colsep M01 & \colsep M02 & \colsep M03 & \colsep M04 & \colsep M05 & \colsep M06 & \colsep M07 & \colsep M08 \\
          Alveolar  &        18.6 &        0    &        12.0 &        0    &         1.7 &        0    &        0    &        0    &        0    &        0    &        0    &        0    &        0    &        0    &        15.3 &        13.8 \\
          Retroflex &        31.3 &        38.3 &        31.9 &        22.9 &        28.5 &        14.1 &        17.3 &        23.5 &        24.6 &        49.2 &        38.1 &        35.8 &        29.3 &        33.5 &        42.2 &        23.6 \\
        \end{tabular}
      \end{block}
      \vspace{-2.5cm}
      % -- Column 2 ---------------------------------------------------
      \begin{columns}[t]

        \begin{column}{0.485\linewidth}

          % -- Block 2-1
          \begin{block}{Classification results}
            \hspace{-0.9cm}\textbf{Structure of the classification model:}
            \begin{itemize}
              \item Multinomial logistic regression on the three sibilants in the aCa context
              \item 20 predictors (RMS, PeakF, LFT, HFT, M1--M4, VC/CV F2 and F3 transition coefficients), all z-score normalized by speaker
            \end{itemize}
            
            \begin{columns}
              \begin{column}{0.45\linewidth}
                \textbf{Model patterns in the aCa environment:}
                \begin{itemize}                
                  \item Palatal--retroflex model confusions predominate
                  \item Model confusions between alveolar and retroflex categories are rare
                \end{itemize}
              \end{column}
              \begin{column}{0.55\linewidth}
                
                \bigskip
                \bigskip

              \Large
            \begin{tabular}{lrrr}
              \toprule
            & \colsep alv. & \colsep ret. & \colsep pal. \\
            \midrule
            alv.    & 96.1 & 1.2  & 2.7  \\
            ret.    & 0.4  & 69.2 & 30.3 \\
            pal.    & 3.8  & 30.1 & 66.2 \\
            \bottomrule
            \end{tabular}
              \end{column}
            \end{columns}

            \bigskip
            \bigskip
            
            \textbf{Effects of lexical characteristics:}
            \begin{itemize}
              \item Model accuracy was significantly higher on items with sibilant-contrast neighbors ($e^\beta = 1.386$, $z = 10.74$, $p < 0.001$), controlling for lexical frequency and neighborhood density
              \item Lexical frequency had a significant negative effect ($e^\beta = 0.89$, $z = -13.09$, $p < 0.001$), meaning lower frequency words were associated with higher model accuracy in distinguishing sibilant place of articulation
            \end{itemize}

            \bigskip

          \end{block}
       
        \end{column}

        % -- empty separator column -------------------------------------

        \begin{column}{0.01\linewidth}        

        \end{column}

        % -- Column 3 ---------------------------------------------------
        \begin{column}{0.485\linewidth}

          % -- Block 3.1
          \begin{block}{Discussion}
            \begin{itemize}
              \item The present data, combined with the general sparsity of minimal pairs in the Telugu lexicon,\ssref{8} point toward a sibilant system which is more reliably comprised of two categories than three
                \begin{itemize}
                  \item Notably, following the recording many speakers indicated that while they were taught three distinct pronunciations in school, they are only able to perceive or produce two
                  \item Speakers also have an awareness of which dialects are more or less likely to show the palatal $\rightarrow$ alveolar alternation
                \end{itemize}
              \item Further examination of item-specific patterns is needed to account for the lexical variability in palatal similarity to alveolars and retroflexes
            \end{itemize}
            
          \end{block}

          % -- Block 3.2
          \begin{block}{References}
            \ssref{1}Krishnamurti, B. (2003);~ 
            \ssref{2}Masica, C. P. (1993);~ 
            \ssref{3}Sjoberg, A. (1962);~ 
            \ssref{4}Bhaskararao, P., \& Ray, A. (2017);~ 
            \ssref{5}Maddieson, I., \& Precoda, K. (1990);~
            \ssref{6}\.{Z}ygis, M., \& Padgett, J. (2010);~
            \ssref{7}Li, M., \& Zhang, J. (2017);~
            \ssref{8}Baker \textit{et al.} (2002)
          \end{block}
			\vspace{-1cm}
          % -- Block 3.3
          \begin{block}{Acknowledgements}
            Thanks to Indranil Dutta and the Phonetics Laboratory at EFLU, Hyderabad for facilitating the recordings, and to the members of the KU Experimental Research Seminar for their helpful feedback.
          \end{block}

        \end{column}%3

      \end{columns}

    \end{column}
    
  \end{columns}


\end{frame}

\end{document}

