%%%%%%%%%%%%%%%%%%%%%%%%%%%%%%%%%%%%%%%%%%%%%%%%%%%%%%%%%%%%%%%%%
% Filename: template.Rnw
%   Author: Paul Johnson
%
%%%%%%%%%%%%%%%%%%%%%%%%%%%%%%%%%%%%%%%%%%%%%%%%%%%%%%%%%%%%%%%%%
%%%%%%%%%%%%%%%%%%%%%%%%%%%%%%%%%%%%%%%%%%%%%%%%%%%%%%%%%%%%%%%%%
\documentclass[10pt,english]{beamer}
\usepackage{lmodern}
\renewcommand{\sfdefault}{lmss}
\renewcommand{\ttdefault}{lmtt}
\usepackage[T1]{fontenc}
\usepackage[latin9]{inputenc}
\usepackage{listings}
\setcounter{secnumdepth}{3}
\setcounter{tocdepth}{3}
\usepackage{url}
\usepackage{graphicx}
\usepackage{babel}
\makeatletter
%%%%%%%%%%%%%%%%%%%%%%%%%%%%%% Textclass specific LaTeX commands.
\usepackage{Sweavel}
 \newenvironment{topcolumns}{\begin{columns}[t]}{\end{columns}}

%%%%%%%%%%%%%%%%%%%%%%%%%%%%%% User specified LaTeX commands.
\usepackage{dcolumn}
\usepackage{booktabs}
\usepackage{multicol}

% use 'handout' to produce handouts
%\documentclass[handout]{beamer}
\usepackage{wasysym}
\usepackage{pgfpages}
\newcommand{\vn}[1]{\mbox{{\it #1}}}\newcommand{\vb}{\vspace{\baselineskip}}\newcommand{\vh}{\vspace{.5\baselineskip}}\newcommand{\vf}{\vspace{\fill}}\newcommand{\splus}{\textsf{S-PLUS}}\newcommand{\R}{\textsf{R}}


\usepackage{graphicx}
\usepackage{listings}
\lstset{tabsize=2, breaklines=true,style=Rstyle}
%\usetheme{Warsaw}
% or ...

%\setbeamercovered{transparent}
% or whatever (possibly just delete it)

\usetheme{Antibes}


%Switch comment character to turn on/off \pause commands given as \pausealt
%\newcommand{\pausealt}{\par }
\newcommand{\pausealt}{\pause}


\usepackage{color}
\definecolor{gray1}{gray}{0.75}

\newlength{\figurewidth}
\figurewidth \textwidth  % This is for rectangular graphs
\newlength{\figurewidthB}
\figurewidthB .7\textwidth  % This is for square graphs


\expandafter\def\expandafter\insertshorttitle\expandafter{%
  \insertshorttitle\hfill\insertframenumber\,/\,\inserttotalframenumber}

%=============================================================================


\title[Regression Methods]
{R Regression Methods}

\subtitle{Interrogate R Output Objects}

\author[Johnson] {Paul E. Johnson}

\institute[University of Kansas]{Center for Research Methods and Data Analysis \\ University of Kansas}

\date[2012]{2012}

\subject{regression}

%====================================

\begin{document}


% In document Latex options:
\fvset{listparameters={\setlength{\topsep}{0em}}}

\def\Sweavesize{\scriptsize}
\def\Rcolor{\color{black}}
\def\Rbackground{\color[gray]{0.95}}

\begin{Schunk}
\begin{Sinput}
 ## Other settings I like
 options(device = pdf)
 options(useFancyQuotes = FALSE) 
 op <- par() 
 pjmar <- c(5.1, 5.1, 1.5, 2.1) 
 options(SweaveHooks=list(fig=function() par(mar=pjmar, ps=12)))
 pdf.options(onefile=F,family="Times",pointsize=12)
\end{Sinput}
\end{Schunk}




\begin{frame}
  \titlepage
\end{frame}



\begin{frame}
\frametitle{Outline}

\tableofcontents{}

\end{frame}

%==================================================

\section{Methods}
\begin{frame}[containsverbatim]
  \frametitle{Methods: Things To Do ``To'' a Regression Object}





\begin{Schunk}
\begin{Sinput}
 bush1 <- glm(pres04 ~ partyid + sex + owngun, data=dat, family=binomial(link=logit))
\end{Sinput}
\end{Schunk}


\begin{description}
  \item [pres04] Kerry,  Bush
  \item [partyid]  Factor with 7 levels, SD $\rightarrow$ SR
  \item [sex]    Male, Female
  \item [owngun] Yes, No
\end{description}
\end{frame}

\begin{frame}[containsverbatim]
  \frametitle{Just for the Record, The Data Preparation Steps Were $\ldots$}

\begin{Schunk}
\begin{Sinput}
 preslev <- levels(dat$pres04)
 dat$pres04[dat$pres04 %in% preslev[3:10]]<- NA
 dat$pres04 <- factor(dat$pres04)
 levels(dat$pres04) <- c("Kerry", "Bush")
 plev <- levels(dat$partyid)
 dat$partyid[dat$partyid %in% plev[8]] <- NA
 dat$partyid <- factor(dat$partyid)
 levels(dat$partyid) <- c("Strong Dem.", "Dem.", "Ind. Near Dem.", "Independent", "Ind. Near Repub.", "Repub.", "Strong Repub.")
 dat$owngun[ dat$owngun == "REFUSED"] <- NA
 levels(dat$sex) <- c("Male","Female")
 dat$owngun <- relevel(dat$owngun, ref="NO")
\end{Sinput}
\end{Schunk}

\end{frame}


\begin{frame}[containsverbatim,allowframebreaks]
  \frametitle{First, Find Out What You Got}


\begin{Schunk}
\begin{Sinput}
 attributes(bush1)
\end{Sinput}
\begin{Soutput}
$names
 [1] "coefficients"      "residuals"        
 [3] "fitted.values"     "effects"          
 [5] "R"                 "rank"             
 [7] "qr"                "family"           
 [9] "linear.predictors" "deviance"         
[11] "aic"               "null.deviance"    
[13] "iter"              "weights"          
[15] "prior.weights"     "df.residual"      
[17] "df.null"           "y"                
[19] "converged"         "boundary"         
[21] "model"             "na.action"        
[23] "call"              "formula"          
[25] "terms"             "data"             
[27] "offset"            "control"          
[29] "method"            "contrasts"        
[31] "xlevels"          

$class
[1] "glm" "lm" 
\end{Soutput}
\end{Schunk}


\end{frame}


\begin{frame}[containsverbatim]
  \frametitle{Understanding attributes}
  \begin{itemize}
  \item If  you see \$, it means you have an S3 object
  \item That means you can just ``take'' values out of the object with
    the dollar sign operator using commands like

\begin{Schunk}
\begin{Sinput}
 bush1$coefficients
\end{Sinput}
\begin{Soutput}
            (Intercept)             partyidDem. 
                 -3.571                   1.910 
  partyidInd. Near Dem.      partyidIndependent 
                  1.456                   3.464 
partyidInd. Near Repub.           partyidRepub. 
                  5.468                   6.031 
   partyidStrong Repub.               sexFemale 
                  7.191                   0.049 
              owngunYES 
                  0.642 
\end{Soutput}
\end{Schunk}

  \end{itemize}
\end{frame}

\begin{frame}[containsverbatim]
  \frametitle{R Core Team Warns against \$ Access}
  \begin{itemize}
  \item A usage like this works

\begin{Schunk}
\begin{Sinput}
 ## plot( ...whatever..., type="n", axes = F)
\end{Sinput}
\end{Schunk}


 \item But it might not work in the future, if the internal contents
   of the glm object were to change

 \item We should instead use the "extractor method"

\begin{Schunk}
\begin{Sinput}
 library(rockchalk)
\end{Sinput}
\end{Schunk}

  \item Challenge: finding/remembering the extractor functions.
  \item Especially difficult because some VERY important extractor
    functions don't show up using usual methods of searching for them
    (AIC, coefficients)
  \end{itemize}
\end{frame}


\begin{frame}[containsverbatim]
  \frametitle{Double-Check the glm Object's Class}

  \begin{itemize}
  \item Ask the object what class it is from

\begin{Schunk}
\begin{Sinput}
 class(bush1)
\end{Sinput}
\begin{Soutput}
[1] "glm" "lm" 
\end{Soutput}
\end{Schunk}


  \end{itemize}
\end{frame}


\begin{frame}[containsverbatim,allowframebreaks]
  \frametitle{Ask R What Methods are declared to apply to a ``glm'' Object}

\begin{Schunk}
\begin{Sinput}
 methods(class = "glm")
\end{Sinput}
\begin{Soutput}
 [1] add1.glm*           anova.glm          
 [3] confint.glm*        cooks.distance.glm*
 [5] deviance.glm*       drop1.glm*         
 [7] effects.glm*        extractAIC.glm*    
 [9] family.glm*         formula.glm*       
[11] influence.glm*      logLik.glm*        
[13] model.frame.glm     nobs.glm*          
[15] predict.glm         print.glm          
[17] residuals.glm       rstandard.glm      
[19] rstudent.glm        summary.glm        
[21] vcov.glm*           weights.glm*       

   Non-visible functions are asterisked
\end{Soutput}
\end{Schunk}


\end{frame}



\begin{frame}[containsverbatim,allowframebreaks]
  \frametitle{Check methods for ``lm'' class}

\begin{Schunk}
\begin{Sinput}
 methods(class = "lm")
\end{Sinput}
\begin{Soutput}
 [1] add1.lm*           alias.lm*         
 [3] anova.lm           case.names.lm*    
 [5] confint.lm*        cooks.distance.lm*
 [7] deviance.lm*       dfbeta.lm*        
 [9] dfbetas.lm*        drop1.lm*         
[11] dummy.coef.lm*     effects.lm*       
[13] extractAIC.lm*     family.lm*        
[15] formula.lm*        hatvalues.lm      
[17] influence.lm*      kappa.lm          
[19] labels.lm*         logLik.lm*        
[21] model.frame.lm     model.matrix.lm   
[23] nobs.lm*           plot.lm           
[25] predict.lm         print.lm          
[27] proj.lm*           qr.lm*            
[29] residuals.lm       rstandard.lm      
[31] rstudent.lm        simulate.lm*      
[33] summary.lm         variable.names.lm*
[35] vcov.lm*          

   Non-visible functions are asterisked
\end{Soutput}
\end{Schunk}


\end{frame}



%%% Commented out 2012-06-06 before brief impromptu presentation
% \begin{frame}[containsverbatim]
%   \frametitle{Do You Wonder How ``They'' Do ``That''?}
%   \begin{itemize}

%     \item At some point, you realize that the help page is not
%       detailed enough.  You may need to see the Actual Code

%     \item Darth said ``Use the Source, Luke!''

%       If you want to know ``what a function does'', the
%       best option is to download the ACTUAL SOURCE CODE and read it!
%     \end{itemize}
% \end{frame}


% \begin{frame}[containsverbatim]
%   \frametitle{Can See Some Code Within an R Session}
%   \begin{itemize}
%   \item In the ``old days'', you could easily see a function's
%       ``code'' by typing its name (i.e., omit the parentheses).

%       Ex: q used to show all of the steps in shutting down.

%     \item Today, in R 2.11, when I type q I see:

% \begin{Schunk}\begin{Soutput}
% > q
% function (save = "default", status = 0, runLast = TRUE)
% .Internal(quit(save, status, runLast))
% <environment: namespace:base>
%   \end{Soutput}
% \end{Schunk}

% \end{itemize}
% \end{frame}

% \begin{frame}[containsverbatim]
%   \frametitle{Some Functions Still Show Their Code}
%   \begin{itemize}

%    \item Some very informative examples. Try:
%       \begin{itemize}
%         \item \texttt{ > lm \#(or stats::lm)}
%         \item \texttt{ > glm \#(or stats::glm)}
%         \item \texttt{ > termplot}
%       \end{itemize}

%     \item Generic method output not so useful. Try:
%       \begin{itemize}
%       \item \texttt{ > predict}
%       \item \texttt{ > plot}
%       \end{itemize}

%     \end{itemize}
%   \end{frame}


\begin{frame}
  \frametitle{Looking Into the Class Hierarchy}
  \begin{itemize}

  \item Functions are always located inside packages.  With R, several
    packages are supplied and are automatically searched for methods.

   \item Read the source code for some of  your favorite functions.

\begin{Schunk}
\begin{Sinput}
 library(car)
 m1 <- lm(statusquo ~ income + age, data = Chile)
\end{Sinput}
\end{Schunk}


   \item For functions in packages that are loaded, typing its name
   (without telling R what package it lives in) will show its contents.
 \end{itemize}
\end{frame}

\begin{frame}
  \frametitle{Functions, Methods and Hidden Methods}
  \begin{itemize}
  \item Methods are ALSO FOUND if we ask for them explicitly
    with their namespace (and two colons)..

\begin{Schunk}
\begin{Sinput}
 library(car)
 m1 <- lm(statusquo ~ income + age, data = Chile)
\end{Sinput}
\end{Schunk}

   Result should be identical to previous code.

 \item Hidden methods: Functions that are not ``exported'' by the package writer remain hidden

 \item functions used by package author, but they don't want create
   confusion by having users access them directly

 \item You can see code for hidden methods if you use three colons.

\begin{Schunk}
\begin{Sinput}
 summary(agam2, dispersion=1/myshape2$alpha)
\end{Sinput}
\begin{Soutput}
Call:
glm(formula = yobs2 ~ I(1/xseq), family = Gamma(link = "inverse"), 
    start = c(2, 4), control = glm.control(maxit = 100))

Deviance Residuals: 
    Min       1Q   Median       3Q      Max  
-2.5708  -0.8150  -0.2532   0.3325   2.5156  

Coefficients:
            Estimate Std. Error z value Pr(>|z|)    
(Intercept)  0.26736    0.03352   7.976 1.51e-15 ***
I(1/xseq)    2.80621    0.34735   8.079 6.54e-16 ***
---
Signif. codes:  0 '***' 0.001 '**' 0.01 '*' 0.05 '.' 0.1 ' ' 1

(Dispersion parameter for Gamma family taken to be 0.6756965)

    Null deviance: 654.43  on 799  degrees of freedom
Residual deviance: 599.09  on 798  degrees of freedom
AIC: 2479.4

Number of Fisher Scoring iterations: 7
\end{Soutput}
\begin{Sinput}
 
\end{Sinput}
\end{Schunk}

\end{itemize}

\end{frame}

% ______________________________

\section{Interrogate Models}


\begin{frame}[containsverbatim,allowframebreaks]
  \frametitle{The First Method Used is usually \texttt{summary()}}

\begin{Schunk}
\begin{Sinput}
 summary(bush1)
\end{Sinput}
\begin{Soutput}
Call:
glm(formula = pres04 ~ partyid + sex + owngun, family = binomial(link = logit), 
    data = dat)

Deviance Residuals: 
   Min      1Q  Median      3Q     Max  
-2.941  -0.488   0.163   0.390   2.683  

Coefficients:
                        Estimate Std. Error z value
(Intercept)              -3.5712     0.3934   -9.08
partyidDem.               1.9103     0.3972    4.81
partyidInd. Near Dem.     1.4559     0.4348    3.35
partyidIndependent        3.4642     0.4105    8.44
partyidInd. Near Repub.   5.4677     0.5073   10.78
partyidRepub.             6.0307     0.4502   13.39
partyidStrong Repub.      7.1908     0.6213   11.57
sexFemale                 0.0488     0.1928    0.25
owngunYES                 0.6424     0.1937    3.32
                        Pr(>|z|)    
(Intercept)              < 2e-16 ***
partyidDem.              1.5e-06 ***
partyidInd. Near Dem.    0.00081 ***
partyidIndependent       < 2e-16 ***
partyidInd. Near Repub.  < 2e-16 ***
partyidRepub.            < 2e-16 ***
partyidStrong Repub.     < 2e-16 ***
sexFemale                0.80006    
owngunYES                0.00091 ***
---
Signif. codes:  0 '***' 0.001 '**' 0.01 '*' 0.05 '.' 0.1 ' ' 1 

(Dispersion parameter for binomial family taken to be 1)

    Null deviance: 1721.9  on 1242  degrees of freedom
Residual deviance:  764.0  on 1234  degrees of freedom
  (3267 observations deleted due to missingness)
AIC: 782

Number of Fisher Scoring iterations: 6
\end{Soutput}
\end{Schunk}


\end{frame}


\begin{frame}[containsverbatim,allowframebreaks]
  \frametitle{Summary Object}

  Create a Summary Object
\begin{Schunk}
\begin{Sinput}
 sb1 <- summary(bush1)
 attributes(sb1)
\end{Sinput}
\begin{Soutput}
$names
 [1] "call"           "terms"          "family"        
 [4] "deviance"       "aic"            "contrasts"     
 [7] "df.residual"    "null.deviance"  "df.null"       
[10] "iter"           "na.action"      "deviance.resid"
[13] "coefficients"   "aliased"        "dispersion"    
[16] "df"             "cov.unscaled"   "cov.scaled"    

$class
[1] "summary.glm"
\end{Soutput}
\end{Schunk}


 My deviance is

\begin{Schunk}
\begin{Sinput}
 sb1$deviance
\end{Sinput}
\begin{Soutput}
[1] 764
\end{Soutput}
\end{Schunk}


\end{frame}


\begin{frame}[containsverbatim,allowframebreaks]
  \frametitle{The coef Enigma}

  \begin{itemize}
  \item \texttt{coef()} is the same as \texttt{coefficients()}

  \item Note the Bizarre Truth:
    \begin{enumerate}
    \item that the ``coef'' function returns
something different when it is applied to a model object
\begin{Schunk}
\begin{Sinput}
 coef(bush1)
\end{Sinput}
\begin{Soutput}
            (Intercept)             partyidDem. 
                 -3.571                   1.910 
  partyidInd. Near Dem.      partyidIndependent 
                  1.456                   3.464 
partyidInd. Near Repub.           partyidRepub. 
                  5.468                   6.031 
   partyidStrong Repub.               sexFemale 
                  7.191                   0.049 
              owngunYES 
                  0.642 
\end{Soutput}
\end{Schunk}


Than is returned from a summary object.

\begin{Schunk}
\begin{Sinput}
 coef(sb1)
\end{Sinput}
\begin{Soutput}
                        Estimate Std. Error z value
(Intercept)               -3.571       0.39   -9.08
partyidDem.                1.910       0.40    4.81
partyidInd. Near Dem.      1.456       0.43    3.35
partyidIndependent         3.464       0.41    8.44
partyidInd. Near Repub.    5.468       0.51   10.78
partyidRepub.              6.031       0.45   13.39
partyidStrong Repub.       7.191       0.62   11.57
sexFemale                  0.049       0.19    0.25
owngunYES                  0.642       0.19    3.32
                        Pr(>|z|)
(Intercept)              1.1e-19
partyidDem.              1.5e-06
partyidInd. Near Dem.    8.1e-04
partyidIndependent       3.2e-17
partyidInd. Near Repub.  4.3e-27
partyidRepub.            6.5e-41
partyidStrong Repub.     5.6e-31
sexFemale                8.0e-01
owngunYES                9.1e-04
\end{Soutput}
\end{Schunk}

\end{enumerate}
\end{itemize}
\end{frame}






\begin{frame}[containsverbatim, allowframebreaks]
  \frametitle{\texttt{anova()}}
  \begin{itemize}
    \item You can apply \texttt{anova()} to just one model
    \item That gives a ``stepwise'' series of comparisons (not very useful)

\begin{Schunk}
\begin{Sinput}
 anova(bush1, test="Chisq")
\end{Sinput}
\begin{Soutput}
Analysis of Deviance Table

Model: binomial, link: logit

Response: pres04

Terms added sequentially (first to last)


        Df Deviance Resid. Df Resid. Dev Pr(>Chi)    
NULL                     1242       1722             
partyid  6      947      1236        775  < 2e-16 ***
sex      1        0      1235        775  0.97862    
owngun   1       11      1234        764  0.00087 ***
---
Signif. codes:  0 '***' 0.001 '**' 0.01 '*' 0.05 '.' 0.1 ' ' 1 
\end{Soutput}
\end{Schunk}

\end{itemize}
\end{frame}

% _____________________________

\begin{frame}
  \frametitle{But anova Very Useful to Compare 2 Models}

  Here's the basic procedure:

  \begin{enumerate}
  \item Fit 1 big model, ``mod1''
  \item Exclude some variables to create a smaller model, ``mod2''
  \item Run \texttt{anova()} to compare:

      anova(mod1, mod2, test=''Chisq'')

   \item If resulting test statistic is far from 0, it means the big
     model really is better and you should keep those variables in there.
   \end{enumerate}

   Quick Reminder:

   \begin{itemize}
   \item In an OLS model, this is would be an F test for the
     hypothesis that the coefficients for omitted parameters are all
     equal to 0.
   \item In a model estimated by maximum likelihood, it is a
     likelihood ratio test with df= number of omitted parameters.
   \end{itemize}
 \end{frame}


% ______________________________

\begin{frame}[containsverbatim,allowframebreaks]
  \frametitle{But there's an anova ``Gotcha''}


\begin{Schunk}
  \begin{Soutput}
> anova(bush0, bush1, test="Chisq")
Error in anova.glmlist(c(list(object), dotargs),
  dispersion = dispersion,  :
  models were not all fitted to the same size of dataset
\end{Soutput}
\end{Schunk}

  What the Heck?
\end{frame}

% _____________________________

\begin{frame}[containsverbatim]
  \frametitle{\texttt{anova()} Gotcha, cont.}
  \begin{itemize}
    \item Explanation: Listwise Deletion of Missing Values causes this.

      Missings cause sample sizes to differ when variables change.

    \item One Solution: Fit both models on same data.
    \begin{enumerate}
      \item Fit the ``big model'' (one with most variables)

\begin{lstlisting}
mod1 <- glm(y~ x1+ x2 + x3 + (more variables), data=dat, family=binomial)
\end{lstlisting}


      \item Fit the ``smaller Model'' with the data extracted from
        the fit of the previous model (model.frame(mod1), extractor for mod1\$model) as the data frame

\begin{lstlisting}
 mod2 <- glm(y~  x3 + (some variables), data=model.frame(mod1), family=binomial)\end{lstlisting}

   \item After that, anova() will work

     \end{enumerate}
   \end{itemize}
\end{frame}

\begin{frame}[containsverbatim]
  \frametitle{Example anova()}
  \begin{itemize}


  \item Here's the big model
\begin{Schunk}
\begin{Sinput}
  bush3 <- glm(pres04 ~ partyid + sex + owngun + race + wrkslf + realinc + polviews , data=dat, family=binomial(link=logit))
\end{Sinput}
\end{Schunk}


 \item Here's the small model
\begin{Schunk}
\begin{Sinput}
  bush4 <- glm(pres04 ~ partyid +  owngun + race + polviews , data=model.frame(bush3), family=binomial(link=logit))
\end{Sinput}
\end{Schunk}


 \end{itemize}
 \end{frame}


%___________________________________


\begin{frame}[containsverbatim]
  \frametitle{\texttt{anova()}: The Big Reveal!}
  \begin{itemize}

  \item anova:
\begin{Schunk}
\begin{Sinput}
  anova(bush3, bush4, test="Chisq")
\end{Sinput}
\begin{Soutput}
Analysis of Deviance Table

Model 1: pres04 ~ partyid + sex + owngun + race + wrkslf + realinc + polviews
Model 2: pres04 ~ partyid + owngun + race + polviews
  Resid. Df Resid. Dev Df Deviance Pr(>Chi)
1      1044        589                     
2      1047        593 -3     -4.1     0.25
\end{Soutput}
\end{Schunk}


  \item Conclusion: the big model is not statistically significantly
    better than the small model
  \item Same as: Can't reject the null hypothesis that $\beta_j$=0 for
    all omitted parameters
 \end{itemize}
\end{frame}


\begin{frame}
  \frametitle{Interesting Use of anova}
  \begin{itemize}
    \item Consider the fit for ``polviews'' in bush3 (recall
      ``extremely liberal'' is the reference category, the intercept)
    \end{itemize}


\begin{tabular}{l|ccccccc}
\hline
label:& lib. & slt. lib. & mod. & sl. con. & con. & extr. con. \tabularnewline
\hline
mle($\hat{\beta}$): & 0.41 & 1.3  & 1.8* & 2.5* & 2.6* & 3.1*\tabularnewline
\hline
se: & 0.88 & 0.83 & 0.79 & 0.83 & 0.84 & 1.2\tabularnewline
\hline
\end{tabular}

* $p \leq 0.05$

\begin{itemize}
\item I wonder: are all ``conservatives'' the same? Do we really
  need separate parameter estimates for those respondents?
\end{itemize}

\end{frame}

%______________________________

\begin{frame}[containsverbatim]
  \frametitle{Use \texttt{anova()} To Test the Recoding}

  \begin{enumerate}
    \item Make a New Variable for the New Coding
\begin{Schunk}
\begin{Sinput}
 dat$newpolv <- dat$polviews
 (levnpv <- levels(dat$newpolv))
\end{Sinput}
\begin{Soutput}
[1] "EXTREMELY LIBERAL"    "LIBERAL"             
[3] "SLIGHTLY LIBERAL"     "MODERATE"            
[5] "SLGHTLY CONSERVATIVE" "CONSERVATIVE"        
[7] "EXTRMLY CONSERVATIVE"
\end{Soutput}
\begin{Sinput}
 dat$newpolv[dat$newpolv %in% levnpv[5:7]] <- levnpv[6]
\end{Sinput}
\end{Schunk}

\end{enumerate}

\begin{itemize}
  \item Effect is to set slight and extreme conservatives into the
    conservative category
  \end{itemize}

\end{frame}

%_________________________________

\begin{frame}[containsverbatim]
  \frametitle{Better Check newpolv}


\begin{Schunk}
\begin{Sinput}
 dat$newpolv <- factor(dat$newpolv)
 table(dat$newpolv)
\end{Sinput}
\begin{Soutput}
EXTREMELY LIBERAL           LIBERAL 
              139               524 
 SLIGHTLY LIBERAL          MODERATE 
              517              1683 
     CONSERVATIVE 
             1470 
\end{Soutput}
\end{Schunk}


\end{frame}

%_________________________________

\begin{frame}[containsverbatim]
  \frametitle{Neat anova thing, cont.}

  \begin{enumerate}
  \item Fit a new regression model, replacing polviews with newpolv


\begin{Schunk}
\begin{Sinput}
 bush5 <- glm(pres04 ~ partyid + sex + owngun + race + wrkslf + realinc + newpolv , data=dat, family=binomial(link=logit))
\end{Sinput}
\end{Schunk}

 \item Use \texttt{anova()} to test:

\begin{Schunk}
\begin{Sinput}
 anova(bush3, bush5, test="Chisq")
\end{Sinput}
\begin{Soutput}
Analysis of Deviance Table

Model 1: pres04 ~ partyid + sex + owngun + race + wrkslf + realinc + polviews
Model 2: pres04 ~ partyid + sex + owngun + race + wrkslf + realinc + newpolv
  Resid. Df Resid. Dev Df Deviance Pr(>Chi)
1      1044        589                     
2      1046        589 -2   -0.431     0.81
\end{Soutput}
\end{Schunk}

\end{enumerate}
\begin{itemize}
\item Apparently, all conservatives really are alike :)
\item A similar test for liberals is left to the reader!
\end{itemize}
\end{frame}


%_______________________________________________


\begin{frame}[containsverbatim]
  \frametitle{\texttt{drop1} Relieves Tedium}

  \begin{itemize}
  \item \texttt{drop1()} repeats the \texttt{anova()} procedure,
    removing each variable one-at-a-time.

\begin{Schunk}
\begin{Sinput}
 drop1(bush3, test="Chisq")
\end{Sinput}
\begin{Soutput}
Single term deletions

Model:
pres04 ~ partyid + sex + owngun + race + wrkslf + realinc + polviews
         Df Deviance AIC LRT Pr(>Chi)    
<none>           589 627                 
partyid   6      951 977 362  < 2e-16 ***
sex       1      589 625   0    0.991    
owngun    1      592 628   4    0.050 .  
race      2      618 652  30  3.6e-07 ***
wrkslf    1      592 628   4    0.054 .  
realinc   1      589 625   0    0.761    
polviews  6      628 654  40  5.7e-07 ***
---
Signif. codes:  0 '***' 0.001 '**' 0.01 '*' 0.05 '.' 0.1 ' ' 1 
\end{Soutput}
\end{Schunk}


   \item Recall ``Chisq'' $\Leftrightarrow$ L.L.R test.
   \end{itemize}
 \end{frame}



\begin{frame}[containsverbatim,allowframebreaks]
  \frametitle{Variance-Covariance Matrix of $\hat{\beta}$}
\begin{Schunk}
\begin{Sinput}
 outreg(list(ols1, bushideoglm1, bushideoProbit1), tight=T, showAIC=T, modelLabels=c("OLS","Logit","Probit"))
\end{Sinput}
\begin{tabular}{*{4}{l}}
\hline
     &  OLS&  Logit&  Probit\tabularnewline
   & Estimate & Estimate & Estimate \tabularnewline
    & (S.E.) & (S.E.) & (S.E.)\tabularnewline
 \hline
 \hline
  (Intercept)    &-0.248*** &-3.952*** &-2.304*** \tabularnewline
     &  (0.045) &  (0.317) &  (0.173)\tabularnewline
  V023022    &0.181*** &0.952*** &0.556*** \tabularnewline
     &  (0.010) &  (0.070) &  (0.038)\tabularnewline
 \hline
 N & 809 & 809 & 809 \tabularnewline
 RMSE             &0.415   &       &       \tabularnewline
 $R^2$             &0.304   &       &       \tabularnewline
 Deviance         &       &833.913   &836.019   \tabularnewline
 $-2LLR (Model \chi^2)$  &  &   276.885*** &   274.778*** \tabularnewline
 AIC & 876.564 & 837.913 & 840.019\tabularnewline
 \hline
\hline
 
 \multicolumn{4}{c}{${*  p}\le 0.05$${*\!\!*  p}\le 0.01$${*\!\!*\!\!*  p}\le 0.001$}\tabularnewline
 \end{tabular}\end{Schunk}


These will match the ``SE'' column in the summary of bush1
\begin{Schunk}
\begin{Sinput}
 wildGuess <- lm(xsq ~ x, data = dat) 
 wildGuessResid <- resid(wildGuess)
\end{Sinput}
\end{Schunk}

\end{frame}



\begin{frame}[containsverbatim]
  \frametitle{Heteroskedasticity-consistent Standard Errors?}

  Variants of the
  Huber-White ``heteroskedasticity-consistent'' (slang: robust)
  covarance matrix are available in ``car'' and ``sandwich''.

  \begin{itemize}

    \item  hccm() in car works for linear models only

    \item vcovHC in the ``sandwich'' package returns a matrix of
      estimates. One should certainly read ?vcovHC and the associated literature.

\begin{Schunk}
\begin{Sinput}
 ps3b <- plotSlopes(mod1, plotx = "x1", modx = "x3", modxVals = c("right"), interval = "confidence")
\end{Sinput}
\end{Schunk}

\end{itemize}
\end{frame}


\begin{frame}[containsverbatim]
  \frametitle{The heteroskedasticity consistent standard errors of the $\hat{\beta}$  are:}

\begin{Schunk}
\begin{Sinput}
 t(sqrt(diag(myvcovHC)))
\end{Sinput}
\begin{Soutput}
     (Intercept) partyidDem.
[1,]      0.4013      0.3988
     partyidInd. Near Dem. partyidIndependent
[1,]                0.4394             0.4158
     partyidInd. Near Repub. partyidRepub.
[1,]                  0.5079        0.4535
     partyidStrong Repub. sexFemale owngunYES
[1,]               0.6262    0.1946    0.1941
\end{Soutput}
\end{Schunk}

\end{frame}



\begin{frame}[containsverbatim,allowframebreaks]
  \frametitle{Compare those:}

\begin{columns}
  \column{3cm}

The HC and ordinary standard errors are almost identical:
  \column{8cm}
\begin{Schunk}
\begin{Sinput}
 dat$fh_pr_mc <- drop(scale(dat$fh_pr, scale = FALSE))
 newdf$fh_pr_mc <- plotSeq(dat$fh_pr_mc, 25)
\end{Sinput}
\end{Schunk}

\includegraphics{plots/t-038}
\end{columns}

\end{frame}

\begin{frame}[containsverbatim, allowframebreaks]
  \frametitle{Multicollinearity Diagnostics}

  \begin{itemize}
    \item VIF (Variance Inflation Factors) available in ``car''
    \item rockchalk has ``mcDiagnose''
\begin{Schunk}
\begin{Sinput}
  library(tweedie)
\end{Sinput}
\end{Schunk}

\end{itemize}
\end{frame}

%___________________________________

\begin{frame}[containsverbatim]
\frametitle{plot.lm (plot.glm) produces Diagnostics}

  Run plot() on the model object for a quick diagnostic analysis.

  Example:
\begin{Schunk}
\begin{Sinput}
 dat$xlog <- log(dat$x)
 dat$ylog <- log(dat$y)
 m1 <- lm(ylog ~ xlog, data = dat)
 head(model.matrix(m1))
\end{Sinput}
\end{Schunk}



%%<<echo=T,include=F>>=


\end{frame}

\begin{frame}
  \frametitle{Here's a Scatterplot with OLS Fit}

\includegraphics[width=8cm]{plots/t-olsabline10}
\end{frame}


\begin{frame}
  \frametitle{Output from plot(myolsmod)}

\includegraphics[width=9cm]{plots/t-diag09}

\end{frame}


\begin{frame}[plain]
  \frametitle{Output from plot.glm Difficult To Read}

\includegraphics[width=9cm]{plots/t-diag10}

\end{frame}




\begin{frame}[containsverbatim,allowframebreaks]
   \frametitle{\texttt{influence()} Function Digs up the Diagnostics}

\begin{Schunk}
\begin{Sinput}
 ib1 <- influence(bush1)
 head(ib1$hat)
\end{Sinput}
\begin{Soutput}
       1        4        5        9       10 
0.003941 0.003941 0.004117 0.003941 0.005226 
      11 
0.005226 
\end{Soutput}
\begin{Sinput}
 head(ib1$coefficients)
\end{Sinput}
\begin{Soutput}
   (Intercept) partyidDem. partyidInd. Near Dem.
1   -0.0052361    0.005286             0.0052149
4   -0.0052361    0.005286             0.0052149
5   -0.0059698    0.005023             0.0051036
9   -0.0052361    0.005286             0.0052149
10  -0.0005007    0.019143             0.0007462
11   0.0001594   -0.006095            -0.0002376
   partyidIndependent partyidInd. Near Repub.
1           0.0052232               0.0053054
4           0.0052232               0.0053054
5           0.0051290               0.0052763
9           0.0052232               0.0053054
10          0.0006130              -0.0007269
11         -0.0001952               0.0002315
   partyidRepub. partyidStrong Repub.  sexFemale
1      0.0053094            5.274e-03 -0.0004822
4      0.0053094            5.274e-03 -0.0004822
5      0.0052130            5.165e-03  0.0009737
9      0.0053094            5.274e-03 -0.0004822
10    -0.0008014           -2.216e-04  0.0080812
11     0.0002552            7.056e-05 -0.0025732
   owngunYES
1   0.000635
4   0.000635
5   0.000730
9   0.000635
10 -0.010400
11  0.003312
\end{Soutput}
\begin{Sinput}
 head(ib1$sigma)
\end{Sinput}
\begin{Soutput}
     1      4      5      9     10     11 
0.7871 0.7871 0.7871 0.7871 0.7853 0.7870 
\end{Soutput}
\begin{Sinput}
 head(ib1$dev.res)
\end{Sinput}
\begin{Soutput}
      1       4       5       9      10      11 
-0.2413 -0.2413 -0.2355 -0.2413  1.8942 -0.6031 
\end{Soutput}
\begin{Sinput}
 head(ib1$pear.res)
\end{Sinput}
\begin{Soutput}
      1       4       5       9      10      11 
-0.1718 -0.1718 -0.1677 -0.1718  2.2390 -0.4466 
\end{Soutput}
\end{Schunk}

\end{frame}

\begin{frame}[containsverbatim,allowframebreaks]
 \frametitle{\texttt{influence.measures()} A bigger collection of influence measures}

 From influence.measures, DFBETAS for each parameter, DFFITS, covariance ratios, Cook's distances and the diagonal elements of the hat matrix.


\begin{Schunk}
\begin{Sinput}
 imb1 <- influence.measures(bush1)
 attributes(imb1)
\end{Sinput}
\begin{Soutput}
$names
[1] "infmat" "is.inf" "call"  

$class
[1] "infl"
\end{Soutput}
\begin{Sinput}
 colnames(imb1$infmat)
\end{Sinput}
\begin{Soutput}
 [1] "dfb.1_"   "dfb.prD." "dfb.pIND" "dfb.prtI"
 [5] "dfb.pINR" "dfb.prR." "dfb.pSR." "dfb.sxFm"
 [9] "dfb.oYES" "dffit"    "cov.r"    "cook.d"  
[13] "hat"     
\end{Soutput}
\begin{Sinput}
 head(imb1$infmat)
\end{Sinput}
\begin{Soutput}
      dfb.1_ dfb.prD.   dfb.pIND   dfb.prtI
1  -0.016910  0.01691  0.0152357  0.0161655
4  -0.016910  0.01691  0.0152357  0.0161655
5  -0.019279  0.01607  0.0149105  0.0158739
9  -0.016910  0.01691  0.0152357  0.0161655
10 -0.001621  0.06137  0.0021851  0.0019015
11  0.000515 -0.01950 -0.0006943 -0.0006042
     dfb.pINR   dfb.prR.   dfb.pSR.  dfb.sxFm
1   0.0132875  0.0149821  0.0107838 -0.003177
4   0.0132875  0.0149821  0.0107838 -0.003177
5   0.0132145  0.0147101  0.0105602  0.006417
9   0.0132875  0.0149821  0.0107838 -0.003177
10 -0.0018248 -0.0022668 -0.0004541  0.053377
11  0.0005798  0.0007202  0.0001443 -0.016960
    dfb.oYES    dffit  cov.r    cook.d      hat
1   0.004164 -0.01932 1.0106 1.303e-05 0.003941
4   0.004164 -0.01932 1.0106 1.303e-05 0.003941
5   0.004787 -0.01928 1.0108 1.297e-05 0.004117
9   0.004164 -0.01932 1.0106 1.303e-05 0.003941
10 -0.068361  0.17528 0.9704 2.941e-03 0.005226
11  0.021721 -0.05569 1.0083 1.170e-04 0.005226
\end{Soutput}
\begin{Sinput}
 summary(imb1)
\end{Sinput}
\begin{Soutput}
Potentially influential observations of
	 glm(formula = pres04 ~ partyid + sex + owngun, family = binomial(link = logit),      data = dat) :

     dfb.1_ dfb.prD. dfb.pIND dfb.prtI dfb.pINR
10    0.00   0.06     0.00     0.00     0.00   
13   -0.03   0.00     0.00     0.00     0.01   
54    0.00   0.06     0.00     0.00     0.00   
81    0.22  -0.18    -0.17    -0.18    -0.15   
118   0.00   0.06     0.00     0.00     0.00   
156   0.00   0.06     0.00     0.00     0.00   
189   0.06   0.06     0.00    -0.01    -0.01   
445   0.00   0.06     0.00     0.00     0.00   
589   0.06   0.06     0.00    -0.01    -0.01   
605   0.00   0.06     0.00     0.00     0.00   
664   0.19  -0.19    -0.17    -0.18    -0.15   
704   0.05   0.00     0.11    -0.01    -0.01   
833   0.01   0.00     0.00     0.00     0.00   
904   0.20  -0.23    -0.21    -0.22    -0.17   
986  -0.04   0.00     0.00     0.00     0.01   
987  -0.01   0.00     0.12     0.00     0.00   
1120 -0.04   0.00     0.00     0.00     0.01   
1161  0.06   0.06     0.00    -0.01    -0.01   
1215  0.05   0.00     0.11    -0.01    -0.01   
1227  0.01   0.00     0.00     0.00     0.00   
1292 -0.04   0.00     0.00     0.00    -0.21   
1298 -0.01   0.00     0.12     0.00     0.00   
1322 -0.01   0.00     0.12     0.00     0.00   
1564 -0.05   0.00     0.13     0.01     0.01   
1603  0.19  -0.19    -0.17    -0.18    -0.15   
1606  0.02   0.00     0.00     0.00    -0.22   
1624  0.00   0.06     0.00     0.00     0.00   
1737  0.02   0.00     0.00     0.00    -0.22   
1758 -0.05   0.00     0.13     0.01     0.01   
1784  0.01   0.00     0.00     0.00     0.00   
1797  0.00   0.06     0.00     0.00     0.00   
1805  0.01   0.00     0.00     0.00     0.00   
1812  0.01   0.00     0.00     0.00     0.00   
1846  0.00   0.06     0.00     0.00     0.00   
1943 -0.04   0.00     0.00     0.00    -0.21   
2002 -0.05   0.00     0.13     0.01     0.01   
2029  0.02   0.00     0.00     0.00    -0.22   
2097 -0.04   0.00     0.00     0.00    -0.21   
2119  0.00   0.06     0.00     0.00     0.00   
2126  0.03   0.00     0.00     0.00    -0.01   
2143  0.06   0.06     0.00    -0.01    -0.01   
2146  0.00   0.00     0.00     0.00     0.00   
2174  0.00   0.06     0.00     0.00     0.00   
2259  0.05   0.00     0.11    -0.01    -0.01   
2315 -0.01   0.00     0.12     0.00     0.00   
2327  0.00   0.06     0.00     0.00     0.00   
2405  0.02   0.00     0.00     0.00    -0.22   
2486  0.00   0.00     0.00     0.00     0.00   
2487  0.00   0.00     0.00     0.00     0.00   
2508 -0.04   0.00     0.00     0.00    -0.21   
2616 -0.01   0.00     0.12     0.00     0.00   
2651 -0.05   0.00     0.13     0.01     0.01   
2817  0.05   0.00     0.11    -0.01    -0.01   
2823 -0.05   0.00     0.13     0.01     0.01   
2832  0.00   0.06     0.00     0.00     0.00   
2855  0.00   0.06     0.00     0.00     0.00   
3057  0.20  -0.23    -0.21    -0.22    -0.17   
3078  0.00   0.06     0.00     0.00     0.00   
3180  0.06   0.06     0.00    -0.01    -0.01   
3212  0.01   0.00     0.00     0.00     0.00   
3282  0.01   0.00     0.12     0.00     0.00   
3334  0.01   0.00     0.00     0.00     0.00   
3415  0.01   0.00     0.00     0.00     0.00   
3454  0.01   0.00     0.00     0.00     0.00   
3510  0.06   0.06     0.00    -0.01    -0.01   
3548  0.00   0.00     0.00     0.00    -0.19   
3564  0.04   0.00     0.00     0.00    -0.01   
3718  0.01   0.00     0.12     0.00     0.00   
3769 -0.05   0.00     0.13     0.01     0.01   
3823 -0.01   0.00     0.12     0.00     0.00   
3890 -0.01   0.00     0.12     0.00     0.00   
4113  0.24  -0.22    -0.21    -0.22    -0.18   
4199  0.01   0.00     0.12     0.00     0.00   
4225  0.24  -0.22    -0.21    -0.22    -0.18   
4239  0.00   0.06     0.00     0.00     0.00   
4274  0.00   0.06     0.00     0.00     0.00   
4334  0.06   0.06     0.00    -0.01    -0.01   
4364  0.00   0.00     0.00     0.00     0.00   
4436  0.22  -0.18    -0.17    -0.18    -0.15   
4471  0.01   0.00     0.00     0.00     0.00   
     dfb.prR. dfb.pSR. dfb.sxFm dfb.oYES dffit  
10    0.00     0.00     0.05    -0.07     0.18  
13    0.01    -0.22     0.06     0.04    -0.29_*
54    0.00     0.00     0.05    -0.07     0.18  
81   -0.17    -0.12    -0.07    -0.05     0.22  
118   0.00     0.00     0.05    -0.07     0.18  
156   0.00     0.00     0.05    -0.07     0.18  
189  -0.01    -0.01    -0.12    -0.08     0.21  
445   0.00     0.00     0.05    -0.07     0.18  
589  -0.01    -0.01    -0.12    -0.08     0.21  
605   0.00     0.00     0.05    -0.07     0.18  
664  -0.17    -0.12     0.04    -0.05     0.21  
704  -0.01     0.00    -0.10    -0.08     0.24  
833   0.00    -0.22    -0.04     0.03    -0.28_*
904  -0.19    -0.14     0.05     0.08     0.27_*
986  -0.12     0.00     0.09     0.05    -0.23  
987   0.00     0.00     0.07    -0.07     0.23  
1120 -0.12     0.00     0.09     0.05    -0.23  
1161 -0.01    -0.01    -0.12    -0.08     0.21  
1215 -0.01     0.00    -0.10    -0.08     0.24  
1227 -0.12     0.00    -0.06     0.04    -0.22  
1292  0.01     0.00     0.09     0.05    -0.33_*
1298  0.00     0.00     0.07    -0.07     0.23  
1322  0.00     0.00     0.07    -0.07     0.23  
1564  0.01     0.01     0.09     0.10     0.26_*
1603 -0.17    -0.12     0.04    -0.05     0.21  
1606  0.00     0.00    -0.08     0.04    -0.32_*
1624  0.00     0.00     0.05    -0.07     0.18  
1737  0.00     0.00    -0.08     0.04    -0.32_*
1758  0.01     0.01     0.09     0.10     0.26_*
1784 -0.12     0.00    -0.06     0.04    -0.22  
1797  0.00     0.00     0.05    -0.07     0.18  
1805 -0.12     0.00    -0.06     0.04    -0.22  
1812 -0.12     0.00    -0.06     0.04    -0.22  
1846  0.00     0.00     0.05    -0.07     0.18  
1943  0.01     0.00     0.09     0.05    -0.33_*
2002  0.01     0.01     0.09     0.10     0.26_*
2029  0.00     0.00    -0.08     0.04    -0.32_*
2097  0.01     0.00     0.09     0.05    -0.33_*
2119  0.00     0.00     0.05    -0.07     0.18  
2126 -0.01    -0.18    -0.04    -0.06    -0.23  
2143 -0.01    -0.01    -0.12    -0.08     0.21  
2146 -0.11     0.00     0.06    -0.08    -0.20  
2174  0.00     0.00     0.05    -0.07     0.18  
2259 -0.01     0.00    -0.10    -0.08     0.24  
2315  0.00     0.00     0.07    -0.07     0.23  
2327  0.00     0.00     0.05    -0.07     0.18  
2405  0.00     0.00    -0.08     0.04    -0.32_*
2486  0.00    -0.18     0.04    -0.05    -0.23  
2487 -0.11     0.00     0.06    -0.08    -0.20  
2508  0.01     0.00     0.09     0.05    -0.33_*
2616  0.00     0.00     0.07    -0.07     0.23  
2651  0.01     0.01     0.09     0.10     0.26_*
2817 -0.01     0.00    -0.10    -0.08     0.24  
2823  0.01     0.01     0.09     0.10     0.26_*
2832  0.00     0.00     0.05    -0.07     0.18  
2855  0.00     0.00     0.05    -0.07     0.18  
3057 -0.19    -0.14     0.05     0.08     0.27_*
3078  0.00     0.00     0.05    -0.07     0.18  
3180 -0.01    -0.01    -0.12    -0.08     0.21  
3212 -0.12     0.00    -0.06     0.04    -0.22  
3282  0.00     0.00    -0.09     0.09     0.26_*
3334 -0.12     0.00    -0.06     0.04    -0.22  
3415 -0.12     0.00    -0.06     0.04    -0.22  
3454 -0.12     0.00    -0.06     0.04    -0.22  
3510 -0.01    -0.01    -0.12    -0.08     0.21  
3548  0.00     0.00     0.07    -0.10    -0.30_*
3564 -0.11     0.00    -0.06    -0.09    -0.20  
3718  0.00     0.00    -0.09     0.09     0.26_*
3769  0.01     0.01     0.09     0.10     0.26_*
3823  0.00     0.00     0.07    -0.07     0.23  
3890  0.00     0.00     0.07    -0.07     0.23  
4113 -0.20    -0.14    -0.08     0.07     0.27_*
4199  0.00     0.00    -0.09     0.09     0.26_*
4225 -0.20    -0.14    -0.08     0.07     0.27_*
4239  0.00     0.00     0.05    -0.07     0.18  
4274  0.00     0.00     0.05    -0.07     0.18  
4334 -0.01    -0.01    -0.12    -0.08     0.21  
4364 -0.11     0.00     0.06    -0.08    -0.20  
4436 -0.17    -0.12    -0.07    -0.05     0.22  
4471 -0.12     0.00    -0.06     0.04    -0.22  
     cov.r   cook.d hat  
10    0.97_*  0.00   0.01
13    0.93_*  0.03   0.01
54    0.97_*  0.00   0.01
81    0.93_*  0.02   0.00
118   0.97_*  0.00   0.01
156   0.97_*  0.00   0.01
189   0.97_*  0.00   0.01
445   0.97_*  0.00   0.01
589   0.97_*  0.00   0.01
605   0.97_*  0.00   0.01
664   0.93_*  0.01   0.00
704   0.96_*  0.01   0.01
833   0.93_*  0.03   0.01
904   0.95_*  0.01   0.01
986   0.95_*  0.01   0.01
987   0.96_*  0.01   0.01
1120  0.95_*  0.01   0.01
1161  0.97_*  0.00   0.01
1215  0.96_*  0.01   0.01
1227  0.95_*  0.01   0.01
1292  0.97_*  0.01   0.02
1298  0.96_*  0.01   0.01
1322  0.96_*  0.01   0.01
1564  0.98    0.01   0.01
1603  0.93_*  0.01   0.00
1606  0.97_*  0.01   0.01
1624  0.97_*  0.00   0.01
1737  0.97_*  0.01   0.01
1758  0.98    0.01   0.01
1784  0.95_*  0.01   0.01
1797  0.97_*  0.00   0.01
1805  0.95_*  0.01   0.01
1812  0.95_*  0.01   0.01
1846  0.97_*  0.00   0.01
1943  0.97_*  0.01   0.02
2002  0.98    0.01   0.01
2029  0.97_*  0.01   0.01
2097  0.97_*  0.01   0.02
2119  0.97_*  0.00   0.01
2126  0.91_*  0.03   0.00
2143  0.97_*  0.00   0.01
2146  0.94_*  0.01   0.00
2174  0.97_*  0.00   0.01
2259  0.96_*  0.01   0.01
2315  0.96_*  0.01   0.01
2327  0.97_*  0.00   0.01
2405  0.97_*  0.01   0.01
2486  0.91_*  0.03   0.00
2487  0.94_*  0.01   0.00
2508  0.97_*  0.01   0.02
2616  0.96_*  0.01   0.01
2651  0.98    0.01   0.01
2817  0.96_*  0.01   0.01
2823  0.98    0.01   0.01
2832  0.97_*  0.00   0.01
2855  0.97_*  0.00   0.01
3057  0.95_*  0.01   0.01
3078  0.97_*  0.00   0.01
3180  0.97_*  0.00   0.01
3212  0.95_*  0.01   0.01
3282  0.98    0.01   0.01
3334  0.95_*  0.01   0.01
3415  0.95_*  0.01   0.01
3454  0.95_*  0.01   0.01
3510  0.97_*  0.00   0.01
3548  0.96_*  0.01   0.01
3564  0.94_*  0.01   0.00
3718  0.98    0.01   0.01
3769  0.98    0.01   0.01
3823  0.96_*  0.01   0.01
3890  0.96_*  0.01   0.01
4113  0.95_*  0.02   0.01
4199  0.98    0.01   0.01
4225  0.95_*  0.02   0.01
4239  0.97_*  0.00   0.01
4274  0.97_*  0.00   0.01
4334  0.97_*  0.00   0.01
4364  0.94_*  0.01   0.00
4436  0.93_*  0.02   0.00
4471  0.95_*  0.01   0.01
\end{Soutput}
\end{Schunk}


     Can get component columns directly with 'dfbetas', 'dffits', 'covratio' and 'cooks.distance'.
\end{frame}



\begin{frame}[containsverbatim,allowframebreaks]
 \frametitle{But if You Want dfbeta, Not dfbetas, Why Not Ask?}


\begin{Schunk}
\begin{Sinput}
 dfb1 <- dfbeta(bush1)
 colnames(dfb1)
\end{Sinput}
\begin{Soutput}
[1] "(Intercept)"            
[2] "partyidDem."            
[3] "partyidInd. Near Dem."  
[4] "partyidIndependent"     
[5] "partyidInd. Near Repub."
[6] "partyidRepub."          
[7] "partyidStrong Repub."   
[8] "sexFemale"              
[9] "owngunYES"              
\end{Soutput}
\begin{Sinput}
 head(dfb1)
\end{Sinput}
\begin{Soutput}
   (Intercept) partyidDem. partyidInd. Near Dem.
1   -0.0052361    0.005286             0.0052149
4   -0.0052361    0.005286             0.0052149
5   -0.0059698    0.005023             0.0051036
9   -0.0052361    0.005286             0.0052149
10  -0.0005007    0.019143             0.0007462
11   0.0001594   -0.006095            -0.0002376
   partyidIndependent partyidInd. Near Repub.
1           0.0052232               0.0053054
4           0.0052232               0.0053054
5           0.0051290               0.0052763
9           0.0052232               0.0053054
10          0.0006130              -0.0007269
11         -0.0001952               0.0002315
   partyidRepub. partyidStrong Repub.  sexFemale
1      0.0053094            5.274e-03 -0.0004822
4      0.0053094            5.274e-03 -0.0004822
5      0.0052130            5.165e-03  0.0009737
9      0.0053094            5.274e-03 -0.0004822
10    -0.0008014           -2.216e-04  0.0080812
11     0.0002552            7.056e-05 -0.0025732
   owngunYES
1   0.000635
4   0.000635
5   0.000730
9   0.000635
10 -0.010400
11  0.003312
\end{Soutput}
\end{Schunk}


      I wondered what dfbetas does. You can see for yourself. Look at
      the code. Run:

\begin{Schunk}\begin{Soutput}
    >  stats:::dfbetas.lm
  \end{Soutput}
\end{Schunk}

\end{frame}


% _____________________________________________

\begin{frame}
  \frametitle{\texttt{predict()} with newdata}

  \begin{itemize}
    \item If you run this:

      \texttt{predict(bush5)}

    R calculates $X\hat{\beta}$, a ``linear predictor'' value for each row in your dataframe

  \item See ``\texttt{?predict.glm}.''

  \item We ask for predicted probabilities like so

    \texttt{predict(bush5, type="response")}

    and you still get one prediction for each line in the data.
  \end{itemize}
\end{frame}



\begin{frame}[containsverbatim]
  \frametitle{Use predict to calculate with ``for example'' values}
  \begin{itemize}
  \item Create ``example'' dataframes and get probabilities for
    hypothetical cases.

    \begin{Schunk}
    \begin{Sinput}
      mydf <- # Pretend there are some commands, for example
    \end{Sinput}
  \end{Schunk}
\item Run that new example data frame through the predict function

\begin{Schunk}
\begin{Sinput}
 cor(wildGuessResid, xpoly[,2])
\end{Sinput}
\begin{Soutput}
[1] 1
\end{Soutput}
\end{Schunk}


\end{itemize}
\end{frame}



%%  Commented-out 2012-06-06 to shorten presentation

% \begin{frame}[containsverbatim]
%   \frametitle{Create the New Data Frame}

% <<predict10, echo=T,include=F>>=
% nd <- bush5$model
% colnames(nd)
% mynewdf <- expand.grid(levels(nd$partyid), levels(nd$newpolv))
% colnames(mynewdf) <- c("partyid","newpolv")
% mynewdf$sex <- levels(nd$sex)[1]
% mynewdf$owngun <- levels(nd$owngun)[1]
% mynewdf$race <- levels(nd$race)[1]
% mynewdf$wrkslf <- levels(nd$wrkslf)[1]
% mynewdf$realinc <- mean(nd$realinc)
% mynewdf$newpred <- predict(bush5, newdata=mynewdf, type="response")
% levels(mynewdf$newpolv) <- c("Ex.L","L","SL","M","C")
% @
% \footnotesize{
% \input{plots/t-predict10}
% }

% \end{frame}

% \begin{frame}[containsverbatim]
%   \frametitle{Make Table of Predicted Probabilities}
% <<predict20,include=F,echo=T>>=
% library(gdata)
% newtab <- aggregate.table(mynewdf$newpred, by1=mynewdf$partyid, by2=mynewdf$newpolv, FUN=I)
% @

% \input{plots/t-predict20}

% <<predict30, echo=F,results=tex>>=
% library(memisc)
% toLatex(newtab)
% @
% \end{frame}

% \begin{frame}
%   \frametitle{Or Perhaps You Would Like A Figure?}

% <<pred90,fig=T,include=F>>=
% prebynewpol <- unstack(mynewdf, newpred~newpolv)
% matplot(prebynewpol,type="l",xaxt="n",xlab="Political Party Identification", ylab="Pred. Prob(Bush)")
% axis(1, at=1:7, labels=c("SD","D","ID","I","IR","R","SR"))
% legend("topleft", legend=c("Extreme Liberal","Liberal","Slight Liberal","Moderate","Conservative"),col=1:5,lty=1:5)
% @

% \includegraphics[width=10cm]{plots/t-pred90}
% \end{frame}


% \begin{frame}[containsverbatim]
%   \frametitle{How Could You Make That Figure?}

%   \input{plots/t-pred90}
% \end{frame}

% %===================================================



\begin{frame}[containsverbatim]
  \frametitle{Termplot: Plotting The Linear Predictor}


\begin{Schunk}
\begin{Sinput}
 termplot(bush1,terms=c("partyid"))
\end{Sinput}
\end{Schunk}

\includegraphics[width=10cm]{plots/t-termplot10}
\end{frame}


\begin{frame}[containsverbatim]
  \frametitle{Termplot: Some of the Magic is Lost on a Logistic Model}


\begin{Schunk}
\begin{Sinput}
 summary(mod1)
\end{Sinput}
\begin{Soutput}
Call:
lm(formula = th.bush.kerry ~ V045117 + V043116 + V043210 + V043213 + 
    V045145X + V041109A, data = mydta1)

Residuals:
     Min       1Q   Median       3Q      Max 
-120.570  -18.767   -0.394   19.031  123.140 

Coefficients:
                                     Estimate Std. Error t value Pr(>|t|)    
(Intercept)                          -54.6475     8.0604  -6.780 2.37e-11 ***
V045117L                             -10.7877     7.9306  -1.360  0.17414    
V045117SL                              2.3754     7.9326   0.299  0.76468    
V045117M                               5.6121     7.8190   0.718  0.47312    
V045117SC                             10.1405     8.2569   1.228  0.21977    
V045117C                              17.4988     8.3414   2.098  0.03624 *  
V045117EC                             26.3985     9.7830   2.698  0.00712 ** 
V043116WD                             24.6052     4.0323   6.102 1.65e-09 ***
V043116ID                             22.3647     3.7648   5.940 4.28e-09 ***
V043116I                              40.6049     5.1647   7.862 1.26e-14 ***
V043116IR                             65.2119     4.5902  14.207  < 2e-16 ***
V043116WR                             67.2391     4.5154  14.891  < 2e-16 ***
V043116SR                             82.3482     4.7217  17.440  < 2e-16 ***
V043210No                              7.9106     2.6149   3.025  0.00257 ** 
V043210Med                             6.7806     5.8401   1.161  0.24598    
V043213Same                           17.7009     2.8209   6.275 5.79e-10 ***
V043213Better                         25.0827     3.2776   7.653 5.81e-14 ***
V045145X2. Very good                  -7.6227     2.5281  -3.015  0.00265 ** 
V045145X3. Somewhat good             -14.5053     3.3873  -4.282 2.08e-05 ***
V045145X4. Not very good             -14.6720     6.1407  -2.389  0.01712 *  
V045145X7. Don't feel anything {VOL} -26.2376     8.6676  -3.027  0.00255 ** 
V041109AF                              0.2839     2.1898   0.130  0.89687    
---
Signif. codes:  0 '***' 0.001 '**' 0.01 '*' 0.05 '.' 0.1 ' ' 1

Residual standard error: 29.95 on 781 degrees of freedom
  (409 observations deleted due to missingness)
Multiple R-squared:  0.7122,	Adjusted R-squared:  0.7045 
F-statistic: 92.04 on 21 and 781 DF,  p-value: < 2.2e-16
\end{Soutput}
\end{Schunk}

\includegraphics[width=10cm]{plots/t-termplot20}
\end{frame}



\begin{frame}[containsverbatim]
  \frametitle{Termplot: But If You Had Some Continuous Data, Watch Out!}

\begin{Schunk}
\begin{Sinput}
 termplot(myolsmod, terms=c("x"), partial.resid = T, se = T)
\end{Sinput}
\end{Schunk}

\includegraphics[width=10cm]{plots/t-termplot30}
\end{frame}




\begin{frame}[containsverbatim]
  \frametitle{\texttt{termplot()} works because $\ldots$}

  \begin{itemize}
  \item termplot doesn't make calculations, it uses the
    ``\texttt{predict}'' method associated with a model object.
  \item \texttt{predict} is a generic method, it doesn't do any work either!
  \item Actual work gets done by methods for models,
    \texttt{predict.lm} or  \texttt{predict.glm}.
  \item You can leave out the ``terms'' option, termplot will
    cycle through all of the predictors in the model.
  \end{itemize}
\end{frame}

\begin{frame}[containsverbatim]
  \frametitle {Why Termplot is Not the End of the Story}
  \begin{itemize}
  \item Termplot draws $X\hat{\beta}$, the linear predictor.
  \item Maybe we want predicted probabilities instead.
  \item Maybe we want predictions for certain case types: \texttt{ termplot} allows the predict implementation to decide which
    values of the inputs will be used.
  \item A regression expert will quickly conclude that a really
    great graph may require direct use of the \texttt{predict}
    method for the model object.
  \end{itemize}
\end{frame}




% % ===================================================
% \section{Output}

% \begin{frame}[containsverbatim]
%   \frametitle{You Will Want to Use \LaTeX{}     After You See This}

%   \begin{itemize}
%     \item How do you get regression tables out of your project?
%     \item Do you go through error-prone copying, pasting, typing,
%       tabling, etc?
%     \item What if your software could produce a finished
%       publishable table?
%     \end{itemize}
%   \end{frame}

% <<echo=F, results=hide>>=
% library(rockchalk)
% @

% \begin{frame}[containsverbatim]
%   \frametitle{}
%   \begin{itemize}
%     \item Years ago, I wrote a function ``outreg''


% <<echo=T, results=hide, include=F>>=
% library(rockchalk)
% @

% <<output10,echo=T, eval=F, include=F>>=
% outreg(bush1, tight=F, modelLabels=c("Bush Logistic"))
% @


%     \item This command:

% <<output30,echo=T,eval=T,include=F>>=
% <<output10>>
% @
% \input{plots/t-output10}
%    \item Produces the output on the next slide
%   \end{itemize}
% \end{frame}

% \begin{frame}[plain,containsverbatim]

% <<output40,echo=F,include=F,results=tex>>=
% <<output10>>
% @

% \input{plots/t-output40}
% \end{frame}




% \begin{frame}[containsverbatim]
%   \frametitle{Polish that up}

%   \begin{itemize}
%     \item you can beautify the variable labels, either by specifying
%       them in the outreg command or editing the table output.
%     \item outreg produces Latex that looks like this in the R session output.

% <<output41,echo=F,include=F>>=
% <<output10>>
% @
% \input{plots/t-output41}

% \end{itemize}
% \end{frame}


% \begin{frame}
%   \frametitle{Push Several Models Into One Wide Table}

% <<output50,echo=T, eval=F, include=F>>=
% outreg(list(bush1,bush4,bush5),  modelLabels=c("bush1","bush4","bush5"))
% @
% \input{plots/t-output50}

%   Sorry, I had to split this manually across 3 slides :(

% \end{frame}


% \begin{frame}[plain,allowframebreaks]

% <<output51,echo=F,include=F,results=tex>>=
% <<output50>>
% @

% \footnotesize{
% \input{plots/t-output51}
% }
% \end{frame}

% %__________________________________


% \begin{frame}
%   \frametitle {R Packages for Producing Regression Output}

%   \begin{itemize}
%   \item memisc: works well, further from final form than outreg

%   \item xtable: incomplete output, but latex or HTML works

%   \item apsrtable: very similar to outreg

%   \item Hmisc ``latex'' function

%   \end{itemize}
% \end{frame}


% \begin{frame}[containsverbatim,allowframebreaks]

% <<output70,echo=T,results=tex>>=
% library(xtable)
% tabout1 <- xtable(bush1)
% print(tabout1, type="latex")

% @


% \end{frame}

% \begin{frame}[containsverbatim]
%   \frametitle{If you Can't Shake the MS Word ``Habit''}

%   The best you can do is HTML output, which you can copy paste-special
%   into a document.

% <<output80,echo=T>>=
% print(xtable(summary(bush1)), type="HTML")
% @
% \end{frame}

% \begin{frame}[containsverbatim,allowframebreaks]
%   \frametitle{memisc mtable is nice for comparing models (except for
%     verbosity of parameter labels)}

% <<echo=T,eval=T>>=
%  library(memisc)
%  mtable(bush1,bush4,bush5)
% @
% \end{frame}


% \begin{frame}[containsverbatim,allowframebreaks]
%   \frametitle{memisc toLatex}

% \footnotesize{
% <<output84, echo=T,results=tex>>=
% toLatex(mtable(bush1))
% @
% }
% \end{frame}

% \begin{frame}
%   \frametitle{Relable Levels to Truncate Output}

%   \begin{itemize}
%     \item We could have to edit that output A LOT


%     \item Hack the Labels down
% <<echo=T>>=
% levels(dat$partyid) <- c("SD","D","ID","I","IR","R","SR")
% levels(dat$polviews) <- c("EL","L","SL","M","SC","C","EC")
% levels(dat$newpolv) <- c("EL","L","SL","M","C")
% levels(dat$wrkslf) <- c("Yes","No")
% @

%     \item Re-run the models

% <<echo=T>>=

% bush1 <- glm(pres04 ~ partyid + sex + owngun, data=dat, family=binomial(link=logit))

%  bush3 <- glm(pres04 ~ partyid + sex + owngun + race + wrkslf + realinc + polviews , data=dat, family=binomial(link=logit))
%  bush4 <- glm(pres04 ~ partyid +  owngun + race + polviews , data=bush3$model, family=binomial(link=logit))
% bush5 <- glm(pres04 ~ partyid + sex + owngun + race + wrkslf + realinc + newpolv , data=dat, family=binomial(link=logit))
% @
% \end{itemize}
% \end{frame}

% %___________________________

% \begin{frame}[containsverbatim,allowframebreaks,plain]

% <<output85, echo=T,include=F,results=tex>>=
% toLatex(mtable(bush1,bush4,bush5))
% @
% \footnotesize{
% \input{plots/t-output85}
% }



% \end{frame}


%===================================================
\end{document}
