\makeatletter
\def\input@path{{/home/pauljohn/TrueMounted/ps/SVN-guides/Rcourse/regression-glm-2//}}
\makeatother
\documentclass[10pt,english]{beamer}
\usepackage{lmodern}
\renewcommand{\sfdefault}{lmss}
\renewcommand{\ttdefault}{lmtt}
\usepackage[T1]{fontenc}
\usepackage[utf8]{inputenc}
\usepackage{listings}
\setcounter{secnumdepth}{3}
\setcounter{tocdepth}{3}

\makeatletter
%%%%%%%%%%%%%%%%%%%%%%%%%%%%%% Textclass specific LaTeX commands.
\usepackage{Sweavel}


%%%%%%%%%%%%%%%%%%%%%%%%%%%%%% User specified LaTeX commands.
\usepackage{dcolumn}
\usepackage{booktabs}

% use 'handout' to produce handouts
%\documentclass[handout]{beamer}
\usepackage{wasysym}
\usepackage{pgfpages}
\newcommand{\vn}[1]{\mbox{{\it #1}}}\newcommand{\vb}{\vspace{\baselineskip}}\newcommand{\vh}{\vspace{.5\baselineskip}}\newcommand{\vf}{\vspace{\fill}}\newcommand{\splus}{\textsf{S-PLUS}}\newcommand{\R}{\textsf{R}}


\usepackage{graphicx}
\usepackage{listings}
\lstset{tabsize=2, breaklines=true,style=Rstyle}
\usetheme{Antibes}
% or ...

%\setbeamercovered{transparent}
% or whatever (possibly just delete it)


% In document Latex options:
\fvset{listparameters={\setlength{\topsep}{0em}}}
\def\Sweavesize{\normalsize}
\def\Rcolor{\color{black}}
\def\Rbackground{\color[gray]{0.95}}

\mode<presentation>

\newcommand\makebeamertitle{\frame{\maketitle}}%

\setbeamertemplate{frametitle continuation}[from second]
\renewcommand\insertcontinuationtext{...}

\expandafter\def\expandafter\insertshorttitle\expandafter{%
 \insertshorttitle\hfill\insertframenumber\,/\,\inserttotalframenumber}

\makeatother

\usepackage{babel}



\begin{document}

% In document Latex options:
\fvset{listparameters={\setlength{\topsep}{0em}}}

\def\Sweavesize{\normalsize}
\def\Rcolor{\color{black}}
\def\Rbackground{\color[gray]{0.97}}

\begin{Schunk}
\begin{Sinput}
 ## Other settings I like
 options(device = pdf)
 options(useFancyQuotes = FALSE) 
 op <- par() 
 pjmar <- c(5.1, 5.1, 1.5, 2.1) 
 options(SweaveHooks=list(fig=function() par(mar=pjmar, ps=12)))
 pdf.options(onefile=F,family="Times",pointsize=12)
\end{Sinput}
\end{Schunk}



\title[glm2]{After Fitting Regressions}


\author{Paul E. Johnson\inst{1} \and \inst{2}}


\institute[K.U.]{\inst{1}Department of Political Science\and \inst{2}Center for
Research Methods and Data Analysis, University of Kansas}


\date[2012]{2012}

\makebeamertitle



\AtBeginSubsection[]{

  \frame<beamer>{

    \frametitle{Outline}

    \tableofcontents[currentsection,currentsubsection]

  }

}



% %%%%%%%%%%%%%%%%%%%%%%%%%%%%%%%%%%%%%%%%%%%%%%%%%%%%%%%%%%%%%%%%%
% % Filename: plotting1.Rnw
% %   Author: Pascal Deboeck Paul Johnson
% %
% %%%%%%%%%%%%%%%%%%%%%%%%%%%%%%%%%%%%%%%%%%%%%%%%%%%%%%%%%%%%%%%%%
% % Revision History --
% %  2010-05-13 initial release
% %%%%%%%%%%%%%%%%%%%%%%%%%%%%%%%%%%%%%%%%%%%%%%%%%%%%%%%%%%%%%%%%%

% \documentclass[English]{beamer}

% \usetheme{Antibes}

% \usepackage[utf8]{inputenc} % or whatever
% %\usepackage{times}
% \usepackage[T1]{fontenc}
% \usepackage{lmodern}
% \usepackage{multimedia}
% \usepackage{multicol}
% %%\usepackage{apacite}
% % Or whatever. Note that the encoding and the font should match. If T1
% % does not look nice, try deleting the line with the fontenc.
% \usepackage{mathptmx}
% \usepackage{color}
% \usepackage{amsmath}
% \usepackage{amssymb}
% \usepackage{url}
% \usepackage{listings}
% \usepackage{booktabs}
% \usepackage{dcolumn}

% \usepackage[english]{babel} % or whatever

% %\usepackage{geometry}
% \usepackage{multicol}

% %%%just testing \setbeamersize{text margin left=0.5cm}

% %=============================================================================

% %Switch comment character to turn on/off \pause commands given as \pausealt
% %\newcommand{\pausealt}{\par }
% \newcommand{\pausealt}{\pause}


% \usepackage{color}
% \definecolor{gray1}{gray}{0.75}

% \newlength{\figurewidth}
% \figurewidth \textwidth  % This is for rectangular graphs
% \newlength{\figurewidthB}
% \figurewidthB .7\textwidth  % This is for square graphs



% \expandafter\def\expandafter\insertshorttitle\expandafter{%
%   \insertshorttitle\hfill\insertframenumber\,/\,\inserttotalframenumber}


% %=============================================================================
% %

%====================================


%suggested by Ihaka's note on Improving Sweave
%http://www.stat.auckland.ac.nz/~stat782/downloads/Sweave-customisation.pdf
%%%tighten up text output from R
%
% \DefineVerbatimEnvironment{Sinput}{Verbatim}{xleftmargin=0em,fontsize=\footnotesize}
% \DefineVerbatimEnvironment{Soutput}{Verbatim}{xleftmargin=0em,fontsize=\footnotesize}
% \DefineVerbatimEnvironment{Scode}{Verbatim}{xleftmargin=0em,fontsize=\footnotesize]}
% \fvset{listparameters={\setlength{\topsep}{0pt}}}
% \renewenvironment{Schunk}{\vspace{\topsep}}{\vspace{\topsep}}
% %shorten line length
% <<echo=false>>=
% system("rm -rf figs; mkdir figs")
% @
% \SweaveOpts{prefix.string=figs/t,split=T, ae=F,height=4,width=6}
% <<Roptions, echo=F>>=
% options(width=60, continue="  ")
% ###Leave less white space margin  at top of R figures
% op <- par() ## save par for later in case needed to restore
% pjmar <- c(5.1, 4.1, 2, 2.1)
% options(SweaveHooks=list(fig=function() par(mar=pjmar)))
% ###Sweave appears to ignore following settings 2010-03-20
% ps.options(horizontal=F, onefile=F, family="Times",    paper="special", height=4, width=6 )
% pdf.options(onefile=F,family="Times",paper="special",height=4,width=6)
% options(papersize="special")
% @



\begin{frame}
  \titlepage
\end{frame}



\begin{frame}
\frametitle{Outline}

\tableofcontents{}

\end{frame}

%==================================================

\section{Methods}
\begin{frame}[containsverbatim]
  \frametitle{Methods: Things To Do ``To'' a Regression Object}





\begin{Schunk}
\begin{Sinput}
 bush1 <- glm(pres04 ~ partyid + sex + owngun, data=dat, family=binomial(link=logit))
\end{Sinput}
\end{Schunk}


\begin{description}
  \item [pres04] Kerry,  Bush
  \item [partyid]  Factor with 7 levels, SD $\rightarrow$ SR
  \item [sex]    Male, Female
  \item [owngun] Yes, No
\end{description}
\end{frame}

\begin{frame}[containsverbatim]
  \frametitle{Just for the Record, The Data Preparation Steps Were $\ldots$}

\begin{Schunk}
\begin{Sinput}
 preslev <- levels(dat$pres04)
 dat$pres04[dat$pres04 %in% preslev[3:10]]<- NA
 dat$pres04 <- factor(dat$pres04)
 levels(dat$pres04) <- c("Kerry", "Bush")
 plev <- levels(dat$partyid)
 dat$partyid[dat$partyid %in% plev[8]] <- NA
 dat$partyid <- factor(dat$partyid)
 levels(dat$partyid) <- c("Strong Dem.", "Dem.", "Ind. Near Dem.", "Independent", "Ind. Near Repub.", "Repub.", "Strong Repub.")
 dat$owngun[ dat$owngun == "REFUSED"] <- NA
 levels(dat$sex) <- c("Male","Female")
 dat$owngun <- relevel(dat$owngun, ref="NO")
\end{Sinput}
\end{Schunk}

\end{frame}


\begin{frame}[containsverbatim,allowframebreaks]
  \frametitle{First, Find Out What You Got}


\begin{Schunk}
\begin{Sinput}
 attributes(bush1)
\end{Sinput}
\begin{Soutput}
$names
 [1] "coefficients"      "residuals"        
 [3] "fitted.values"     "effects"          
 [5] "R"                 "rank"             
 [7] "qr"                "family"           
 [9] "linear.predictors" "deviance"         
[11] "aic"               "null.deviance"    
[13] "iter"              "weights"          
[15] "prior.weights"     "df.residual"      
[17] "df.null"           "y"                
[19] "converged"         "boundary"         
[21] "model"             "na.action"        
[23] "call"              "formula"          
[25] "terms"             "data"             
[27] "offset"            "control"          
[29] "method"            "contrasts"        
[31] "xlevels"          

$class
[1] "glm" "lm" 
\end{Soutput}
\end{Schunk}


\end{frame}


\begin{frame}[containsverbatim]
  \frametitle{Understanding attributes}
  \begin{itemize}
  \item If  you see \$, it means you have an S3 object
  \item That means you can just ``take'' values out of the object with
    the dollar sign operator using commands like

\begin{Schunk}
\begin{Sinput}
 bush1$coefficients
\end{Sinput}
\begin{Soutput}
            (Intercept)             partyidDem. 
                 -3.571                   1.910 
  partyidInd. Near Dem.      partyidIndependent 
                  1.456                   3.464 
partyidInd. Near Repub.           partyidRepub. 
                  5.468                   6.031 
   partyidStrong Repub.               sexFemale 
                  7.191                   0.049 
              owngunYES 
                  0.642 
\end{Soutput}
\end{Schunk}


\item That "crude" approach is discouraged. We should instead use
    "extractor methods"

    \texttt{coefficients(bush1)}

  \item Challenge: finding/remembering the extractor functions.
  \end{itemize}
\end{frame}


\begin{frame}[containsverbatim]
  \frametitle{Just Making Sure About the Object's Class}

  \begin{itemize}
  \item Ask the object what class it is from

\begin{Schunk}
\begin{Sinput}
 class(bush1)
\end{Sinput}
\begin{Soutput}
[1] "glm" "lm" 
\end{Soutput}
\end{Schunk}


  \end{itemize}
\end{frame}


\begin{frame}[containsverbatim,allowframebreaks]
  \frametitle{ Ask What Methods Apply to a ``glm'' Object}

\begin{Schunk}
\begin{Sinput}
 methods(class = "glm")
\end{Sinput}
\begin{Soutput}
 [1] add1.glm*           anova.glm          
 [3] confint.glm*        cooks.distance.glm*
 [5] deviance.glm*       drop1.glm*         
 [7] effects.glm*        extractAIC.glm*    
 [9] family.glm*         formula.glm*       
[11] influence.glm*      logLik.glm*        
[13] model.frame.glm     nobs.glm*          
[15] predict.glm         print.glm          
[17] residuals.glm       rstandard.glm      
[19] rstudent.glm        summary.glm        
[21] vcov.glm*           weights.glm*       

   Non-visible functions are asterisked
\end{Soutput}
\end{Schunk}


\end{frame}



\begin{frame}[containsverbatim,allowframebreaks]
  \frametitle{Check methods for ``lm'' class}

\begin{Schunk}
\begin{Sinput}
 confint(mod1)
\end{Sinput}
\begin{Soutput}
                   2.5 %     97.5 %
(Intercept) 27.165443886 40.2839844
GRIP         0.007898305  0.1242813
\end{Soutput}
\end{Schunk}


\end{frame}




\begin{frame}[containsverbatim]
  \frametitle{Do You Wonder How ``They'' Do ``That''?}
  \begin{itemize}

    \item At some point, you realize that the help page is not
      detailed enough.  You may need to see the Actual Code

    \item Darth said ``Use the Source, Luke!''

      If you want to know ``what a function does'', the
      best option is to download the ACTUAL SOURCE CODE and read it!
    \end{itemize}
\end{frame}


\begin{frame}[containsverbatim]
  \frametitle{Can See Some Code Within an R Session}
  \begin{itemize}
  \item In the ``old days'', you could easily see a function's
      ``code'' by typing its name (i.e., omit the parentheses).

      Ex: q used to show all of the steps in shutting down.

    \item Today, in R 2.11, when I type q I see:

\begin{Schunk}\begin{Soutput}
> q
function (save = "default", status = 0, runLast = TRUE)
.Internal(quit(save, status, runLast))
<environment: namespace:base>
  \end{Soutput}
\end{Schunk}

\end{itemize}
\end{frame}

\begin{frame}[containsverbatim]
  \frametitle{Some Functions Still Show Their Code}
  \begin{itemize}

   \item Some very informative examples. Try:
      \begin{itemize}
        \item \texttt{ > lm \#(or stats::lm)}
        \item \texttt{ > glm \#(or stats::glm)}
        \item \texttt{ > termplot}
      \end{itemize}

    \item Generic method output not so useful. Try:
      \begin{itemize}
      \item \texttt{ > predict}
      \item \texttt{ > plot}
      \end{itemize}

    \end{itemize}
  \end{frame}


\begin{frame}
  \frametitle{Looking Into the Class Hierarchy}
  \begin{itemize}

  \item In many cases, you can only find what you need if you give
      the ``function'' name and the name of the ``class'' separated by
      a period.

   \item Try:
      \begin{itemize}
      \item \texttt{ > predict.lm }

      \item \texttt{ > predict.glm}
      \end{itemize}

    \item Many methods are inside ``namespaces'' and you can't see
      their code without some extra effort.
      \begin{itemize}
      \item namespace::method will often be useful

      \item Three colons needed for ``hidden methods''

        stats:::weights.glm
      \end{itemize}

    \item Many times I have doublechecked this detailed posting by
      Prof. Brian Ripley on this question:

      \url{http://tolstoy.newcastle.edu.au/R/help/05/09/12506.html}

    \end{itemize}

\end{frame}

% ______________________________

\section{Interrogate Models}


\begin{frame}[containsverbatim,allowframebreaks]
  \frametitle{The First Method Used is usually \texttt{summary()}}

\begin{Schunk}
\begin{Sinput}
 summary(bush1)
\end{Sinput}
\begin{Soutput}
Call:
glm(formula = pres04 ~ partyid + sex + owngun, family = binomial(link = logit), 
    data = dat)

Deviance Residuals: 
   Min      1Q  Median      3Q     Max  
-2.941  -0.488   0.163   0.390   2.683  

Coefficients:
                        Estimate Std. Error z value
(Intercept)              -3.5712     0.3934   -9.08
partyidDem.               1.9103     0.3972    4.81
partyidInd. Near Dem.     1.4559     0.4348    3.35
partyidIndependent        3.4642     0.4105    8.44
partyidInd. Near Repub.   5.4677     0.5073   10.78
partyidRepub.             6.0307     0.4502   13.39
partyidStrong Repub.      7.1908     0.6213   11.57
sexFemale                 0.0488     0.1928    0.25
owngunYES                 0.6424     0.1937    3.32
                        Pr(>|z|)    
(Intercept)              < 2e-16 ***
partyidDem.              1.5e-06 ***
partyidInd. Near Dem.    0.00081 ***
partyidIndependent       < 2e-16 ***
partyidInd. Near Repub.  < 2e-16 ***
partyidRepub.            < 2e-16 ***
partyidStrong Repub.     < 2e-16 ***
sexFemale                0.80006    
owngunYES                0.00091 ***
---
Signif. codes:  0 '***' 0.001 '**' 0.01 '*' 0.05 '.' 0.1 ' ' 1 

(Dispersion parameter for binomial family taken to be 1)

    Null deviance: 1721.9  on 1242  degrees of freedom
Residual deviance:  764.0  on 1234  degrees of freedom
  (3267 observations deleted due to missingness)
AIC: 782

Number of Fisher Scoring iterations: 6
\end{Soutput}
\end{Schunk}


\end{frame}


\begin{frame}[containsverbatim,allowframebreaks]
  \frametitle{Summary Object}

  Create a Summary Object
\begin{Schunk}
\begin{Sinput}
 sb1 <- summary(bush1)
 attributes(sb1)
\end{Sinput}
\begin{Soutput}
$names
 [1] "call"           "terms"          "family"        
 [4] "deviance"       "aic"            "contrasts"     
 [7] "df.residual"    "null.deviance"  "df.null"       
[10] "iter"           "na.action"      "deviance.resid"
[13] "coefficients"   "aliased"        "dispersion"    
[16] "df"             "cov.unscaled"   "cov.scaled"    

$class
[1] "summary.glm"
\end{Soutput}
\end{Schunk}


 My deviance is

\begin{Schunk}
\begin{Sinput}
 sb1$deviance
\end{Sinput}
\begin{Soutput}
[1] 764
\end{Soutput}
\end{Schunk}


\end{frame}


\begin{frame}[containsverbatim,allowframebreaks]
  \frametitle{The coef Enigma}

  \begin{itemize}
  \item \texttt{coef()} is the same as \texttt{coefficients()}

  \item Note the Bizarre Truth:
    \begin{enumerate}
    \item that the ``coef'' function returns
something different when it is applied to a model object
\input{plots/t-sum45}

Than is returned from a summary object.

\begin{Schunk}
\begin{Sinput}
 coef(sb1)
\end{Sinput}
\begin{Soutput}
                        Estimate Std. Error z value
(Intercept)               -3.571       0.39   -9.08
partyidDem.                1.910       0.40    4.81
partyidInd. Near Dem.      1.456       0.43    3.35
partyidIndependent         3.464       0.41    8.44
partyidInd. Near Repub.    5.468       0.51   10.78
partyidRepub.              6.031       0.45   13.39
partyidStrong Repub.       7.191       0.62   11.57
sexFemale                  0.049       0.19    0.25
owngunYES                  0.642       0.19    3.32
                        Pr(>|z|)
(Intercept)              1.1e-19
partyidDem.              1.5e-06
partyidInd. Near Dem.    8.1e-04
partyidIndependent       3.2e-17
partyidInd. Near Repub.  4.3e-27
partyidRepub.            6.5e-41
partyidStrong Repub.     5.6e-31
sexFemale                8.0e-01
owngunYES                9.1e-04
\end{Soutput}
\end{Schunk}

\end{enumerate}
\end{itemize}
\end{frame}






\begin{frame}[containsverbatim]
  \frametitle{\texttt{anova()}}
  \begin{itemize}
    \item You can apply \texttt{anova()} to just one model
    \item That gives a ``stepwise'' series of comparisons (not very useful)

\begin{Schunk}
\begin{Sinput}
 anova(bush1, test="Chisq")
\end{Sinput}
\begin{Soutput}
Analysis of Deviance Table

Model: binomial, link: logit

Response: pres04

Terms added sequentially (first to last)


        Df Deviance Resid. Df Resid. Dev Pr(>Chi)    
NULL                     1242       1722             
partyid  6      947      1236        775  < 2e-16 ***
sex      1        0      1235        775  0.97862    
owngun   1       11      1234        764  0.00087 ***
---
Signif. codes:  0 '***' 0.001 '**' 0.01 '*' 0.05 '.' 0.1 ' ' 1 
\end{Soutput}
\end{Schunk}

\end{itemize}
\end{frame}

% _____________________________

\begin{frame}
  \frametitle{But anova Very Useful to Compare 2 Models}

  Here's the basic procedure:

  \begin{enumerate}
  \item Fit 1 big model, ``mod1''
  \item Exclude some variables to create a smaller model, ``mod2''
  \item Run \texttt{anova()} to compare:

      anova(mod1, mod2, test=''Chisq'')

   \item If resulting test statistic is far from 0, it means the big
     model really is better and you should keep those variables in there.
   \end{enumerate}

   Quick Reminder:

   \begin{itemize}
   \item In an OLS model, this is would be an F test for the
     hypothesis that the coefficients for omitted parameters are all
     equal to 0.
   \item In a model estimated by maximum likelihood, it is a
     likelihood ratio test with df= number of omitted parameters.
   \end{itemize}
 \end{frame}


% ______________________________

\begin{frame}[containsverbatim,allowframebreaks]
  \frametitle{But there's an anova ``Gotcha''}


%<<>>=
%anova(bush0, bush1, test="Chisq")
%@

\begin{Schunk}
  \begin{Soutput}
> anova(bush0, bush1, test="Chisq")
Error in anova.glmlist(c(list(object), dotargs),
  dispersion = dispersion,  :
  models were not all fitted to the same size of dataset
\end{Soutput}
\end{Schunk}

  What the Heck?
\end{frame}

% _____________________________

\begin{frame}
  \frametitle{\texttt{anova()} Gotcha, cont.}
  \begin{itemize}
    \item Explanation: Listwise Deletion of Missing Values causes this.

      Missings cause sample sizes to differ when variables change.

    \item One Solution: Fit both models on same data.
    \begin{enumerate}
      \item Fit the ``big model'' (one with most variables)


mod1 <- glm(y~ x1+ x2 + x3 + $\ldots$, data=dat, family=binomial)


      \item Fit the ``smaller Model'' with the data extracted from
        the fit of the previous model (mod1\$model) as the data frame

mod2 <- glm(y~  x3 + $\ldots$, data=mod1\$model, family=binomial)

     \item After that, anova() will work

     \end{enumerate}

     \item Hasten to add: more elaborate treatment of missingness is
       often called for.

   \end{itemize}
\end{frame}

\begin{frame}[containsverbatim]
  \frametitle{Example anova()}
  \begin{itemize}


  \item Here's the big model
\begin{Schunk}
\begin{Sinput}
  bush3 <- glm(pres04 ~ partyid + sex + owngun + race + wrkslf + realinc + polviews , data=dat, family=binomial(link=logit))
\end{Sinput}
\end{Schunk}


 \item Here's the small model
\begin{Schunk}
\begin{Sinput}
  bush4 <- glm(pres04 ~ partyid +  owngun + race + polviews , data=model.frame(bush3), family=binomial(link=logit))
\end{Sinput}
\end{Schunk}


 \end{itemize}
 \end{frame}


%___________________________________


\begin{frame}[containsverbatim]
  \frametitle{\texttt{anova()}: The Big Reveal!}
  \begin{itemize}

  \item anova:
\begin{Schunk}
\begin{Sinput}
  anova(bush3, bush4, test="Chisq")
\end{Sinput}
\begin{Soutput}
Analysis of Deviance Table

Model 1: pres04 ~ partyid + sex + owngun + race + wrkslf + realinc + polviews
Model 2: pres04 ~ partyid + owngun + race + polviews
  Resid. Df Resid. Dev Df Deviance Pr(>Chi)
1      1044        589                     
2      1047        593 -3     -4.1     0.25
\end{Soutput}
\end{Schunk}


  \item Conclusion: the big model is not statistically significantly
    better than the small model
  \item Same as: Can't reject the null hypothesis that $\beta_j$=0 for
    all omitted parameters
 \end{itemize}
\end{frame}


\begin{frame}
  \frametitle{Interesting Use of anova}
  \begin{itemize}
    \item Consider the fit for ``polviews'' in bush3 (recall
      ``extremely liberal'' is the reference category, the intercept)
    \end{itemize}


\begin{tabular}{l|ccccccc}
\hline
label:& lib. & slt. lib. & mod. & sl. con. & con. & extr. con. \tabularnewline
\hline
mle($\hat{\beta}$): & 0.41 & 1.3  & 1.8* & 2.5* & 2.6* & 3.1*\tabularnewline
\hline
se: & 0.88 & 0.83 & 0.79 & 0.83 & 0.84 & 1.2\tabularnewline
\hline
\end{tabular}

* $p \leq 0.05$

\begin{itemize}
\item I wonder: are all ``conservatives'' the same? Do we really
  need separate parameter estimates for those respondents?
\end{itemize}

\end{frame}

%______________________________

\begin{frame}[containsverbatim]
  \frametitle{Use \texttt{anova()} To Test the Recoding}

  \begin{enumerate}
    \item Make a New Variable for the New Coding
\begin{Schunk}
\begin{Sinput}
 dat$newpolv <- dat$polviews
 (levnpv <- levels(dat$newpolv))
\end{Sinput}
\begin{Soutput}
[1] "EXTREMELY LIBERAL"    "LIBERAL"             
[3] "SLIGHTLY LIBERAL"     "MODERATE"            
[5] "SLGHTLY CONSERVATIVE" "CONSERVATIVE"        
[7] "EXTRMLY CONSERVATIVE"
\end{Soutput}
\begin{Sinput}
 dat$newpolv[dat$newpolv %in% levnpv[5:7]] <- levnpv[6]
\end{Sinput}
\end{Schunk}

\end{enumerate}

\begin{itemize}
  \item Effect is to set slight and extreme conservatives into the
    conservative category
  \end{itemize}

\end{frame}

%_________________________________

\begin{frame}[containsverbatim]
  \frametitle{Better Check newpolv}


\begin{Schunk}
\begin{Sinput}
 dat$newpolv <- factor(dat$newpolv)
 table(dat$newpolv)
\end{Sinput}
\begin{Soutput}
EXTREMELY LIBERAL           LIBERAL 
              139               524 
 SLIGHTLY LIBERAL          MODERATE 
              517              1683 
     CONSERVATIVE 
             1470 
\end{Soutput}
\end{Schunk}


\end{frame}

%_________________________________

\begin{frame}[containsverbatim]
  \frametitle{Neat anova thing, cont.}

  \begin{enumerate}
  \item Fit a new regression model, replacing polviews with newpolv


\begin{Schunk}
\begin{Sinput}
 bush5 <- glm(pres04 ~ partyid + sex + owngun + race + wrkslf + realinc + newpolv , data=dat, family=binomial(link=logit))
\end{Sinput}
\end{Schunk}

 \item Use \texttt{anova()} to test:

\begin{Schunk}
\begin{Sinput}
 anova(bush3, bush5, test="Chisq")
\end{Sinput}
\begin{Soutput}
Analysis of Deviance Table

Model 1: pres04 ~ partyid + sex + owngun + race + wrkslf + realinc + polviews
Model 2: pres04 ~ partyid + sex + owngun + race + wrkslf + realinc + newpolv
  Resid. Df Resid. Dev Df Deviance Pr(>Chi)
1      1044        589                     
2      1046        589 -2   -0.431     0.81
\end{Soutput}
\end{Schunk}

\end{enumerate}
\begin{itemize}
\item Apparently, all conservatives really are alike :)
\item A similar test for liberals is left to the reader!
\end{itemize}
\end{frame}


%_______________________________________________


\begin{frame}[containsverbatim]
  \frametitle{\texttt{drop1} Relieves Tedium}

  \begin{itemize}
  \item \texttt{drop1()} repeats the \texttt{anova()} procedure,
    removing each variable one-at-a-time.

\begin{Schunk}
\begin{Sinput}
 pm1 <- predictOMatic(m2, divider = "std.dev.", n = 5 )
 print(pm1, digits=3)
\end{Sinput}
\begin{Soutput}
$x1
             fit      lwr      upr fit$se.fit    x1       x2         x3           x4 x5
(m-2sd) 15.90005 15.20211 16.59799  0.3556653 30.14 50.46198 0.01624596 -0.005435391  A
(m-sd)  17.95194 17.48503 18.41884  0.2379302 40.22 50.46198 0.01624596 -0.005435391  A
(m)     20.00383 19.64839 20.35927  0.1811297 50.30 50.46198 0.01624596 -0.005435391  A
(m+sd)  22.05572 21.59513 22.51630  0.2347110 60.38 50.46198 0.01624596 -0.005435391  A
(m+2sd) 24.10761 23.41811 24.79710  0.3513612 70.46 50.46198 0.01624596 -0.005435391  A

$x2
             fit      lwr      upr fit$se.fit    x2       x1         x3           x4 x5
(m-2sd) 16.13013 15.43087 16.82939  0.3563376 30.48 50.30392 0.01624596 -0.005435391  A
(m-sd)  18.06718 17.59934 18.53503  0.2384095 40.47 50.30392 0.01624596 -0.005435391  A
(m)     20.00424 19.64880 20.35968  0.1811294 50.46 50.30392 0.01624596 -0.005435391  A
(m+sd)  21.94130 21.48158 22.40102  0.2342707 60.45 50.30392 0.01624596 -0.005435391  A
(m+2sd) 23.87836 23.18996 24.56676  0.3508045 70.44 50.30392 0.01624596 -0.005435391  A

$x3
             fit      lwr      upr fit$se.fit    x3       x1       x2           x4 x5
(m-2sd) 20.28135 19.59667 20.96603  0.3489078 -1.88 50.30392 50.46198 -0.005435391  A
(m-sd)  20.14272 19.68520 20.60023  0.2331447 -0.93 50.30392 50.46198 -0.005435391  A
(m)     20.00408 19.64861 20.35955  0.1811429  0.02 50.30392 50.46198 -0.005435391  A
(m+sd)  19.86544 19.39650 20.33438  0.2389676  0.97 50.30392 50.46198 -0.005435391  A
(m+2sd) 19.72681 19.02683 20.42678  0.3566999  1.92 50.30392 50.46198 -0.005435391  A

$x4
             fit      lwr      upr fit$se.fit    x4       x1       x2         x3 x5
(m-2sd) 20.36103 19.66262 21.05944  0.3559037 -2.07 50.30392 50.46198 0.01624596  A
(m-sd)  20.18322 19.71672 20.64972  0.2377237 -1.04 50.30392 50.46198 0.01624596  A
(m)     20.00542 19.64997 20.36086  0.1811334 -0.01 50.30392 50.46198 0.01624596  A
(m+sd)  19.82761 19.36507 20.29015  0.2357070  1.02 50.30392 50.46198 0.01624596  A
(m+2sd) 19.64980 18.95668 20.34293  0.3532108  2.05 50.30392 50.46198 0.01624596  A

$x5
             fit      lwr      upr fit$se.fit x5       x1       x2         x3           x4
A (70%) 20.00463 19.64919 20.36007  0.1811289  A 50.30392 50.46198 0.01624596 -0.005435391
B (30%) 20.07446 19.53116 20.61776  0.2768619  B 50.30392 50.46198 0.01624596 -0.005435391

attr(,"pnames")
[1] "x1" "x2" "x3" "x4" "x5"
\end{Soutput}
\end{Schunk}


   \item Recall ``Chisq'' $\Leftrightarrow$ L.L.R test.
   \end{itemize}
 \end{frame}




\begin{frame}[containsverbatim]
  \frametitle{Termplot: Plotting The Linear Predictor}


\begin{tabular}{*{3}{l}}
 \hline
                &\multicolumn{2}{c}{Bush-Kerry Thermometer Differential}   \\
                &Estimate &(S.E.) \\
 \hline
 \hline
  (Intercept)    & -54.647*  &   (8.06) \\
  Liberal        & -10.788  &   (7.931) \\
  Slight Lib.    &  2.375  &   (7.933) \\
  Moderate       &  5.612  &   (7.819) \\
  Slight Con.    &  10.141  &   (8.257) \\
  Conservative   &  17.499*  &   (8.341) \\
  Ext. Cons.     &  26.398*  &   (9.783) \\
  Weak Dem.      &  24.605*  &   (4.032) \\
  Indep. Lean Dem.   &  22.365*  &   (3.765) \\
  Independent    &  40.605*  &   (5.165) \\
  Indep. Lean Rep.   &  65.212*  &   (4.59) \\
  Weak Rep.      &  67.239*  &   (4.515) \\
  Str. Repub.    &  82.348*  &   (4.722) \\
  No Gay Marr    &  7.911*  &   (2.615) \\
  V043210Med     &  6.781  &   (5.84) \\
  Econ. Same     &  17.701*  &   (2.821) \\
  Econ. Better   &  25.083*  &   (3.278) \\
  V045145X2. Very good   & -7.623*  &   (2.528) \\
  V045145X3. Somewhat good   & -14.505*  &   (3.387) \\
  V045145X4. Not very good   & -14.672*  &   (6.141) \\
  V045145X7. Don't feel anything {VOL}   & -26.238*  &   (8.668) \\
  Female         &  0.284  &   (2.19) \\
 \hline 
 N                &803      &       \\
 RMSE            &29.95        & \\
 $R^2$           &0.712        & \\
 adj $R^2$       &0.704        & \\
 \hline
 \hline
 
 \multicolumn{2}{l}{${*}  p \le 0.05$   }\\
 \end{tabular}
\includegraphics[width=10cm]{plots/t-termplot10}
\end{frame}


\begin{frame}[containsverbatim]
  \frametitle{Termplot: Some of the Magic is Lost on a Logistic Model}


\begin{Schunk}
\begin{Sinput}
 termplot(bush1,terms=c("partyid"), partial.resid = T, se = T)
\end{Sinput}
\end{Schunk}

\includegraphics[width=10cm]{plots/t-termplot20}
\end{frame}



\begin{frame}[containsverbatim]
  \frametitle{Termplot: But If You Had Some Continuous Data, Watch Out!}

\begin{Schunk}
\begin{Sinput}
 termplot(myolsmod, terms=c("x"), partial.resid = T, se = T)
\end{Sinput}
\end{Schunk}

\includegraphics[width=10cm]{plots/t-termplot30}
\end{frame}




\begin{frame}[containsverbatim]
  \frametitle{\texttt{termplot()} works because $\ldots$}

  \begin{itemize}
  \item termplot doesn't make calculations, it uses the
    ``\texttt{predict}'' method associated with a model object.
  \item \texttt{predict} is a generic method, it doesn't do any work either!
  \item Actual work gets done by methods for models,
    \texttt{predict.lm} or  \texttt{predict.glm}.
  \item You can leave out the ``terms'' option, termplot will
    cycle through all of the predictors in the model.
  \end{itemize}
\end{frame}

\begin{frame}[containsverbatim]
  \frametitle {Why Termplot is Not the End of the Story}
  \begin{itemize}
  \item Termplot draws $X\hat{\beta}$, the linear predictor.
  \item Maybe we want predicted probabilities instead.
  \item Maybe we want predictions for certain case types: \texttt{ termplot} allows the predict implementation to decide which
    values of the inputs will be used.
  \item A regression expert will quickly conclude that a really
    great graph may require direct use of the \texttt{predict}
    method for the model object.
  \end{itemize}
\end{frame}


% _____________________________________________

\begin{frame}
  \frametitle{\texttt{predict()} with newdata}

  \begin{itemize}
    \item If you run this:

      \texttt{predict(bush5)}

    R calculates $X\hat{\beta}$, a ``linear predictor'' value for each row in your dataframe

  \item See ``\texttt{?predict.glm}.''

  \item We ask for predicted probabilities like so

    \texttt{predict(bush5, type="response")}

    and you still get one prediction for each line in the data.
  \end{itemize}
\end{frame}



\begin{frame}
  \frametitle{Use predict to calculate with ``for example'' values}
  \begin{itemize}
  \item Create ``example'' dataframes and get probabilities for
    hypothetical cases.

    \texttt{ > mydf <- \# Pretend there are some commands \\ \#to
      create an example data frame}

  \item Run that new example data frame through the predict function
    \texttt{ > predict(bush5, newdata=mydf, type="response"}

   \end{itemize}
 \end{frame}





\begin{frame}[containsverbatim]
  \frametitle{Create the New Data Frame}

\footnotesize{
\input{plots/t-predict10}
}

\end{frame}

\begin{frame}[containsverbatim]
  \frametitle{Make Table of Predicted Probabilities}

\input{plots/t-predict20}

\input{plots/t-predict30}
\end{frame}

\begin{frame}
  \frametitle{Or Perhaps You Would Like A Figure?}


\includegraphics[width=10cm]{plots/t-pred90}
\end{frame}


\begin{frame}[containsverbatim]
  \frametitle{How Could You Make That Figure?}

  \input{plots/t-pred90}
\end{frame}

%===================================================


\begin{frame}[containsverbatim,allowframebreaks]
  \frametitle{Covariance of $\hat{\beta}$}
\input{plots/t-038}

These will match the ``SE'' column in the summary of bush1
\begin{Schunk}
\begin{Sinput}
 library(rockchalk)
 mcDiagnose(bush1)
\end{Sinput}
\begin{Soutput}
The following auxiliary models are being estimated and returned in a list:
partyidDem. ~ `partyidInd. Near Dem.` + partyidIndependent + 
    `partyidInd. Near Repub.` + partyidRepub. + `partyidStrong Repub.` + 
    sexFemale + owngunYES
<environment: 0x3eb4560>
`partyidInd. Near Dem.` ~ partyidDem. + partyidIndependent + 
    `partyidInd. Near Repub.` + partyidRepub. + `partyidStrong Repub.` + 
    sexFemale + owngunYES
<environment: 0x3eb4560>
partyidIndependent ~ partyidDem. + `partyidInd. Near Dem.` + 
    `partyidInd. Near Repub.` + partyidRepub. + `partyidStrong Repub.` + 
    sexFemale + owngunYES
<environment: 0x3eb4560>
`partyidInd. Near Repub.` ~ partyidDem. + `partyidInd. Near Dem.` + 
    partyidIndependent + partyidRepub. + `partyidStrong Repub.` + 
    sexFemale + owngunYES
<environment: 0x3eb4560>
partyidRepub. ~ partyidDem. + `partyidInd. Near Dem.` + partyidIndependent + 
    `partyidInd. Near Repub.` + `partyidStrong Repub.` + sexFemale + 
    owngunYES
<environment: 0x3eb4560>
`partyidStrong Repub.` ~ partyidDem. + `partyidInd. Near Dem.` + 
    partyidIndependent + `partyidInd. Near Repub.` + partyidRepub. + 
    sexFemale + owngunYES
<environment: 0x3eb4560>
sexFemale ~ partyidDem. + `partyidInd. Near Dem.` + partyidIndependent + 
    `partyidInd. Near Repub.` + partyidRepub. + `partyidStrong Repub.` + 
    owngunYES
<environment: 0x3eb4560>
owngunYES ~ partyidDem. + `partyidInd. Near Dem.` + partyidIndependent + 
    `partyidInd. Near Repub.` + partyidRepub. + `partyidStrong Repub.` + 
    sexFemale
<environment: 0x3eb4560>
Drum roll please! 

And your R_j Squareds are (auxiliary Rsq)
            partyidDem.   partyidInd. Near Dem. 
                0.39471                 0.31465 
     partyidIndependent partyidInd. Near Repub. 
                0.26782                 0.22589 
          partyidRepub.    partyidStrong Repub. 
                0.40933                 0.38675 
              sexFemale               owngunYES 
                0.02243                 0.03130 
The Corresponding VIF, 1/(1-R_j^2)
            partyidDem.   partyidInd. Near Dem. 
                  1.652                   1.459 
     partyidIndependent partyidInd. Near Repub. 
                  1.366                   1.292 
          partyidRepub.    partyidStrong Repub. 
                  1.693                   1.631 
              sexFemale               owngunYES 
                  1.023                   1.032 
Bivariate Correlations for design matrix 
                        partyidDem.
partyidDem.                    1.00
partyidInd. Near Dem.         -0.17
partyidIndependent            -0.15
partyidInd. Near Repub.       -0.13
partyidRepub.                 -0.23
partyidStrong Repub.          -0.21
sexFemale                      0.07
owngunYES                     -0.06
                        partyidInd. Near Dem.
partyidDem.                             -0.17
partyidInd. Near Dem.                    1.00
partyidIndependent                      -0.11
partyidInd. Near Repub.                 -0.10
partyidRepub.                           -0.18
partyidStrong Repub.                    -0.16
sexFemale                               -0.02
owngunYES                               -0.04
                        partyidIndependent
partyidDem.                          -0.15
partyidInd. Near Dem.                -0.11
partyidIndependent                    1.00
partyidInd. Near Repub.              -0.08
partyidRepub.                        -0.15
partyidStrong Repub.                 -0.14
sexFemale                            -0.03
owngunYES                             0.04
                        partyidInd. Near Repub.
partyidDem.                               -0.13
partyidInd. Near Dem.                     -0.10
partyidIndependent                        -0.08
partyidInd. Near Repub.                    1.00
partyidRepub.                             -0.13
partyidStrong Repub.                      -0.12
sexFemale                                 -0.04
owngunYES                                  0.00
                        partyidRepub.
partyidDem.                     -0.23
partyidInd. Near Dem.           -0.18
partyidIndependent              -0.15
partyidInd. Near Repub.         -0.13
partyidRepub.                    1.00
partyidStrong Repub.            -0.22
sexFemale                       -0.04
owngunYES                        0.04
                        partyidStrong Repub.
partyidDem.                            -0.21
partyidInd. Near Dem.                  -0.16
partyidIndependent                     -0.14
partyidInd. Near Repub.                -0.12
partyidRepub.                          -0.22
partyidStrong Repub.                    1.00
sexFemale                              -0.03
owngunYES                               0.11
                        sexFemale owngunYES
partyidDem.                  0.07     -0.06
partyidInd. Near Dem.       -0.02     -0.04
partyidIndependent          -0.03      0.04
partyidInd. Near Repub.     -0.04      0.00
partyidRepub.               -0.04      0.04
partyidStrong Repub.        -0.03      0.11
sexFemale                    1.00     -0.11
owngunYES                   -0.11      1.00
\end{Soutput}
\end{Schunk}

\end{frame}



\begin{frame}[containsverbatim]
  \frametitle{Heteroskedasticity-consistent Standard Errors?}

  Variants of the
  Huber-White ``heteroskedasticity-consistent'' (slang: robust)
  covarance matrix are available in ``car'' and ``sandwich''.

  \begin{itemize}

    \item  hccm() in car works for linear models only

    \item vcovHC in the ``sandwich'' package returns a matrix of
      estimates. One should certainly read ?vcovHC and the associated literature.

\begin{Schunk}
\begin{Sinput}
 newdata(m2, predVals = c("income"))
\end{Sinput}
\begin{Soutput}
  income      age sex
1    7.5 38.54872   F
2   15.0 38.54872   F
3   35.0 38.54872   F
\end{Soutput}
\begin{Sinput}
 newdata(m2, predVals = c("income"), n = 5)
\end{Sinput}
\begin{Soutput}
  income      age sex
1    2.5 38.54872   F
2    7.5 38.54872   F
3   15.0 38.54872   F
4   35.0 38.54872   F
5  200.0 38.54872   F
\end{Soutput}
\begin{Sinput}
 ## income data not continuous, barely numeric
 sort(unique(Chile$income))
\end{Sinput}
\begin{Soutput}
[1]   2.5   7.5  15.0  35.0  75.0 125.0 200.0
\end{Soutput}
\begin{Sinput}
 newdata(m2, predVals = list(income = "table", "sex" = c("M","F")), n = 5)
\end{Sinput}
\begin{Soutput}
   income sex      age
1    15.0   M 38.54872
2    35.0   M 38.54872
3     7.5   M 38.54872
4    75.0   M 38.54872
5     2.5   M 38.54872
6    15.0   F 38.54872
7    35.0   F 38.54872
8     7.5   F 38.54872
9    75.0   F 38.54872
10    2.5   F 38.54872
\end{Soutput}
\begin{Sinput}
 newdata(m2, predVals = list(income = "quantile", age = "std.dev."), n = 3)
\end{Sinput}
\begin{Soutput}
  income   age sex
1    7.5 23.79   F
2   15.0 23.79   F
3   35.0 23.79   F
4    7.5 38.55   F
5   15.0 38.55   F
6   35.0 38.55   F
7    7.5 53.31   F
8   15.0 53.31   F
9   35.0 53.31   F
\end{Soutput}
\end{Schunk}

\end{itemize}
\end{frame}


\begin{frame}[containsverbatim]
  \frametitle{The heteroskedasticity consistent standard errors of the $\hat{\beta}$  are:}

\begin{Schunk}
\begin{Sinput}
 predictOMatic(m2)
\end{Sinput}
\begin{Soutput}
$income
     income      age sex        fit
0%      2.5 38.54872   F 0.02609745
25%     7.5 38.54872   F 0.03191981
50%    15.0 38.54872   F 0.04065334
75%    35.0 38.54872   F 0.06394277
100%  200.0 38.54872   F 0.25608057

$age
     age   income sex         fit
0%    18 33.87586   F -0.10021358
25%   26 33.87586   F -0.03681408
50%   36 33.87586   F  0.04243529
75%   49 33.87586   F  0.14545948
100%  70 33.87586   F  0.31188316

$sex
        sex   income      age         fit
F (50%)   F 33.87586 38.54872  0.06263375
M (50%)   M 33.87586 38.54872 -0.08383037

attr(,"pnames")
[1] "income" "age"    "sex"   
\end{Soutput}
\end{Schunk}

\end{frame}



\begin{frame}[containsverbatim,allowframebreaks]
  \frametitle{Compare those:}

\begin{columns}
  \column{3cm}

The HC and ordinary standard errors are almost identical:
  \column{8cm}
\begin{Schunk}
\begin{Sinput}
 m2 <- lm(log(y) ~ log(x), data = dat)
 head(model.matrix(m2))
\end{Sinput}
\end{Schunk}

\includegraphics{plots/t-042}
\end{columns}

\end{frame}


\begin{frame}[containsverbatim]
\frametitle{Tons of Diagnostic Information}

  Run plot() on the model object for a quick view.

  Example: plot(myolsmod)




\end{frame}

\begin{frame}[plain]

\includegraphics[width=10cm]{plots/t-diag09}

\end{frame}


\begin{frame}[plain]
  \frametitle{Tough to read the glm plot, IMHO$\ldots$}

\includegraphics[width=10cm]{plots/t-diag10}

\end{frame}




\begin{frame}[containsverbatim,allowframebreaks]
   \frametitle{\texttt{influence()} Function Digs up the Diagnostics}

\begin{Schunk}
\begin{Sinput}
  table(x)
\end{Sinput}
\begin{Soutput}
x
  0   1   2   3   4   5   6 
396 357 187  46  11   2   1 
\end{Soutput}
\end{Schunk}

\end{frame}

\begin{frame}[containsverbatim,allowframebreaks]
 \frametitle{\texttt{influence.measures()} A bigger collection of influence measures}

 From influence.measures, DFBETAS for each parameter, DFFITS, covariance ratios, Cook's distances and the diagonal elements of the hat matrix.


\begin{Schunk}
\begin{Sinput}
 dfb1 <- dfbeta(bush1)
 colnames(dfb1)
\end{Sinput}
\begin{Soutput}
[1] "(Intercept)"            
[2] "partyidDem."            
[3] "partyidInd. Near Dem."  
[4] "partyidIndependent"     
[5] "partyidInd. Near Repub."
[6] "partyidRepub."          
[7] "partyidStrong Repub."   
[8] "sexFemale"              
[9] "owngunYES"              
\end{Soutput}
\begin{Sinput}
 head(dfb1)
\end{Sinput}
\begin{Soutput}
   (Intercept) partyidDem. partyidInd. Near Dem.
1   -0.0052361    0.005286             0.0052149
4   -0.0052361    0.005286             0.0052149
5   -0.0059698    0.005023             0.0051036
9   -0.0052361    0.005286             0.0052149
10  -0.0005007    0.019143             0.0007462
11   0.0001594   -0.006095            -0.0002376
   partyidIndependent partyidInd. Near Repub.
1           0.0052232               0.0053054
4           0.0052232               0.0053054
5           0.0051290               0.0052763
9           0.0052232               0.0053054
10          0.0006130              -0.0007269
11         -0.0001952               0.0002315
   partyidRepub. partyidStrong Repub.  sexFemale
1      0.0053094            5.274e-03 -0.0004822
4      0.0053094            5.274e-03 -0.0004822
5      0.0052130            5.165e-03  0.0009737
9      0.0053094            5.274e-03 -0.0004822
10    -0.0008014           -2.216e-04  0.0080812
11     0.0002552            7.056e-05 -0.0025732
   owngunYES
1   0.000635
4   0.000635
5   0.000730
9   0.000635
10 -0.010400
11  0.003312
\end{Soutput}
\end{Schunk}


     Can get component columns directly with ‘dfbetas’, ‘dffits’, ‘covratio’ and ‘cooks.distance’.
\end{frame}




\begin{frame}[containsverbatim,allowframebreaks]
 \frametitle{But if You Want dfbeta, Not dfbetas, Why Not Ask?}


\begin{Schunk}
\begin{Sinput}
  dat$grpfac1 <- factor(dat$grp)
  str(dat$grpfac1)
\end{Sinput}
\begin{Soutput}
 Factor w/ 3 levels "1","2","3": 1 2 1 3 1 1 2 2 2 1 ...
\end{Soutput}
\begin{Sinput}
  with(dat, table(grpfac1, grp))
\end{Sinput}
\begin{Soutput}
       grp
grpfac1 1 2 3
      1 6 0 0
      2 0 7 0
      3 0 0 6
\end{Soutput}
\end{Schunk}


      I wondered what dfbetas does. You can see for yourself. Look at
      the code. Run:

\begin{Schunk}\begin{Soutput}
    >  stats:::dfbetas.lm
  \end{Soutput}
\end{Schunk}

\end{frame}



% > influencePlot(bush1)

% opens up an interactive plot--user left-clicks on
% the observations return their row number
% \end{frame}

% ===================================================
\section{Output}

\begin{frame}[containsverbatim]
  \frametitle{You Will Want to Use \LaTeX{}     After You See This}

  \begin{itemize}
    \item How do you get regression tables out of your project?
    \item Do you go through error-prone copying, pasting, typing,
      tabling, etc?
    \item What if your software could produce a finished
      publishable table?
    \end{itemize}
  \end{frame}

\begin{Schunk}
\begin{Sinput}
  dat$grpfac1 <- factor(dat$grp, labels = c("Number1", "Number2", "Number3"))
  str(dat$grpfac1)
\end{Sinput}
\begin{Soutput}
 Factor w/ 3 levels "Number1","Number2",..: 1 2 1 3 1 1 2 2 2 1 ...
\end{Soutput}
\begin{Sinput}
  with(dat, table(grp, grpfac1))
\end{Sinput}
\begin{Soutput}
   grpfac1
grp Number1 Number2 Number3
  1       6       0       0
  2       0       7       0
  3       0       0       6
\end{Soutput}
\end{Schunk}


\begin{frame}[containsverbatim]
  \frametitle{}
  \begin{itemize}
    \item Years ago, I wrote a function ``outreg''





    \item This command:

\input{plots/t-output10}
   \item Produces the output on the next slide
  \end{itemize}
\end{frame}

\begin{frame}[plain,containsverbatim]


\input{plots/t-output40}
\end{frame}




\begin{frame}[containsverbatim]
  \frametitle{Polish that up}

  \begin{itemize}
    \item you can beautify the variable labels, either by specifying
      them in the outreg command or editing the table output.
    \item outreg produces Latex that looks like this in the R session output.

\input{plots/t-output41}

\end{itemize}
\end{frame}


\begin{frame}
  \frametitle{Push Several Models Into One Wide Table}

\input{plots/t-output50}

  Sorry, I had to split this manually across 3 slides :(

\end{frame}


\begin{frame}[plain,allowframebreaks]


\footnotesize{
\input{plots/t-output51}
}
\end{frame}

%__________________________________


\begin{frame}
  \frametitle {R Packages for Producing Regression Output}

  \begin{itemize}
  \item memisc: works well, further from final form than outreg

  \item xtable: incomplete output, but latex or HTML works

  \item apsrtable: very similar to outreg

  \item Hmisc ``latex'' function

  \end{itemize}
\end{frame}


\begin{frame}[containsverbatim,allowframebreaks]

\input{plots/t-output70}


\end{frame}

\begin{frame}[containsverbatim]
  \frametitle{If you Can't Shake the MS Word ``Habit''}

  The best you can do is HTML output, which you can copy paste-special
  into a document.

\input{plots/t-output80}
\end{frame}

\begin{frame}[containsverbatim,allowframebreaks]
  \frametitle{memisc mtable is nice for comparing models (except for
    verbosity of parameter labels)}

\begin{Schunk}
\begin{Sinput}
   x <- c("Y","N","Y","Y","F","N")
  is.factor(x)
\end{Sinput}
\begin{Soutput}
[1] FALSE
\end{Soutput}
\begin{Sinput}
  is.character(x)
\end{Sinput}
\begin{Soutput}
[1] TRUE
\end{Soutput}
\end{Schunk}

\end{frame}


\begin{frame}[containsverbatim,allowframebreaks]
  \frametitle{memisc toLatex}

\footnotesize{
\input{plots/t-output84}
}
\end{frame}

\begin{frame}
  \frametitle{Relable Levels to Truncate Output}

  \begin{itemize}
    \item We could have to edit that output A LOT


    \item Hack the Labels down
\begin{Schunk}
\begin{Sinput}
 myxm <- round(myxm, 2)
 myxsd <- round(myxsd, 2)
\end{Sinput}
\end{Schunk}


    \item Re-run the models

\begin{Schunk}
\begin{Sinput}
  levels(xf1)
\end{Sinput}
\begin{Soutput}
[1] "F" "N" "Y"
\end{Soutput}
\end{Schunk}

\end{itemize}
\end{frame}

%___________________________

\begin{frame}[containsverbatim,allowframebreaks,plain]

\footnotesize{
\input{plots/t-output85}
}



\end{frame}


%===================================================
\end{document}
