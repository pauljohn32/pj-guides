\batchmode
\makeatletter
\def\input@path{{/home/pauljohn/SVN/SVN-guides/Rcourse/regression-plots-1//}}
\makeatother
\documentclass[10pt,english]{beamer}
\usepackage{lmodern}
\renewcommand{\sfdefault}{lmss}
\renewcommand{\ttdefault}{lmtt}
\usepackage[T1]{fontenc}
\usepackage[utf8]{inputenc}
\usepackage{listings}
\setcounter{secnumdepth}{3}
\setcounter{tocdepth}{3}

\makeatletter
%%%%%%%%%%%%%%%%%%%%%%%%%%%%%% Textclass specific LaTeX commands.
\usepackage{Sweavel}
 \def\lyxframeend{} % In case there is a superfluous frame end
 \newenvironment{topcolumns}{\begin{columns}[t]}{\end{columns}}

%%%%%%%%%%%%%%%%%%%%%%%%%%%%%% User specified LaTeX commands.
\usepackage{dcolumn}
\usepackage{booktabs}

% use 'handout' to produce handouts
%\documentclass[handout]{beamer}
\usepackage{wasysym}
\usepackage{pgfpages}
\newcommand{\vn}[1]{\mbox{{\it #1}}}\newcommand{\vb}{\vspace{\baselineskip}}\newcommand{\vh}{\vspace{.5\baselineskip}}\newcommand{\vf}{\vspace{\fill}}\newcommand{\splus}{\textsf{S-PLUS}}\newcommand{\R}{\textsf{R}}


\usepackage{graphicx}
\usepackage{listings}
\lstset{tabsize=2, breaklines=true,style=Rstyle}
\usetheme{Antibes}
% or ...

%\setbeamercovered{transparent}
% or whatever (possibly just delete it)


% In document Latex options:
\fvset{listparameters={\setlength{\topsep}{0em}}}
\def\Sweavesize{\normalsize} 
\def\Rcolor{\color{black}} 
\def\Rbackground{\color[gray]{0.95}}

\mode<presentation>

\newcommand\makebeamertitle{\frame{\maketitle}}%

\setbeamertemplate{frametitle continuation}[from second]
\renewcommand\insertcontinuationtext{...}

\expandafter\def\expandafter\insertshorttitle\expandafter{%
 \insertshorttitle\hfill\insertframenumber\,/\,\inserttotalframenumber}

\makeatother

\usepackage{babel}
\begin{document}

% In document Latex options:
\fvset{listparameters={\setlength{\topsep}{0em}}}

\def\Sweavesize{\scriptsize} 
\def\Rcolor{\color{black}} 
\def\Rbackground{\color[gray]{0.97}}

\begin{Schunk}
\begin{Sinput}
 ## Other settings I like
 options(device = pdf)
 options(useFancyQuotes = FALSE) 
 op <- par() 
 pjmar <- c(5.1, 5.1, 1.5, 2.1) 
 options(SweaveHooks=list(fig=function() par(mar=pjmar, ps=12)))
 pdf.options(onefile=F,family="Times",pointsize=12)
\end{Sinput}
\end{Schunk}



\title[Descriptive]{Regression Presentations: Plots }


\author{Paul E. Johnson\inst{1} \and \inst{2}}


\institute[K.U.]{\inst{1}Department of Political Science\and \inst{2}Center for
Research Methods and Data Analysis, University of Kansas}


\date[2013]{2013}

\makebeamertitle

\lyxframeend{}

\AtBeginSubsection[]{
  \frame<beamer>{ 
    \frametitle{Outline}   
    \tableofcontents[currentsection,currentsubsection] 
  }
}

\begin{frame}

\frametitle{Presenting Regressions}

We Need
\begin{itemize}
\item Nice Looking Regression Tables (that's a separate set of notes. Look
for regression-tables-1)
\item Calculate \& plot predicted values and confidence intervals
\item This lecture is about 2-d plots
\item For 3-d plots, look for plots-3d
\end{itemize}
\end{frame}

\begin{frame}
\frametitle{Orientation \#1}
\begin{itemize}
\item One of the primary points of emphasis in the S \& R languages is the
development of a consistent framework for the estimation and probing
of regression models.
\item We fit a model, then ``interrogate'' it.
\item A predict method should be supplied with \emph{any }regression model. 

\begin{itemize}
\item provides predicted values (and, hopefully, confidence intervals) for
user-specified input values. 
\item If there is no predict method, then we are in the stone age. We have
to manually extract the slope coefficients (\texttt{coef(m1)}), create
a matrix of interesting predictor values (\texttt{X}), and calculate
\texttt{X \%*\% coef(m1)}. 
\end{itemize}
\end{itemize}
\end{frame}

\begin{frame}
\frametitle{predict method: key arguments}

predict() is a generic function, each regression model is responsible
to implement predict.''model''()
\begin{description}
\item [{object:}] a fitted regression model
\item [{newdata:}] values of predictors for which predictions are sought
\item [{interval:}] possibly a confidence or prediction interval (may not
be available, depending on model type)
\item [{type:}] Predictions may be supplied on various scales. In GLM,
for example ``response'' or ``link''
\end{description}
\end{frame}

\begin{frame}
\frametitle{The creation of a "newdata" object is a struggle}
\begin{itemize}
\item If \texttt{X \%*\% coef(m1)} is the stone age, then predict with
newdata is the golden age of Rock and Roll. 
\item Various suggestions have been offered to streamline the creation of
newdata objects and calculate predictions. These things put us in
the Space Age.

\begin{description}
\item [{Space~Age:}] technological things work well, except when they
don't, and we all feel helpless when that happens.
\end{description}
\item Because the Space Age tools sometimes fail, we still teach the ``older
ways'' of predict and newdata. But we wish we didn't have to because
it makes the students cry.
\end{itemize}
\end{frame}

\begin{frame}[containsverbatim, allowframebreaks]
\frametitle{Glimpse the Space Age}
\begin{itemize}
\item Let's stop right here to get an orientation. Lets all run this:


\begin{Schunk}
\begin{Sinput}
 help.search("inverse gaussian")
 
\end{Sinput}
\end{Schunk}


\item In the following, I want to help you understand what's going on in
there and why it is useful.
\end{itemize}
\end{frame}


\lyxframeend{}\section{Data}

\begin{frame}

\frametitle{Example Data Used in these Notes}
\begin{itemize}
\item American National Election Study (2004), recoding details follow.
\item Chile dataset in car package
\item Simulated data (rockchalk package genCorrelatedData)
\end{itemize}
\end{frame}

\begin{frame}[containsverbatim,allowframebreaks]
\frametitle{Re-use the Coding for the American National Election Study 2004}


\inputencoding{latin9}\begin{lstlisting}[basicstyle={\footnotesize}]
## V043038     B1a. Feeling Thermometer: GW Bush
## V043039     B1b. Feeling Thermometer: John Kerry
## V043210     R1. R position on gay marriage
## V043213     S3. National economy better/worse since GW Bush took ofc
## V045117     G4a. Liberal/conservative 7-point scale: self-placement
## V045145X    H5x. Summary: Pre-Post US flag makes R feel
## V041109A    HHListing.9a. Respondent gender
## V043116     J1x. Summary: R party ID
## V043250     Y1x. Summary: Respondent age
\end{lstlisting}
\inputencoding{utf8}

\end{frame}

\begin{frame}[containsverbatim]
\frametitle{Thermometer Scores for Bush and Kerry}


\includegraphics[width=10cm]{plots/t-anes10}

\end{frame}

\begin{frame}[containsverbatim,allowframebreaks]
\frametitle{Create a New Dependent Variable}


The difference in thermometer scores:

\begin{Schunk}
\begin{Sinput}
 mydta1$th.bush.kerry <- mydta1$V043038 - mydta1$V043039
\end{Sinput}
\end{Schunk}



Clean up a bunch of variables \& value labels

\begin{Schunk}
\begin{Sinput}
 ##Party
 mydta1$V043116 <- mydta1$V043116[, drop = TRUE]
 levels(mydta1$V043116) <- c("SD","WD","ID","I","IR","WR","SR","O")
 mydta1$V043116[ mydta1$V043116 %in% levels(mydta1$V043116)[8] ] <- NA
 mydta1$V043116 <- mydta1$V043116[, drop = TRUE]
 table(mydta1$V043116)
\end{Sinput}
\begin{Soutput}
 SD  WD  ID   I  IR  WR  SR 
203 179 210 118 138 154 193 
\end{Soutput}
\begin{Sinput}
 ##IDEO
 mydta1$V045117 <- mydta1$V045117[ , drop = TRUE]
 levels(mydta1$V045117) <- c("EL","L","SL","M","SC","C","EC")
 table(mydta1$V045117)
\end{Sinput}
\begin{Soutput}
 EL   L  SL   M  SC   C  EC 
 20 103 125 279 143 166  31 
\end{Soutput}
\begin{Sinput}
 ##Gender
 levels(mydta1$V041109A) <- c("M","F")
 ## Gay Marriage
 levels(mydta1$V043210) 
\end{Sinput}
\begin{Soutput}
[1] "1. Should be allowed"                                    "3. Should not be allowed"                               
[3] "5. Should not be allowed to marry but should be allowed" "VOL"                                                    
[5] "8. Don't know"                                           "9. Refused"                                             
\end{Soutput}
\begin{Sinput}
 mydta1$V043210[ mydta1$V043210 %in% levels(mydta1$V043210)[4:10] ] <-NA  
 mydta1$V043210 <- mydta1$V043210[, drop = TRUE]
 levels(mydta1$V043210) <- c("Allow","No","Med") 
 ## Economy
 mydta1$V043213 <- mydta1$V043213[ , drop = TRUE]
 l <- levels(mydta1$V043213)
 econnew <- factor(mydta1$V043213, levels=c(l[2],l[3],l[1]), labels=c("Worse","Same","Better"))
 table(mydta1$V043213, econnew)
\end{Sinput}
\begin{Soutput}
             econnew
              Worse Same Better
  1. Better       0    0    190
  3. Worse      668    0      0
  5. The same     0  343      0
\end{Soutput}
\begin{Sinput}
 mydta1$V043213 <- econnew
 rm(econnew)
 ##Flag
 mydta1$V045145X <- mydta1$V045145X[, drop = TRUE] 
\end{Sinput}
\end{Schunk}


\end{frame}


\lyxframeend{}\section{Plotting Regressions (2d)}

\begin{frame}[containsverbatim]
\frametitle{Plotting Regressions: Here is the General Plan}

\begin{topcolumns}%{}


\column{5cm}
\begin{itemize}
\item Put a continuous predictor on the horizontal axis ($x1_{i}$)
\item Put the dependent variable on the vertical axis ($y_{i}$)
\item Calculate $\hat{y}_{i}=\hat{b}_{0}+\hat{b}_{1}x1_{i}$ and plot that
``predicted line''
\end{itemize}

\column{7cm}

\includegraphics[width=7cm]{plots/t-plot10}
\end{topcolumns}%{}
\end{frame}


\lyxframeend{}\subsection{abline}

\begin{frame}[containsverbatim]
\frametitle{Previous plot requires 3 Easy Steps (That \emph{Almost Every} R user has taken)}

1) lm 2) plot 3) abline.

Example:

\begin{Schunk}
\begin{Sinput}
 ## Step 1
 mod1age <- lm(th.bush.kerry ~ V043250, data = mydta1)
 ## Step 2
 plot(th.bush.kerry ~ V043250, type = "p", xlab = "V403250: Age", ylab = "Bush-Kerry thermometers", data = mydta1, col = gray(.7))
 ## Step 3
 abline(mod1age, lwd = 1.5)
\end{Sinput}
\end{Schunk}


\end{frame}

\begin{frame}[containsverbatim, allowframebreaks]
\frametitle{abline: behind the scenes.}
\begin{itemize}
\item abline() is a general purpose ``straight line drawing tool''. It
can receive input in the form of coefficients, or, if you provide
it with a regression object, it will notice and try to get the coefficients
from the regression object.
\item At the command line, run abline, you'll see it. Look for ``if (is.object(a)
|| is.list(a)) \{''
\end{itemize}
\begin{Schunk}
\begin{Sinput}
 abline
\end{Sinput}
\begin{Soutput}
function (a = NULL, b = NULL, h = NULL, v = NULL, reg = NULL, 
    coef = NULL, untf = FALSE, ...) 
{
    int_abline <- function(a, b, h, v, untf, col = par("col"), 
        lty = par("lty"), lwd = par("lwd"), ...) .External.graphics(C_abline, 
        a, b, h, v, untf, col, lty, lwd, ...)
    if (!is.null(reg)) {
        if (!is.null(a)) 
            warning("'a' is overridden by 'reg'")
        a <- reg
    }
    if (is.object(a) || is.list(a)) {
        p <- length(coefa <- as.vector(coef(a)))
        if (p > 2) 
            warning(gettextf("only using the first two of %d regression coefficients", 
                p), domain = NA)
        islm <- inherits(a, "lm")
        noInt <- if (islm) 
            !as.logical(attr(stats::terms(a), "intercept"))
        else p == 1
        if (noInt) {
            a <- 0
            b <- coefa[1L]
        }
        else {
            a <- coefa[1L]
            b <- if (p >= 2) 
                coefa[2L]
            else 0
        }
    }
    if (!is.null(coef)) {
        if (!is.null(a)) 
            warning("'a' and 'b' are overridden by 'coef'")
        a <- coef[1L]
        b <- coef[2L]
    }
    int_abline(a = a, b = b, h = h, v = v, untf = untf, ...)
    invisible()
}
<bytecode: 0x338ea60>
<environment: namespace:graphics>
\end{Soutput}
\end{Schunk}


\end{frame}

\begin{frame}[containsverbatim]
\frametitle{Getting away from abline()}

This lecture, it turns out, is mostly about getting away from the
limitations of abline()!
\begin{itemize}
\item limitations of abline()

\begin{enumerate}
\item Doesn't work when there are more predictors.
\item Doesn't work with nonlinear models or glm or ....
\end{enumerate}
\item The R paradigm suggests instead a 3 step sequence.

\begin{enumerate}
\item newdata: create an exemplar data set of interesting values for which
we might want predictions
\item predict w/newdata: calculate predictions
\item plot!
\end{enumerate}
\end{itemize}
\end{frame}

\begin{frame}[containsverbatim]
\frametitle{We want to produce a plot like this}



\begin{Schunk}
\begin{Sinput}
 plotSlopes(mod1age, plotx = "V043250", lwd = 0.3)
\end{Sinput}
\end{Schunk}


\includegraphics[width=11cm]{plots/t-rcps10}

\end{frame}

\begin{frame}[containsverbatim]
\frametitle{Request the 95\% Confidence Interval}


\begin{Schunk}
\begin{Sinput}
 plotSlopes(mod1age, plotx = "V043250", interval = "confidence", lwd = 0.3)
\end{Sinput}
\end{Schunk}


\includegraphics[width=11cm]{plots/t-rcps20}

\end{frame}

\begin{frame}[containsverbatim, allowframebreaks]
\frametitle{The newdata -> predict -> plot }

I made that with rockchalk::plotSlopes(). In a model with only 1 predictor,
it is pretty easy to make a plot like that.
\begin{enumerate}
\item Create a set of ``focal values'': possible values of ``x'' for
which we need predictions. 

\begin{itemize}
\item In a one-predictor linear model, ``two points define a straight line''.
We might as well take the target values at the edges of the data: 

\begin{itemize}
\item \inputencoding{latin9}
\begin{lstlisting}
V03250r <- range(mydta1$V03250, na.rm = TRUE)
\end{lstlisting}
\inputencoding{utf8}
\end{itemize}
\item However, if we want confidence intervals, we'll need an evenly spaced
sequence of points, something like

\begin{itemize}
\item \inputencoding{latin9}
\begin{lstlisting}
V03250r <- rockchalk::plotSeq(mydta1$V03250, length.out = 40)	
\end{lstlisting}
\inputencoding{utf8}
\end{itemize}
\end{itemize}
\item Create a new data frame using the new sequence as the independent
variable


\inputencoding{latin9}\begin{lstlisting}
ndf <- data.frame(V03250 = V03250r)
\end{lstlisting}
\inputencoding{utf8}

\item Use predict to get predictions, one for each row in the new data frame.


\inputencoding{latin9}\begin{lstlisting}
predict(m0, newdata = ndf)
\end{lstlisting}
\inputencoding{utf8}


or possibly


\inputencoding{latin9}\begin{lstlisting}
predict(m0, newdata = ndf, interval = "confidence")
\end{lstlisting}
\inputencoding{utf8}

\end{enumerate}
\end{frame}

\begin{frame}[containsverbatim, allowframebreaks]
\frametitle{This Code Would Work}


\begin{Schunk}
\begin{Sinput}
 ndf <- data.frame(V043250 = range(mydta1$V043250))
 ndf$fit <- predict(mod1age, newdata = ndf )
 head(ndf)
\end{Sinput}
\begin{Soutput}
  V043250       fit
1      18 -3.524611
2      90  9.742260
\end{Soutput}
\begin{Sinput}
 plot(th.bush.kerry ~ V043250, data = mydta1, xlab = "V403250: Age", ylab = "th.bush.kerry = Bush-Kerry thermometers",  col = gray(.7)) ## Scatterplot
 lines(fit ~ V043250, data = ndf, lwd = 1.5)
\end{Sinput}
\end{Schunk}


\end{frame}


\lyxframeend{}\subsection{Confidence Intervals from predict}

\begin{frame}[containsverbatim]
\frametitle{Confidence Intervals}
\begin{itemize}
\item predict methods for some regression models can calculate 95\% intervals
of various type.
\item lm: 

\begin{itemize}
\item interval=''confidence'' to obtain the 95\% confidence intervals
for the estimate of the expected value of $y$ given the predictors
\item interval=''predict'' to obtain the 95\% confidence intervals for
the outcome $y_{i}$ given the predictors.
\end{itemize}
\item When intervals are requested, predict returns a matrix with columns
called


\inputencoding{latin9}\begin{lstlisting}
fit lwr upr
\end{lstlisting}
\inputencoding{utf8}


Predicted (``fitted'') values, and the lower and upper limits of
the intervals.

\item R's predict.glm does not supply intervals (complicated statistical
controversy behind that), but they will provide standard errors of
fitted values, from which we can create our own confidence intervals
\end{itemize}
\end{frame}

\begin{frame}[containsverbatim]
\frametitle{Simple example to plot Confidence and Prediction Intervals}


\def\Sweavesize{\scriptsize}
\begin{Schunk}
\begin{Sinput}
 x <- rnorm(100, m = 50, s = 20)
 y <- 3 + 0.2 * x + 15 * rnorm(100)
 mp1 <- lm(y ~ x)
 ndf <- data.frame(x = plotSeq(x,40))
 p1 <- as.data.frame(predict(mp1, interval="conf", newdata = ndf))
 p2 <- as.data.frame(predict(mp1, interval="pred", newdata = ndf))
 plot(x, y, ylim = magRange(y, 1.3))
 lines(ndf$x, p1$fit, col = "black", lwd=3)
 lines(ndf$x, p1$lwr, col = "red", lwd = 3, lty = 4)
 lines(ndf$x, p1$upr, col = "red", lwd = 4, lty = 4)
 lines(ndf$x, p2$lwr, col = "purple", lwd = 4, lty = 3)
 lines(ndf$x, p2$upr, col = "purple", lwd = 4, lty = 3)
 legend("topleft", legend = c("predicted","Conf. Interval", "Pred. Interval"), col = c("black", "red", "purple"), lty = c(1, 4,3), cex = 0.8, lwd = 3)
\end{Sinput}
\end{Schunk}


\end{frame}

\begin{frame}[containsverbatim]
\frametitle{Difference b/t Conf and Pred Intervals}

\includegraphics[width=11cm]{plots/t-plint10}

\end{frame}

\begin{frame}[containsverbatim]
\frametitle{Confidence Intervals for the Bush-Kerry Regression}




\begin{Schunk}
\begin{Sinput}
 ndf1 <- data.frame(V043250 = range(mydta1$V043250))
 mod1pred <- predict(mod1age, newdata = ndf1, interval="confidence")
 ndf1 <- cbind(ndf1, mod1pred)
 lines(fit ~ V043250, data = ndf1, lwd = 3, lty = 1)
 lines(lwr ~ V043250, data = ndf1, lwd = 2, lty = 2)
 lines(upr ~ V043250, data = ndf1, lwd = 2, lty = 2)
 legend("topleft", legend = c("predicted","95% CI"), lty = c(1,2), lwd = c(3,2), bg = "white")
\end{Sinput}
\end{Schunk}


\end{frame}

\begin{frame}
\frametitle{The Fitted (Predicted) Value and 95\% Confidence Interval}

\includegraphics[width=11cm]{plots/t-plot17}

\end{frame}


\lyxframeend{}\section{Multiple Predictors}


\lyxframeend{}\subsection{newdata}

\begin{frame}[containsverbatim]
\frametitle{What's that "newdata" argument in predict?}
\begin{itemize}
\item Run predict without 'newdata', what do you get?

\begin{itemize}
\item one prediction for each row in the data, same as output from fitted()


\begin{Schunk}
\begin{Sinput}
 fit <- predict(mod1age)
 head(fit)
\end{Sinput}
\begin{Soutput}
          1           2           3           4           5           6 
 2.37177573  1.81898943 -0.02363155  6.24127979  4.58292091  2.92456202 
\end{Soutput}
\begin{Sinput}
 head(fitted(mod1age))
\end{Sinput}
\begin{Soutput}
          1           2           3           4           5           6 
 2.37177573  1.81898943 -0.02363155  6.24127979  4.58292091  2.92456202 
\end{Soutput}
\end{Schunk}


\end{itemize}
\item That does not simplify anything. 
\end{itemize}
\end{frame}

\begin{frame}[containsverbatim]
\frametitle{The Transition to Multiple Regression}
\begin{itemize}
\item We want a prediction for a particular combination of predictors, to
answer ``what would happen if?'' questions.

\begin{itemize}
\item e.g., I need to know what young left-handed women feel about something
\end{itemize}
\item How to choose particular values of predictors? 

\begin{itemize}
\item quantiles?
\item mean +/- standard deviation?
\end{itemize}
\item Making a newdata objects that work requires some discipline because
values must be included for all variables.
\end{itemize}
\end{frame}

\begin{frame}
\frametitle{Building a newdata object}
\begin{itemize}
\item Step 1. For each variable in the model, create a collection of one
or more example values for which predictions are to be calculated
(can use quantiles, means, medians, or anything else).
\item Step 2. R's expand.grid() will create a ``mix and match'' data frame. 


\begin{Schunk}
\begin{Sinput}
 focalValues <- data.frame(x1 = c(1, 2, 3, 4), x2 = c(100,200))
 mAndM <- expand.grid(focalValues)
 mAndM[1:8, ]
\end{Sinput}
\begin{Soutput}
  x1  x2
1  1 100
2  2 100
3  3 100
4  4 100
5  1 200
6  2 200
7  3 200
8  4 200
\end{Soutput}
\end{Schunk}


\end{itemize}
\end{frame}

\begin{frame}[containsverbatim]
\frametitle{Tricky Thing about newdata argument}

predict(m1, newdata = mAndM) will fail if
\begin{itemize}
\item newdata does not include one named column for each variable included
in the model (here, m1)
\item any values for factor variables included in mAndM are not valid labels. 
\item The model was specified in an unexpected way.

\begin{description}
\item [{BAD:}] \inputencoding{latin9}\lstinline!m1 <- lm(dat$y ~ dat$x1 + dat$x2)!\inputencoding{utf8}
\item [{GOOD:}] \inputencoding{latin9}\lstinline!m1 <- lm(y ~ x1 + x2, data = dat)!\inputencoding{utf8}
\end{description}
\item The model includes transformations like \texttt{as.numeric(x)} or
\texttt{as.factor(x)}. These complicate the creation of newdata quite
a bit.
\end{itemize}
\end{frame}

\begin{frame}[containsverbatim, allowframebreaks]
\frametitle{Here's an example that works with the Chile data}
\begin{itemize}
\item Using the Chile data from the car package, with 2 numeric (income,
age) and 1 factor (sex).


\begin{Schunk}
\begin{Sinput}
 require(car)
 data(Chile)
 Chile$income <- Chile$income/1000
 m2 <- lm(statusquo ~ income + age + sex, data=Chile)
\end{Sinput}
\end{Schunk}


\item statusquo : scale score of support for the status quo
\item sex: M or F
\item income (pesos, rescaled to 1000s of pesos)
\item region (5 regions of Chile)
\end{itemize}
\begin{itemize}
\item Could use means, medians, or whatever else strikes our fancy. This
creates just one example row


\begin{Schunk}
\begin{Sinput}
 par(mfcol = c(3, 2))  # Fill columns first
 par(mfrow = c(3, 2))  # Fill rows first
\end{Sinput}
\end{Schunk}


\end{itemize}
\end{frame}

\begin{frame}[containsverbatim, allowframebreaks]
\frametitle{Incidentally, Concerning Factors in newdata}
\begin{itemize}
\item If we spell the factor levels incorrectly, this won't run.
\item Thus there is an argument in favor of not spelling them, but rather
asking for the valid levels and then using them in the newdata 


\begin{Schunk}
\begin{Sinput}
 sexlev <- levels(Chile$sex)
 newdf <- data.frame(income=c(mean(Chile$income, na.rm = T)),  age = c(55), sex = sexlev[1])
 newdf$fit <- predict(m1, newdata = newdf)
\end{Sinput}
\end{Schunk}


\item The valid levels are held in sexlev, and sexlev{[}1{]} is the first
valid level.
\end{itemize}
\end{frame}

\begin{frame}
\frametitle{Getting good results with newdata will require practice}
\begin{itemize}
\item Be patient, it can be made to work.
\item Lets pause for some practice. There are several WorkingExamples that
have newdata elements. I suggest that everybody should step through
number 1, then pick randomly between 2-4.

\begin{enumerate}
\item regression-plot-factor-01.R
\item regression-predictedPlots-01.R
\item regression-predictedPlots-02.R
\item regression-quadratic-1.R
\end{enumerate}
\item Please open one and ``step through'' to see what happens
\end{itemize}
\end{frame}


\lyxframeend{}\section{termplot}

\begin{frame}
\frametitle{termplot is provided with R}
\begin{itemize}
\item Because termplot is provided with R, it is my favorite tool for a
``first quick look'' at the regression model.
\item It

\begin{itemize}
\item has proper displays for both numeric and factor predictors
\item has many arguments for customization of plots
\item Automatically graphs each predictor, one after the other
\end{itemize}
\item Disadvantages

\begin{itemize}
\item does not handle models with interaction effects
\item does not respond correctly to some complicated model formulae
\end{itemize}
\end{itemize}
\end{frame}

\begin{frame}[containsverbatim]
\frametitle{Look at a small model on the Chile data}

\begin{Schunk}
\begin{Sinput}
 ps3b <- plotSlopes(mod1, plotx = "x1", modx = "x3", modxVals = c("right"))
\end{Sinput}
\end{Schunk}


Will prepare a simple line plot for each variable, I'll show you ``income''


\includegraphics[width=10cm]{plots/t-tpchile05}

\end{frame}

\begin{frame}[containsverbatim]
\frametitle{I almost always add se and partial.resid arguments}

Richer display results

\begin{Schunk}
\begin{Sinput}
 predictOMatic(m1, predVals = list("x1" = c(0, 10, 20, 30, 40)), se.fit = TRUE, interval = "confidence")
\end{Sinput}
\begin{Soutput}
  x1        fit        lwr      upr fit$se.fit
1  0 -1.0800436 -3.0955963 0.935509  1.0156642
2 10  0.8916973 -0.5613787 2.344773  0.7322247
3 20  2.8634382  1.9246751 3.802201  0.4730554
4 30  4.8351792  4.2252686 5.445090  0.3073422
5 40  6.8069201  6.0429997 7.570841  0.3849498
\end{Soutput}
\end{Schunk}



\includegraphics[width=10cm]{plots/t-tpchile10}

\end{frame}

\begin{frame}
\frametitle{Look at a small model on the Chile data}


\includegraphics[width=10cm]{plots/t-tpchile20}

\end{frame}

\begin{frame}
\frametitle{Look at a small model on the Chile data}


\includegraphics[width=10cm]{plots/t-tpchile30}

\end{frame}

\begin{frame}[containsverbatim,allowframebreaks]
\frametitle{Run a big model}


\begin{Schunk}
\begin{Sinput}
 m3 <- lm (th.bush.kerry ~ V045117 + V043116 + V043210 + V043213 + V045145X + V041109A, data = mydta1)
\end{Sinput}
\end{Schunk}

\begin{itemize}
\item The interesting thing about this is that the model generates a huge
set of coefficients, but
\item termplot notices which variable corresponds to each set and makes
a graceful plot
\end{itemize}
\end{frame}

\begin{frame}[containsverbatim,allowframebreaks]
\frametitle{The Fitted Regression Model}

\scriptsize{
\begin{Schunk}
\begin{Sinput}
 termplot(bush1,terms=c("partyid"))
\end{Sinput}
\end{Schunk}

}

\end{frame}

\begin{frame}[containsverbatim]
\frametitle{termplot will try to show these to you, one at a time}
\begin{itemize}
\item You run 


\inputencoding{latin9}\begin{lstlisting}
termplot(m3, se = TRUE, partial.resid = TRUE)
\end{lstlisting}
\inputencoding{utf8}


to see all of them. 


I have to ask ``by name'' in order to put this presentation together

\item se=T prints the 95\% ``confidence intervals'' for the predicted
values
\item partial.resid=T plots the ``partial residuals''
\end{itemize}
\end{frame}

\begin{frame}[containsverbatim]
\frametitle{termplot: Ideology}


\includegraphics[width=10cm]{plots/t-termplot40}

\end{frame}

\begin{frame}[containsverbatim]
\frametitle{termplot: Party ID}


\includegraphics[width=10cm]{plots/t-termplot41}

\end{frame}

\begin{frame}[containsverbatim]
\frametitle{termplot: Gay Marriage}


\includegraphics[width=10cm]{plots/t-termplot42}

\end{frame}

\begin{frame}[containsverbatim]
\frametitle{termplot: National Economy}


\includegraphics[width=10cm]{plots/t-termplot43}

\end{frame}

\begin{frame}[containsverbatim]
\frametitle{termplot: The Flag}


\includegraphics[width=10cm]{plots/t-termplot44}

\end{frame}

\begin{frame}[containsverbatim]
\frametitle{termplot: Gender}


\includegraphics[width=10cm]{plots/t-termplot45}

\end{frame}

\begin{frame}[containsverbatim]
\frametitle{termplot: Gender w/o partial residuals}


\includegraphics[width=10cm]{plots/t-termplot46}

\end{frame}

\begin{frame}
\frametitle{Frustrating Detour: Why is the Vertical Axis called "partial"?}
\begin{itemize}
\item The ?termplot manual is difficult to understand, the meaning of a
``term predictions'' and ``partial residual'' is not stated. 
\item So I went and read the code. (Then we are more sympathetic to the
manual writer, because this is a complicated idea.)
\item Summary of the idea: Aim is to plot $y_{i}$ on $X1_{i}$, after ``subtracting
out'' effects of $X2_{i}$\textrm{, $X3{}_{i}$}
\item A ``Partial Residual'' is what is left unaccounted for by one variable
when the predictive effects of all of the other input variables have
been taken into account. 
\item If you are satisfied with that ``story'', skip the next few slides.
Otherwise, be brave and keep trying.
\end{itemize}
\end{frame}

\begin{frame}[containsverbatim, allowframebreaks]
\frametitle{Huh? What is the Partial Predicted Value?}
\begin{itemize}
\item Consider model \textrm{}\inputencoding{latin9}
\begin{lstlisting}
mod <- lm(y ~ X1 + X2 + X3, data = dat)
\end{lstlisting}
\inputencoding{utf8}
\item The ordinary ``predicted value'' is 


\[
\hat{y_{i}}=\hat{b}_{0}+\hat{b}_{1}X1_{i}+\hat{b}_{2}X2_{i}+\hat{b}_{3}X3_{i}
\]


\item The ``mean prediction'' is: 
\[
\hat{b}_{0}+\hat{b}_{1}\overline{X1}+\hat{b}_{2}\overline{X2}+\hat{b}_{3}\overline{X3}
\]

\item One old math professor trick is to add and subtract the same thing
from one side of an equation and then rearrange so that some new insight
flows out. Here, I will add and subtract the mean prediction from
the predicted value.
\item After re-grouping, we see the ``full model'' is the sum of the ``mean
prediction'' and ``term'' predictions
\begin{eqnarray}
\hat{y}_{i}= & \{\hat{b}_{0}+\hat{b}_{1}\overline{X1}+\hat{b}_{2}\overline{X2}+\hat{b}_{3}\overline{X3}\}\nonumber \\
 & +\{\hat{b}_{1}(X1_{i}-\overline{X1})+\hat{b}_{1}(X2_{i}-\overline{X2})+\hat{b}_{3}(X3_{i}-\overline{X3})\}
\end{eqnarray}

\item In R, predict(mod, type=''terms'') returns the last ``term predictions''
as separate columns:
\end{itemize}
\begin{equation}
\begin{array}{cccccc}
 & X1 &  & X2 &  & X3\\
 & \overbrace{\hat{b}_{1}(X1_{i}-\overline{X1})} & , & \overbrace{\hat{b}_{2}(X2_{i}-\overline{X2})} & , & \overbrace{\hat{b}_{3}(X3_{i}-\overline{X3})}
\end{array}
\end{equation}

\begin{itemize}
\item These are the predicted values used by termplot in its internal calculations
\end{itemize}
\end{frame}

\begin{frame}
\frametitle{And a partial residual is...}
\begin{itemize}
\item begin with the ``full model residual''
\end{itemize}
\begin{equation}
y_{i}-\hat{y}_{i}=y_{i}-\{\hat{b}_{0}+\hat{b}_{1}X1_{i}+\hat{b}_{2}X2_{i}+\hat{b}_{3}X3_{i}\}
\end{equation}

\begin{itemize}
\item Get the \textrm{$partial\, residual_{i}$} by replacing $X1_{i}$
with $\overline{X1}$: \textrm{
\begin{equation}
y-\{\hat{b}_{0}+\hat{b}_{1}\overline{X1}+\hat{b}_{2}X2_{i}+\hat{b}_{3}X3_{i}\}
\end{equation}
}

\begin{itemize}
\item If we ``do not allow variations in $X1$ to play a predictive role,
what remains unexplained?''
\end{itemize}
\item Another interpretation (See the source code in ``residuals.lm''):

\begin{itemize}
\item the partial residual = ``full residual'' + ``term prediction''
\begin{eqnarray*}
partial\, residual(X1_{i}) & =y_{i} & -\{\hat{b}_{0}+\hat{b}_{1}X1_{i}+\hat{b}_{2}X2_{i}+\hat{b}_{3}X3_{i}\}\\
 &  & +\{\hat{b}_{1}(X1_{i}-\overline{X1})\}\\
 & =y_{i}-\{ & \hat{b}_{0}+\hat{b}_{1}\overline{X1}+\hat{b}_{2}X2_{i}+\hat{b}_{3}X3_{i}\}
\end{eqnarray*}

\end{itemize}
\end{itemize}
\end{frame}

\begin{frame}[containsverbatim,allowframebreaks]
\frametitle{Reminder}
\begin{itemize}
\item termplot is a wonderful tool, but it suffers from one crippling flaw:
it does handle interaction terms.
\item That lead to

\begin{itemize}
\item many hours of teaching about predict and newdata in our intro regression
class
\item the creation of the rockchalk package to automate the management of
some common cases.
\end{itemize}
\end{itemize}
\end{frame}


\lyxframeend{}\section{rockchalk}

\begin{frame}[containsverbatim,allowframebreaks]
\frametitle{rockchalk functions}
\begin{description}
\item [{plotSlopes}] was the first effort there, and later it became apparent
that a general approach to automatically create newdata objects would
be a tremendous help (and allows predictOMatic to ``just work'').
\item [{newdata:}] scan regression model and construct ``newdata'' objects
that can be used by predict methods


This is user friendly, flexible. The focal values of several predictors
can be selected (by quantile, by standard deviations, a sequence that
walks across the observed values).

\item [{predictOMatic:}] Calculates predicted values for various predictor
values
\item [{plotCurves:}] handles nonlinear formulae and interactions among
nonlinear terms.
\end{description}
\end{frame}

\begin{frame}[containsverbatim,allowframebreaks]
\frametitle{newdata()}
\begin{itemize}
\item newdata() tries to do the right thing, allowing easy customization.
\item Consider the Chile data, with statusquo \textasciitilde{} income +
age + sex
\end{itemize}
\begin{Schunk}
\begin{Sinput}
 dat$xlog <- log(dat$x)
 dat$ylog <- log(dat$y)
 m1 <- lm(ylog ~ xlog, data = dat)
 head(model.matrix(m1))
\end{Sinput}
\end{Schunk}


\end{frame}

\begin{frame}[containsverbatim,allowframebreaks]
\frametitle{predictOMatic shows marginal effects}

If no particulars are requested, predictOMatic shows the marginal
effects of each separate predictor, keeping the other values at their
means (numeric variables) or modes (factor variables).

\begin{Schunk}
\begin{Sinput}
 dat$fh_pr_mc <- drop(scale(dat$fh_pr, scale = FALSE))
 newdf$fh_pr_mc <- plotSeq(dat$fh_pr_mc, 25)
\end{Sinput}
\end{Schunk}


\begin{Schunk}
\begin{Sinput}
 predictOMatic(m1mc, predVals = c("educationc"))
\end{Sinput}
\begin{Soutput}
  educationc type      fit
1  -4.415102   bc 26.09849
2  -2.350102   bc 35.93543
3  -0.190102   bc 46.22492
4   1.959898   bc 56.46677
5   5.174898   bc 71.78190
\end{Soutput}
\end{Schunk}


\end{frame}

\begin{frame}[containsverbatim,allowframebreaks]
\frametitle{plotSlopes uses that framework}
\begin{itemize}
\item We want to estimate an interactive model, by changing this:


\textrm{}\inputencoding{latin9}
\begin{lstlisting}
th.bush.kerry ~ V043250 + V041109A
\end{lstlisting}
\inputencoding{utf8}


$V041109A$ is respondent gender, coded ``M'' and ``F''. $V043250$
is age.

\item To this:


\textrm{}\inputencoding{latin9}
\begin{lstlisting}
th.bush.kerry ~ V043250 * V041109A
\end{lstlisting}
\inputencoding{utf8}

\item We can plot an age, thermometer line for each ``value'' of gender

\begin{itemize}
\item one line for ``M''
\item one line for ``F''. 
\end{itemize}
\end{itemize}
\end{frame}

\begin{frame}[containsverbatim,allowframebreaks]
\frametitle{plotSlopes uses that information}


\includegraphics[width=10cm]{plots/t-plotslopes10}

\end{frame}

\begin{frame}[containsverbatim,allowframebreaks]
\frametitle{How Did I Make That Plot?}
\begin{itemize}
\item Students thought it was very tedious to calculate predictions for
various moderators
\item TaDa! plotSlopes rockchalk simplifies this work.


\inputencoding{latin9}\begin{lstlisting}
plotSlopes(m2, plotx = "numericPredictor", modx = "aModeratorVariable")
\end{lstlisting}
\inputencoding{utf8}

\item It figures out which values of aModeratorVariable should be considered,
calculates predicted values for subgroups and draws them.
\item It returns an object that includes the newdata object.
\end{itemize}
\begin{Schunk}
\begin{Sinput}
 m6 <- lm(th.bush.kerry ~ V043250 * V041109A, data=mydta1)
 plotSlopes(m6, plotx = "V043250", modx = "V041109A", plotPoints = FALSE, plotLegend = FALSE,  xlab = "Age", ylab = "Bush-Kerry Thermometer", ylim = c(-100,150), col = c("black","gray60"), llwd=c(2,3))
 legend("topleft", legend = levels(mydta1$V041109A), col=c("black","gray60"), lty=c(1,2), lwd=c(2,3))
\end{Sinput}
\end{Schunk}


\end{frame}

\begin{frame}[containsverbatim,allowframebreaks]
\frametitle{Sometimes We Want To See The Points}


\includegraphics[width=8cm]{plots/t-plotslopes20}

Notice the non-beautified ``automatic'' legend

\end{frame}

\begin{frame}[containsverbatim, allowframebreaks]
\frametitle{Write that out completely, without plotSlopes}


\begin{Schunk}
\begin{Sinput}
 m6 <- lm(th.bush.kerry~V043250 + V041109A, data=mydta1)
 mycols <- c(gray(.7), rgb(0.7, 0.1, 0.1), "black", "red") 
 V043250r <- range(mydta1$V043250)
 newdf1 <- expand.grid(V043250 = V043250r,  V041109A = levels(mydta1$V041109A))
 newdf1$fit <- predict(m6, newdata = newdf1)
 plot(th.bush.kerry ~ V043250, type = "n", data = mydta1, xlab = "V043250: Age (years)", ylab = "Bush-Kerry Thermometer Differential")
 points(th.bush.kerry ~ V043250, data = mydta1,  lwd = 0.6, col = ifelse(V041109A %in% c("M"), mycols[1], mycols[2]))
 by(newdf1, newdf1$V041109A, function(x) lines(fit~V043250, data = x, col = mycols[2+as.numeric(x$V041109A)], lty = as.numeric(x$V041109A), lwd = 3 ))
\end{Sinput}
\begin{Soutput}
newdf1$V041109A: M
NULL
------------------------------------------------------------------------------------------------------------------------ 
newdf1$V041109A: F
NULL
\end{Soutput}
\begin{Sinput}
 legend("topleft", legend = c("Male", "Female"), col = mycols[3:4], lty = c(1,2), bg = "white", lwd = 3)
\end{Sinput}
\end{Schunk}


\end{frame}

\begin{frame}[containsverbatim]
\frametitle{Plot Predicted values by Gender}

\includegraphics[width=10cm]{plots/t-plot22B}

\end{frame}

\begin{frame}[containsverbatim]
\frametitle{Chile data: statusquo $\sim$ income * region + sex}


\def\Sweavesize{\scriptsize}
\begin{Schunk}
\begin{Sinput}
 termplot(chile1F, partial.resid=TRUE, se=TRUE, ylabs="Statusquo Support")
\end{Sinput}
\end{Schunk}


\end{frame}

\begin{frame}
\frametitle{Gender, Income, and Chilean Voter Attitudes}

\includegraphics[width=10cm]{plots/t-chile40}

\end{frame}

\begin{frame}[containsverbatim, allowframebreaks]
\frametitle{}


\begin{Schunk}
\begin{Sinput}
 m6ps <- plotSlopes(m6, plotx = "income", modx = "region", ylim = magRange(Chile$statusquo, c(1, 1.3)), llwd = c(2, 2), cex = 1.0)
 m6ps$newdata
\end{Sinput}
\begin{Soutput}
   region income sex         fit         lwr        upr
1      SA    2.5   F -0.23158507 -0.32180224 -0.1413679
2       S    2.5   F  0.22786977  0.12974686  0.3259927
3       C    2.5   F  0.02405915 -0.08384606  0.1319644
4       N    2.5   F  0.11756374 -0.03728760  0.2724151
5       M    2.5   F  0.14317264 -0.14128334  0.4276286
6      SA  200.0   F  0.25558282  0.03236837  0.4787973
7       S  200.0   F  0.31625024 -0.07781512  0.7103156
8       C  200.0   F  0.05668369 -0.32438301  0.4377504
9       N  200.0   F  0.76059507  0.12329258  1.3978975
10      M  200.0   F  2.13398804  0.67646454  3.5915115
\end{Soutput}
\end{Schunk}


\includegraphics[width=10cm]{plots/t-chile70}

\end{frame}


\lyxframeend{}\section{effects}

\begin{frame}[containsverbatim]
\frametitle{effects Package}
\begin{itemize}
\item There are several other R packages that help us to explore predicted
values.
\item One of the first packages for this purpose was effects, by John Fox,
Sanford Weisberg, and others.
\item rockchalk is doing same/similar set of calculations, but user interface
of rockchalk is (perhaps, just IMHO) more flexible \& pleasant.
\item plots from effects are not so nice as plots from termplot.
\item Other packages that have a similar orientation are rms and Zelig
\end{itemize}
\end{frame}

\begin{frame}[containsverbatim]
\frametitle{effects Package}



\begin{Schunk}
\begin{Sinput}
 library(effects)
 mod1.ae <- allEffects(mod1)
 plot(mod1.ae, "V041109A")
\end{Sinput}
\end{Schunk}


\end{frame}

\begin{frame}[containsverbatim]
\frametitle{Effects Plot: Gender}

\includegraphics[width=10cm]{plots/t-effects11}

\end{frame}

\begin{frame}[containsverbatim]
\frametitle{effects Package}



\begin{Schunk}
\begin{Sinput}
 plot(mod1.ae, "V043116", xlab="V043116: Party Identification")
\end{Sinput}
\end{Schunk}


\end{frame}

\begin{frame}[containsverbatim]
\frametitle{Effects Plot: Party Identification}

\includegraphics[width=10cm]{plots/t-effects21}

\end{frame}

\begin{frame}[containsverbatim]
\frametitle{Effects Plot: Ideology}



\includegraphics[width=10cm]{plots/t-effects25}

\end{frame}

\begin{frame}[containsverbatim]
\frametitle{Effects Plot: Gay Marriage}



\includegraphics[width=10cm]{plots/t-effects30}

\end{frame}


\lyxframeend{}\section{Ordinal versus Numeric}

\begin{frame}[containsverbatim]
\frametitle{R Emphasizes Factors}
  \begin{itemize}
    \item Importing Data: Any non-numeric score -> "factor" treatment
    \item Age-old question: Is a 7 point scale numeric or merely ordinal?
    \item Plots \& regressions seem tidier if we use numeric scorings
    \item Possible to formally test the numeric versus the factor scorings
  \end{itemize}
\end{frame}

\begin{frame}[containsverbatim]
\frametitle{Create Numeric Predictors}




\begin{Schunk}
\begin{Sinput}
 mydta1$V045117n <- as.numeric(mydta1$V045117)
 mydta1$V043116n <- as.numeric(mydta1$V043116)
 mydta1$V043210n <- as.numeric(mydta1$V043210)
 mydta1$V043213n <- as.numeric(mydta1$V043213)
 mydta1$V045145Xn <- as.numeric(mydta1$V045145X)
 m3n <- lm (th.bush.kerry ~ V045117n + V043116n + V043210n + V043213n + V045145Xn + V041109A, data=mydta1) 
\end{Sinput}
\end{Schunk}

\begin{Schunk}
\begin{Sinput}
 outreg(m3n, tight=FALSE, modelLabels=c("Numeric Predictors"), varLabels=list(V045117n="Ideology", V043116n="Party ID", V043210n="AntiGay", V043213n="Economy", V045145Xn="Flag Love", V041109AF="Female"))
\end{Sinput}
\end{Schunk}


\end{frame}

\begin{frame}[containsverbatim]
\frametitle{Beautiful Outreg Output}

\scriptsize{
\begin{tabular}{*{3}{l}}
 \hline
                &\multicolumn{2}{c}{Numeric Predictors}   \\
                &Estimate &(S.E.) \\
 \hline
 \hline
  (Intercept)    & -99.868*  &   (5.825) \\
  Ideology       &  6.739*  &   (1.021) \\
  Party ID       &  13.161*  &   (0.695) \\
  AntiGay        &  7.207*  &   (2.201) \\
  Economy        &  13.721*  &   (1.605) \\
  Flag Love      & -6.665*  &   (1.249) \\
  Female         &  0.437  &   (2.169) \\
 \hline 
 N                &803      &       \\
 RMSE            &30.469        & \\
 $R^2$           &0.696        & \\
 adj $R^2$       &0.694        & \\
 \hline
 \hline
 
 \multicolumn{2}{l}{${*}  p \le 0.05$   }\\
 \end{tabular}
}

\end{frame}

\begin{frame}[containsverbatim]
\frametitle{termplot: Ideology}


\includegraphics[width=10cm]{plots/t-numer20}

\end{frame}

\begin{frame}[containsverbatim]
\frametitle{Compare to ordinary R plot(bush-kerry,ideology)}


\includegraphics[width=10cm]{plots/t-numer21}

\end{frame}

\begin{frame}[containsverbatim]
\frametitle{termplot: Party ID}


\includegraphics[width=10cm]{plots/t-numer30}

\end{frame}

\begin{frame}
\frametitle{Compare to ordinary R plot(bush-kerry, party))}


\includegraphics[width=10cm]{plots/t-numer40}

\end{frame}

\begin{frame}[containsverbatim]
\frametitle{Should you use the Numeric Model?}
\begin{itemize}
\item My humble opinion: not without checking first
\item Look at ``termplot'' output to see if the relationship really is
linear
\item Conduct hypothesis tests
\end{itemize}
\end{frame}

\begin{frame}[containsverbatim]
\frametitle{An F test will work (these are nested models)}


\begin{Schunk}
\begin{Sinput}
 anova(m3,m3n, test="F")
\end{Sinput}
\begin{Soutput}
Analysis of Variance Table

Model 1: th.bush.kerry ~ V045117 + V043116 + V043210 + V043213 + V045145X + 
    V041109A
Model 2: th.bush.kerry ~ V045117n + V043116n + V043210n + V043213n + V045145Xn + 
    V041109A
  Res.Df    RSS  Df Sum of Sq      F    Pr(>F)    
1    781 700542                                   
2    796 738959 -15    -38417 2.8553 0.0002215 ***
---
Signif. codes:  0 '***' 0.001 '**' 0.01 '*' 0.05 '.' 0.1 ' ' 1
\end{Soutput}
\end{Schunk}


\end{frame}

\begin{frame}[containsverbatim]
\frametitle{What does that Significance Test Mean?}
\begin{itemize}
\item Something's wrong, at last one variable should not be treaded as a
linear effect.
\item Difficult to say with confidence what to do next. 
\item I would look at individual variables.
\end{itemize}
\end{frame}

\begin{frame}[containsverbatim,allowframebreaks]
\frametitle{Party, for example}




\begin{Schunk}
\begin{Sinput}
 mod2 <- lm (th.bush.kerry ~ V045117n + V043116 + V043210n + V043213n + V045145Xn + V041109A, data=mydta1) 
 mod2n <- lm (th.bush.kerry ~ V045117n + V043116n + V043210n + V043213n + V045145Xn + V041109A, data=mydta1) 
\end{Sinput}
\end{Schunk}

\begin{Schunk}
\begin{Sinput}
 outreg(list(mod2, mod2n), tight=FALSE, modelLabels=c("Party Factor","Numeric Predictors"), varLabels=list(V045117n="Ideology", V043116="Party ID", V043210n="AntiGay", V043213n="Economy", V045145Xn="Flag Love", V041109AF="Female"))
\end{Sinput}
\end{Schunk}


\end{frame}

\begin{frame}[containsverbatim]
\frametitle{Beautiful Outreg Output}

\scriptsize{
\begin{tabular}{*{5}{l}}
 \hline
                &\multicolumn{2}{c}{Party Factor} &\multicolumn{2}{c}{Numeric Predictors}   \\
                &Estimate &(S.E.)  &Estimate &(S.E.) \\
 \hline
 \hline
  (Intercept)    & -87.181*  &   (6.154) & -99.868*  &   (5.825) \\
  Ideology       &  6.067*  &   (1.023) &  6.739*  &   (1.021) \\
  V043116WD      &  25.878*  &   (3.986)  & .       &         \\
  V043116ID      &  23.6*  &   (3.716)  & .       &         \\
  V043116I       &  41.993*  &   (5.017)  & .       &         \\
  V043116IR      &  66.697*  &   (4.505)  & .       &         \\
  V043116WR      &  68.496*  &   (4.393)  & .       &         \\
  V043116SR      &  83.971*  &   (4.654)  & .       &         \\
  AntiGay        &  6.611*  &   (2.178) &  7.207*  &   (2.201) \\
  Economy        &  13.211*  &   (1.591) &  13.721*  &   (1.605) \\
  Flag Love      & -6.376*  &   (1.238) & -6.665*  &   (1.249) \\
  Female         &  0.666  &   (2.17) &  0.437  &   (2.169) \\
  V043116n        & .       &         &  13.161*  &   (0.695) \\
 \hline 
 N                &803      &      &803      &       \\
 RMSE            &30.043        &&30.469        & \\
 $R^2$           &0.707        &&0.696        & \\
 adj $R^2$       &0.703        &&0.694        & \\
 \hline
 \hline
 
 \multicolumn{2}{l}{${*}  p \le 0.05$   }\\
 \end{tabular}
}

\end{frame}

\begin{frame}[containsverbatim]
\frametitle{An F test will work}


\begin{Schunk}
\begin{Sinput}
 anova(mod2, mod2n, test="F")
\end{Sinput}
\begin{Soutput}
Analysis of Variance Table

Model 1: th.bush.kerry ~ V045117n + V043116 + V043210n + V043213n + V045145Xn + 
    V041109A
Model 2: th.bush.kerry ~ V045117n + V043116n + V043210n + V043213n + V045145Xn + 
    V041109A
  Res.Df    RSS Df Sum of Sq      F    Pr(>F)    
1    791 713920                                  
2    796 738959 -5    -25039 5.5485 4.945e-05 ***
---
Signif. codes:  0 '***' 0.001 '**' 0.01 '*' 0.05 '.' 0.1 ' ' 1
\end{Soutput}
\end{Schunk}


\end{frame}


\lyxframeend{}\section{3d Plots}

\begin{frame}[containsverbatim]
\frametitle{Kinds of 3d Plot}
\begin{itemize}
\item Please see the separate lecture notes on plotting in 3 dimensions.
After inserting that material into this file, it simply became too
unwieldy for editing.
\end{itemize}
\end{frame}
\end{document}
