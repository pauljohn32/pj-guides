%%%%%%%%%%%%%%%%%%%%%%%%%%%%%%%%%%%%%%%%%%%%%%%%%%%%%%%%%%%%%%%%%
% Filename: template.Rnw
%   Author: Paul Johnson
%
%%%%%%%%%%%%%%%%%%%%%%%%%%%%%%%%%%%%%%%%%%%%%%%%%%%%%%%%%%%%%%%%%
%%%%%%%%%%%%%%%%%%%%%%%%%%%%%%%%%%%%%%%%%%%%%%%%%%%%%%%%%%%%%%%%%
\documentclass[11pt,english]{beamer}
\usepackage{lmodern}
\renewcommand{\sfdefault}{lmss}
\renewcommand{\ttdefault}{lmtt}
\usepackage[T1]{fontenc}
\usepackage[utf8]{inputenc}
\usepackage{listings}
\setcounter{secnumdepth}{3}
\setcounter{tocdepth}{3}
\usepackage{url}
\usepackage{graphicx}


\makeatletter
%%%%%%%%%%%%%%%%%%%%%%%%%%%%%% Textclass specific LaTeX commands.
\usepackage{Sweavel}
 \newenvironment{topcolumns}{\begin{columns}[t]}{\end{columns}}

%%%%%%%%%%%%%%%%%%%%%%%%%%%%%% User specified LaTeX commands.
\usepackage{dcolumn}
\usepackage{booktabs}

% use 'handout' to produce handouts
%\documentclass[handout]{beamer}
\usepackage{wasysym}
\usepackage{pgfpages}
\newcommand{\vn}[1]{\mbox{{\it #1}}}\newcommand{\vb}{\vspace{\baselineskip}}\newcommand{\vh}{\vspace{.5\baselineskip}}\newcommand{\vf}{\vspace{\fill}}\newcommand{\splus}{\textsf{S-PLUS}}\newcommand{\R}{\textsf{R}}


\usepackage{graphicx}
\usepackage{listings}
\lstset{tabsize=2, breaklines=true,style=Rstyle}

%
%In document Latex options:
\fvset{listparameters={\setlength{\topsep}{0em}}}
\def\Sweavesize{\scriptsize}
\def\Rcolor{\color{black}}
\def\Rbackground{\color[gray]{0.95}}

% \usetheme{Warsaw}
% or ...

%\setbeamercovered{transparent}
% or whatever (possibly just delete it)

\mode<presentation>
  \usetheme{KU}
  \usecolortheme{dolphin} %dark blues

\setbeamertemplate{frametitle continuation}[from second]
\renewcommand\insertcontinuationtext{...}

%%for presentations
\expandafter\def\expandafter\insertshorttitle\expandafter{%
 \insertshorttitle\hfill\insertframenumber\,/\,\inserttotalframenumber}

%=============================================================================
%



\title[Writing Functions I] % (short title, use only with long paper titles)
{Writing Functions In R}

\subtitle{Necessity Really is a Mother}

\author[Johnson] { Paul E. Johnson\inst{1}\inst{2}}

\institute[University of Kansas]{\inst{1}
  University of Kansas, Department of
  Political Science \inst{2} Center for Research Methods and Data Analysis}

\date[\today] % (optional, should be abbreviation of conference name)
{\today}

\subject{R functions}

%====================================

\begin{document}

\AtBeginSection[]{
  \frame<beamer>{
    \frametitle{Outline}
    \tableofcontents[currentsection,currentsubsection]
  }
}




% In document Latex options:


\begin{Schunk}
\begin{Sinput}
 ## Other settings I like
 options(device = pdf)
 options(useFancyQuotes = FALSE) 
 op <- par() 
 pjmar <- c(5.1, 5.1, 1.5, 2.1) 
 options(SweaveHooks=list(fig=function() par(mar=pjmar, ps=12)))
 pdf.options(onefile=F,family="Times",pointsize=12)
\end{Sinput}
\end{Schunk}




\begin{frame}
  \titlepage
\end{frame}



\begin{frame}
\frametitle{Outline}

\tableofcontents{}

\end{frame}


%_______________________________

\section{Arguments}

\begin{frame}[containsverbatim]
  \frametitle{Function Anatomy}
  Arguments: The Arguments of a function

  \begin{Sinput}
    someWork <- function(what1, what2, what3)
  \end{Sinput}

  what1, what2, and what3 become ``local variables'' inside the function.

  \begin{itemize}
  \item named arguments must begin with letters, may include numbers,
    or ``\_'' or ``-'' or ``.'', but no ``\!'' or ``,'' or ``*''
  \end{itemize}
\end{frame}


\begin{frame}[containsverbatim]
  \frametitle{Arguments: named and unnamed R variables}
  \begin{Sinput}
    someWork <-  function(what1, what2, what3, ...)
  \end{Sinput}

  \begin{itemize}
  \item In R, everything is an ``object''. what1, what2, what3 can be \emph{ANYTHING!}
  \item That is a blessing and a curse
    \begin{itemize}
    \item Blessing: Flexibility! Let an argument be a vector, matrix,
      list, whatever. The function declaration does not care.
    \item Curse: Difficulty managing the infinite array of possible
      ways users might break your functions.
    \end{itemize}
  \item It is your choice, whether your function should receive
    \begin{itemize}
    \item  2 integers (x, y), or
    \item  one vector of 2 integers (x)
    \item For examples of this, look at the arguments of
      plot.default, arrows, segments, lines, and matplot.
  \end{itemize}
\end{itemize}
\end{frame}


\begin{frame}[containsverbatim]
  \frametitle{Peculiar spellings like $\ldots$}
  Functions may have an argument ``$\ldots$''
  \begin{itemize}
  \item It seems funny, but ``...'' is actually a word made up of
    \emph{three legal} characters
  \item If the caller includes any named arguments that are not
    \texttt{what1}, \texttt{what2}, or \texttt{what3}, then the R
    runtime will put them in a list and give them to ``...''
  \item The ``...'' argument requires special effort.
  \end{itemize}
\end{frame}



\begin{frame}[containsverbatim]
  \frametitle{Function Arguments: R is VERY Lenient}
  \begin{itemize}
      \item R is lenient. Perhaps too lenient!
        \begin{itemize}
        \item Arguments are not type-defined. In
          \begin{Sinput}
            myF <- function(x1, x2){
          \end{Sinput}
          x1 could be an integer vector, a character string, a regression model, or anything

        \item Function writer may declare default values, but need
          not. These are acceptable definitions

\begin{Sinput}
someWork <- function(what1, what2, what3, what4, what5)

someWork <- function(what1 = 0, what2 = NULL, what3 = c(1,2,3), what4 = 3 * what1, what5)

someWork <- function(what1 = 0, what2, what3, what4 = 3 * what1, what5)
\end{Sinput}

\end{itemize}

\end{itemize}
\end{frame}



\begin{frame}[containsverbatim]
  \frametitle{R is Lenient toward Users as Well}
  \begin{itemize}

\item R is lenient on format of function calls

\begin{Sinput}
someWork(1)
someWork(what=1, someObject)
someWork(what5 = fred, what4 = jim, what3 = joe)
\end{Sinput}
    \begin{itemize}
    \item Not all parameters must be provided
    \item Partial argument matching
    \end{itemize}
  \end{itemize}
\end{frame}


\begin{frame}[containsverbatim]
  \frametitle{R is Lenient About Undefined Variables}

  \begin{exampleblock}{This is a convenience and a curse}
    Variables inside functions might not be resolved the way you expect.

  \begin{itemize}
  \item If a variable is used, but not defined inside the function, R
    will ``look outward'' for it
  \item This is ``lexical scope'' in action. The runtime engine
    searches through a sequence of ``frames'', ending up at the user's
    workspace (which is the Global Environment).

  \end{itemize}
   \end{exampleblock}
\end{frame}


\begin{frame}[containsverbatim]
  \frametitle{Example of the Undefined Variable Problem}
  \begin{itemize}

  \item Suppose ``dat'' exists in your workspace.

  \item Here is a function that will find ``dat'', even if that's not
    what you intend.

\begin{Sinput}
myFn <- function(x, y, z){
    dat1 <- 2 * x + 3 * y
    res <- sqrt(dat)
}
\end{Sinput}

Note my typographical error (dat where dat1 should be). The function
should crash, but it will not.
   \item R will find the wrong ``dat'' and the result we get will be unhelpful

   \item In my experience, this is the single most important cause of
     hard-to-find flaws in user code.

  \end{itemize}
\end{frame}



\begin{frame}[containsverbatim]
  \frametitle{Inside the function, Check Arguments}

  \begin{itemize}
  \item Check and Re-format

    \begin{itemize}
    \item ``argument checking'': diagnose what arguments the
      user provided
    \item Figure out if they are ``correct'' for what the function
      requires.
    \end{itemize}
  \item You choose how sympathetic you want to be toward users. Do you
    want to accept incomplete input and re-format it? (In
    rockchalk/R/genCorrelatedData.R, find my ``re-formatter''
    functions like \texttt{makeVec()},  \texttt{vech2mat()}).

  \end{itemize}
\end{frame}


\begin{frame}[containsverbatim]
  \frametitle{Arguments: When to Use Defaults?}

  \begin{itemize}
    \item I don't know, but $\ldots$
  \item As an R beginner, I took a very conservative strategy of
    putting defaults on everything!

\begin{Sinput}
function(what1 = NULL, what2 = NULL, what3 = 3, what4 = ``a'')
\end{Sinput}

  \item That way, if a user forgets to provide
    ``what3'', then the system will not go looking for it.
  \item If the defaults usually work, this is concise, easy to read.

  \item Most functions in R base code don't set defaults for all, or
    even most, variables.

  \item Instead, we manage arguments with more delicacy.  Insisting on
    NULL defaults is ``throwing the baby out with the bath water.''
 \end{itemize}
\end{frame}



\begin{frame}[containsverbatim, allowframebreaks]
  \frametitle{Functions that Help while Checking  Arguments}

  \begin{exampleblock}{missing()}
    \begin{itemize}
    \item Inside a function, \texttt{missing()} is a way to
      ask if the user supplied a value for an argument.

\begin{Sinput}
doSomething <- function(what1, what2, what3, what4){
   if (missing(what1)) stop(``you forgot to specify what1'')
}
\end{Sinput}

   \item I've found this conservative approach to be an
     error-preventer. If x is not provided, or if the user gave us a
     NULL, we better adapt!
\begin{Sinput}
if (missing(x) || is.null(x)) ## do something
\end{Sinput}
\end{itemize}
\end{exampleblock}


\begin{exampleblock}{Type Checks}
     There are many ``is.'' functions. Ask if x is a
  certain type of thing:
  \begin{description}
  \item [is.vector()] TRUE or FALSE: are you a vector?
  \item [is.list()] TRUE or FALSE: are you a list?
  \item [is.numeric()] TRUE or FALSE: are you numeric?
  \item You get the general idea, I hope
  \end{description}
\end{exampleblock}


\begin{exampleblock}{Look at all of these is. functions!}
  \begin{Soutput}
is.array         is.loaded              is.object
is.atomic        is.logical             is.ordered
is.call          is.matrix              is.package_version
is.character     is.mts                 is.pairlist
is.complex       is.na                  is.primitive
is.data.frame    is.na<-                is.qr
is.double        is.na.data.frame       is.R
is.element       is.na<-.default        is.raster
is.empty.model   is.na<-.factor         is.raw
is.environment   is.name                is.recursive
is.expression    is.nan                 is.relistable
is.factor        is.na.numeric_version  is.single
is.finite        is.na.POSIXlt          is.stepfun
is.function      is.null                is.symbol
is.infinite      is.numeric             is.table
is.integer       is.numeric.Date        is.ts
is.language      is.numeric.difftime    is.tskernel
is.leaf          is.numeric.POSIXt      is.unsorted
is.list          is.numeric_version     is.vector
\end{Soutput}
\end{exampleblock}

\begin{alertblock}{Size Checks}
  \begin{description}
  \item [length()]
  \item [nrow(), ncol()] number of rows (columns) from a matrix
  \item [NROW(), NCOL()] Works if input is matrix, but will treat a
    vector as a one column matrix.
  \item [dim()] Returns 2 dimensional vector, works with matrices and arrays
  \end{description}
\end{alertblock}

\begin{alertblock}{Stop, Warn}
  \begin{description}
  \item[stop(), stopifnot()] Ways to abend the function
  \item[warning] Will return, but throws a message onto the R
    warning log. User can run warnings() to see messages.
  \end{description}
\end{alertblock}

\end{frame}

\section{dots. I see dots}

\begin{frame}[containsverbatim, allowframebreaks]
  \frametitle{dot management strategies I use}
  \begin{itemize} 
  \item Get the dots. The R parser notices all named
    arguments that seem to be ``unclaimed'' by other arguments.
  \item If you have an argument named ``...'', those arguments
    go into a list named ``...''.
  \item So you should run:
    \begin{Sinput}
      dots <- list(...)
    \end{Sinput}
 
  \end{itemize}
\end{frame} 


\begin{frame}[containsverbatim, allowframebreaks]
  \frametitle{Inspect and filter the dots}
  \begin{itemize} 
  \item In some functions, I need to pass in ``...'' that are
    aimed at several different subroutines.
  \item examples
    \begin{enumerate}
      \item rockchalk::plotSlopes
      \item kutils::peek
      \end{enumerate}
    \item Find out what arguments a function might want:
  \item So you should run:
    \begin{Sinput}
      formals(function_name_here)
      ## Possibly just
      names(formals(function_name_here))
    \end{Sinput}
    or read the documentation
  \end{itemize}
\end{frame} 

\begin{frame}[containsverbatim, allowframebreaks]
  \frametitle{Manually pick-and-choose items out of dots}
  
  \begin{itemize}
  \item I will type in the names of the arguments I
    know I want to keep
  \item Put them (and their values) into a separate list
  \item then use ``modifyList'' to place the user-provided
    dots ``on top'' of my favorite settings.
  \end{itemize}
  
  \begin{Sinput}
    drawHist <- function(x, normal = TRUE, pois = FALSE, 
    kde = TRUE, nbinom = FALSE, ...){
      dots <- list(...)
      pjfavoriteDefaults <- list(right = FALSE, prob = TRUE, 
      border = gray(.80), include.lowest = TRUE, 
      plot = FALSE)
      
    histNames <- c("breaks", "freq", "probability", 
                   "include.lowest", "right", "density", 
                   "angle", "col", "border", "main", 
                    "xlim", "ylim", "xlab", "ylab", "axes", 
                    "plot", "labels", "nclass")
    argsForHist <- dots[ names(dots) [names(dots) %in% histNames]]
    argsForHist <- modifyList(pjfavoriteDefaults, argsForHist)
    \end{Sinput}
    
    \begin{itemize}
    \item Then we may want to purge those particular arguments
      from the dots list (not possibly needed for other functions).
      This prevents accidentally sending an argument for hist 
      to an lm function, for example.
    \end{itemize}
\end{frame}
  

\begin{frame}[containsverbatim, allowframebreaks]
  \frametitle{Trust the formals}
  
   \begin{itemize}
     \item Typing in the names seems stupid.
     \item If the R functions change, then this code will be outdated.
     \item If I'm in a gambling mood, then I'll not type
       in all of the names and I'll use the formals function
       to get the names of arguments. Be careful.
    \item Example usage
   \begin{Sinput}
     histNames <- names(formals(function_name_here))
   \end{Sinput}

  \item
    After that, you may have to clean out ones you
    don't want to keep. Observe:
\begin{Schunk}
\begin{Sinput}
 names(formals(hist.default))
\end{Sinput}
\begin{Soutput}
 [1] "x"              "breaks"         "freq"           "probability"    "include.lowest" "right"          "density"        "angle"          "col"            "border"         "main"          
[12] "xlim"           "ylim"           "xlab"           "ylab"           "axes"           "plot"           "labels"         "nclass"         "warn.unused"    "..."           
\end{Soutput}
\end{Schunk}

  \item Quick way to get the argument names into a vector:
    
\begin{Schunk}
\begin{Sinput}
 dput(names(formals(hist.default)))
\end{Sinput}
\begin{Soutput}
c("x", "breaks", "freq", "probability", "include.lowest", "right", 
"density", "angle", "col", "border", "main", "xlim", "ylim", 
"xlab", "ylab", "axes", "plot", "labels", "nclass", "warn.unused", 
"...")
\end{Soutput}
\end{Schunk}

   and then edit that for use in a function. Cut out the dots!   
 \end{itemize}
\end{frame}


\begin{frame}[containsverbatim, allowframebreaks]
  \frametitle{Pruning argument lists}
\begin{itemize}  
  \item Usually, I want to check for some arguments and
    keep them for use, as described above.
    
  \item In order to explain this fully, I'd need you to understand the
    use of R's ``do.call'', which is a deep thought that we should
    probably investigate in a separate talk.
    
  \item I understood the beauty of that in this example code:
    \url{http://pj.freefaculty.org/R/WorkingExamples/efficiency-stackListItems-01.R}

  \item There is quite a bit of discussion of do.call in the essay
    ``Rchaeology'' that is included in the \texttt{rockchalk} package.
    
  \end{itemize}
\end{frame}

\begin{frame}[containsverbatim, allowframebreaks]
  \frametitle{The kutils::peek example}
  
  \begin{itemize}
  \item The peek function has 2 nested functions
    \begin{itemize}
    \item one for creating histograms
    \item  one for creating barplots
    \end{itemize}
    
  \item The dots argument may be users to pass in
    arguments intended for hist, barplot, or pdf.
  \end{itemize}
  
\def\Sweavesize{\tiny}
\begin{Sinput}
peek <-
    function(dat, sort = TRUE, file = NULL, textout = FALSE, ask, ...,
             xlabstub = "kutils peek: ", freq = FALSE,
             histargs = list(probability = !freq),
             barargs = list(horiz = TRUE, las = 1))
{
    quickhist <- function(i){
        args <- list(xlab = paste0(xlabstub, i),
                     main = i)
        histargz <- modifyList(args, dots)
        histargz <- modifyList(histargz, histargs)
        qqq <- modifyList(list(x = dat[ , i]), histargz)
        h1 <- do.call("hist", qqq)
        if (textout){
            df1 <- data.frame("midpoints" = h1$mids, "density" = h1$density)
            cat(i, "\n")
            print(df1)
        }
    }
    
    quickbar <- function(i){
        targs <- list(dat[ , i], exclude = NULL,
                      dnn =  if (!freq) "Proportion" else "Frequency")
        targz <- modifyList(targs, dotsForTable)
        ## don't allow hist or barplot arguments to to to table, silences warning
        ## targz[c(names.par, names.barplot.unique, names.hist.unique)] <- NULL
        t1 <- do.call("table", targz)
        if (!freq) t1 <- t1/margin.table(t1)
        names(t1) <- ifelse(is.na(names(t1)), "NA", names(t1))
        ## decision 1: brute force chop
        ## names(t1) <- shorten(names(t1), k = LIMIT, unique = TRUE)
        names(t1) <- sapply(names(t1), stringbreak, k = LIMIT)
        args <- list(t1, main = i,
                     xlab = paste(xlabstub,  i, if (freq)"Frequencies" else "(Proportion)"))
        barargz <- modifyList(args, dots)
        barargz <- modifyList(barargz, barargs)
        ## Remove args that would have gone to hist
        barargz[names.hist.unique] <- NULL
        par.old <- par(no.readonly = TRUE)
        ## if longest name exceeds guess of space in margin,
        ## then pad the margin left side.
        marinch <- par("mai")
        marrequired <- max(strwidth(names(t1), units = "inches"))
        if (marrequired + 0.35 > marinch[2]) {
            marinch[2] <- marrequired + 0.5
            par(xpd = TRUE)
            par("mai" = marinch)
        }
        do.call("barplot", barargz)
                
        par(par.old)
        if (textout) {
            cat(i, "\n")
            print(t1)
        }
    }

    ## limit on character strings for barplots
    LIMIT <- 30
    
    ## We get lots of warnings about inappropriate arguments to functions.
    ## Focus on most likely objections by getting names unique to
    ## hist and removing them from the bar portion, and
    ## names unique to bar and removing from the hist portion
    names.par <- names(par())
    names.hist <- removeMatches(names(formals(hist.default)), "...")
    names.barplot <- removeMatches(names(formals(barplot.default)), "...")
    names.barplot.unique <- names.barplot[!names.barplot %in% names.hist]
    names.hist.unique <- names.hist[!names.hist %in% names.barplot]
    
    varType <- function(x){
        ifelse(is.numeric(x), "numeric",
        ifelse(is.character(x) | is.factor(x), "factor", "noneoftheabove"))
    }

    if (is.atomic(dat)){
        xclean <- xname <- deparse(substitute(dat))
        dat <- as.data.frame(dat)
        if (length(grep("\\$", xname) > 0)){
            xsplit <- unlist(strsplit(xname, "\\$"))
            xclean <- xsplit[length(xsplit)]
        }
        colnames(dat) <- xclean
    } else {
        if (!is.data.frame(dat)) dat <- as.data.frame(dat)
    }
    
    namez <- colnames(dat)
    namez <- if(sort) sort(namez) else namez
    
    colTypes <- sapply(dat, varType)
    colTypes <- colTypes[namez]

    if (length(colTypes[colTypes == "noneoftheabove"]) > 0) {
        cat(paste("These variables are being omitted because they are",
                  "neither numbers nor can they be interpreted as factors: \n"))
        cat(paste(names(colTypes[colTypes == "noneoftheabove"]), "\n"))
        ## remove unrecognized types
        colTypes <- colTypes[colTypes != "noneoftheabove"]
    }
     
    dots <- list(...)
      
    ## copy args for table, remove from dots
    tableFormals <- c("exclude", "dnn", "useNA", "deparse.level")
    dotsForTable <- dots[tableFormals[tableFormals %in% names(dots)]]
    dots[names(dotsForTable)] <- NULL

    ## copy args for pdf, remove from dots
    pdfFormals <- c("width", "height", "onefile", "family", "title", "fonts",
                    "version", "paper", "encoding", "bg", "fg", "pointsize",
                    "pagecentre", "colormodel", "useDingbats", "useKerning",
                    "fillOddEven", "compress")
    dotsForPDF <- dots[pdfFormals[pdfFormals %in% names(dots)]]

    deviceFormals <- c("width", "height")
    dotsForDevice <- dots[deviceFormals[deviceFormals %in% names(dots)]]
    dots[names(dotsForPDF)] <- NULL

     
    
    ## Unless user sets ask, we assume TRUE if there is no file argument
    if (missing(ask)) {
        if (is.null(file)) ask <- TRUE else ask <- FALSE
    }
    
    if (!is.null(file)){
        if (!is.character(file)) stop("Sorry, file has to be a character string")
        pdfargs <- list(file = file, onefile = TRUE)
        pdfargz <- modifyList(pdfargs, dotsForPDF)
        do.call("pdf", pdfargz)
    } else {
        do.call("dev.new", dotsForDevice)
    }
    
    if (is.null(file) || isTRUE(ask)) devAskNewPage(TRUE)
    
    for (i in names(colTypes)){
        if ( colTypes[i] == "numeric" ){
            quickhist(i)
        } else {
            quickbar(i)
        }
    }
    if (!is.null(file)) dev.off()
    namez
}
\end{Sinput}
  
  
  
  
\end{frame}

%___________________________

\section{Returns}
\begin{frame}[containsverbatim, allowframebreaks]
  \frametitle{The Return Has To Be Singular}
  \begin{itemize}
  \item When  you use a function, it is necessary to ``catch'' the
    output with a single object name, as in

\begin{Schunk}
\begin{Sinput}
 newthing <- doubleMe(32)
 newthing
\end{Sinput}
\begin{Soutput}
[1] 64
\end{Soutput}
\begin{Sinput}
 is.numeric(newthing)
\end{Sinput}
\begin{Soutput}
[1] TRUE
\end{Soutput}
\begin{Sinput}
 is.vector(newthing)
\end{Sinput}
\begin{Soutput}
[1] TRUE
\end{Soutput}
\end{Schunk}


  \item We expect ``doubleMe(32)'' should return 64, and it does.

  \item The ``vectorization for free'' gives us a hint of what is to follow.

\begin{Schunk}
\begin{Sinput}
 newthing <- doubleMe(c(1,2,3,4,5))
 newthing
\end{Sinput}
\begin{Soutput}
[1]  2  4  6  8 10
\end{Soutput}
\begin{Sinput}
 is.numeric(newthing)
\end{Sinput}
\begin{Soutput}
[1] TRUE
\end{Soutput}
\begin{Sinput}
 is.vector(newthing)
\end{Sinput}
\begin{Soutput}
[1] TRUE
\end{Soutput}
\end{Schunk}



\item R allows us to return one ``thing'', but ``thing'' can be a
  rich, informative thing!
\end{itemize}

\end{frame}

%______________________________

\begin{frame}[containsverbatim]
  \frametitle{Generalization. Return a list}
  \begin{itemize}

  \item A list may include numbers, characters, vectors ,etc
  \item or data frames or other lists
  \item Read code for function ``lm''
  \item Almost ALL interesting R functions return a list of elements.
  \end{itemize}
\end{frame}


\begin{frame}[containsverbatim,allowframebreaks]
  \frametitle{Hide details in attributes}
  \begin{itemize}
    \item R objects can have attributes, which are, well, anything.
      
    \item Ever run \texttt{attributes(some\_object\_name)}?
      
    \item In S3, an object's class is simply determined by an attribute.
  \end{itemize}
  
\begin{Schunk}
\begin{Sinput}
 x <- c(1, 2, 3, 4, 6)
 y <- c(44, 12, 44, 22, 4)
 attributes(x)
\end{Sinput}
\begin{Soutput}
NULL
\end{Soutput}
\begin{Sinput}
 attr(x, "hi") <- paste("Oh, my hearty friend, I greet  you warmly", rnorm(6))
 attr(x, "areg") <- lm(y ~ x)
 attributes(x)
\end{Sinput}
\begin{Soutput}
$hi
[1] "Oh, my hearty friend, I greet  you warmly 0.585528817843856"  "Oh, my hearty friend, I greet  you warmly 0.709466017509524"  "Oh, my hearty friend, I greet  you warmly -0.109303314681054"
[4] "Oh, my hearty friend, I greet  you warmly -0.453497173462763" "Oh, my hearty friend, I greet  you warmly 0.605887455840394"  "Oh, my hearty friend, I greet  you warmly -1.81795596770373" 

$areg

Call:
lm(formula = y ~ x)

Coefficients:
(Intercept)            x  
     44.919       -6.162  
\end{Soutput}
\begin{Sinput}
 summary(attr(x, "areg"))
\end{Sinput}
\begin{Soutput}
Call:
lm(formula = y ~ x)

Residuals:
      1       2       3       4       5 
  5.243 -20.595  17.568   1.730  -3.946 

Coefficients:
            Estimate Std. Error t value Pr(>|t|)  
(Intercept)   44.919     15.216   2.952   0.0599 .
x             -6.162      4.188  -1.471   0.2376  
---
Signif. codes:  0 '***' 0.001 '**' 0.01 '*' 0.05 '.' 0.1 ' ' 1

Residual standard error: 16.11 on 3 degrees of freedom
Multiple R-squared:  0.4191,	Adjusted R-squared:  0.2255 
F-statistic: 2.165 on 1 and 3 DF,  p-value: 0.2376
\end{Soutput}
\end{Schunk}

    
 
\end{frame}



% \begin{frame}[containsverbatim]
%   \frametitle{Check Point: revise your function}

%   \begin{itemize}
%   \item Create ``myGreatFunction2'' by revising
%     ``myGreatFunction''. Make it return a vector of 3 values: the
%     maximum, the minimum, and the median.
%   \item Generate your own input data, x1, like so

% <<echo=T, include=T, eval=F>>=
% x1 <- rnorm(10000, m=7, sd=19)
% @
%   \item After you've written myGreatFunction, use it:
% <<echo=T, include=T, eval=F>>=
% myGreatFunction(x1)
% @
%   \item Now stress test your function by changing x1

% <<echo=T, include=T, eval=F>>=
% x1[c(13, 44, 99, 343, 555)]  <- NA
% myGreatFunction(x1)
% @
% \end{itemize}
% \end{frame}



\begin{frame}[containsverbatim, allowframebreaks]
  \frametitle{Check Point: Run this example function}
  Admittedly, this is a dumb example, but $\ldots$
  \begin{itemize}
  \item This function returns a regression object
\begin{Schunk}
\begin{Sinput}
 aRegMod <- function(xin1, xin2, yout){ lm(yout ~ xin1 + xin2) }
\end{Sinput}
\end{Schunk}

  \item Generate some data, run \texttt{aRegMod()}
\begin{Schunk}
\begin{Sinput}
 dat <- rockchalk::genCorrelatedData(N = 100, rho = 0.3, beta = c(1, 1.0, -1.1, 0.0))
 m1 <- with(dat, aRegMod(x1, x2, y))
\end{Sinput}
\end{Schunk}


  \item m1 is a ``single object''
  \item Run ``class(m1)'',  ``attributes(m1)'', ``summary(m1)'', `str(m1)'',
  \end{itemize}
\end{frame}

%________________________________________

%% \begin{frame}[containsverbatim, allowframebreaks]
%%   \frametitle{Almost All Substantial R Functions Return A Diverse List of Items}
%%   \begin{itemize}
%%   \item An R function can return an R ``list'' object.  Basically,
%%     that means a combination of anything.

%%   \item Look at the end of R's glm function, for example. After it has
%%     done a bunch of calculations, it has an object ``fit'' and then
%%     more and more details are wedged together with ``fit''

%% \begin{Sinput}
%% fit <- eval(call(if (is.function(method)) "method" else method,
%%     x = X, y = Y, weights = weights, start = start, etastart = etastart,
%%     mustart = mustart, offset = offset, family = family,
%%     control = control, intercept = attr(mt, "intercept") >
%%         0L))
%% if (length(offset) && attr(mt, "intercept") > 0L) {
%%     fit$null.deviance <- eval(call(if (is.function(method)) "method" else method,
%%         x = X[, "(Intercept)", drop = FALSE], y = Y, weights = weights,
%%         offset = offset, family = family, control = control,
%%         intercept = TRUE))$deviance
%% }
%% if (model)
%%     fit$model <- mf
%% fit$na.action <- attr(mf, "na.action")
%% if (x)
%%     fit$x <- X
%% if (!y)
%%     fit$y <- NULL
%% fit <- c(fit, list(call = call, formula = formula, terms = mt,
%%     data = data, offset = offset, control = control, method = method,
%%     contrasts = attr(X, "contrasts"), xlevels = .getXlevels(mt,
%%         mf)))
%% class(fit) <- c(fit$class, c("glm", "lm"))
%% fit
%% \end{Sinput}

%% \end{itemize}
%% \end{frame}

%% % $


%===================================================
\end{document}
