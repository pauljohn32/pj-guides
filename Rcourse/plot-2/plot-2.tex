%%%%%%%%%%%%%%%%%%%%%%%%%%%%%%%%%%%%%%%%%%%%%%%%%%%%%%%%%%%%%%%%%
% Filename: template.Rnw
%   Author: Paul Johnson
%
%%%%%%%%%%%%%%%%%%%%%%%%%%%%%%%%%%%%%%%%%%%%%%%%%%%%%%%%%%%%%%%%%
%%%%%%%%%%%%%%%%%%%%%%%%%%%%%%%%%%%%%%%%%%%%%%%%%%%%%%%%%%%%%%%%%
\documentclass[11pt,english]{beamer}
\usepackage{lmodern}
\renewcommand{\sfdefault}{lmss}
\renewcommand{\ttdefault}{lmtt}
\usepackage[T1]{fontenc}
\usepackage[utf8]{inputenc}
\usepackage{listings}
\setcounter{secnumdepth}{3}
\setcounter{tocdepth}{3}
\usepackage{url}
\usepackage{graphicx}

\makeatletter
%%%%%%%%%%%%%%%%%%%%%%%%%%%%%% Textclass specific LaTeX commands.
\usepackage{Sweavel}
 \newenvironment{topcolumns}{\begin{columns}[t]}{\end{columns}}

%%%%%%%%%%%%%%%%%%%%%%%%%%%%%% User specified LaTeX commands.
\usepackage{dcolumn}
\usepackage{booktabs}
\usepackage{multicol}

% use 'handout' to produce handouts
%\documentclass[handout]{beamer}
\usepackage{wasysym}
\usepackage{pgfpages}
\newcommand{\vn}[1]{\mbox{{\it #1}}}\newcommand{\vb}{\vspace{\baselineskip}}\newcommand{\vh}{\vspace{.5\baselineskip}}\newcommand{\vf}{\vspace{\fill}}\newcommand{\splus}{\textsf{S-PLUS}}\newcommand{\R}{\textsf{R}}


\usepackage{graphicx}
\usepackage{listings}
\lstset{tabsize=2, breaklines=true,style=Rstyle}

\usetheme{Antibes}

%Switch comment character to turn on/off \pause commands given as \pausealt
%\newcommand{\pausealt}{\par }
\newcommand{\pausealt}{\pause}


\usepackage{color}
\definecolor{gray1}{gray}{0.75}

\newlength{\figurewidth}
\figurewidth \textwidth  % This is for rectangular graphs
\newlength{\figurewidthB}
\figurewidthB .7\textwidth  % This is for square graphs


\expandafter\def\expandafter\insertshorttitle\expandafter{%
  \insertshorttitle\hfill\insertframenumber\,/\,\inserttotalframenumber}

%=============================================================================
%

\title[Plotting II] % (short title, use only with long paper titles)
{Plotting II}

\subtitle{Important Ingredients in the Best Plot Recipes}

\author[Johnson] {Paul E. Johnson\inst{1,2}}

\institute[University of Kansas]{\inst{1} Department of Political Science \\
  University of Kansas \and \inst{2}Center for Research Methods and Data Analysis \\ University of Kansas} % (optional, but mostly needed)

\date[2013] % (optional, should be abbreviation of conference name)
{2013}

\subject{Plotting (2)}

%=============================================================================
%=============================================================================
\begin{document}




% In document Latex options:
\fvset{listparameters={\setlength{\topsep}{0em}}}

\def\Sweavesize{\scriptsize}
\def\Rcolor{\color{black}}
\def\Rbackground{\color[gray]{0.95}}

\begin{Schunk}
\begin{Sinput}
 ## Other settings I like
 options(device = pdf)
 options(useFancyQuotes = FALSE) 
 op <- par() 
 pjmar <- c(5.1, 5.1, 1.5, 2.1) 
 options(SweaveHooks=list(fig=function() par(mar=pjmar, ps=12)))
 pdf.options(onefile=F,family="Times",pointsize=12)
\end{Sinput}
\end{Schunk}



\begin{frame}
  \titlepage
\end{frame}

%=============================================================================



\begin{frame}
\frametitle{Outline}

\tableofcontents{}

\end{frame}





%=============================================================================


\section{Saving Plots}

\begin{frame}
  \frametitle{The Screen as an Output Device}

  \begin{itemize}
  \item You make a nice plot and view it on the screen.

  \item The ``window'' is a device, as far as R is concerned.
  \item You can create more than one plotting device. Try:
\begin{Schunk}
\begin{Sinput}
 dev.new()
\end{Sinput}
\end{Schunk}

or
\begin{Schunk}
\begin{Sinput}
 dev.new(height = 3, width = 7)
\end{Sinput}
\end{Schunk}

   That should create a new ``screen'' device and the next plot
   command you run will draw on it.

  \item Possible to have several devices, and alternate among them.

  \item What's my current device? Run:
\begin{Schunk}
\begin{Sinput}
 dev.cur()
\end{Sinput}
\end{Schunk}

  \item Related functions: ``dev.next()'' and ``dev.set()'' and ``dev.off()''
  \end{itemize}
\end{frame}


%_________________________
\begin{frame}
  \frametitle{A Printer is also a Device}

  \begin{itemize}

  \item Send the current device's output to the default printer

\begin{Schunk}
\begin{Sinput}
 dev.print(paper="letter", height=4, width=4)
\end{Sinput}
\end{Schunk}


\item Here's our major problem.  If the screen device is a different
  size \& shape, then copying or printing it to another device \&
  shape will usually create a ugly result.

\item Users frequently note that a plot looks great if they draw it on
  a 8 inch by 6 inch plot, but when they try to print that in a
  smaller section of paper, the result is horrible.

  \item Furthermore, on-screen devices do not always gracefully resize
    when we stretch and resize them.

\end{itemize}
\end{frame}

% _____________________________
\begin{frame}
  \frametitle{Save pdf, postscript, png, etc}
  \begin{itemize}

  \item Things we call ``file types'' or ``formats'' are \textbf{devices} in R.

  \item Recall the ``window'' where you see the graph is a ``device''
    in the eyes of R.

  \item Problems arise translating ``that'' device into a
    file-formatted device.

    \begin{itemize}
      \item Screen devices are generally ``square'', say 7'' by 7''
      \item  A screen device output sized for 7'' x 7'' will look
      bad if it is written into a 6 inch x 4 inch output device.
      \item This is a major problem with Rstudio, because it
        constrains the graphic device to be a particular rectangle on
        the screen.
    \end{itemize}
  \end{itemize}
\end{frame}

%__________________________________________

\begin{frame}
  \frametitle {Many Devices Are Available}

  \begin{itemize}

    \item Today, I almost always save in  \texttt{pdf} or \texttt{png}
      devices. \texttt{pdf} is scalable, \texttt{png} is for web pages.
    \item Many other excellent devices exist, I'm considering studying
      up on \texttt{svg} or \texttt{tikz}
    \item \texttt{xfig} device creates output that can be revised in
      xfig (Godfather of free line art software). That's useful sometimes.
    \item \texttt{pdf} is now R's default recommendation (was
      postscript when I started)

  \end{itemize}
\end{frame}


\begin{frame}[containsverbatim]
  \frametitle {Ask your system what it has.}

  \begin{itemize}
    \item Run
\begin{Schunk}
\begin{Sinput}
 ?Devices
\end{Sinput}
\end{Schunk}

    should list the base devices and some other possibilities your
    system \emph{might} have.

    \item Check your particular system for extra devices.
\begin{Schunk}
\begin{Sinput}
 capabilities()
\end{Sinput}
\end{Schunk}

Today I see
     \end{itemize}

\def\Sweavesize{\scriptsize}
\begin{Schunk}
\begin{Soutput}
jpeg      png     tiff    tcltk      X11     aqua http/ftp  sockets
TRUE     TRUE     TRUE     TRUE     TRUE    FALSE     TRUE     TRUE
libxml     fifo   cledit    iconv      NLS  profmem    cairo
TRUE     TRUE     TRUE     TRUE     TRUE     TRUE     TRUE
\end{Soutput}
\end{Schunk}

\end{frame}

%_____________________________________

\begin{frame}
  \frametitle {``But I Can Just Point and Click,'' you say...}

  Almost all of my students don't pay attention to me on this. They
  use a ``graphical environment'' with a pull down menu
  ``File Save As''.  I say STOP that.

  Why?
  \begin{itemize}
  \item \textbf{Quality} The result will be bad if the output device
    shape and size is not similar to the screen device. Better to
    create a ``certain'' sized device and let R write on it.
  \item \textbf{Reproducibility}: Pointing and Clicking does not equal
    reproducible output. You can't ``give'' that to students \& co-authors
  \item \textbf{Speed}: Suppose you need to scan through 200 datasets
    and make 3 plots each. Do you really want to point-and-click
    through there? Really?
  \end{itemize}
\end{frame}

%________________________________________

\begin{frame}
  \frametitle {The Step-by-Step Procedure}

  Your Code will
  \begin{enumerate}
  \item Create an output device (pdf, png, etc)
  \item Run the commands that create the figure (will not show on the screen)

  \item Turn off the device in order to finish off the file.
  \end{enumerate}

  While you are ``experimenting'' with code, don't execute
  code in steps 1 and 3, so you can see output on screen. If you see
  my example code with \texttt{if (SAVEME) pdf(file =
    "somefile.pdf", ...)}, you will know what what I mean.

  Check WorkingExample scripts like ``glm-logit-unbalanced-1.R'' or ``glm-logit-compareLinks-1.R''

\end{frame}

%________________________________________



\begin{frame}
  \frametitle {Save Files from the R Script}

  \begin{itemize}

    \item I try to remember to have this ``boiler plate'' at the top
      of every R program that I write when I might need to save graphs
\begin{Schunk}
\begin{Sinput}
 pdf.options(onefile = FALSE, family = "Times", paper = "special", height = 4, width = 6)
 ps.options(horizontal = FALSE, onefile = FALSE, family = "Times", paper = "special", height = 4, width = 6 )
 options(papersize="special")
\end{Sinput}
\end{Schunk}


    \item These set my taste for the ``postscript'' and ``pdf'' output devices.

    \item \emph{IF} I remember those, then saving a file is not so tedious.
  \end{itemize}
\end{frame}

%________________________________________________________
\begin{frame}
  \frametitle {Explain those arguments}
  \begin{description}
    \item [horizontal = FALSE]: I prefer ``portrait'' output
    \item [onefile = FALSE]: Don't try to put all plots into a single file (I
      want separate images in separate files)
    \item [onefile = FALSE]: vital for postscript devices. Otherwise, no
      bounding box is included and the result is not true EPS  (Encapsulated Postscript)
    \item [paper = "special"]: Helps with pdf output. Otherwise, it
      assumes all pdf images are full page size
    \item [family="Times"]: I prefer to see figures with fonts that
      match the text.
  \end{description}
\end{frame}


%________________________________________________________
\begin{frame}[containsverbatim]
  \frametitle{Typical Experience}
  \begin{itemize}
  \item First, begin with the process of creating a graph. Run it on
    screen until you get it right, and then ``wrap'' your commands
    in this way:

\begin{Schunk}
\begin{Sinput}
 pdf(file = "myimage.pdf")
 ##or postscript(file = "myimage.eps")
 ##"put your plot commands here")
 dev.off()   #vital to save your file!
\end{Sinput}
\end{Schunk}


  \item BIG WARNING 1. If you ``turn on'' the file device, you
    will not get to see the output on the screen in the R graphic device.
  \item BIG WARNING 2. If you ``turn on'' an output device, and forget
    to run dev.off(), you will cause lots of trouble for yourself. Try
    it and see.

  \end{itemize}
\end{frame}


%___________________________________


\begin{frame}[containsverbatim]
  \frametitle {If You Forget the Boilerplate}
  \begin{itemize}
  \item You omit pdf.options, then you need to write in details each time, e.g.

  \begin{Schunk}
\begin{Sinput}
 pdf(file = "myfavorite1.pdf", onefile = FALSE, family = "Times", paper = "special", height = 4, width = 6)
 #hist(x) or whatever
 dev.off()
\end{Sinput}
\end{Schunk}


    \item Instead of

  \begin{Schunk}
\begin{Sinput}
 pdf(file="myfavorite1.pdf")
 #hist(x) or whatever
 dev.off()
\end{Sinput}
\end{Schunk}


  \item Even if you did set pdf.options, you can customize


  \begin{Schunk}
\begin{Sinput}
 pdf(file="myfavorite1.pdf", height=8, width=7)
 #hist(x) or whatever
 dev.off()
\end{Sinput}
\end{Schunk}


  \end{itemize}
\end{frame}


%___________________________________


\begin{frame}[containsverbatim]
  \frametitle {Conditionalize Code To Make This Less Tedious}
  \begin{itemize}
  \item Do your on-screen work like this
\begin{Schunk}
\begin{Sinput}
 par(mfcol=c(3,1))
 for (i in 1:3){
      minx <- 20 + 50 * (i-1)
      xvals <- seq(minx,300, length.out=1000)
      gam <- dgamma(xvals, shape = 2, scale = 1)
      igaus <- dinvgauss(xvals, mu = 2, lambda = 5)
      plot(xvals,gam, type="l", lty = 1, main="")
      lines(xvals, igaus, lty = 2)
      legend(150, 0.7*max(gam), c("gamma(sh=2,sc=1)", "inv gauss(mu=2,lambda=5)"), lty = 1:2,)
  }
\end{Sinput}
\end{Schunk}


    \item To save figures in files, change
    \begin{verbatim}
SAVEME <- TRUE
     \end{verbatim}

      and then run the whole script over again.
  \end{itemize}
\end{frame}

%______________________________________________

\begin{frame}[containsverbatim]
  \frametitle {Another Way To Make This Less Tedious}
  \begin{itemize}
  \item Let R number the figures for you.

  \item Put this in the program BEFORE creating plots

\begin{Schunk}
\begin{Sinput}
 xpoly <- poly(dat$x, 2)
 head(xpoly)
\end{Sinput}
\begin{Soutput}
               1         2
[1,] -0.09043700 0.1234184
[2,] -0.08802246 0.1136961
[3,] -0.08764475 0.1121995
[4,] -0.08763258 0.1121514
[5,] -0.08749189 0.1115957
[6,] -0.08724543 0.1106244
\end{Soutput}
\begin{Sinput}
 (pcorr <- cor(xpoly[, 1], xpoly[ ,2]))
\end{Sinput}
\begin{Soutput}
[1] -1.599071e-16
\end{Soutput}
\end{Schunk}


  \item When finished, tell R to stop creating output files.
\begin{Schunk}
\begin{Sinput}
 head(model.matrix(mod1))
\end{Sinput}
\begin{Soutput}
  (Intercept) schtypeprivate
1           1              1
3           1              0
5           1              0
6           1              1
7           1              0
8           1              0
\end{Soutput}
\end{Schunk}


   \item When you run this program, no graphs will ``show on the screen.''
   \item Each graph you draw will go in a separate file,
     \texttt{myfile001.pdf}, \texttt{myfile002.pdf},
     \texttt{myfile003.pdf}.

  \item The downside here is that all figures have to have same
    height, width
  \end{itemize}
\end{frame}

%_______________________________________________

\begin{frame}
  \frametitle{Example Code that Writes a Bunch of Figures}


\begin{Schunk}
\begin{Sinput}
 for (i in 1:10){
    fn <- paste("fig", i, ".pdf", sep="")
    mm <- 3*i; msd <- i^2;
    mtitle <- paste("Histogram", i, "m =", mm, "sd =", msd)
    pdf(file = fn)
    hist(rnorm(1000, mean = 3*i, sd = i^2),  main = mtitle)
    dev.off()
   }
\end{Sinput}
\end{Schunk}


\end{frame}


%___________________________________________
\section{Layout}

\begin{frame}[containsverbatim]
  \frametitle{Brief Note on par}
  \begin{itemize}
  \item Many settings that affect graphs are sitting ``out there''
    in memory
  \item To see them, run
\begin{Schunk}
\begin{Sinput}
 par()
\end{Sinput}
\end{Schunk}

  \item To see a particular one, run
\begin{Schunk}
\begin{Sinput}
 plot( prestige ~ education, data=Prestige, col=Prestige$type)
 m7 <- lm(prestige ~ type / education, data=Prestige)
 nd2 <- expand.grid(education=range(Prestige$education), type = levels(Prestige$type))
 nd2$pred <- predict(m7, newdata=nd2)
 for(i in 1:3){
  with(nd2[nd2$type %in% levels(nd2$type)[i], ], 
     lines(education, pred, col=i, lty=i, lwd=2))
  }
 legend("bottomright", legend=levels(Prestige$type), lty=1:3, col=1:3, lwd=2)
 outreg(m7, tight=F)
\end{Sinput}
\end{Schunk}

  \item To set one, run
\begin{Schunk}
\begin{Sinput}
 attributes(s3d)
\end{Sinput}
\begin{Soutput}
$names
[1] "xyz.convert" "points3d"    "plane3d"     "box3d"      
\end{Soutput}
\end{Schunk}

  \end{itemize}
\end{frame}

%_______________________________
\begin{frame}[containsverbatim]
  \frametitle{Brief Note on par}
  \begin{itemize}
  \item Danger: those settings will remain in effect for the
    duration of the device.
  \item When you close that device, all of the changes are
    ``forgotten'' and par() for that device reverts.
    \begin{itemize}
    \item Need to re-run if you change devices
    \item Many people recommend getting in the routine of saving par
      settings and replacing them when done.
\begin{Schunk}
\begin{Sinput}
 plot(salary ~ age, data = mydf)
 plot(salary ~ age, data = mydf, cex = 2)
 plot(salary ~ age, data = mydf, col = "pink", cex = 2, pch = 16)
 plot(salary ~ age, data = mydf, col = as.numeric(sex), cex = salary, pch = 16)
 plot(salary ~ age, data = mydf, type = "l")
 plot(salary ~ age, data = mydf, type = "b")
 plot(salary ~ age, data = mydf, type = "h")
 plot(salary ~ age, data = mydf, type = "s")
 plot(salary ~ age, data = mydf, type = "c")
 text(salary ~ age, data = mydf, labels = 1:8)
\end{Sinput}
\end{Schunk}

    \item Seems somewhat unnecessary to me because you can close
      the device and ``start fresh''

    \end{itemize}
  \end{itemize}
\end{frame}

%___________________________________________

\begin{frame}[containsverbatim]
  \frametitle{par: mfcol and mfrow}
  \begin{itemize}
  \item Either of these will cause R to subdivide a plotting device into 3 rows
\begin{Schunk}
\begin{Sinput}
 dat <- genCorrelatedData(1000, stde=5)
 dat$x3 <- rnorm(1000)
 dat$x4 <- rnorm(1000)
 dat$x5 <- factor(c( rep("A", 700), rep("B", 300)))
 m2 <- lm( y ~ x1 + x2 + x3 + x4 + x5, data=dat)
 m2.data <- model.data(m2)
 newdata1 <- newdata(m2, predVals = list(x1 = "quantile"))
 newdata1$fit <- predict(m2, newdata=newdata1)
 print(newdata1, digits = 3)
\end{Sinput}
\begin{Soutput}
    x1   x2     x3       x4 x5  fit
1 43.3 50.5 0.0162 -0.00544  A 18.6
2 50.2 50.5 0.0162 -0.00544  A 20.0
3 57.2 50.5 0.0162 -0.00544  A 21.4
\end{Soutput}
\end{Schunk}

  \item Either of these will subdivide a plotting device into 3 rows, 2 columns
\begin{Schunk}
\begin{Sinput}
 fit <- list(call = call, call1 = object$call, call2 = object$call1, nmis = object$nmis, m = m, qhat = qhat, u = u, qbar = qbar, ubar = ubar, b = b, t = t, r = r, dfcom = dfcom, df = df,         fmi = fmi, lambda = lambda)
\end{Sinput}
\end{Schunk}

  \end{itemize}
\end{frame}


%_______________________________
\begin{frame}[containsverbatim]
  \frametitle{mfrow fills rows, mfcol fills columns}
  \begin{itemize}

  \item The order of ``drawing'' is the difference between mfcol and mfrow

  \item  Suppose 6 plots are A, B, C, D, E, F
  \end{itemize}

\begin{topcolumns}
  \column{6cm}
  \begin{description}
    \item [mfcol]  fills the columns first
      \begin{tabular}{|c|c|}
        \hline
        A & D \tabularnewline
        \hline
        \hline
        B & E\tabularnewline
        \hline
         \hline
        C & F\tabularnewline
        \hline
         \hline
      \end{tabular}
    \end{description}

   \column{6cm}
   \begin{description}

   \item [mfrow]  fills the rows first

     \begin{tabular}{|c|c|}
       \hline
       A & B \tabularnewline
       \hline
       \hline
       C & D\tabularnewline
       \hline
       \hline
       E & F\tabularnewline
       \hline
       \hline
     \end{tabular}
   \end{description}
 \end{topcolumns}
\end{frame}

%_______________________________
\begin{frame}[containsverbatim]
  \frametitle{Example: to fit horizontal screen}

\begin{Schunk}
\begin{Sinput}
 x <- rpois(1000, lambda = 108);
 y <- rnorm(1000, m = x, sd = 10);
 par(mfcol = c(1,3))
 hist(x, main = "Default hist(x)")
 hist(y, main = "Default hist(y)")
 plot(x, y, main = "Scatterplot x, y")
\end{Sinput}
\end{Schunk}


\end{frame}


%_______________________________
\begin{frame}[containsverbatim]

\includegraphics[totalheight=65mm]{plots/t-mfcol10}


\end{frame}

%_______________________________
\begin{frame}[containsverbatim]
  \frametitle{When you are finished...}

  Either:
  \begin{itemize}
  \item Reset the option manually

    par(mfcol=c(1,1))

  \item Or close the device

    dev.off()

  \end{itemize}
\end{frame}

%________________________________

\begin{frame}
  \frametitle{Some R Functions Use \texttt{mfcol} Behind the Scenes}
  \begin{itemize}
  \item Suppose you have a factor variable like ``race'' or ``sex''
  \item coplot generates separate plot for each value of the factor
\begin{Schunk}
\begin{Sinput}
 example(coplot)
\end{Sinput}
\end{Schunk}


\end{itemize}
\end{frame}


%_______________________________
\begin{frame}[containsverbatim]
  \frametitle{R's Layout() Function Offers More Control}

  \begin{itemize}
  \item Check the current layout:

    \texttt{"layout.show(1)"}

    It should put a numeral ``1'' on a screen device.
  \item layout wants you to specify an R matrix to
    \begin{itemize}
    \item divide the space by rows and columns, and
    \item let R know which figure will be drawn in which area
    \end{itemize}

  \item Stupid example

    \begin{equation}
      mymat=\left[\begin{array}{cc}
          1 & 1\\
          1 & 1 \\
        \end{array}\right]\end{equation}


    Meaning: divide my plotting area into 4 parts, but let
    figure ``1'' use all 4 parts.


  \end{itemize}
\end{frame}

%____________________________________________

\begin{frame}[containsverbatim]

\begin{Schunk}
\begin{Sinput}
 mymat <- matrix( c(1, 1, 1, 1), ncol = 2)
 layout(mat = mymat)
 layout.show(1)
\end{Sinput}
\end{Schunk}

\includegraphics[totalheight=65mm]{plots/t-layout10}
\end{frame}

%______________________________________

\begin{frame}[containsverbatim]

\includegraphics[totalheight=65mm]{plots/t-layout11}

\end{frame}

%___________________________________

\begin{frame}[containsverbatim]
\begin{Schunk}
\begin{Sinput}
 mymat <- matrix(c(0,2,1,2), ncol = 2)
 layout(mat = mymat)
 layout.show(2)
\end{Sinput}
\end{Schunk}

\includegraphics[totalheight=65mm]{plots/t-layout20}

\end{frame}

%____________________________

\begin{frame}[containsverbatim]
  \frametitle{R Squishes the Figures to Fit!}

\includegraphics[totalheight=65mm]{plots/t-layout21}
\end{frame}

\begin{frame}
  \frametitle{Layout 5 images in a ``rainbow'' shape}

  \begin{equation}
    mymat=\left[\begin{array}{ccc}
        1 & 1 & 1\\
        2 & 0 & 3\\
        4 & 0 & 5\end{array}\right]\end{equation}


\begin{Schunk}
\begin{Sinput}
 mymat <- matrix(c(1,2,4,1,0,0,1,3,5), nrow = 3)
 layout(mat = mymat)
 layout.show(5)
\end{Sinput}
\end{Schunk}


\end{frame}

%_______________________________________________

\begin{frame}[containsverbatim]
  \frametitle{You Need 5 Plots for 5 Indicated Spaces}

\begin{Schunk}
\begin{Sinput}
 layout(mat = mymat)
 hist(x, main = "Default hist(x)")
 hist(y, main = "Default hist(y)")
 plot(x, y, main = "Scatterplot x, y")
 plot(x, y, main = "Top Right Corner of Scatter", xlim = c(mean(x), max(x)), ylim = c(mean(y),max(y)))
 plot(x, y, main = "Bottom Left Corner of Scatter", xlim = c(min(x), mean(x)), ylim = c(min(y), mean(y)))
\end{Sinput}
\end{Schunk}



\end{frame}

%_____________________________________________

\begin{frame}[containsverbatim]
  \frametitle{Voila}
  \includegraphics[totalheight=65mm]{plots/t-layout31}

\end{frame}

%_______________________________________________

\begin{frame}
  \frametitle{Cautionary Comment about Layout}
  \begin{itemize}
    \item Layout affords the ability to fill a page with an
      arbitrarily complicated array of figures.
    \item I worry that some of this effort is wasted because some/many
      journals do not want a full page layout of graphs. If they will
      give you the space at all, they probably want the individual
      figures in separate graphs that can be arranged according to
      their style.
    \item Exception would be layout used to link together inseparable
      drawings, e.g. ``rug'' plots below scatterplots.
  \end{itemize}
\end{frame}

%___________________________________________
\section{Lattice Layouts}

\begin{frame}
  \frametitle{Trellis Graphics with the lattice Package}

  \begin{itemize}
  \item Trellis graphics in S/S+ are highly prized (natural scientists
    especially)
  \item Main idea: make it easy/simple to create many plots that
    subdivide a sample and illustrate for subgroups (coplot, but a
    fully worked out framework)
  \item Deepayan Sarkar contributed the lattice package for R
  \item Some see lattice graphs as a complete replacement for
    existing 2D plotting framework (e.g., Doug Bates).
  \item I am more cautious, possibly because
    \begin{itemize}
    \item lattice graphics came along after I had mastered the others
    \item seems tougher to get individual pieces ``just right''
    \item see previous concern about difficulty of layout in publication.
    \end{itemize}
  \end{itemize}
\end{frame}

%___________________________________________

\begin{frame}
  \frametitle{xyplot is the workhorse generic function}
  \begin{itemize}
    \item xyplot will accept a formula and figure out what to do with it.
    \item After a vertical bar, | , factors indicate desired subgroups.


  \end{itemize}

\end{frame}

% _______________________________________________

\begin{frame}[containsverbatim]




\begin{Schunk}
\begin{Sinput}
 xyplot( hrs1 ~ age | sex , data=dat)
\end{Sinput}
\end{Schunk}


\includegraphics[totalheight=65mm]{plots/t-xyplot11}

\end{frame}

%_____________________________

\begin{frame}[containsverbatim]



\begin{Schunk}
\begin{Sinput}
 histogram(~income06  | sex, data=dat)
\end{Sinput}
\end{Schunk}


\includegraphics[totalheight=65mm]{plots/t-lathist31}

\end{frame}

%_____________________________

\begin{frame}[containsverbatim]



\begin{Schunk}
\begin{Sinput}
 histogram(~I(numwomen + nummen)  | sex, data = dat)
\end{Sinput}
\end{Schunk}


\includegraphics[totalheight=65mm]{plots/t-lathist41}


\end{frame}

%_____________________________

\begin{frame}[containsverbatim]



\input{plots/t-lathist50}

\includegraphics[totalheight=65mm]{plots/t-lathist51}

\end{frame}

%_____________________________

\begin{frame}[containsverbatim]




\begin{Schunk}
\begin{Sinput}
 histogram(~I(numwomen+nummen)  | sex, data=dat, xlim=c(0,100), breaks=100)
\end{Sinput}
\end{Schunk}


\includegraphics[totalheight=65mm]{plots/t-lathist61}


\end{frame}

%_____________________________

\begin{frame}[containsverbatim]

  Appears we just need to cut some extreme values.

\begin{Schunk}
\begin{Sinput}
 dat$numwomen[dat$numwomen > 100] <- NA
 dat$nummen[dat$nummen > 100] <- NA
\end{Sinput}
\end{Schunk}


\end{frame}

%_____________________________

\begin{frame}[containsverbatim]



\input{plots/t-lathist70}

\includegraphics[totalheight=65mm]{plots/t-lathist71}



\end{frame}

%______________________________

\begin{frame}[containsverbatim]





\begin{Schunk}
\begin{Sinput}
 histogram(~I(numwomen+nummen)  | sex * sexsex5, data=dat)
\end{Sinput}
\end{Schunk}

\includegraphics[totalheight=65mm]{plots/t-lathist81}


\end{frame}


%_____________________________

\begin{frame}[containsverbatim]



\begin{Schunk}
\begin{Sinput}
 histogram(~I(numwomen+nummen)  | sexsex5 * sex , data=dat)
\end{Sinput}
\end{Schunk}

\includegraphics[totalheight=65mm]{plots/t-lathist91}

\end{frame}

%_________________________________________
\section{Plotmath}

\begin{frame}
  \frametitle{The plotmath Framework}

  \begin{itemize}
  \item R has a ``\LaTeX like'' math markup called ``plotmath''
  \item Create an R ``expression'', then when plotted, math will appear!
  \item read \texttt{?plotmath}
    \item run \texttt{example(plotmath)}

  \end{itemize}
\end{frame}

%__________________________________

\begin{frame}[containsverbatim]
  \frametitle{Simple example}


\input{plots/t-pm10}

\end{frame}

%__________________________________

\begin{frame}[containsverbatim]
  \frametitle{Simple example}


\includegraphics[totalheight=65mm]{plots/t-pm11}
\end{frame}

%____________________________

\begin{frame}
  \frametitle{A Few plotmath Tips}
  \begin{itemize}
    \item Two Equal Signs (== outputs =)
  %%  \item $\[\]$ marks subscripts, ^ for superscripts
    \item Want * to show? Type \%*\%
    \item Want $\cdot$ ? Type  cdot

  \end{itemize}
\end{frame}

%____________________________

\begin{frame}[containsverbatim]
  \frametitle{The Only Really Big Gotcha Is...}

  \begin{itemize}
  \item Blending calculations and math.
  \item This does not work:
\begin{Sinput}
x <- 10
text(4, 5, expression(paste("x is ", x)))
\end{Sinput}

  \item Instead, a more elaborate procedure is required.
  \item One method is bquote


  \end{itemize}
\end{frame}

%________________________

\begin{frame}
  \frametitle{bquote example usage}


\input{plots/t-pm20}

\end{frame}


%_________________________________
\begin{frame}[containsverbatim]


\includegraphics[totalheight=65mm]{plots/t-pm21}

\end{frame}

%_______________________________
\begin{frame}
  \frametitle{I've Been Proposing This as a plotmath Example}

















\includegraphics[totalheight=70mm]{plots/t-pm91}

\end{frame}

%____________________________________________

\begin{frame}
  \frametitle{Set Parameters, Create $x$, $y$, Make Title}
\input{plots/t-pm90A}

\end{frame}
%____________________________________________

\begin{frame}
  \frametitle{Create Plot and axes}
\begin{Schunk}
\begin{Sinput}
 ### xpd needed to allow writing outside strict box of graph
 par(xpd=TRUE, ps=10)
 plot(myx, myDensity, type="l", xlab="x", ylab="Probability Density ", main=myTitle1, axes=FALSE)
 axis(2, pos= mu - 3.6*sigma)
 axis(1, pos=0)
\end{Sinput}
\end{Schunk}


\end{frame}

%________________________________________

\begin{frame}[plain]

\includegraphics[totalheight=70mm]{plots/t-pm92AB}
\end{frame}

%____________________________________________

\begin{frame}
  \frametitle{Label Lower Axis, Draw Line Across Range}
\begin{Schunk}
\begin{Sinput}
 lines(c(myx[1],myx[length(myx)]),c(0,0)) ### closes off axes
 # bquote creates an expression that text plotters can use
 t1 <-  bquote( mu== .(mu))
 # Find a value of myx that is "very close to" mu
 centerX <- max(which (myx <= mu))
 # plot light vertical line under peak of density
 lines( c(mu, mu), c(0, myDensity[centerX]), lty= 14, lwd=.2)
 # label the mean in the bottom margin
 mtext(bquote( mu == .(mu)), 1, at=mu, line=-1)
\end{Sinput}
\end{Schunk}


\end{frame}


%________________________________________

\begin{frame}[plain]

\includegraphics[totalheight=70mm]{plots/t-pm92C}
\end{frame}
%____________________________________________

\begin{frame}
  \frametitle{Mark Interval from mean to $\sigma$}
\begin{Schunk}
\begin{Sinput}
 ### find position 20% "up" vertically, to use for arrow coordinate
 ss = 0.2 * max(myDensity)
 # Insert interval to represent width of one sigma
 arrows( x0=mu, y0= ss, x1=mu+sigma, y1=ss, code=3, angle=90, length=0.1)
 ### Write the value of sigma above that interval
 t2 <-  bquote( sigma== .(round(sigma,2)))
 text( mu+0.5*sigma, 1.15*ss, t2)
\end{Sinput}
\end{Schunk}


\end{frame}



%________________________________________

\begin{frame}[plain]

\includegraphics[totalheight=70mm]{plots/t-pm92D}
\end{frame}

%____________________________________________

\begin{frame}
  \frametitle{Create formula for Normal Distribution}
\begin{Schunk}
\begin{Sinput}
 ### Create a formula for the Normal
 normalFormula <- expression (f(x) == frac (1, sigma* sqrt(2*pi)) * e^{~~ - ~~ frac(1,2)~~ bgroup("(", frac(x-mu,sigma),")")^2})
 # Draw the Normal formula
 text ( mu + 0.5*sigma, max(myDensity)- 0.10 * max(myDensity),  normalFormula, pos=4)
\end{Sinput}
\end{Schunk}


\end{frame}

%________________________________________

\begin{frame}[plain]

\includegraphics[totalheight=70mm]{plots/t-pm92E}
\end{frame}

%____________________________________________

\begin{frame}
  \frametitle{Write Left Side Critical Value Below Axis}
\begin{Schunk}
\begin{Sinput}
 ### Theory says we should have 2.5% of the area to the left of: -1.96 * sigma.
 ### Find the X coordinate of that "critical value"
 criticalValue <- mu -1.96 * sigma
 criticalValue <- round(criticalValue, 2)
 ### Then grab all myx values that are "to the left" of that critical value.
 specialX <-  myx[myx <= criticalValue]
 ### mark the critical value in the graph
 text ( criticalValue, 0 ,
       label=criticalValue, pos=1)
\end{Sinput}
\end{Schunk}


\end{frame}

%________________________________________

\begin{frame}[plain]

\includegraphics[totalheight=70mm]{plots/t-pm92F}
\end{frame}

%____________________________________________

\begin{frame}
  \frametitle{Use polygon to Shade the Area Under Curve on Left}
\begin{Schunk}
\begin{Sinput}
 ### Take sequence parallel to values of myx inside critical region
 specialY <- myDensity[myx < criticalValue]
 #  Polygon makes a nice shaded illustration
 polygon(c(specialX[1], specialX, specialX[length(specialX )]), c(0, specialY, 0), density=c(-110),col="lightgray" )
\end{Sinput}
\end{Schunk}


\end{frame}


%________________________________________

\begin{frame}[plain]

\includegraphics[totalheight=70mm]{plots/t-pm92G}
\end{frame}

%____________________________________________

\begin{frame}
  \frametitle{Make Calculations about Area on Left Side}
\begin{Schunk}
\begin{Sinput}
 ### I want to insert annotation about area on left side.
 shadedArea <- round(pnorm(mu - 1.96 * sigma, mean=mu, sd=sigma),4)
 al1 <- bquote(Prob(x <= .(criticalValue)))
 al2 <- bquote(phantom(0) == F( .(criticalValue) ))
 al3 <- bquote(phantom(0) == .(round(shadedArea,3)))
 ### Hard to position this text "just right"
 ### Have tried many ideas, this may be least bad.
 ### Get center position in shaded area
 medX <- median(specialX)
 ### density at that center point:
 denAtMedX <- myDensity[indexMed <- max(which(specialX < medX))]
\end{Sinput}
\end{Schunk}


\end{frame}

%________________________________________

\begin{frame}
  \frametitle{Write Annotation Above Shaded Area on Left}
  \begin{Schunk}
\begin{Sinput}
 text(medX, denAtMedX+0.0055, labels = al1)
 text(medX, denAtMedX+0.004, labels = al2)
 text(medX, denAtMedX+0.0025, labels = al3)
 ### point from text toward shaded area
 arrows( x0 = medX, y0 = myDensity[indexMed]+0.002,
        x1 = mu-2.5 *sigma, y1 = 0.40*myDensity[length(specialX)], length = 0.1)
\end{Sinput}
\end{Schunk}

\end{frame}
%________________________________________

\begin{frame}[plain]

\includegraphics[totalheight=70mm]{plots/t-pm92H}
\end{frame}
%____________________________________________

\begin{frame}
  \frametitle{Insert arrow and text from mean to $-1.96 \sigma$}
\begin{Schunk}
\begin{Sinput}
 ss <- 0.1 * max(myDensity)
 ### Mark interval from mu to mu-1.96*sigma
 arrows( x0 = mu, y0 = ss, x1 = mu-1.96*sigma, y1 = ss, code = 3, angle = 90, length = 0.1)
 ### Put text above interval
 text( mu - 2.0*sigma,1.15*ss, bquote(paste(.(criticalValue), phantom(1)==mu-1.96 * sigma,sep="")), pos=4)
\end{Sinput}
\end{Schunk}


\end{frame}

%________________________________________

\begin{frame}[plain]

\includegraphics[totalheight=70mm]{plots/t-pm92I}
\end{frame}
%____________________________________________

\begin{frame}
  \frametitle{Shade right side critical region}

\begin{Schunk}
\begin{Sinput}
 criticalValue <- mu +1.96 * sigma
 criticalValue <- round(criticalValue, 2)
 ### Then grab all myx values that are "to the left" of that critical value.
 specialX <-  myx[myx >= criticalValue]
 ### mark the critical value in the graph
 text ( criticalValue, 0 , label= paste(criticalValue), pos=1)
 ### Take sequence parallel to values of myx inside critical region
 specialY <- myDensity[myx > criticalValue]
 #  Polygon makes a nice shaded illustration
 polygon(c(specialX[1], specialX, specialX[length(specialX )]), c(0, specialY, 0), density=c(-110), col = "lightgray" )
\end{Sinput}
\end{Schunk}


\end{frame}
%________________________________________

\begin{frame}[plain]

\includegraphics[totalheight=70mm]{plots/t-pm92J}
\end{frame}
%____________________________________________

\begin{frame}
  \frametitle{Gather data on right side shaded area}
\begin{Schunk}
\begin{Sinput}
 shadedArea <- round(pnorm(mu + 1.96 * sigma, mean = mu, sd = sigma, lower.tail = F),4)
 ### Insert simpler comment on right side.
 
 al2 <- bquote(phantom(0) == 1 - F( .(criticalValue) ))
 al3 <- bquote(phantom(0) == .(round(shadedArea,3)))
\end{Sinput}
\end{Schunk}


\end{frame}
%________________________________________

% \begin{frame}[plain]

% <<pm92K,fig=T,echo=T,include=F,height=6, width=6>>=
% <<pm90A>>
% <<pm90B>>
% <<pm90C>>
% <<pm90D>>
% <<pm90E>>
% <<pm90F>>
% <<pm90G>>
% <<pm90H>>
% <<pm90I>>
% <<pm90J>>
% <<pm90K>>
% @
% \includegraphics[totalheight=65mm]{plots/t-pm92K}
% \end{frame}
%____________________________________________

\begin{frame}
  \frametitle{Annotate right shaded area}
\begin{Schunk}
\begin{Sinput}
 ### Hard to position this text "just right"
 ### Have tried many ideas, this may be least bad.
 ### Get center position in shaded area
 medX <- median(specialX)
 ### density at that center point:
 denAtMedX <- myDensity[indexMed <- max(which(specialX < medX))]
 text(medX, denAtMedX+0.004, labels = al2)
 text(medX, denAtMedX+0.0025, labels = al3)
 ### point from text toward shaded area
 arrows(x0 = medX, y0 = myDensity[indexMed]+0.002, x1 = mu+2.5 *sigma, y1 = 0.40*myDensity[length(specialX)] ,   length = 0.1)
\end{Sinput}
\end{Schunk}


\end{frame}
%________________________________________

\begin{frame}[plain]

\includegraphics[totalheight=70mm]{plots/t-pm92L}
\end{frame}
%____________________________________________

\begin{frame}
  \frametitle{Insert arrow from mean to $1.96 \sigma$}
\input{plots/t-pm90M}

\end{frame}
%________________________________________

\begin{frame}[plain]


\includegraphics[totalheight=70mm]{plots/t-pm92M}

\end{frame}
%__________________________

% ==========================================================

\end{document}
