\begin{Schunk}
\begin{Sinput}
 ###Set mu and sigma at your pleasure:
 mu <- 10.03
 sigma <- 12.5786786
 myx <- seq( mu - 3.5*sigma,  mu+ 3.5*sigma, length.out=500)
 myDensity <- dnorm(myx,mean=mu,sd=sigma)
 # It is challenging to combine plotmath with values of mu and sigma in one expression.
 # Either bquote or substitute can be used.  First use bquote:
 
 myTitle1 <- bquote (paste("x ~ Normal(", mu== .(round(mu,2)), ',', sigma== .(round(sigma,2)),")") )
 ### Using substitute:
 ### myTitle1 <-  substitute( "x ~ Normal" ~~ group( "(", list(mu==mu1, sigma^2==sigma2)#, ")") ,  list(mu1=round(mu,2), sigma2=round(sigma^2,2)))
 
 ### Or combine the two:
 ### t1 <- substitute ( mu == a ,  list (a = mu))
 ### t2 <- substitute ( sigma == a, list(a = round(sigma,2)))
 ### myTitle1 <- bquote(paste("x ~ Normal(", .(t1),",", .(t2),")" ) )
 ### xpd needed to allow writing outside strict box of graph
 par(xpd=TRUE, ps=10)
 plot(myx, myDensity, type="l", xlab="x", ylab="Probability Density ", main=myTitle1, axes=FALSE)
 axis(2, pos= mu - 3.6*sigma)
 axis(1, pos=0)
 lines(c(myx[1],myx[length(myx)]),c(0,0)) ### closes off axes
 # bquote creates an expression that text plotters can use
 t1 <-  bquote( mu== .(mu))
 # Find a value of myx that is "very close to" mu
 centerX <- max(which (myx <= mu))
 # plot light vertical line under peak of density
 lines( c(mu, mu), c(0, myDensity[centerX]), lty= 14, lwd=.2)
 # label the mean in the bottom margin
 mtext(bquote( mu == .(mu)), 1, at=mu, line=-1)
\end{Sinput}
\end{Schunk}
