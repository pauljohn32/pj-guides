%% LyX 2.0.3 created this file.  For more info, see http://www.lyx.org/.
%% Do not edit unless you really know what you are doing.
\documentclass[10pt,english]{beamer}
\usepackage{lmodern}
\renewcommand{\sfdefault}{lmss}
\renewcommand{\ttdefault}{lmtt}
\usepackage[T1]{fontenc}
\usepackage[latin9]{inputenc}
\usepackage{listings}
\setcounter{secnumdepth}{3}
\setcounter{tocdepth}{3}
\usepackage{url}

\makeatletter
%%%%%%%%%%%%%%%%%%%%%%%%%%%%%% Textclass specific LaTeX commands.
\usepackage{Sweavel}
 \def\lyxframeend{} % In case there is a superfluous frame end
 \long\def\lyxframe#1{\@lyxframe#1\@lyxframestop}%
 \def\@lyxframe{\@ifnextchar<{\@@lyxframe}{\@@lyxframe<*>}}%
 \def\@@lyxframe<#1>{\@ifnextchar[{\@@@lyxframe<#1>}{\@@@lyxframe<#1>[]}}
 \def\@@@lyxframe<#1>[{\@ifnextchar<{\@@@@@lyxframe<#1>[}{\@@@@lyxframe<#1>[<*>][}}
 \def\@@@@@lyxframe<#1>[#2]{\@ifnextchar[{\@@@@lyxframe<#1>[#2]}{\@@@@lyxframe<#1>[#2][]}}
 \long\def\@@@@lyxframe<#1>[#2][#3]#4\@lyxframestop#5\lyxframeend{%
   \frame<#1>[#2][#3]{\frametitle{#4}#5}}

%%%%%%%%%%%%%%%%%%%%%%%%%%%%%% User specified LaTeX commands.
\usepackage{dcolumn}
\usepackage{booktabs}

% use 'handout' to produce handouts
%\documentclass[handout]{beamer}
\usepackage{wasysym}
\usepackage{pgfpages}
\newcommand{\vn}[1]{\mbox{{\it #1}}}\newcommand{\vb}{\vspace{\baselineskip}}\newcommand{\vh}{\vspace{.5\baselineskip}}\newcommand{\vf}{\vspace{\fill}}\newcommand{\splus}{\textsf{S-PLUS}}\newcommand{\R}{\textsf{R}}


\usepackage{graphicx}
\usepackage{listings}
\lstset{tabsize=2, breaklines=true,style=Rstyle}
%\usetheme{Warsaw}
% or ...

%\setbeamercovered{transparent}
% or whatever (possibly just delete it)

\mode<presentation>
{
  \usetheme{KU}
  \usecolortheme{dolphin} %dark blues
}

% In document Latex options:
\fvset{listparameters={\setlength{\topsep}{0em}}}
\def\Sweavesize{\normalsize}
\def\Rcolor{\color{black}}
\def\Rbackground{\color[gray]{0.95}}

\newcommand\makebeamertitle{\frame{\maketitle}}%

\setbeamertemplate{frametitle continuation}[from second]
\renewcommand\insertcontinuationtext{...}

%\usepackage{handoutWithNotes}
%\pgfpagesuselayout{3 on 1 with notes}[letterpaper, border shrink=5mm]

\expandafter\def\expandafter\insertshorttitle\expandafter{%
 \insertshorttitle\hfill\insertframenumber\,/\,\inserttotalframenumber}

\makeatother

\usepackage{babel}
\begin{document}

% In document Latex options:
\fvset{listparameters={\setlength{\topsep}{0em}}}

\def\Sweavesize{\normalsize}
\def\Rcolor{\color{black}}
\def\Rbackground{\color[gray]{0.95}}

\begin{Schunk}
\begin{Sinput}
 ## Other settings I like
 options(device = pdf)
 options(useFancyQuotes = FALSE) 
 op <- par() 
 pjmar <- c(5.1, 5.1, 1.5, 2.1) 
 options(SweaveHooks=list(fig=function() par(mar=pjmar, ps=12)))
 pdf.options(onefile=F,family="Times",pointsize=12)
\end{Sinput}
\end{Schunk}



\title{Tidbits }


\subtitle[Descriptive]{Parting Wisdom }


\author{Paul E. Johnson\inst{1} \and Pascal Deboeck\inst{2}}


\institute[K.U.]{\inst{1}Department of Political Science\and \inst{2}Dept. of Psychology,
University of Kansas}


\date[2012]{2013}

\makebeamertitle

\lyxframeend{}

\begin{frame}[containsverbatim]
\frametitle{Don't Forget: R Documents Itself}

\begin{Schunk}
\begin{Sinput}
 help.start() 
\end{Sinput}
\end{Schunk}

\begin{itemize}
\item launches a browser (if it can find one) that overviews the

\begin{itemize}
\item delivered documents

\begin{itemize}
\item Introduction to R (a book)
\item R FAQ
\end{itemize}
\item package documents
\end{itemize}
\item Remember also package ``vignettes'', pdf ``article'' documents
that are delivered with many packages.
\end{itemize}
\end{frame}

\begin{frame}[containsverbatim]
\frametitle{RSiteSearch}
\begin{itemize}
\item help.search(``regression'')

\begin{itemize}
\item is OK if you are just looking for installed functions
\end{itemize}
\item RSiteSearch(``regression'')

\begin{itemize}
\item opens a browser with options on what gets searched
\end{itemize}
\end{itemize}
\end{frame}

\begin{frame}[containsverbatim]
\frametitle{RSiteSearch}
\begin{itemize}
\item Google searches for R material may help, but may also lead to bad
advice from people who don't know more than you do.
\item Two solutions

\begin{enumerate}
\item Join r-help, or at least read/search its archives
\item Read Stack Overflow\end{enumerate}
\begin{itemize}
\item http://stackoverflow.com/questions/tagged/r
\end{itemize}
\end{itemize}
\end{frame}

\begin{frame}[containsverbatim]
\frametitle{Feel the Power of the Source, Luke}
\begin{itemize}
\item For example, go to \url{http://rweb.quant.ku.edu/cran}. You can get
the R source code (e.g, R-3.0.1.tar.gz)

\begin{itemize}
\item Explore in src/library
\end{itemize}
\item When you are perplexed with a package, download source ``tarball''
\item They are easily downloaded from CRAN package listing. Look for ``Contributed
extension packages''
\item All packages follow a standard format, with the R code in the R folder,
and the documentation and examples are under inst.
\end{itemize}
\end{frame}

\begin{frame}[containsverbatim]
\frametitle{Ways to step through the Source code}
\begin{itemize}
\item Don't be afraid to debug the R source code, or the code for any package
\item Easiest way: Run


\begin{lstlisting}
> debug(lm)
\end{lstlisting}


\item Then watch what happens. Try and see!
\end{itemize}
\begin{lstlisting}
> x <- rnorm(100); y <- rnorm(100)
> m1 <- lm(y ~ x)
\end{lstlisting}

\begin{itemize}
\item Type ``n'' to step into the code. After than, ``n'' (or just hit
Enter) to step to next command
\item Type ``c'' to continue and finish function
\item While in debug mode, run commands to inspect data and variables.


\begin{lstlisting}
ls()
\end{lstlisting}


\item You will see the state of all variables ``inside'' the function.
\end{itemize}
\end{frame}


\begin{frame}[containsverbatim]
  \frametitle{debug package}
\begin{itemize}
\item Slightly fancier mtrace function in ``debug'' package (Mark Bravington)
\end{itemize}
\begin{Schunk}
\begin{Sinput}
 plot(y ~ x, data = mydata)
\end{Sinput}
\end{Schunk}

\begin{itemize}
\item pops open a ``code browser'' window
\item Hit return in R window to go step-by-step.
\end{itemize}
\end{frame}

\begin{frame}[containsverbatim]
\frametitle{How to not Betray yourself as a Stranger in R-land}
\begin{itemize}
\item Refer to ``packages'', not libraries. library() is a function that
opens packages.
\item Use ``<-'', not ``='' for assignments
\item Avoid unnecessary ``for loops,'' use automatic ``vectorization''
instead or (l-s)apply instead
\end{itemize}
\end{frame}

\begin{frame}[containsverbatim]
\frametitle{How to ask a question}
\begin{itemize}
\item We usually become tired and frustrated, and then send out emails like
``I can't make XYZ work''.
\item Generally, those messages are not helpful because we need to know
EXACTLY what you tried and we need to know WHAT SYSTEM and WHICH PACKAGES
you have.
\item r-help has a (longish) ``posting guide'' (that strains my patience).
\item The best advice is this: every time you ask a question, provide:

\begin{itemize}
\item Output from R command ``sessionInfo()''.
\item a COMPLETE and run-able example of the ``problem''
\end{itemize}
\item Informative subject heading (NOT ``Need R help, please'')
\end{itemize}
\end{frame}

%% \begin{frame}[containsverbatim]
%% \frametitle{R Bible has Four Books}
%% \begin{itemize}
%% \item VR: William Venables and Brian Ripley, \emph{Modern Applied Statistics
%% with S }(package: MASS)

%% \begin{itemize}
%% \item Ripley has been a tireless code contributor and R maintainer. The
%% R authorities expect you have read MASS.
%% \end{itemize}
%% \item PB: Pinheiro and Bates, \emph{Mixed Effects Models in S and S-Plus}
%% (packages: nlme, lme4)

%% \begin{itemize}
%% \item If you want to talk about ``hierarchical models'', learn to think
%% of it in the way R folks think of it
%% \end{itemize}
%% \item F: John Fox, \emph{Applied Regression Analysis, Linear Models, and
%% Related Methods} (and the \emph{Companion to Applied Regression})
%% (packages: car, Rcmdr)
%% \item H: Frank Harrell, \emph{Regression Modeling Strategies} (packages,
%% rms Hmisc)

%% \begin{itemize}
%% \item Original author of Proc Logistic in SAS
%% \end{itemize}
%% \end{itemize}
%% \end{frame}

%% \begin{frame}[containsverbatim]
%% \frametitle{There are Lots of Ways to Estimate the "Same" Model}
%% \begin{description}
%% \item [{OLS:}] stats::lm, rms::ols
%% \item [{ordinal:}] MASS::polr, rms::lrm, ordinal::clm, VGAM::vglm
%% \item [{multinomial:}] MASS::multinom, VGAM::vglm\end{description}
%% \begin{itemize}
%% \item These are generally compatible, but different as well. If you have
%% trouble with the ones provided with R, then r-help is an appropriate
%% venue to ask.
%% \item If you are using other packages, you should seek out the author or
%% the discussion forums that they want you to follow.
%% \end{itemize}
%% \end{frame}

%% \begin{frame}[containsverbatim]
%% \frametitle{\url{http://pj.freefaculty.org/guides}}
%% \begin{itemize}
%% \item Now you know me, you might understand these stat write-ups.


%% Better-than-average writeups on OLS, logistic/probit models, GLM,
%% and Distributions.

%% \item If you use Linux or are trying to do High Performance (``Cluster'')
%% Computing, \url{http://crmda.ku.edu/computing}
%% \item If you are trying to administer Windows with Stat programs, I've got
%% a growing collection of tips as well in the WinStat Admin pages. (Notepad++,
%% Emacs -> R)
%% \end{itemize}
%% \end{frame}

%% \begin{frame}[containsverbatim]
%% \frametitle{Have a Nice Summer}

%% Don't forget to wear sun block.

%% \end{frame}


%% \lyxframeend{}


\end{document}
