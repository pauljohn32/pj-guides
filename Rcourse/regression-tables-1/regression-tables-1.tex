\batchmode
\makeatletter
\def\input@path{{/home/pauljohn/SVN/SVN-guides/Rcourse/regression-tables-1//}}
\makeatother
\documentclass[11pt,english]{beamer}
\usepackage{lmodern}
\renewcommand{\sfdefault}{lmss}
\renewcommand{\ttdefault}{lmtt}
\usepackage[T1]{fontenc}
\usepackage[utf8]{inputenc}
\usepackage{listings}
\setcounter{secnumdepth}{3}
\setcounter{tocdepth}{3}
\usepackage{babel}
\usepackage{url}
\ifx\hypersetup\undefined
  \AtBeginDocument{%
    \hypersetup{unicode=true}
  }
\else
  \hypersetup{unicode=true}
\fi

\makeatletter
%%%%%%%%%%%%%%%%%%%%%%%%%%%%%% Textclass specific LaTeX commands.
\usepackage{Sweavel}
 \def\lyxframeend{} % In case there is a superfluous frame end
 \long\def\lyxframe#1{\@lyxframe#1\@lyxframestop}%
 \def\@lyxframe{\@ifnextchar<{\@@lyxframe}{\@@lyxframe<*>}}%
 \def\@@lyxframe<#1>{\@ifnextchar[{\@@@lyxframe<#1>}{\@@@lyxframe<#1>[]}}
 \def\@@@lyxframe<#1>[{\@ifnextchar<{\@@@@@lyxframe<#1>[}{\@@@@lyxframe<#1>[<*>][}}
 \def\@@@@@lyxframe<#1>[#2]{\@ifnextchar[{\@@@@lyxframe<#1>[#2]}{\@@@@lyxframe<#1>[#2][]}}
 \long\def\@@@@lyxframe<#1>[#2][#3]#4\@lyxframestop#5\lyxframeend{%
   \frame<#1>[#2][#3]{\frametitle{#4}#5}}

%%%%%%%%%%%%%%%%%%%%%%%%%%%%%% User specified LaTeX commands.
\usepackage{dcolumn}
\usepackage{booktabs}

% use 'handout' to produce handouts
%\documentclass[handout]{beamer}
\usepackage{wasysym}
\usepackage{pgfpages}
\newcommand{\vn}[1]{\mbox{{\it #1}}}\newcommand{\vb}{\vspace{\baselineskip}}\newcommand{\vh}{\vspace{.5\baselineskip}}\newcommand{\vf}{\vspace{\fill}}\newcommand{\splus}{\textsf{S-PLUS}}\newcommand{\R}{\textsf{R}}


\usepackage{graphicx}
\usepackage{listings}
\lstset{tabsize=2, breaklines=true,style=Rstyle}
\usetheme{Antibes}
% or ...

%\setbeamercovered{transparent}
% or whatever (possibly just delete it)


% In document Latex options:
\fvset{listparameters={\setlength{\topsep}{0em}}}
\def\Sweavesize{\normalsize} 
\def\Rcolor{\color{black}} 
\def\Rbackground{\color[gray]{0.95}}

\mode<presentation>

\newcommand\makebeamertitle{\frame{\maketitle}}%

\setbeamertemplate{frametitle continuation}[from second]
\renewcommand\insertcontinuationtext{...}

\expandafter\def\expandafter\insertshorttitle\expandafter{%
 \insertshorttitle\hfill\insertframenumber\,/\,\inserttotalframenumber}

\usepackage{hyperref}
\hypersetup{colorlinks=true,citecolor=blue}

\makeatother

\begin{document}

% In document Latex options:
\fvset{listparameters={\setlength{\topsep}{0em}}}

\def\Sweavesize{\normalsize} 
\def\Rcolor{\color{black}} 
\def\Rbackground{\color[gray]{0.90}}

\begin{Schunk}
\begin{Sinput}
 ## Other settings I like
 options(device = pdf)
 options(useFancyQuotes = FALSE) 
 op <- par() 
 pjmar <- c(5.1, 5.1, 1.5, 2.1) 
 options(SweaveHooks=list(fig=function() par(mar=pjmar, ps=12)))
 pdf.options(onefile=F,family="Times",pointsize=12)
\end{Sinput}
\end{Schunk}



\title[Descriptive]{Regression Presentations: Tables }


\author{Paul E. Johnson\inst{1} \and \inst{2}}


\institute[K.U.]{\inst{1}Department of Political Science\and \inst{2}Center for
Research Methods and Data Analysis, University of Kansas}


\date[2013]{2013}

\makebeamertitle

\lyxframeend{}

\AtBeginSubsection[]{
  \frame<beamer>{ 
    \frametitle{Outline}   
    \tableofcontents[currentsection,currentsubsection] 
  }
}

\begin{frame}

\frametitle{Presenting OLS To The Masses}
\begin{itemize}
\item We Nice Looking Regression Tables
\item Better to output a close-to-final result from R

\begin{itemize}
\item Reduces typographical errors
\item Easy to re-run estimates and produce another automatic table.
\end{itemize}
\end{itemize}
\end{frame}


\lyxframeend{}\section{Data}

\begin{frame}[containsverbatim,allowframebreaks]
\frametitle{Get Some Data from Nat. Election Study 2004}

\begin{Schunk}
\begin{Sinput}
 help.search("inverse gaussian")
 
\end{Sinput}
\end{Schunk}


\inputencoding{latin9}\begin{lstlisting}[basicstyle={\footnotesize}]
## V043038     B1a. Feeling Thermometer: GW Bush
## V043039     B1b. Feeling Thermometer: John Kerry
## V043210     R1. R position on gay marriage
## V043213     S3. National economy better/worse since GW Bush took ofc
## V045117     G4a. Liberal/conservative 7-point scale: self-placement
## V045145X    H5x. Summary: Pre-Post US flag makes R feel
## V041109A    HHListing.9a. Respondent gender
## V043116     J1x. Summary: R party ID
## V043250     Y1x. Summary: Respondent age
\end{lstlisting}
\inputencoding{utf8}

\end{frame}

\begin{frame}[containsverbatim]
\frametitle{This was a surprise to me}


\includegraphics[width=10cm]{plots/t-anes10}

\end{frame}

\begin{frame}[containsverbatim,allowframebreaks]
\frametitle{Create a New Dependent Variable}


The difference in thermometer scores:

\def\Sweavesize{\scriptsize}
\begin{Schunk}
\begin{Sinput}
 mydta1$th.bush.kerry <- mydta1$V043038 - mydta1$V043039
\end{Sinput}
\end{Schunk}



Clean up a bunch of variables \& value labels

\begin{Schunk}
\begin{Sinput}
 ##Party
 mydta1$V043116 <- mydta1$V043116[, drop = TRUE]
 levels(mydta1$V043116) <- c("SD","WD","ID","I","IR","WR","SR","O")
 mydta1$V043116[ mydta1$V043116 %in% levels(mydta1$V043116)[8] ] <- NA
 mydta1$V043116 <- mydta1$V043116[, drop = TRUE]
 table(mydta1$V043116)
\end{Sinput}
\begin{Soutput}
 SD  WD  ID   I  IR  WR  SR 
203 179 210 118 138 154 193 
\end{Soutput}
\begin{Sinput}
 ##IDEO
 mydta1$V045117 <- mydta1$V045117[ , drop = TRUE]
 levels(mydta1$V045117) <- c("EL","L","SL","M","SC","C","EC")
 table(mydta1$V045117)
\end{Sinput}
\begin{Soutput}
 EL   L  SL   M  SC   C  EC 
 20 103 125 279 143 166  31 
\end{Soutput}
\begin{Sinput}
 ##Gender
 levels(mydta1$V041109A) <- c("M","F")
 ## Gay Marriage
 levels(mydta1$V043210) 
\end{Sinput}
\begin{Soutput}
[1] "1. Should be allowed"                                    "3. Should not be allowed"                               
[3] "5. Should not be allowed to marry but should be allowed" "VOL"                                                    
[5] "8. Don't know"                                           "9. Refused"                                             
\end{Soutput}
\begin{Sinput}
 mydta1$V043210[ mydta1$V043210 %in% levels(mydta1$V043210)[4:10] ] <-NA  
 mydta1$V043210 <- mydta1$V043210[, drop = TRUE]
 levels(mydta1$V043210) <- c("Allow","No","Med") 
 ## Economy
 mydta1$V043213 <- mydta1$V043213[ , drop = TRUE]
 l <- levels(mydta1$V043213)
 econnew <- factor(mydta1$V043213, levels=c(l[2],l[3],l[1]), labels=c("Worse","Same","Better"))
 table(mydta1$V043213, econnew)
\end{Sinput}
\begin{Soutput}
             econnew
              Worse Same Better
  1. Better       0    0    190
  3. Worse      668    0      0
  5. The same     0  343      0
\end{Soutput}
\begin{Sinput}
 mydta1$V043213 <- econnew
 rm(econnew)
 ##Flag
 mydta1$V045145X <- mydta1$V045145X[, drop = TRUE] 
\end{Sinput}
\end{Schunk}


\end{frame}


\lyxframeend{}\section{Making Regression Tables}

\begin{frame}
\frametitle{What Should a Regression Table Look Like?}
\begin{itemize}
\item It needs 

\begin{itemize}
\item coefficients
\item standard errors (or t-values, possibly)
\item model diagnostics like $N$ and $R^{2}$ and so forth.
\end{itemize}
\item I want it to be easy to generate nice looking tables automatically
to make it easy to prepare presentations and class notes.
\end{itemize}
\end{frame}


\begin{frame}
  \frametitle {R Functions/Packages for Producing Regression Output}
  \begin{itemize}
  \item outreg: An R function I prepared 2006, now in "rockchalk" package
  \item memisc: works well, output not quite "presentation ready"
  \item xtable: incomplete output, but latex or HTML works 
  \item apsrtable: very similar to outreg
  \item Hmisc ``latex'' function
  \item texreg: a new regression table making framework
  \end{itemize}    
\end{frame}

\begin{frame}[containsverbatim]
\frametitle{Are Your A LaTeX User?}
\begin{itemize}
\item Some scientists say it is required, same with math. Many of the smart
people I know prepare documents in \LaTeX{}. 
\item if using MS Word or similar makes you feel like you are ``finger
painting'' to format material, \LaTeX{} may be the right thing for
you.
\item My \LaTeX{} notes page: \url{http://pj.freefaculty.org/latex}
\item My \LaTeX{} lecture notes and example documents are linked on that
site. 
\end{itemize}
Maybe we should stop and talk that over. Maybe I should show you my
\href{http://pj.freefaculty.org/guides/Computing-HOWTO/LatexAndLyx/LaTeX-General-1/LaTeX-lecture-1.pdf}{LaTeX Overview}

\end{frame}

\begin{frame}[containsverbatim]
\frametitle{Automatic Tables are Easiest for LaTeX users}
\begin{itemize}
\item \LaTeX{} is structured document ``markup'', and most table-making
packages in R cater to \LaTeX{} users, rather than other folks. 
\item However, for the MS Word (Libre Office) addicted, all is not lost.
Several packages now can export to HTML, which Word is often able
to import gracefully. 
\item I'm preparing these lectures using Sweave, which means that R runs
and inserts the tables in automatically, I literally never touch them.
\end{itemize}
\end{frame}

\begin{frame}
\frametitle{Consider making 50 sets of tables, one for each student}
\begin{itemize}
\item In my regression class, I create random data sets that are individualized
to the students, and then I need to show them what their results ought
to be. 
\item I could write 50 MS Word documents, one for each student, or
\item I DID make a \LaTeX{} based script that automated the production of
one report per student. Please inspect the result:


\url{http://pj.freefaculty.org/stat/ps706/pj-test2}

\item Please examine one or two of those files and then tell me it would
be easier to use MS Word...
\end{itemize}
\end{frame}


\lyxframeend{}\section{rockchalk: outreg}

\begin{frame}[containsverbatim]
\frametitle{Here's what To Do}

Load the rockchalk package, in which the outreg function resides.



\begin{Schunk}
\begin{Sinput}
 mod1age <- lm(th.bush.kerry ~ V043250, data = mydta1)
 outreg(mod1age, tight = FALSE, modelLabels = c("Age as Predictor"))
\end{Sinput}
\end{Schunk}


\end{frame}

\begin{frame}[containsverbatim]
\frametitle{Produces this LaTeX Markup}


\begin{Schunk}
\begin{Soutput}
\begin{tabular}{*{3}{l}}
 \hline
                &\multicolumn{2}{c}{Age as Predictor}   \\
                &Estimate &(S.E.) \\
 \hline
 \hline
  (Intercept)    & -6.841  &   (4.596) \\
  V043250        &  0.184*  &   (0.092) \\
 \hline 
 N                &1191      &       \\
 RMSE            &53.885        & \\
 $R^2$           &0.003        & \\
 \hline
 \hline
 
 \multicolumn{2}{l}{${*}  p \le 0.05$   }\\
 \end{tabular}
\end{Soutput}
\end{Schunk}


\end{frame}

\begin{frame}[containsverbatim]
\frametitle{Which LaTeX Renders as}

[1] "whew, there is a DF"
\begin{tabular}{*{3}{l}}
\hline
                  & \multicolumn{2}{c}{Age as Predictor}   \tabularnewline
                   &Estimate  &(S.E.)  \tabularnewline
 \hline
 \hline
   (Intercept)     &-6.841   &   (4.596) \tabularnewline
   V043250         &0.184*   &   (0.092) \tabularnewline
 \hline
 N                 &1191       &        \tabularnewline
 RMSE             &53.885         & \tabularnewline
 $R^2$             &0.003         & \tabularnewline
 \hline
\hline
 
 \multicolumn{3}{c}{${*  p}\le 0.05$${*\!\!*  p}\le 0.01$${*\!\!*\!\!*  p}\le 0.001$}\tabularnewline
 \end{tabular}

\end{frame}

\begin{frame}[containsverbatim]
\frametitle{Add Gender}


\begin{Schunk}
\begin{Sinput}
 ## Run a new regression
 mod2age <- lm(th.bush.kerry ~ V043250 + V041109A, data = mydta1)
 ## Put 2 regressions in same table
 outreg(list(mod1age, mod2age), tight = TRUE, modelLabels = c("Age Only","Age With Gender"))
\end{Sinput}
\end{Schunk}


My terminology: ``tight'' means coefficients and standard errors
vertically aligned

\end{frame}

\begin{frame}[containsverbatim]
\frametitle{Output To LaTEX}

[1] "whew, there is a DF"
[1] "whew, there is a DF"
\begin{tabular}{*{3}{l}}
\hline
                  & Age Only & Age With Gender   \tabularnewline
                   &Estimate   &Estimate \tabularnewline
                 &(S.E.)   &(S.E.) \tabularnewline
 \hline
 \hline
   (Intercept)     &-6.841 &-3.085 \tabularnewline
                 &(4.596)   &(4.831)  \tabularnewline
   V043250         &0.184* &0.191* \tabularnewline
                 &(0.092)   &(0.092)  \tabularnewline
   V041109AF         & .      &-7.713* \tabularnewline
                 &         &(3.123)  \tabularnewline
 \hline
 N                 &1191       &1191       \tabularnewline
 RMSE             &53.885   &53.770   \tabularnewline
 $R^2$             &0.003   &0.008   \tabularnewline
 adj $R^2$         &0.003   &0.007   \tabularnewline
 \hline
\hline
 
 \multicolumn{3}{c}{${*  p}\le 0.05$${*\!\!*  p}\le 0.01$${*\!\!*\!\!*  p}\le 0.001$}\tabularnewline
 \end{tabular}

\end{frame}

\begin{frame}[containsverbatim]
\frametitle{Recent updates to outreg}
\begin{itemize}
\item outreg was the first useful R function I created, I was distributing
it (sans packaging) since 2006.
\item The rockchalk package includes it now, I've made some ``user convenience''
changes.

\begin{itemize}
\item easier customization of model ``header'' labels and variable names
\item easier to customize the selection of ``goodness of fit'' indicators
in the bottom of the table
\item I'm not an alpha 0.05 insister anymore, you can choose 1 or more alpha
levels (with stars!)
\end{itemize}
\end{itemize}
\end{frame}

\begin{frame}
\frametitle{outreg can create html file output}
\begin{itemize}
\item This is a brand new feature in outreg 1.8 (June, 2013)

\begin{itemize}
\item outreg2HTML() receives outreg results and converts into Web markup.
\item Word 2010 will not ``paste special'' the HTML markup I generate,
but it can ``Insert -> File'' and it absorbs the HTML markup in
a reasonable way. You can finger paint to customize.
\item Not as nice looking or as automatic as \LaTeX{}, but I would use it
if somebody made me use MS Word.
\end{itemize}
\end{itemize}
\end{frame}

\begin{frame}
\frametitle{Lets put that to the test}

\begin{Schunk}
\begin{Sinput}
 library(car)
 m1 <- lm(statusquo ~ income + age, data = Chile)
\end{Sinput}
\end{Schunk}


I've copied the file ``or1-test.html'' to a Website, where you can
download it for yourself and try to import it into a word processor
of your choice. \href{http://pj.freefaculty.org/R/or1-test.html}{http://pj.freefaculty.org/R/or1-test.html}

\end{frame}


\lyxframeend{}\section{xtable}

\begin{frame}
\frametitle{The xtable package is as old as R itself}
\begin{itemize}
\item In the old old days, xtable was ``THE'' one to use for \LaTeX{}
output
\item I still use it to create quick tables of some summary output, particularly
output from rockchalk::summarizeNumeric. (There's an example in each
file in pj-test2 mentioned above).
\end{itemize}
\end{frame}

\begin{frame}[containsverbatim]
\frametitle{xtable is Nice Too}


This code:

\begin{Schunk}
\begin{Sinput}
 library(xtable)
 tabout1 <- xtable(mod2age)
 print(tabout1, type="latex")
\end{Sinput}
\end{Schunk}


\end{frame}

\begin{frame}[containsverbatim]
\frametitle{Generates A LaTeX Table}


% latex table generated in R 3.1.1 by xtable 1.7-3 package
% Thu Aug  7 23:19:24 2014
\begin{table}[ht]
\centering
\begin{tabular}{rrrrr}
  \hline
 & Estimate & Std. Error & t value & Pr($>$$|$t$|$) \\ 
  \hline
(Intercept) & -3.0850 & 4.8314 & -0.64 & 0.5232 \\ 
  V043250 & 0.1913 & 0.0917 & 2.09 & 0.0371 \\ 
  V041109AF & -7.7134 & 3.1232 & -2.47 & 0.0137 \\ 
   \hline
\end{tabular}
\end{table}

Maybe I'm missing something, but this is not really a presentable,
finished table.

\end{frame}

\begin{frame}[containsverbatim]
\frametitle{xtable: HTML Markup Output}


This code:

\begin{Schunk}
\begin{Sinput}
 print(tabout1, type="HTML")
\end{Sinput}
\end{Schunk}


\end{frame}

\begin{frame}[containsverbatim]
\frametitle{Produces HTML Markup that another program can Absorb}


\def\Sweavesize{\scriptsize}
\begin{Schunk}
\begin{Soutput}
<!-- html table generated in R 3.1.1 by xtable 1.7-3 package -->
<!-- Thu Aug  7 23:19:24 2014 -->
<TABLE border=1>
<TR> <TH>  </TH> <TH> Estimate </TH> <TH> Std. Error </TH> <TH> t value </TH> <TH> Pr(&gt;|t|) </TH>  </TR>
  <TR> <TD align="right"> (Intercept) </TD> <TD align="right"> -3.0850 </TD> <TD align="right"> 4.8314 </TD> <TD align="right"> -0.64 </TD> <TD align="right"> 0.5232 </TD> </TR>
  <TR> <TD align="right"> V043250 </TD> <TD align="right"> 0.1913 </TD> <TD align="right"> 0.0917 </TD> <TD align="right"> 2.09 </TD> <TD align="right"> 0.0371 </TD> </TR>
  <TR> <TD align="right"> V041109AF </TD> <TD align="right"> -7.7134 </TD> <TD align="right"> 3.1232 </TD> <TD align="right"> -2.47 </TD> <TD align="right"> 0.0137 </TD> </TR>
   </TABLE>
\end{Soutput}
\end{Schunk}


\end{frame}


\lyxframeend{}\section{memisc}

\begin{frame}[containsverbatim]
\frametitle{And a nice LaTeX converter "toLatex"}
\begin{itemize}
\item [memisc] (by Martin Elf) is a general purpose social science package
that has many excellent features
\item It is the ONLY program I use to make presentable ``Cross Tabulation
Tables''
\item It has excellent support for SPSS user therapy and post-addiction
recovery.
\item So far as I know, memisc was the first R package to offer the special
features that other packages are now emulating.

\begin{itemize}
\item Easy ability to incorporate results from several regression models
in a single table
\item Easy ability to export that result to a presentable \LaTeX{} format.
\end{itemize}
\end{itemize}
\end{frame}

\begin{frame}[containsverbatim,allowframebreaks]
\frametitle{memisc: "mtable" Feature was a Major Breakthrough}


\def\Sweavesize{\scriptsize}
\begin{Schunk}
\begin{Sinput}
 outreg(list("Age Only" = mod1age, "Age With Gender" = mod2age), tight = FALSE)
\end{Sinput}
\end{Schunk}


\end{frame}

\begin{frame}[containsverbatim]
\frametitle{memisc also offers a nice LaTeX converter "toLatex"}


\def\Sweavesize{\scriptsize}
\begin{Schunk}
\begin{Sinput}
 toLatex(mtable(mod1age,mod2age,mod3age))
\end{Sinput}
\end{Schunk}


\end{frame}

\begin{frame}[containsverbatim,allowframebreaks]
\frametitle{And the Result}


\scriptsize{
%%%%%%%%%%%%%%%%%%%%%%%%%%%%%%%%%%%%%%%%%%%%%%%%%%%%%%%%%%%%%%%%%%%%%%%%%%%%%%%%%%%%%%%%%%%%%
%
% Calls:
% mod1age:  lm(formula = th.bush.kerry ~ V043250, data = mydta1) 
% mod2age:  lm(formula = th.bush.kerry ~ V043250 + V041109A, data = mydta1) 
% mod3age:  lm(formula = th.bush.kerry ~ V043250 * V041109A, data = mydta1) 
%
%%%%%%%%%%%%%%%%%%%%%%%%%%%%%%%%%%%%%%%%%%%%%%%%%%%%%%%%%%%%%%%%%%%%%%%%%%%%%%%%%%%%%%%%%%%%%
\begin{tabular}{lcD{.}{.}{7}cD{.}{.}{7}cD{.}{.}{7}}
\toprule
&&\multicolumn{1}{c}{mod1age} && \multicolumn{1}{c}{mod2age} && \multicolumn{1}{c}{mod3age}\\
\midrule
(Intercept)                     &  &   -6.841    &&   -3.085    &&   -6.536   \\
                                &  &   (4.596)   &&   (4.831)   &&   (6.692)  \\
V043250                         &  &  0.184^{*}  &&  0.191^{*}  &&    0.265   \\
                                &  &   (0.092)   &&   (0.092)   &&   (0.135)  \\
V041109A: F/M                   &  &             && -7.713^{*}  &&   -1.268   \\
                                &  &             &&   (3.123)   &&   (9.194)  \\
V043250 $\times$ V041109A: F/M &  &             &&             &&   -0.137   \\
                                &  &             &&             &&   (0.184)  \\
\midrule
R-squared                       &  &       0.003 &&       0.008 &&       0.009\\
adj. R-squared                  &  &       0.003 &&       0.007 &&       0.006\\
sigma                           &  &      53.885 &&      53.770 &&      53.780\\
F                               &  &       4.028 &&       5.072 &&       3.565\\
p                               &  &       0.045 &&       0.006 &&       0.014\\
Log-likelihood                  &  &   -6437.301 &&   -6434.252 &&   -6433.973\\
Deviance                        &  & 3452404.627 && 3434769.503 && 3433162.607\\
AIC                             &  &   12880.603 &&   12876.504 &&   12877.946\\
BIC                             &  &   12895.850 &&   12896.834 &&   12903.359\\
N                               &  &    1191     &&    1191     &&    1191    \\
\bottomrule
\end{tabular}
}

\end{frame}

\begin{frame}[containsverbatim,allowframebreaks]
\frametitle{Critique}
\begin{itemize}
\item ``Too Much'' information at the bottom of the plot. I'd rather have
the default be minimal and allow users to ask for more.
\item No symbol key to indicate what {*} {*}{*} mean
\item Variable Labels difficult to understand
\end{itemize}
\end{frame}


\lyxframeend{}\section{Other packages}


\lyxframeend{}\lyxframe{rms has offered \LaTeX{} for 20 years}
\begin{itemize}
\item The package now known as ``rms'' (Regression Modeling Strategies,
by Frank Harrell) has been offering latex as a generic function for
20 years, even before R existed (because S did exist).
\item Has many latex functions customized to the regression fitting routines
in rms
\end{itemize}

\lyxframeend{}\lyxframe{apsrtable}
\begin{itemize}
\item Prepared by a then-graduate student at my Alma Mater Washington University
in St. Louis.
\item Perhaps the title is unfortunate because it suggests that it is intended
for the American Political Science Review, but that is not its only
target audience.
\end{itemize}

\lyxframeend{}\lyxframe{texreg}
\begin{itemize}
\item The most recent entry in the regression table sweepstakes, by Philip
Leifeld
\item It has specialized converters designed for many of the popular kinds
of regression model, including many in packages I've never heard of.

\begin{itemize}
\item At first I had admiration and awe at the dedication to write separate
``back end'' functions for all of the different kinds of regression
\item Later, I became angry that regression package writers do not standardize
their object structures to be compatible with glm() and lm() in R
base, necessitating a nearly heroic effort to translate each one,
one by one.
\item After that, I decided it is crazy to cater to inconsistent package
writers and tried to generalize outreg to accomodate that diversity
without writing 100 separate little functions for 100 separate little
regression packages.
\end{itemize}
\end{itemize}

\lyxframeend{}

\begin{frame}[containsverbatim]
\frametitle{Example texreg usage}

\begin{Schunk}
\begin{Sinput}
 library(texreg)
 to1 <- texreg(list(mod1age,mod2age), return.string = TRUE, use.packages = FALSE, booktabs = TRUE, dcolumn = TRUE)
\end{Sinput}
\end{Schunk}

\begin{itemize}
\item One criticism: user interface to texreg almost unbelievably filled
up with complicated looking options. Perhaps not for beginners in
R or \LaTeX{}.
\end{itemize}
\end{frame}

\begin{frame}[containsverbatim]
\frametitle{Example texreg output}

\begin{Schunk}
\begin{Sinput}
 par(mfcol = c(3, 2))  # Fill columns first
 par(mfrow = c(3, 2))  # Fill rows first
\end{Sinput}
\end{Schunk}

\begin{itemize}
\item Another criticism: Who presents a regression without RMSE (standard
error of estimate = sigma...)?
\end{itemize}
\end{frame}

\begin{frame}[containsverbatim]
\frametitle{I wondered what that LaTeX Markup would look like}

\def\Sweavesize{\scriptsize}

\begin{Schunk}
\begin{Sinput}
 cat(to1)
\end{Sinput}
\begin{Soutput}
\begin{table}
\begin{center}
\begin{tabular}{l D{.}{.}{4.3}@{} D{.}{.}{4.3}@{} }
\toprule
            & \multicolumn{1}{c}{Model 1} & \multicolumn{1}{c}{Model 2} \\
\midrule
(Intercept) & -6.84    & -3.09     \\
            & (4.60)   & (4.83)    \\
V043250     & 0.18^{*} & 0.19^{*}  \\
            & (0.09)   & (0.09)    \\
V041109AF   &          & -7.71^{*} \\
            &          & (3.12)    \\
\midrule
R$^2$       & 0.00     & 0.01      \\
Adj. R$^2$  & 0.00     & 0.01      \\
Num. obs.   & 1191     & 1191      \\
\bottomrule
\multicolumn{3}{l}{\scriptsize{$^{***}p<0.001$, $^{**}p<0.01$, $^*p<0.05$}}
\end{tabular}
\caption{Statistical models}
\label{table:coefficients}
\end{center}
\end{table}
\end{Soutput}
\end{Schunk}


\end{frame}

\begin{frame}[containsverbatim]
\frametitle{One Extra Large Attraction for texreg}
\begin{itemize}
\item The author, unlike others, has exerted himself to learn what CSS formatting
and HTML magic MS Word can tolerate, and the function htmlreg() creates
an output file that Word can import and it will look almost as good
as in \LaTeX{}.
\item On my system (where Word runs in an emulation layer for Linux), the
resulting table did look nice, but it would tolerate no finger painting
by me. I was not able to adjust anything. But that's probably more
of a reflection on me than on texreg.
\end{itemize}
\end{frame}


\lyxframeend{}
\end{document}
