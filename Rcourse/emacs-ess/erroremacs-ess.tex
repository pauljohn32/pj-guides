\documentclass[11pt,english]{beamer}
\usepackage{lmodern}
\renewcommand{\sfdefault}{lmss}
\renewcommand{\ttdefault}{lmtt}
\usepackage[T1]{fontenc}
\usepackage[latin9]{inputenc}
\usepackage{listings}
\setcounter{secnumdepth}{3}
\setcounter{tocdepth}{3}
\usepackage{url}
\usepackage{graphicx}

\makeatletter

%%%%%%%%%%%%%%%%%%%%%%%%%%%%%% LyX specific LaTeX commands.
%% Because html converters don't know tabularnewline
\providecommand{\tabularnewline}{\\}

%%%%%%%%%%%%%%%%%%%%%%%%%%%%%% Textclass specific LaTeX commands.
\usepackage{Sweavel}

\newenvironment{topcolumns}{\begin{columns}[t]}{\end{columns}}
\newcommand{\code}[1]{\texttt{#1}}

%%%%%%%%%%%%%%%%%%%%%%%%%%%%%% User specified LaTeX commands.
\usepackage{dcolumn}
\usepackage{booktabs}


% use 'handout' to produce handouts
% \documentclass[handout]{beamer}
\usepackage{wasysym}
\usepackage{pgfpages}

% \setbeamercovered{transparent}
% or whatever (possibly just delete it)

% In document Latex options:
\fvset{listparameters={\setlength{\topsep}{0em}}}
\def\Sweavesize{\small} 
\def\Rcolor{\color{black}} 
\def\Rbackground{\color[gray]{0.95}}

\usepackage{graphicx}
\usepackage{listings}
\lstset{tabsize=2, breaklines=true,style=Rstyle}
\usetheme{Antibes}
% or ...

% \setbeamercovered{transparent}
% or whatever (possibly just delete it)

% \mode<presentation>
% {
% \usetheme{KU}
% \usecolortheme{dolphin} %dark blues
% }


%%   not for article, but for presentation
\mode<presentation>
\newcommand\makebeamertitle{\frame{\maketitle}}

\setbeamertemplate{frametitle continuation}[from second]
\renewcommand\insertcontinuationtext{...}

\expandafter\def\expandafter\insertshorttitle\expandafter{%
  \insertshorttitle\hfill\insertframenumber\,/\,\inserttotalframenumber}

\makeatother

\usepackage{babel}


\begin{document}

% \beamerdefaultoverlayspecification{<+->}

\selectlanguage{english}%
\AtBeginSection[]{
  \frame<beamer>{ 
    \frametitle{Outline}   
    \tableofcontents[currentsection,currentsubsection] 
  }
}


\date[2015]{2015}

\makebeamertitle


\begin{frame} 
\frametitle{Outline}

\tableofcontents{}

\end{frame}

\section{Why Use Emacs?}

\begin{frame}[containsverbatim, allowframebreaks]
  \frametitle{Why Use Emacs? Multidimensional Availability}
  \begin{itemize}
  \item All Platforms (Windows, Mac, Linux, Unix, Atari,...)
  \item Across Time: Will Always Exist (because eager fanatics maintain it)
  \item Free \& Open Source
  \end{itemize}
\end{frame}

\begin{frame}[containsverbatim, allowframebreaks]
  \frametitle{Why Use Emacs? Super Tools}
  \begin{itemize}
  \item Emacs has ``major modes'' for most languages and types of files
    (C, C++, Lisp, R, SAS, Stata, \LaTeX{}, English, $\ldots$). 
  \item Incredibly powerful text management tools

    \begin{itemize}
    \item ``compare'' documents or buffers
    \item Regular expression search and replace (even across many files)
    \item Copy, paste, insert columns
    \end{itemize}
  \item Enormous power to edit very large files
  \item At some point in the future, you may find that the only editor
    that is capable for a particular project is Emacs. Prepare for
    that time by using Emacs for other projects as well! 
  \end{itemize}
\end{frame}

\begin{frame}[containsverbatim, allowframebreaks]
  \frametitle{Check the R FAQ}
  \begin{itemize}
  \item R FAQ, by Kurt Hornik (Version 2.15.2012-09-19):
    \begin{lstlisting}[basicstyle={\small}, xleftmargin=0in]
6.2 Should I run R from within Emacs?

Yes, definitely. Inferior R mode provides a 
readline/history mechanism, object name 
completion, and syntax-based highlighting 
of the interaction buffer using Font Lock 
mode, as well as a very convenient interface 
to the R help system.

Of course, it also integrates nicely with the 
mechanisms for editing R source using Emacs. 
One can write code in one Emacs buffer and 
send whole or parts of it for execution to R;
this is helpful for both data analysis and 
programming. One can also seamlessly integrate 
with a revision control system, in order to 
maintain a log of changes in your programs 
and data, as well as to allow for the 
retrieval of past versions of the code.  

In addition, it allows you to keep a record 
of your session, which can also be used for 
error recovery through the use of the 
transcript mode.
    \end{lstlisting}

  \end{itemize}
\end{frame}


\section{Emacs Anatomy}

\begin{frame}[containsverbatim]
  \frametitle{Start Emacs. Here's What I See}

  \includegraphics[width=11cm]{importfigs/emacs-start-01.png}

  1: Pull Down Menus 2. Button Bar 3. Status Bar 4. Mini-buffer

\end{frame}

\begin{frame}[containsverbatim]
  \frametitle{Terminology}
  \begin{itemize}
  \item Frame: That ``whole thing''. 
  \item Window: The content display area inside a frame
  \item Buffer: A chunk of ``content,'' (letters and words) that can
    be shown inside a window.
  \item Check for yourself.

    \begin{itemize}
    \item Chose File -> Split Window. You can have 2 ``windows'' showing same
      ``buffer''
    \item Choose File -> New Frame
    \item Choose Menu Buffers to select a buffer to display in the currently
      focused Window
    \end{itemize}
  \end{itemize}
\end{frame}

\begin{frame}[containsverbatim]
  \frametitle{Emacs "Major Mode" system}
  \begin{itemize}
  \item We want the editor to customize itself to the language we are working
    with.
  \item A display of R code should differ from Lisp or Java
  \item Emacs will guess the mode you want from the file extension (if you
    open a file)
  \item Or you can specify the mode while inside the session

  \end{itemize}
\end{frame}

\begin{frame}[containsverbatim]
  \frametitle{Emacs "Major Mode" system}
  \begin{itemize}
  \item Emacs automatically chooses the mode for you, from the file
    name you opened.
  \item Here's how to manually change the mode, if you want to. (M is
    the Meta key, usually ``Alt''

    \begin{itemize}
    \item M-x text-mode
    \item M-x c-mode
    \item M-x tex-mode
    \item M-x R-mode 
    \end{itemize}
  \end{itemize}
\end{frame}

\begin{frame}[containsverbatim]
  \frametitle{Major Modes try to help with Formatting}
  \begin{itemize}
    \item The TAB key should indent a line.
    \item Position the cursor ANYWHERE in a line and hit tab (Not
      necessary to be at start of line).
    \item SHOULD Indent properly.
    \item Usually, languages want a certain number of spaces, and NOT
      the literal TAB to be inserted. Emacs handles that.
    \item The shortcut paragraph re-shape key, M-q, can be configured
      to work very carefully, to re-shape long lines.  It works very
      well in the major modes for LaTeX, for example. Not so well in
      ESS at the current time.
   \end{itemize}
\end{frame}


\begin{frame}[containsverbatim]
  \frametitle{Comment Characters}
  \begin{itemize}
      \item Programming languages differ in the comment symbol
      \item Major modes generally know and can insert and arrange them
        for you.
      \item Hit TAB to force re-indentation
      \item Comment and un-comment blocks.
        \begin{enumerate}
          \item Highlight a block of text
          \item Hit the keystroke M-;. 
          \item In an R file, that inserts two pound signs at front of
            lines
          \item If you insert one pound sign, text will get pushed to the
            right when you hit TAB.
          \item If you insert 3 pound signs, text will get pushed
            flush left
       \end{enumerate}
  \end{itemize}
\end{frame}

\section{No Learning Curve}

\begin{frame}[containsverbatim]
  \frametitle{Understand the History of Emacs}
  \begin{itemize}
  \item Emacs was created in the days before

    \begin{itemize}
    \item mice
    \item multi-windowed ``desktop'' environments
    \item Windows or Macintosh computers
    \end{itemize}
  \item Can still be used in a ``keyboard'' only world

    \begin{itemize}
    \item Combinations using (C) Control, (S) Shift
      (S), and (M) Meta key (usually ``Alt'')
    \item These are are featured in the Emacs tutorial
    \item Note the prevalence of ``Emacs cheat sheets'' floating about on
      the Internet
    \end{itemize}
  \end{itemize}
\end{frame}

\begin{frame}[containsverbatim]
  \frametitle{Properly Understood, the Emacs Learning Curve is NOT STEEP}
  \begin{itemize}
  \item If you have a 
    \begin{itemize}
      \item a mouse, and 
      \item Windows, Mac, Linux, or Unix
    \end{itemize}
   then Emacs will run with pull down menus.
  \item It is tremendously programmable, most people can use Emacs
    comfortably even if they don't customize it.
  \item Settings that seem unusual can usually be changed, and I'll
    handle most of that for you.
  \end{itemize}
\end{frame}

\begin{frame}[containsverbatim, allowframebreaks]
  \frametitle{Don't Listen To People Who Want You To Memorize C-x C-g C-h whatever...}
  \begin{itemize}
  \item You can learn key-stroke combinations later, if you need them.
  \item But you don't need most of them now.
  \item What's easier to remember?

    \begin{itemize}
    \item Click the Edit Menu, Choose select all, or
    \item Type C-x h (Control-x, release both keys, then the letter ``h'')
    \end{itemize}
  \item What's easier to remember?

    \begin{itemize}
    \item Hit the ``page down'' key on the keyboard for a few seconds and
      watch a file fly by, or
    \item Type ``M->'' (Alt, Shift and the greater-than sign) to go to
      the bottom
    \item I usually just hit ``page down'' for a moment. Yesterday I
      had R output that was 150,000 lines and M-> would have been
      useful.
    \end{itemize}
  \end{itemize}
\end{frame}

\begin{frame}[containsverbatim, allowframebreaks]
  \frametitle{Example of Previous}
  \begin{itemize}
  \item Here is an example from ``Experienced Programmers Introduction to
    Emacs'' \url{http://weatherall.4all2u.com/work/emacs.htm}
  \end{itemize}
  \begin{lstlisting}[basicstyle={\small},tabsize=2]
    Now to a more conventional learning order
    C-a 	move to beginning of line
    C-e 	move to end of line
    M-< 	move to the beginning of the file
    M-> 	move to the end of the tutorial.
    C-k 	delete rest of line
    
    <Delete> 	delete the character just before the cursor
    C-d 	delete the next character after the cursor
  \end{lstlisting}

  \begin{itemize}
  \item To do those things, I use the mouse, delete key, and the arrow keys. 
  \item I don't use Control key combinations that are unique to Emacs
    very often.  If I did, I'd feel helpless when I had to use some
    other editor. The HOME and END keys work fine.
    Or arrow keys. I would only need C-a if I had no arrow keys, and no
    mouse!
  \item I don't deny the keystrokes might be faster, \emph{if you remember them}. 
  \item New users should not become preoccupied
    with memorizing C, M and S sequences. Note frequent needs, then
    learn those keystrokes.
  \end{itemize}
\end{frame}

\begin{frame}[containsverbatim, allowframebreaks]
  \frametitle{Another Intimidating Example}

  Useful Emacs bits 'n' bobs \url{http://www.insectnation.org/howto/emacs-tips}

  \begin{lstlisting}
    Here's a few handy key combinations I wish I'd been told about when I started using emacs:

    C-g or ESC ESC ESC: cancel minibrowser session
    C-s: search for text
    C-r: search for text backwards
    C-%: replace text (press space to okay each suggested instance
    C-M-%: replace regex (press space to okay each suggested instance
    C-[space]: place mark (I don't use this...see above re. region selection)
    C-w: cut region (std emacs)
    M-w: copy region (std emacs)
    C-k: cut line
    C-y: paste (yank)
    C-l: recentre buffer window vertically around active line
    C-x C-f: open (find) file in buffer
    C-x C-s: save buffer to file
    C-x C-w: save buffer to new file
    C-x C-z: stop (pause) emacs process (re-start with shell fg)
    C-x k: kill buffer
    C-x b: change active buffer
    C-x 1: display only current buffer window
    C-x 2: split buffer window
    C-x 0: kill current buffer pane
    C-x 4 f: open file in new window
    C-x 4 b: open buffer in new window
    C-x 5 2: open new frame
    C-x 5 0: kill current frame
    C-x 5 f: open file in new frame
    C-x 5 b: open buffer in new frame
    C-x o: switch between active sub-window
    C-x D: enter the very nifty dired-mode
    M-x global-font-lock-mode: toggle syntax highlighting
    M-x perl-mode: a lot of code highlights fairly well in Perl mode
    M-x goto-line: jump to line
    F10: access menus in text mode (via the minibuffer)
    M-x byte-compile-file .emacs: compile .emacs or another elisp file for faster operation
  \end{lstlisting}


\end{frame}

\begin{frame}[containsverbatim, allowframebreaks]
  \frametitle{Avoid the Emacs Tutorial Until Later}
  \begin{itemize}
  \item Emacs provides a nearly complete desktop environment.
  \item The Emacs tutorial emphasizes memorization of keystrokes that
    are not truly necessary with current Emacs.
  \item Instead, \emph{listen to me!} Use my init file. Make Emacs behave
    in a more familiar way.
  \item We don't have to sacrifice on any of the other very powerful features
    of Emacs 
  \item Emacs enthusiasts are able to remember hundreds of key
    combinations, but I can't. 
  \end{itemize}
\end{frame}

\begin{frame}[containsverbatim, allowframebreaks]
  \frametitle{CUA Mode: Good Fix for Common "Problem"}
  \begin{itemize}
  \item People are accustomed to cut and paste keys (from MS or Macintosh):


    \begin{tabular}{|c|c|}
      \hline 
      keystroke &
      result\tabularnewline
      \hline 
      \hline 
      C-x &
      cut\tabularnewline
      \hline
      C-c &
      copy\tabularnewline
      \hline 
      C-v &
      paste\tabularnewline
      \hline 
      C-z &
      undo\tabularnewline
      \hline 
    \end{tabular}

  \item Emacs default keystrokes are


    \begin{tabular}{|c|c|}
      \hline 
      keystroke &
      result\tabularnewline
      \hline 
      \hline 
      C-w &
      cut (``wipe'' in Emacs-speak)\tabularnewline
      \hline 
      M-w &
      copy\tabularnewline
      \hline 
      C-y &
      paste (``yank'' in Emacs-speak)\tabularnewline
      \hline 
      C-\_ &
      undo\tabularnewline
      \hline 
    \end{tabular}

  \item Fix. Turn on ``CUA mode'' in Emacs. Will make C-x, C-v, C-c, C-z
    do what you expect.
  \item I turn on CUA mode in my init file, but now it can also be
    done in a pull down menu. Look under Options.
  \item Potential problem: Emacs has many keystroke combinations
    that use C and M and anything we do to ``override'' them may cause
    trouble. But most of these CUA bugs have been solved.
  \item As an example of a bug, try to read the Emacs Tutorial with CUA mode
    on. (Click the Emacs Help menu, choose Tutorial). The tutorial presumes
    you can type C-v to go to the next ``page'' (in the old-fashioned
    terminal sense of displaying the next ``screen sized chunk''). It
    is necessary to turn off CUA mode to make any progress inside the
    Emacs tutorial.
  \item Of course, the fact that the first thing in the Emacs tutorial is
    how to see the ``next screen'' with C-v illustrates my previous
    point, that the Emacs tutorial is addressed to a much different user
    audience than this presentation.
  \end{itemize}
\end{frame}

\begin{frame}[containsverbatim, allowframebreaks]
  \frametitle{CUA mode Rectangular Selection Super-Power}
  \begin{itemize}
  \item Sometimes, you need to highlight a rectangular column of text
    and copy and paste it.
  \item This is a secret power of CUA mode! Very convenient
  \item Here is the CUA way.

    \begin{itemize}
    \item Move the cursor ``top left'' of a rectangle you want to select
    \item Hit ``C-return''
    \item Use arrow keys to expand rectangle. Highlight will display a rectangle.
    \end{itemize}
  \item Cut, copy, paste \emph{just work!}.
  \end{itemize}
\end{frame}

\begin{frame}[containsverbatim]
  \frametitle{Customizations I Recommend}
  \begin{itemize}
  \item When Emacs starts, it reads a series of configuration files

    \begin{itemize}
    \item Some are ``system-wide''
    \item Some are user specific (in the directory ~/.emacs.d)
    \end{itemize}
  \item I customize the settings thusly.  

    \url{http://pj.freefaculty.org/Software/Emacs/init.el}

  \item An administrator can drop that file in the Emacs install
    folder under site-lisp/site.start.d
  \item A user should create a folder .emacs.d in their home folder
    and drop init.el in there

    \begin{itemize}
    \item Emacs will create ``~/.emacs.d'' folder for you if you use
      pull down menu ``Options -> Save Options''
    \item .emacs.d is a ``hidden'' folder, a file manager may not want
      to show it to you.  Try harder.
    \end{itemize}
  \end{itemize}
\end{frame}

\begin{frame}[containsverbatim, allowframebreaks]
  \frametitle{Here's my Emacs init file (on my laptop 2015-02-04}

  \begin{lstlisting}[basicstyle={\tiny},language=Lisp,numbers=left,numberstyle={\tiny},tabsize=2]
 ;; Paul Johnson

;; UPDATE: Because Emacs ESS changes made this MUCH easier, my
;; re-work gets shorter :=)

;; INSTRUCTIONS. Copy this file into the 
;; folder ~/emacs.d, so its name would be ~/emacs.d/init.el
;; or copy it into the Emacs site-start.d folder.

;; R USER PREVIEW.
;; Here are my special features related to ESS with R.

;; 1. Shift+Enter will send the current line to R, and it will start R
;; if it is not running.  ESS 13-05 chose instead CTL+Enter, which I
;; DO NOT want because it conflicts with CUA mode. 
;; 2. R will assume the current working directory is the document directory.
;; 3. R runs in its own "frame" 
;; 4. Emacs help pops up in its own frame. 


;; JUSTIFICATION.  The intention is to make Emacs work more like a
;; "modern" GUI editor. 
;; See my companion lecture 
;; "Emacs Has No Learning Curve"
;; http://pj.freefaculty.org/guides/Rcourse


;; Paul Johnson <pauljohn@ku.edu>
;; 2012-11-24
;;
;; Conflicts between Emacs-ESS and SAS usage forced me to make
;; some changes. And for no benefit, as SAS mode still not great.
;; I had to cut out a lot of framepop stuff.
;;


;; 2013-07-10 TODO: Find out if this is still necessary, or for
;; which versions of windows.

;; Section I. Windows OS work-arounds
;; 20141222: seems unnecessary now
;;(if (eq system-type 'windows-nt)
;;    (setq use-file-dialog nil)) 
;; There's been a chronic problem with file selection dialogs on Windows
;; Maybe you commment previous out and see if your Windows is fixed.


;; Org-mode with R doesn't work without this
;; http://orgmode.org/worg/org-contrib/babel/how-to-use-Org-Babel-for-R.html
(custom-set-variables
 '(ansi-color-names-vector ["#2e3436" "#a40000" "#4e9a06" "#c4a000" "#204a87" "#5c3566" "#729fcf" "#eeeeec"])
 '(cua-auto-tabify-rectangles nil)
 '(custom-enabled-themes (quote (whiteboard)))
 '(org-babel-load-languages (quote ((emacs-lisp . t) (R . t))))
 '(org-confirm-babel-evaluate nil)
 '(tab-stop-list (quote (8 16 24 32 40 48 56 64 72 80 88 96 104 112 120 4))))

;; Section II. Keyboard and mouse customization

;; IIA: make mouse selection work in the usual Mac/Windows way

(delete-selection-mode t) ; delete seleted text when typing

;; In Linux, I see weirdness in Emacs 24 with paste and clipboard. Confusing!
;; http://stackoverflow.com/questions/13036155/how-to-to-combine-emacs-primary-clipboard-copy-and-paste-behavior-on-ms-windows
(setq select-active-regions t)
(global-set-key [mouse-2] 'mouse-yank-primary)  ; make mouse middle-click only paste from primary X11 selection, not clipboard and kill ring.

;; highlight does not alter KILL ring
(setq mouse-drag-copy-region nil)


;; IIB: keyboard customization

;; CUA mode is helpful not only for copy and paste, but also C-Enter is rectangle select
(cua-mode t) ; windows style binding C-x, C-v, C-c, C-z
(setq cua-auto-tabify-rectangles nil) ;; Don't tabify after rectangle commands
;;20130717(setq cua-keep-region-after-copy t) ;; Selection remains after C-c

;; write line numbers on left of window
;; Caution: if you do this, it makes Emacs much slower!
;;(global-linum-mode 1) ; always show line numbers


;; Section III. Programming conveniences:
(show-paren-mode t) ; light-up matching parens
(global-font-lock-mode t) ; turn on syntax highlighting
(setq text-mode-hook (quote (turn-on-auto-fill text-mode-hook-identify)))

;; Section IV. ESS Emacs Statistics

;; start R in current working directory, don't let R ask user
(setq ess-ask-for-ess-directory nil)

;; ESS 13.05 default C-Ret conflicts with CUA mode rectangular selection.
;; Change shortcut to use Shift-Return
;; Suggested by Vitalie Spinu ESS-help email 2013-05-15
;; Revision suggested 2013-09-30 to co-exist with Windows Emacs
;; and the load order which has ess after user init file.
(eval-after-load "ess-mode"
 '(progn
   (define-key ess-mode-map [(control return)] nil)
   (define-key ess-mode-map [(shift return)] 'ess-eval-region-or-line-and-step))
)

;; create a new frame for each help instance
;; (setq ess-help-own-frame t)
;; If you want all help buffers to go into one frame do:
(setq ess-help-own-frame 'one)

;; I want the *R* process in its own frame
;; This was a broken feature in ESS, fixed now. Helps massively!
(setq inferior-ess-own-frame t)
;;(setq inferior-ess-same-window nil)


;; In Spring 2012, we noticed ESS SAS mode doesn't work well
;; at all on Windows, that lead to removal of lots of stuff
;; I really liked. Even then, we couldn't get much satisfaction.
;;
;; Following was needed for that, otherwise, it is not needed
;; (load "ess-site")
;; (ess-sas-global-unix-keys)


;; ;; ;; Section V. Customize the use of Frames. Try to make new content
;; ;; ;; appear in wholly new frames on screen.
;; ;; ;; 
;; ;; ;; V.A: Discourage Emacs from splitting "frames", encourage it to pop up new
;; ;; ;; frames for new content.
;; ;; ;; see: http://www.gnu.org/software/emacs/elisp/html_node/Choosing-Window.html
;; (setq pop-up-frames t)
;; (setq special-display-popup-frame t)
(setq split-window-preferred-function nil) ;discourage horizontal splits
;; (setq pop-up-windows nil)


;; V.C: Make files opened from the menu bar appear in their own
;; frames. This overrides the default menu bar settings.  Opening an
;; existing file and creating new one in a new frame are the exact
;; same operations.  adapted from Emacs menu-bar.el
(defun menu-find-existing ()
  "Edit the existing file FILENAME."
  (interactive)
  (let* ((mustmatch (not (and (fboundp 'x-uses-old-gtk-dialog)
                              (x-uses-old-gtk-dialog))))
         (filename (car (find-file-read-args "Find file: " mustmatch))))
    (if mustmatch
        (find-file-other-frame filename)
      (find-file filename))))
(define-key menu-bar-file-menu [new-file]
  '(menu-item "Open/Create" find-file-other-frame
	      :enable (menu-bar-non-minibuffer-window-p)
	      :help "Create a new file"))
(define-key menu-bar-file-menu [open-file]
  '(menu-item ,(purecopy "Open File...") menu-find-existing
              :enable (menu-bar-non-minibuffer-window-p)
              :help ,(purecopy "Read an existing file into an Emacs buffer")))


;; V.D  Open directory list in new frame.
(define-key menu-bar-file-menu [dired]
  '(menu-item "Open Directory..." dired-other-frame
	      :help "Read a directory; operate on its files (Dired)"
	      :enable (not (window-minibuffer-p (frame-selected-window menu-updating-frame)))))

;; Section VI: Miscellaneous convenience

;; Remove Emacs "splash screen"
;; http://fuhm.livejournal.com/
(defadvice command-line-normalize-file-name
  (before kill-stupid-startup-screen activate)
  (setq inhibit-startup-screen t))
(setq inhibit-splash-screen t)


;; Show file name in title bar
;; http://www.thetechrepo.com/main-articles/549
(setq frame-title-format "%b - Emacs")

;; I'm right handed, need scroll bar on right (like other programs)
;;(setq scroll-bar-mode-explicit t)
;;(set-scroll-bar-mode `right) 

;; Make Emacs scroll smoothly with down arrow key.
;; 2011-10-14
;; faq 5.45 http://www.gnu.org/s/emacs/emacs-faq.html#Modifying-pull_002ddown-menus
(setq scroll-conservatively most-positive-fixnum)

;; adjust the size of the frames, uncomment this, adjust values
;;(setq default-frame-alist '((width . 90) (height . 65)))


;; Remember password when connected to remote sites via Tramp
;; http://stackoverflow.com/questions/840279/passwords-in-emacs-tramp-mode-editing
;; Emacs "tramp" service (ssh connection) constantly
;; asks for the log in password without this
(setq password-cache-expiry nil)

;; Section : Emacs shells work better
;; http://snarfed.org/why_i_run_shells_inside_emacs
(setq ansi-color-for-comint-mode 'filter)
(setq comint-prompt-read-only t)
(setq comint-scroll-to-bottom-on-input t)
(setq comint-scroll-to-bottom-on-output t)
(setq comint-move-point-for-output t)


(custom-set-faces
 ;; custom-set-faces was added by Custom.
 ;; If you edit it by hand, you could mess it up, so be careful.
 ;; Your init file should contain only one such instance.
 ;; If there is more than one, they won't work right.
 )
  \end{lstlisting}


\end{frame}

\begin{frame}[containsverbatim, allowframebreaks]
  \frametitle{Selection of text}
  \begin{itemize}
  \item The Mouse works!
  \item But most Emacs documentation wants you to act as though you
    don't have one. Use C-space to begin a selection,
    followed by movement keys$\ldots$.  
  \item Emacs defaults changed Emacs-23 and 24 to make Emacs work in a
    more usual way. My my init file has only a minor fiddle now.

    \begin{lstlisting}
      (delete-selection-mode t) ; delete seleted text when typing
    \end{lstlisting}


  \end{itemize}
\end{frame}

\begin{frame}[containsverbatim]
  \frametitle{Shortcuts I do remember}
  \begin{itemize}
  \item There are some keyboard shortcuts I do remember 

    \begin{itemize}
    \item C-s: I search frequently, this is easier than the menu Edit -> Search
      (C-r searches in reverse)
    \item M-\%: Query replace 
    \item C-M-\%: Regular-Expression Query replace (can't live without them)
    \item M-x: execute commands.
    \item C-g : get out of jail free. If you mistakenly start something in the
      minibuffer, C-g gets out of it.
    \item M-q : re-shapes an ill-formed paragraph
    \item M-; : for a selected region, will ``comment'' and ``uncomment''
      all lines.
    \end{itemize}
  \item But for things that I only do once every 6 months, it is easier to
    find them in a menu or type them by name. 
  \end{itemize}
\end{frame}

\begin{frame}[containsverbatim, allowframebreaks]
  \frametitle{Using M-x}
  \begin{itemize}
  \item M-x tells the minibuffer to get ready for a command
  \item TAB completion works inside there, so if you remember the first few
    letters of a command, this is an easy way to do things.
  \item Example: indent code in a region

    \begin{itemize}
    \item Highlight some text and type: \lstinline!M-x indent-region!
    \item After that, the indentation occurs, but Emacs minibuffer reminds me
      I could have used a short cut:


      \begin{lstlisting}
        You can run the command 'indent-region' with C-M-\
      \end{lstlisting}


    \end{itemize}
  \item Repeat previous using TAB completion feature of minibuffer

    \begin{itemize}
    \item Highlight some text and type: \lstinline!M-x inden!


      Stop typing, hit the tab key once or twice. A set of legal completions
      pops up, one of which is ``indent-region''. 

    \item Middle-click the one you want to choose it
    \item Hit enter
    \end{itemize}
  \item Example: accidentally open a read only file. Some file saved from
    R sessions are marked ``read only'' automatically. Tedious! I want
    to edit!. Emacs can try to make it writable.

    \begin{itemize}
    \item Run \lstinline!M-x toggle-read-only!
    \item Emacs minibuffer pops up a reminder 
      \begin{lstlisting}
        You can run the command "toggle-read-only" with C-x C-q
      \end{lstlisting}


      \begin{itemize}
      \item (thanks very much. Which is more memorable. ``toggle-read-only''
        or C-x C-q?)
      \end{itemize}
    \item Maybe the joke's on me here. 

      \begin{itemize}
      \item A reader of the first draft of this presentation pointed out that
        in the Emacs status bar, there is a click-able thing to toggle ``read
        only.''
      \item But it is hidden. Look in the left side of the status bar, for two
        percent signs \%\%
      \item hover the mouse over the first percent sign, and pops up ``buffer
        is read-only. Mouse-1 toggles''
      \item It works fine, \emph{as long as you know its there. }(Reminds me of
        the secret doors in \emph{DOOM)}
      \end{itemize}
    \end{itemize}
  \end{itemize}
\end{frame}

\begin{frame}[containsverbatim, allowframebreaks]
  \frametitle{Tools -> Compare. Example of a "power feature" in Emacs}
  \begin{itemize}
  \item What's the difference between two text files?
  \item Emacs has a great compare tool that relies on the GNU program
    ``diff'', which is available for all systems. Linux and Mac have
    it, Windows people need to install it separately.

  \item There is a Pull Down menu Tools -> Compare (Ediff). Apply to
    buffers or files on disk
    \end{enumerate}
  \item 2 things happen.

    \begin{itemize}
    \item The different parts of the 2 buffers are color highlighted
    \item A small window pops up in which I can Navigate the differences. 

      \begin{itemize}
      \item Type ``n'' to step to the next difference between the buffers, or 
      \item ``p'' for previous.
      \end{itemize}
    \end{itemize}
  \item Don't forget to hit ``q'' to close the compare setup, and then it
    makes you type ``yes'' in full to escape.
  \end{itemize}
\end{frame}


\section{ESS}

\begin{frame}[containsverbatim]
  \frametitle{R-mode is provided by the ESS package}

  \includegraphics[width=11cm]{importfigs/emacs-R-mode.png}

  1. ESS menu 2. Blue R starter (same as M-x R) 3. Send one line to
  R 4. Send selection to R 5. Send current function to R. 

\end{frame}

\begin{frame}[containsverbatim]
  \frametitle{Start an R Session Within Emacs}
  \begin{itemize}
  \item Hit the big blue R button
  \item Hopefully, R starts inside a buffer, which Emacs refers to in the
    buffer list as {*}R{*}
  \item If Emacs can't find R in your system, you need to do some configuration
    work so that the R/bin folder is added to your system path (I have
    instructions for that in the crmda computing documentation)
  \end{itemize}
\end{frame}

\begin{frame}[containsverbatim]
  \frametitle{Some People Prefer "Splits" with Code and *R* like so}

  \includegraphics[width=11cm]{importfigs/emacs-R-splits.png}

  The bottom part is the ESS ``R inferior mode'', a terminal in which
  R is running

\end{frame}

\begin{frame}[containsverbatim]
  \frametitle{I prefer Separate Frames with R}

  \includegraphics[width=13cm]{importfigs/emacs-R-frames.png}

\end{frame}

\begin{frame}[containsverbatim]
  \frametitle{Start Emacs within the Desired Working Directory}
  \begin{itemize}
  \item Don't Start Emacs from an applications menu: It won't know where to
    look for files.
  \item Keep your R files where you want them, and use the File
    Manager or Terminal to open the R file with Emacs.
  \item The easiest way to do that is to copy an R file into the
    desired directory, and then open that file with Emacs. 
  \item Open an R file, Emacs automatically knows to turn on R-mode (ESS Menu
    \& Buttons)
  \end{itemize}
\end{frame}

\begin{frame}[containsverbatim]
  \frametitle{I'm Willing to Fight For Frames}
  \begin{itemize}
  \item I want separate Frames!
  \item Spawning more Frames is very un-Emacs-like because it goes against
    the decades-old tradition of Emacs (during which time one could only
    have one Frame on one terminal).
  \item Much of my Emacs init file is aimed at forcing Emacs to Start Frames
    with new content, rather than simply starting new buffers that
    bump my current content out of the way.
  \item This is an area where reasonable people can disagree. I don't
    want Emacs to be my window manager.
  \end{itemize}
\end{frame}

\begin{frame}[containsverbatim, allowframebreaks]
  \frametitle{Using ESS button bar}
  \begin{itemize}
  \item I often use the big blue R button \includegraphics{importfigs/emacs-ess-BlueR.png}
    to start R. 

    \begin{itemize}
    \item Sometimes I use M-x R, just to prove I still can. (There was a time
      when Emacs for Windows had no button bar, so the M-x R was the only
      way.)
    \end{itemize}
  \item After starting R within Emacs, run ``getwd()'' to make sure the
    working directory is correct.
  \item The function evaluator \includegraphics{importfigs/emacs-ess-Function.png}
    is really handy. While revising a function in code, hit that button
    and Emacs sends the whole function to {*}R{*}.
  \end{itemize}
\end{frame}

\begin{frame}[containsverbatim, allowframebreaks]
  \frametitle{ESS Menu}
  \begin{topcolumns}%{}

    \column{6cm}


    \includegraphics[width=5cm]{importfigs/emacs-ess-Menu.png}


    \column{6cm}
    \begin{itemize}
    \item The ESS menu teaches you the keyboard shortcuts. You can decide if
      you like them.
    \item Eval versus Eval-and-go. 

      \begin{itemize}
      \item Eval commands send instructions to {*}R{*}. 
      \item Eval-and-go send the instructions and transfer the focus to {*}R{*}
      \end{itemize}
    \end{itemize}
  \end{topcolumns}%{}
\end{frame}

\begin{frame}[containsverbatim, allowframebreaks]
  \frametitle{Roxygen}
  \begin{itemize}
  \item Roxygen is a framework for generating R documentation files
  \item In the ``olden days'' (last year), an R package would have separate
    files for R functions and the help files that went with them.
  \item Problem: programmers found it tedious to maintain the separate help
    files
  \item The package roxygen2 (by Hadley Wickham, Peter Danenberg, and Manuel
    Eugster) addresses that by 

    \begin{itemize}
    \item creating a ``language'' for writing documentation inside R code
      files
    \item providing functions to translate the result into documentation files
    \end{itemize}
  \item Even if you aren't writing a package, the Roxygen style might be a
    nice way to prepare your documents. 
  \item Suppose a function is declared like so


    \begin{lstlisting}
      myFabFunc <- function(x, y, z, a){
        x * y * z * a
      }
    \end{lstlisting}


  \item Click on the first line and from the ESS menu, choose Roxygen and
    ``Update/generate template''. 
  \item Observe a skeleton is created in which the function can be described
  \end{itemize}
  \begin{lstlisting}[basicstyle={\small}]

    ##' .. content for \description{} (no empty lines) ..
    ##'
    ##' .. content for \details{} ..
    ##' @title 
    ##' @param x 
    ##' @param y 
    ##' @param z 
    ##' @param a 
    ##' @return 
    ##' @author pauljohn
    myFabFunc <- function(x, y, z, a){
      x * y * z * a
    }
  \end{lstlisting}

  \begin{itemize}
  \item If we then fill in the sections, we will have pretty good documentation
    for the function and it will be package-ready (more or less).
  \end{itemize}
\end{frame}

\begin{frame}[containsverbatim, allowframebreaks]
  \frametitle{ESS Magic I Don't use Menus and Buttons For}
  \begin{itemize}
  \item My shortcut: Shift+Enter sends the current line of code to {*}R{*}.
    I think that's convenient.

    \begin{itemize}
    \item Instead, you could :

      \begin{itemize}
      \item Hit the ESS button with one arrow: \includegraphics{importfigs/emacs-ess-SingleLine.png}
      \item Or type: C-x C-n.
      \end{itemize}
    \end{itemize}
  \item Edit an existing function object:

    \begin{itemize}
    \item C-c C-d


      The mini buffer answers, \lstinline!Object to Edit:!

    \item type the name of some function, such as ``lm''
    \item Emacs opens a buffer called ``username.lm.R'', which shows the code
      for linear models
    \item We can edit that file, and then re-load it into R with
    \item C-c C-l
    \item After that, when you run functions that access lm, your new version
      of that function will be used.
    \item To test this, run \lstinline!C-c C-d lm! and in the beginning insert
      ``\code{browser()}''.
    \item Then run lm, or simply ``example(lm)''. When the computations come
      to the line where you put in ``browser()'', the {*}R{*} terminal
      will stop and give you a chance to interact with the inside of that
      function. 
    \item To me, that is a fun feature. However, I just got an email that indicates
      it is controversial. (See \emph{ESS- Emacs Speaks Statistics}'' Version
      12.04-4, Section 3.4: Philosophies for using ESS. The current ESS
      core team believes that the R code files are the ``real'' stuff
      on which we should be working, not fiddling about with functions in
      memory.)
    \end{itemize}
  \end{itemize}
\end{frame}

\begin{frame}[containsverbatim, allowframebreaks]
  \frametitle{ESS Help Mode}
  \begin{itemize}
  \item While reading a help file in Emacs, it is possible to run the example
    code
  \item Example. In {*}R{*} window, ask for help on lm


    \begin{lstlisting}
      > ?lm
    \end{lstlisting}


  \item Emacs will open the help file and the menu at the top will have an
    ESS-help menu
  \item In the help file, move the cursor to one of the examples at the end
    of the file.
  \item Click the ESS-help menu and choose ``Eval Line'' or one of the other
    options.
  \end{itemize}
\end{frame}

\begin{frame}[containsverbatim, allowframebreaks]
  \frametitle{Transcript Mode. Record Keeping.}
  \begin{itemize}
  \item In the perfect world, here is what happens.

    \begin{enumerate}
    \item Save the R code file.
    \item Close R. Then Re-start R (so we have a clean canvas).
    \item Step through your R code file line-by-line to be sure everything runs
      properly
    \item Save the output file with the extension {*}.Rout (that means it becomes
      a ``transcript'' file).


      Some people use the extension {*}.Rt for transcript files, but for
      me it doesn't work (don't know why, yet)

    \end{enumerate}
  \item As long as you have the R code and the transcript file, you have all
    the records you need for future reference.
  \item What if you are working on an R code file, but you type some commands
    into the {*}R{*} buffer interactively?
  \item I do that all the time. While I'm testing code, I experiment in the
    {*}R{*} buffer, then copy commands back to the R code file. 
  \item I should save the {*}R{*} buffer in a file {*}.Rout. Let's suppose
    I did.
  \item The following problem may arise. 

    \begin{itemize}
    \item I open a new R session
    \item I run the R code file that seemed good, but then it fails!
    \item How can that be? It worked before?
    \end{itemize}
  \item Obviously, something I typed interactively in the {*}R{*} buffer did
    not get copied into the code file properly.
  \item Solution strategy. 

    \begin{itemize}
    \item Open the {*}.Rout file in Emacs. 
    \item Emacs recognizes that this is a transcript file
    \item Emacs won't let me edit that transcript because it is marked ``read
      only'', so I either have to copy it to a new file or M-x toggle-read-only.
    \item Then

      \begin{enumerate}
      \item Select All of the text (either Menu Edit -> Select All, or type ``C-x
        h'')
      \item In the ESS-trans menu, choose ``Clean''.
      \end{enumerate}
    \item Viola! That will erase all the output from that file, and leave behind
      only the commands that were run.
    \end{itemize}
  \item Along the same lines, from a reviewer of these slides, I received
    an email about an ESS feature. Since ESS 12.03, there is an R automatic
    output ``scrub and paste.'' The key sequence is \lstinline!C-u C-u C-y!. 

    \begin{itemize}
    \item Highlight some material in an {*}R{*} output window. It should include
      commands and output.
    \item Copy that selection (To copy, use either 1) M-w or 2) C-c from CUA
      or 3) Edit menu).
    \item Move the focus to an R code file where you want to paste in only the
      R commands, no output. 

      \begin{itemize}
      \item If you do the ordinary paste, with C-y, you get all that messy output.
      \item But this key sequence will ``clean'' and paste the resulting R commands.


        \begin{lstlisting}
          C-u C-u C-y
        \end{lstlisting}


      \item Note, this requires the traditional C-y (``yank'') to paste, not
        the CUA C-v. 
      \end{itemize}
    \end{itemize}
  \end{itemize}
\end{frame}


\section{Conclusion}

\begin{frame}[containsverbatim, allowframebreaks]
  \frametitle{Emacs in Retrospect}
  \begin{itemize}
  \item I think the documentation and publicity that goes with Emacs does
    it great dis-service. 

    \begin{itemize}
    \item Bill Venables, is credited with the quip: ``The first five years
      of Emacs are the worst; after that, it's just difficult.''
    \end{itemize}
  \item When I first tried Emacs, I looked through the tutorial and I thought
    ``this really is trouble.'' 
  \item I've told people that ``Emacs is like Democracy,'' adapting the
    comment that Winston Churchill made to the House of Commons, 1947-11-11,
    ``Democracy is the worst form of government, except for all those
    other forms that have been tried from time to time.``
  \item Almost all of the really smart people I know use Emacs.
  \end{itemize}
\end{frame}

\begin{frame}
  \frametitle{Do I Really Believe There is No Learning Curve?}
  \begin{itemize}
  \item Of course not. 
  \item But I'll get more hits in Google than with my original title, ``Emacs
    Learning Curve 77\% Less Steep Than Previously Believed'' \includegraphics[width=0.5cm]{importfigs/smilie.jpg}
  \item I honestly believe that if people are willing to try Emacs ``my way,''
    they can use it comfortably and benefit from many powerful features. 
  \item It is not necessary to memorize a raft of key sequences, even though
    many Emacs experts do so.
  \end{itemize}
\end{frame}

\begin{frame}
  \frametitle{Emacs is Extensible}
  \begin{itemize}
  \item I'm not denying the fact that 

    \begin{itemize}
    \item Customizing Emacs is possible, enriching, and frustrating (all at
      the same time)
    \end{itemize}
  \item Some parts of Emacs are still difficult for me

    \begin{itemize}
    \item The help menus are still as confusing to me as ever. 
    \item I find the Customization menus still very difficult. 
    \end{itemize}
  \item If a person is willing to use Emacs with the init file I provide,
    using Emacs can be a very reasonable experience. 
  \end{itemize}
\end{frame}

\begin{frame}[containsverbatim]
  \frametitle{Useful Websites}

  Introduction to Emacs: \url{http://cs.earlham.edu/~psg/tutorials/vtwm_emacs/part2.html}

\end{frame}
\end{document}
