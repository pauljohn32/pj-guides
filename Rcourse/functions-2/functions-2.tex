%%%%%%%%%%%%%%%%%%%%%%%%%%%%%%%%%%%%%%%%%%%%%%%%%%%%%%%%%%%%%%%%%
% Filename: template.Rnw
%   Author: Paul Johnson
%
%%%%%%%%%%%%%%%%%%%%%%%%%%%%%%%%%%%%%%%%%%%%%%%%%%%%%%%%%%%%%%%%%
% Revision History --
%  2011-02-08 2011-05-22  2012-09-18
%%%%%%%%%%%%%%%%%%%%%%%%%%%%%%%%%%%%%%%%%%%%%%%%%%%%%%%%%%%%%%%%%
\documentclass[10pt,english]{beamer}
\usepackage{lmodern}
\renewcommand{\sfdefault}{lmss}
\renewcommand{\ttdefault}{lmtt}
\usepackage[T1]{fontenc}
\usepackage[utf8]{inputenc}
\usepackage{listings}
\setcounter{secnumdepth}{3}
\setcounter{tocdepth}{3}
\usepackage{url}
\usepackage{graphicx}


\makeatletter
%%%%%%%%%%%%%%%%%%%%%%%%%%%%%% Textclass specific LaTeX commands.
\usepackage{Sweavel}
 \newenvironment{topcolumns}{\begin{columns}[t]}{\end{columns}}

%%%%%%%%%%%%%%%%%%%%%%%%%%%%%% User specified LaTeX commands.
\usepackage{dcolumn}
\usepackage{booktabs}

% use 'handout' to produce handouts
%\documentclass[handout]{beamer}
\usepackage{wasysym}
\usepackage{pgfpages}
\newcommand{\vn}[1]{\mbox{{\it #1}}}\newcommand{\vb}{\vspace{\baselineskip}}\newcommand{\vh}{\vspace{.5\baselineskip}}\newcommand{\vf}{\vspace{\fill}}\newcommand{\splus}{\textsf{S-PLUS}}\newcommand{\R}{\textsf{R}}


\usepackage{graphicx}
\usepackage{listings}
\lstset{tabsize=2, breaklines=true,style=Rstyle}
%\usetheme{Warsaw}
% or ...

%\setbeamercovered{transparent}
% or whatever (possibly just delete it)

\mode<presentation>
{
  \usetheme{Antibes}
  \usecolortheme{dolphin} %dark blues
}

%
%In document Latex options:
\fvset{listparameters={\setlength{\topsep}{0em}}}
\def\Sweavesize{\normalsize}
\def\Rcolor{\color{black}}
\def\Rbackground{\color[gray]{0.95}}


\setbeamertemplate{frametitle continuation}[from second]
\renewcommand\insertcontinuationtext{...}


%=============================================================================
%


\title[Functions II] % (short title, use only with long paper titles)
{Troubleshooting for R Functions}

\subtitle{Fixing Mistakes, Preventing Them}

\author[Johnson] { Paul E. Johnson\inst{1}\inst{2}}

\institute[University of Kansas]{\inst{1}
  University of Kansas, Department of
  Political Science \inst{2} Center for Research Methods and Data Analysis}

\date[2015] % (optional, should be abbreviation of conference name)
{2015}

\subject{More advanced R function advice}

%====================================

\begin{document}

\AtBeginSection[]{

  \frame<beamer>{ 

    \frametitle{Outline}   

    \tableofcontents[currentsection,currentsubsection] 

  }

}


% In document Latex options:
\fvset{listparameters={\setlength{\topsep}{0em}}}

\def\Sweavesize{\normalsize}
\def\Rcolor{\color{black}}
\def\Rbackground{\color[gray]{0.95}}

\begin{Schunk}
\begin{Sinput}
 ## Other settings I like
 options(device = pdf)
 options(useFancyQuotes = FALSE) 
 op <- par() 
 pjmar <- c(5.1, 5.1, 1.5, 2.1) 
 options(SweaveHooks=list(fig=function() par(mar=pjmar, ps=12)))
 pdf.options(onefile=F,family="Times",pointsize=12)
\end{Sinput}
\end{Schunk}




\begin{frame}
  \titlepage
\end{frame}



\begin{frame}
\frametitle{Outline}

\tableofcontents{}

\end{frame}

%===================================================
\section{debugging}


\begin{frame}[containsverbatim]
  \frametitle{"Debugger", Generally Speaking}
  \begin{itemize}
  \item Most programming languages have debuggers.
    \begin{itemize}
    \item "gdb" is the GNU debugger for C, C++, etc
    \end{itemize}
  \item Key idea: Run a program "inside" the debugger
  \item "break point": halt a program at a place in the code to
    inspect the condition of the variables.
  \item  "back trace": When a program crashes, recover the last few
    commands that were run. 
  \item Many programs can create "core files" that can inspected
    after a program crash
  \end{itemize}
\end{frame}


\begin{frame}[containsverbatim, allowframebreaks]
  \frametitle{R has Interactive Debugging}
  \begin{description}
  \item [debug(func)] An R function that asks the interpreter to stop
    whenever a function named "func" is called. 
    
  \item Caution: The function must be "in" the R session.
    
   \item Example
\begin{Schunk}
\begin{Sinput}
debug(lm)
mod1 <- lm(y1 <- x1 + x2, data = mydata)
\end{Sinput}
\end{Schunk}
   
   \item [browser()] Inserting the function "browser()" into your code creates a
      break point. R interpreter will act as if we had run debug().
      
\begin{Schunk}
\begin{Sinput}
myGiantFn <- function(x, y, z, a, b){
  qq <- x + y
  browser()
  }
\end{Sinput}
\end{Schunk}    
      
      
    \end{description}
\end{frame}


\begin{frame}[containsverbatim]
  \frametitle{To Learn More}  
  \begin{itemize}
  \item  R Core Team, Writing R Extensions (run "help.start()")
  \item  John Chambers, \emph{Software for Data Analysis}
  \item  Hadley Wickham, \emph{Advanced R}, Chap 9
  \end{itemize}
\end{frame}



\begin{frame}[containsverbatim, allowframebreaks]
  \frametitle{What to do inside the debugger}
  \begin{itemize}

  \item Type into the R console
    \begin{description}
    \item [{n <Ret>}] next. Steps "into" the next level of code
    \item [{<Ret>}] Similar to next, but does not step "into" a for loop
    \item [{c<Ret>}] continue (starts R running from there)
    \item [{where}] recent commands (usu. after trouble)
    \item[{Q}] quit debugging
      
   \end{description}
   
  \item Important feature: User can Run R commands
    \begin{itemize}
    \item [ls()] displays currently available object
    \item [print(x)] displays contents of x (same as typing "x")
    \end{itemize}
    
  \item In ESS tracebug mode, use these keys in the source code file
    (M means the Meta (Alt) key. All are capitalized): 
    \begin{itemize}
    \item [{M-N}] Next step
    \item [{M-C}] continue (starts R running from there)
    \item [{M-U}] Step up one frame (get out of for loop)
    \item [{M-Q}] continue (starts R running from there)
    \item see \url{http://ess.r-project.org/Manual/ess.html#ESS-tracebug}
    \end{itemize}
 \end{itemize}
\end{frame}


\begin{frame}[containsverbatim, allowframebreaks]
  \frametitle{debug}

  \begin{itemize}
  \item Try running
  \end{itemize}
\begin{Schunk}
\begin{Sinput}
 set.seed(234234)
 y <- rnorm(100)
 x <- rnorm(100)
 debug(lm)
 m1 <- lm( y ~ x )
\end{Sinput}
\end{Schunk}

 \begin{itemize}
  \item In the R terminal, computation will stop at the start of lm
  \item The whole function will print out, then we step into it
  \end{itemize}
  

\begin{Schunk}
  \begin{Soutput}
> m1 <- lm(y ~ x)
debugging in: lm(y ~ x)
debug: {
    ret.x <- x
    ret.y <- y
    cl <- match.call()
    mf <- match.call(expand.dots = FALSE)
    m <- match(c("formula", "data", "subset", "weights", "na.action",
        "offset"), names(mf), 0L)
    ## SNIP 100s of lines cut
    z
}

  \end{Soutput}
\end{Schunk}
\end{frame}




\begin{frame}[containsverbatim, allowframebreaks]
  \frametitle{Interact with lm in the Debugger/Browser}
\begin{itemize}
   
  \item "n" steps into a lower level of detail (inside "if" or "for")  
  \item Just hit Enter after first "n" to stay in same level
  \begin{Schunk}
    \begin{Soutput}
     
Browse[2]> n
debug at <tmp>#5: ret.x <- x
Browse[2]> n
debug at <tmp>#6: ret.y <- y
Browse[2]> 
debug at <tmp>#7: cl <- match.call()
Browse[2]> 
debug at <tmp>#8: mf <- match.call(expand.dots = FALSE)
Browse[2]> 
debug at <tmp>#9: m <- match(c("formula", "data", "subset", "weights", "na.action", 
    "offset"), names(mf), 0L)
Browse[2]> 
debug at <tmp>#11: mf <- mf[c(1L, m)]
Browse[2]> 
debug at <tmp>#12: mf$drop.unused.levels <- TRUE
%$
 \end{Soutput}
\end{Schunk}     

\item Inspect objects inside the running function
  \begin{Schunk}
    \begin{Soutput}
Browse[2]> ls()
 [1] "cl"          "contrasts"   "data"        "formula"     "method"
 [6] "mf"          "model"       "na.action"   "offset"      "qr"
[11] "ret.x"       "ret.y"       "singular.ok" "subset"      "weights"
[16] "x"           "y"
Browse[2]> model
[1] TRUE
Browse[2]> mf
lm(formula = y ~ x)
Browse[2]> formula
y ~ x
Browse[2]> contrasts
NULL
Browse[2]> method
[1] "qr"
\end{Soutput}
\end{Schunk}
\end{itemize}
\end{frame}


\begin{frame}[containsverbatim, allowframebreaks]
  \frametitle{How do you insert "browser()" into the middle of a function?}
  \begin{itemize}
    
  \item Insert "browser()" in the middle of your function, re-load the function,
    and run the command that calls the function.
    
  \item If you want to insert browser into a function from a package,
    can do in a few ways.
    
   \item R allows "in memory" changes to a function. Consider \texttt{fix()}.

     \begin{Schunk}
       \begin{Sinput}
fix(lm)
       \end{Sinput}
     \end{Schunk}

   \item The program you are using to interact with R will decide how to open that
     function.  
     \begin{itemize}
     \item In an R console in Linux, the
     function opens in vi. Edit, then close to launch it back into the session.
     \item Emacs 24.4 refuses to edit the file until I run
\begin{Sinput}       
M-x server start
\end{Sinput}       
      After that, a new buffer pops up with the code. 
    \item If you want to edit functions in this way, Emacs-ESS
      suggests instead
        \begin{itemize}
          \item \texttt{C-c C-e C-d}  to "dump"
             a function as a text file
          \item \texttt{C-c C-l <Ret>} to load the edited function back.
         \end{itemize}
         
     \item Rstudio pops open a new editor frame. The "save" button
       at the bottom loads the function into the R session.
      \end{itemize}
 \item Changes we make in this way are purely temporary.
 \end{itemize}
\end{frame}


\section{Running Through Errors}

\begin{frame}[containsverbatim, allowframebreaks]
  \frametitle{try and tryCatch}
  \begin{itemize}
    \item Suppose a command inside your function causes a function to fail.
      Ordinarily, R processing will stop
    \item Most programming have some ``try'' ``catch'' logic so that
      commands may fail without destroying the progress of the larger function.
    \item Examples
      \begin{itemize}
        \item Try to run something, ignore error entirely
\begin{Sinput}          
try(file.symlink(normalizePath("../ppimages"), tmpdir), silent = TRUE)
\end{Sinput}
    \item Try to run something, check for class of result
\begin{Sinput}          
h11a <- try(solve(g11),silent=TRUE)

if (inherits(h11a,"try-error")) {
          h11a <- mpinv(g11)
}
\end{Sinput}

   \item tryCatch can put those chores together. This runs lrm, if
     that fails it returns a vector of NA
\begin{Sinput}     
out1 <- tryCatch( lrm( ...whatever ...), error = function(x) {rep(NA, 6)})
\end{Sinput}  
\end{itemize}
\end{itemize}
\end{frame}



\section{Defensive Coding}

\begin{frame}[containsverbatim]
  \frametitle{Advice 1: checkUsage}
  A Description of the Problem
  \begin{itemize}

  \item R searches from inside out for variables and functions
  \item If a function accidentally uses an object from the larger
    environment, you may have a hard-to-find bug. 

  \item Best practice: isolate function from "accidental" access to
    environment variables.
  \end{itemize}
\end{frame}


\begin{frame}[containsverbatim]
  \frametitle{An Example of the Problem}

   \begin{itemize}

  \item Here's a function with an undefined variable "x"
\begin{Schunk}
\begin{Sinput}
 myf <- function(z){
    log(z) + 2*x
 }
\end{Sinput}
\end{Schunk}


  \item It will fail if no x exists
\begin{Schunk}
  \begin{Sinput}
rm(x)
myf(1)
  \end{Sinput}
  \begin{Soutput}
Error in myf(1) : object 'x' not found
  \end{Soutput}
\end{Schunk}

\item But if x happens to be floating about, it succeeds
\begin{Schunk}
\begin{Sinput}
 x <- c(1,2,3)
 myf(1)
\end{Sinput}
\begin{Soutput}
[1] 2 4 6
\end{Soutput}
\end{Schunk}


\item Dangers are obvious, yes? Other people will get errors if you
give them myf().

\end{itemize}
\end{frame}




\begin{frame}[containsverbatim]
  \frametitle{Use codetools::checkUsage}
  \begin{itemize}

  \item install codetools (by Luke Tierney)

  \item Run checkUsage

    \begin{quotation}
       ‘checkUsage’ checks a single R closure.  Options control which
     possible problems to report.  The default settings are moderately
     verbose. $\ldots$

     ‘checkUsageEnv’ and ‘checkUsagePackage’ are convenience functions
     that apply ‘checkUsage’ to all closures in an environment or a
     package.
    \end{quotation}



\begin{Schunk}
\begin{Sinput}
 library(codetools)
 rm(x)
 checkUsage(myf)
\end{Sinput}
\begin{Soutput}
<anonymous>: no visible binding for global variable 'x' (functions-2.Rnw:435)
\end{Soutput}
\end{Schunk}


\end{itemize}
\end{frame}






\section{Profiling (for speed)}


\begin{frame}[containsverbatim]
  \frametitle{Optimization Intuitions and Facts}
  
  \begin{itemize}
  \item New programmers sometimes fixate on simple ideas about what is "fast"
  \item Experienced programmers develop 
    \begin{itemize}
    \item intuitions about what might be "fast"
      \item suspicion that their intuitions are probably wrong
    \end{itemize}
   \item General Advice
       \begin{itemize}
       \item Design code that is clear, well organized, separates
         chores understandably. 
       \item If it really is SLOW, run a profiler to find out why.
      \end{itemize}
   \item Donald Knuth's comment widely cited
      "We should forget about small efficiencies, say about 97\% of the time: premature optimization is the root of all evil."
  \end{itemize}
\end{frame}

\begin{frame}[containsverbatim, allowframebreaks]
  \frametitle{Faster R}
  \begin{itemize}
  \item I started a webpage "Faster R: Things to not forget" about ways to
  make R go faster \url{http://pj.freefaculty.org/blog/?p=122}
  \item Coming away from this, we have a few clear principles and a
    lot of examples.
  \end{itemize}
\end{frame}


\begin{frame}[containsverbatim, allowframebreaks]
  \frametitle{Clear Principles}
   \begin{itemize}
   \item Allocating memory is slow, so do it infrequently if possible
   \item Take advantage of optimizations that the R Core Team has
     built into the R code base
   \item Some functions that "seem" harmless are unnecessary and can
     slow down a function.
   \end{itemize}
 \end{frame}
 

\begin{frame}[containsverbatim, allowframebreaks]
  \frametitle{Vectorizing}
  \begin{itemize}
  \item This is a way of pushing a lot of calculations out of the R
    interpreter over to the C functions
  \item Lets review the example \url{http://pj.freefaculty.org/R/WorkingExamples/efficiency-vectorize.html}
  \end{itemize}
\end{frame}
   
 

\begin{frame}[containsverbatim, allowframebreaks]
  \frametitle{Matrix algebra}
  \begin{itemize}
  \item If you need  $X^TX$, don't write R like
\begin{Sinput}
  t(X) %*% X
\end{Sinput}
  \item DO use crossprod. It can be called with either
\begin{Sinput}
  crossprod(X)
\end{Sinput}   
\begin{Sinput}
  crossprod(X, X)
\end{Sinput}   
  \item Why faster?  Note the result is symmetric.  Hence it is only
    necessary to calculate one triangle
  \item similar function tcrossprod for $X y^T$.
  \end{itemize}
\end{frame}
   

\begin{frame}[containsverbatim, allowframebreaks]
  \frametitle{Similarly, rowSum}
  \begin{itemize}
  \item Novices might write
\begin{Sinput}
  rowtotal <- X[ ,1] + X[ ,2] + X[ ,3] 
\end{Sinput}
  \item Better to ruse
\begin{Sinput}
  rowtotal <-  rowSum(X)
\end{Sinput}
  \item Why? rowSum sends all of the work to the faster C side in one
    block, wherease the first one calls the "+" function over and over.  
    
  \end{itemize}
\end{frame}


\begin{frame}[containsverbatim, allowframebreaks]
  \frametitle{Smarter matrix algebra}
  \begin{itemize}
    \item Wood, Generalized Additive Models, observes that this is
      very slow:
\begin{Sinput}
      A %*% B %*% y
\end{Sinput}
      and this is much faster:
\begin{Sinput}      
      A %*% (B %*% y)
\end{Sinput}
 A and B are NxN, while y is Nx1. 
   \item Each matrix calculation requires a certain number of floating
     point operations.  ($B y$) reduces to an $N x 1$ thing, thus
     eliminating any need to do an $N \times N$ calculation.
  \end{itemize}
\end{frame}

\begin{frame}[containsverbatim, allowframebreaks]
  \frametitle{Examples: Removing columns from a data frame}
  \begin{itemize}
   \item Example: Removing columns from a data frame. 
     \begin{itemize}
     \item This is slow if there are many columns to remove:
 \begin{Sinput}
 dat$A <- NULL
 dat$B <- NULL
 dat$C <- NULL
 dat$D <- NULL
 \end{Sinput}
     
     \item That's slow because data structure has to be repeatedly copied.
     \item Much faster to find columns you want to keep, and keep them
       in one step
\begin{Sinput}
datnames <- colnames(dat)
keepnames <- setdiff(datnames, c("A", "B", "C", "D"))
dat <- dat[ , keepnames]
\end{Sinput} 
\end{itemize}
\end{itemize}
\end{frame}
 
 
 
\begin{frame}[containsverbatim, allowframebreaks]
  \frametitle{Examples: appending rows on bottom of a data frame}

  \begin{itemize}
    \item Many R users are drawn to an idiom like this:
      \begin{Sinput}
dat <- matrix(NA, nrow = 1, ncol = 6)
for(j in 1:100){
    ## some huge calculation creates a row
    x <- massive(y, z)
    dat <- rbind(dat, x)
}
 \end{Sinput}
   \item We find that the repeated use of rbind is poor
   \item One should either 
     \begin{enumerate}
       \item Create a big empty matrix and fill in the rows, or
       \item Create a list of result vectors, and rbind them together
         all at once.
       \end{enumerate}
     \end{itemize}
   Lets discuss the example here: \url{http://pj.freefaculty.org/R/WorkingExamples/efficiency-stackListItems-01.R}
   
\end{frame}


\begin{frame}[containsverbatim, allowframebreaks]
  \frametitle{Notice what they do in BLAS}
  \begin{itemize}
  \item Try to grab memory in big chunks, not little bites.
    
  \item BLAS functions will Avoid creating new matrices
  \item They DO re-use matrices, writing results on top of input
  \end{itemize}
\end{frame}


\begin{frame}[containsverbatim, allowframebreaks]
  \frametitle{Examples: appending rows on bottom of a data frame}
  \begin{itemize}
  \item Matrices generally faster than data frames or lists
  \item An R object with "names" will be processed more slowly
  than the same thing with names

\item Vectorization is Good. The [ function is bad
\item Warning against "for loops" is mostly driven by concern
  about use of "[". (That is, apply or lapply may not actually
  be faster).
\end{itemize}
\end{frame}


\begin{frame}[containsverbatim, allowframebreaks]
  \frametitle{Using the R Profiler}
  \begin{itemize}
    \item Begin with a chunk of R code that seems slow
    \item Before that chunk, insert code like this
 \begin{Sinput}
 Rprof("Jones_Project-2015-1.out")  
 \end{Sinput}
    \item Your R code chunk should run for a "while" so the profiler
      has a chance to study it. Give it a big enough sample of the
      slow function's use so it can diagnose the problem.
     \item After your chunk, add this
 \begin{Sinput}
 Rprof(NULL)
 summaryRprof("Jones_Project-2015-1.out")  
 \end{Sinput}
\end{itemize}
\end{frame}


\begin{frame}[containsverbatim, allowframebreaks]
  \frametitle{Chasing Amelia}
  \begin{itemize}
    \item December 2012, we noticed Amelia (an MI) package became very slow
    \item Alex Schoemann made a test case and we profiled it:
      
   \url{http://pj.freefaculty.org/scraps/profile/prof-puzzle-1.Rout}
   
   
 \end{itemize}
\end{frame}


\begin{frame}[containsverbatim, allowframebreaks]
  \frametitle{system.time() is not the greatest}
  but it is the easiest.
  
  Wrap any code inside
  
  system.time( )
  
  and you'll get back a clock time estimate:
\begin{Sinput}
  > system.time(
+     impA <- amelia(datM, m = 5, idvars="group", p2s = 0)
+     )
   user  system elapsed 
141.805   0.360 142.151 
\end{Sinput}  
\end{frame}




%% \section{Profiling (for memory)}

%% \begin{frame}[containsverbatim]
%%   \frametitle{I'm sorry I did not write anything out yet}
%%   \begin{itemize}
%%     \item I'm so sorry
%%   \end{itemize}
%% \end{frame}


% % ________________________________________
% \begin{frame}[containsverbatim]
%   \frametitle{ff}
%   \begin{itemize}
%   \end{itemize}
% \end{frame}




%% \begin{frame}[containsverbatim, allowframebreaks]
%%   \frametitle{Use "debug" Package }
%%   \begin{itemize}
%%   \item debug package (by Mark V. Bravington (BVR))
%%     \begin{quotation}
%%       Debugger for R functions, with code display,
%%       graceful error recovery, line-numbered conditional
%%       breakpoints, access to exit code, flow control, and
%%       full keyboard input.
%%     \end{quotation}

%%   \end{itemize}
%% \end{frame}


%% \begin{frame}[containsverbatim, allowframebreaks]
%%   \frametitle{\texttt{mtrace()} is like \texttt{browser()} on Steroids}
%%   \begin{itemize}

%%     \item Install the package called "debug"
%%     \item mtrace(lm) has the effect of editing lm and inserting
%%       browser() at the top.

%%     \item Use mtrace like so

%%     \begin{Sinput}
%%       library(debug)
%%       mtrace(lm)
%%       mod <- lm(y ~ x)
%%     \end{Sinput}
%%   \end{itemize}
%% \end{frame}

%% \begin{frame}[containsverbatim]
%%   \frametitle{\texttt{mtrace()}}
%%     \begin{itemize}
%%     \item That "pops open" a new Code Browser window for the lm function
%%     \item In the *R* window, a prompt appears
%%       \begin{Soutput}
%%         D(1) >
%%       \end{Soutput}

%%     \item User should hit the "Enter" key at the  D(1) > prompt, and
%%       \begin{itemize}
%%       \item R runs the next command and
%%       \item  the Code Browser window's highlight bar steps to the next
%%         command that will run.
%%     \end{itemize}

%%   \end{itemize}
%%   \begin{description}
%%   \item [{mtrace.off()}] turns off mtrace processes
%%   \item [{qqq()}]  exit from BVR debugger
%%   \end{description}
%% \end{frame}


%===================================================
\end{document}
